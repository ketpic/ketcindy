%%% title slideluamanualJ
\documentclass{article}
\usepackage{geometry}
\geometry{paperwidth=80mm,paperheight=57mm}
\usepackage{luatexja}
\usepackage{pict2e}
\usepackage{ketpic2e,ketlayer2e}
\usepackage{ketslide}
\usepackage{amsmath,amssymb}
\usepackage{bm,enumerate}
\usepackage{graphicx}
\usepackage{color}
\usepackage[colorlinks=true,linkcolor=blue,filecolor=blue]{hyperref}

\usepackage{ketslide}

\usepackage{qrcode}
\usepackage{setspace}
\def\deg#1{#1^{\circ}}
\def\bs{$\backslash$}
\newcommand{\monthday}{0814}

\definecolor{slidecolora}{cmyk}{0.98,0.13,0,0.43}
\definecolor{slidecolorb}{cmyk}{0.2,0,0,0}
\definecolor{slidecolorc}{cmyk}{0.2,0,0,0}
\definecolor{slidecolord}{cmyk}{0.2,0,0,0}
\definecolor{slidecolore}{cmyk}{0,0,0,0.5}
\definecolor{slidecolorf}{cmyk}{0,0,0,0.5}
\definecolor{slidecolori}{cmyk}{0.98,0.13,0,0.43}
\def\setthin#1{\def\thin{#1}}
\setthin{0}
\newcounter{pagectr}
\setcounter{pagectr}{1}
\newcommand{\slidepage}[1][\monthday-]{%
\setcounter{ketpicctra}{18}%

\begin{layer}{118}{0}
\putnotew{160}{-\theketpicctra.05}{\small#1\thepage/\pageref{pageend}}
\end{layer}

}

\ketslideinit

\setmargin{15}{15}{5}{5}
\pagestyle{empty}

\begin{document}

\begin{layer}{120}{0}
\putnotese{0}{0}{{\Large\bf
\color[cmyk]{0.98,0.13,0,0.43}

\begin{layer}{120}{0}
{\Huge \putnotes{67}{10}{KeTSlideの使い方}}
\putnotes{67}{40}{Setsuo Takato}
\putnotes{67}{50}{KeTCindy Center}
\putnotes{67}{60}{2024.12.23}
\end{layer}

}
}
\end{layer}

\def\mainslidetitley{22}
\def\ketcletter{slidecolora}
\def\ketcbox{slidecolorb}
\def\ketdbox{slidecolorc}
\def\ketcframe{slidecolord}
\def\ketcshadow{slidecolore}
\def\ketdshadow{slidecolorf}
\def\slidetitlex{6}
\def\slidetitlesize{1.3}
\def\mketcletter{slidecolori}
\def\mketcbox{yellow}
\def\mketdbox{yellow}
\def\mketcframe{yellow}
\def\mslidetitlex{62}
\def\mslidetitlesize{2}

\color{black}
\Large\bf\boldmath
\addtocounter{page}{-1}

\renewcommand{\eda}[2][\theedactr]{%
\setcounter{edactr}{#1}
\Ltab{\theedawidth mm}{[#1]\ #2}%
\addtocounter{edactr}{1}%
}
\setthin{0.1}
\def\colorthin{\color[cmyk]{\thin,\thin,\thin,\thin}}
\def\dint{\displaystyle\int}
\newcommand{\dpar}[2]{\dfrac{\partial #1}{\partial #2}}
\newcommand{\dps}{\displaystyle}
\newcommand{\dpint}{\displaystyle\int}
\renewcommand{\slidepage}[1][s]{%
\setcounter{ketpicctra}{18}%
\hypersetup{linkcolor=black}%
\begin{layer}{130}{0}
\putnotee{125}{-\theketpicctra.05}{\small\monthday-\thepage/\pageref{pageend}}
\end{layer}\hypersetup{linkcolor=blue}
}
\newcommand{\setwidth}[1]{\setcounter{txtLiii}{#1}}
\newcommand{\adde}{\addtocounter{enm}{1}}
%%\newcommand{enminit}{\setcounter{enm}{1}}
\newcommand\pnp{(\theenm)}
\newcommand\pt{・}
\renewcommand{\monban}[1][\monthday]{%
#1-\themban\,%
\addtocounter{mban}{1}%
}
\renewcommand{\monbannoadd}[1][\monthday]{%
#1-\themban\,%
}
\newcommand{\eds}[1][1]{\setcounter{edactr}{#1}}%
\newcommand{\footslide}[3][]{\putnotee{5}{#2}{{\color{blue}\small *#1:#3}}}
\newcommand{\addrem}[1]{{\color{blue}{}\scriptsize\;\raisebox{3mm}{*#1}}}
\setcounter{figure}{1}
%%%%%%%%%%%%%%%%

%%%%%%%%%%%%%%%%%%%%

\mainslide{準備}


\slidepage[m]
%%%%%%%%%%%%%%%%

%%%%%%%%%%%%%%%%%%%%

\newslide{\ketcindy のインストール}

\vspace*{18mm}

\slidepage
\enminit
\textinit[120]

\begin{layer}{120}{0}
\addtext{8}{\pnp}{ketcindy homeからketcindyの最新版をダウンロード}\adde
\addtext{8}{\pnp}{doc/ketcinsysettings.cdyを実行}\adde
\end{layer}

%%%%%%%%%%%%%%%%

%%%%%%%%%%%%%%%%%%%%


\mainslide{layer環境と\bs addtext}


\slidepage[m]
%%%%%%%%%%%%%%%%

%%%%%%%%%%%%%%%%%%%%

\newslide{layer環境}

\vspace*{18mm}

\slidepage
\enminit
\textinit[120]

\begin{layer}{120}{0}
\addtext{8}{\ten}{\bs begin\{layer\}\{120\}\{60\}〜\bs end\{layer\}}
\addtext{16}{・}{横120mm,縦60mmの範囲で10mmごとに格子線を引く}
\addtext[8]{16}{・}{KeTSlideではlayer:\hspace{0mm}:\{120\}\{60\}〜end でよい}
\addtext{16}{・}{縦を0とすると格子を描かない}
\addtext{8}{\ten}{layer自体は高さ0のpicture環境}
\addtext{16}{・}{したがって行幅をとらない}
\end{layer}

%%%%%%%%%%%%%%%%

%%%%%%%%%%%%%%%%%%%%


\newslide{putnote}

\vspace*{18mm}

\slidepage
\enminit
\textinit[105]

\begin{layer}{120}{0}
\addtext{8}{\ten}{\bs putnote(s,se,...)\{8\}\{6\}\{要素\}}
\addtext{16}{・}{8,6を基準点として(s,se,...)方向に要素をおく}
\addtext{16}{・}{KeTSlideでは以下のように書いてもよい\\\hspace*{10mm}putnote:\hspace{0mm}:(s,se,...)\{8\}\{6\}\{要素\}}
\addtext[8]{16}{・}{figフォルダに\TeX の描画コードファイル(例えばfigure.tex)を置いたときは以下のように書いてもよい(0.6は縮小率)\\\hspace*{10mm}
putnote:\hspace{0mm}: se\{5\}\{6\}:\hspace{0mm}:figure,0.6}
\end{layer}

%%%%%%%%%%%%%%%

%%%%%%%%%%%%%%%%%%%%


\newslide{\bs addtext (1)}

\vspace*{18mm}

\slidepage
\enminit
\textinit[105]

\begin{layer}{120}{0}
\addtext{8}{\ten}{layer環境の中で文を順に置いていく}
\addtext{16}{}{\bs textinit[105]}
\addtext{24}{\%}{初期位置設定.横幅(default100)を決める}
\addtext{16}{}{layer:\hspace{0mm}:\{120\}\{0\}}
\addtext{16}{}{\bs addtext\{8\}\{\bs ten\}\{文1\}}
\addtext{24}{\%}{書き出しの水平位置 左から8mm}
\addtext{24}{\%}{次の行の書き出しの垂直位置は+8mm}
\addtext{16}{}{\bs addtext\{8\}\{\bs ten\}\{文2\}}
\addtext{16}{}{end}
\end{layer}

%%%%%%%%%%%%%%%

%%%%%%%%%%%%%%%%%%%%


\newslide{\bs addtext (2)}

\vspace*{18mm}

\slidepage
\enminit
\textinit[105]

\begin{layer}{120}{0}
\addtext{8}{\ten}{\bs addtext[8]\{8\}\{\bs ten\}\{文または図\}}
\addtext{16}{\ten[0.5]}{最初のオプション引数[8]は前の行からの改行幅(通常は8mm)への追加分}
\addtext[8]{24}{}{前の行が複数行の時などに用いる}
\addtext{16}{\ten[0.5]}{\bs ten[s]は\$\bs bullet\$のs倍(デフォルトは0.9)}
\addtext{8}{\ten}{番号付けは,enmカウンタで制御}
\addtext{16}{\ten[0.5]}{最初に\bs enminit(=\bs setcounter\{enm\}\{1\})}
\addtext{16}{\ten[0.5]}{\bs addtext\{8\}\{(\bs theenm)\}\{...\}\bs addenm}
\addtext{8}{\ten}{テキストに\bs verbは使えない}
\end{layer}

%%%%%%%%%%%%%%%%

%%%%%%%%%%%%%%%%%%%%


\mainslide{段階的表示}


\slidepage[m]
%%%%%%%%%%%%

%%%%%%%%%%%%%%%%%%%%

\newslide{段階的表示の方法}

\vspace*{18mm}

\slidepage
\enminit
\def\bed{\theenm.}
\textinit[105]

\begin{layer}{120}{0}
\addtext{8}{\bed}{\%repeat=に段階数を入れる(決まってから)}\adde
\addtext{16}{}{{\large\%}repeat=4}
\addtext{8}{\bed}{\Ltab{40mm}{{\large\%}[1]:\hspace{0mm}:要素}1段のみ}\adde
\addtext{8}{\bed}{\Ltab{40mm}{{\large\%}[2,-]:\hspace{0mm}:要素}2段以降}\adde
\addtext{8}{\bed}{\Ltab{40mm}{{\large\%}[-,2]:\hspace{0mm}:要素}2段まで}\adde
\addtext{8}{\bed}{\Ltab{40mm}{{\large\%}[2..4]:\hspace{0mm}:要素}2段から4段まで}\adde
\end{layer}

%%%%%%%%%%%%

%%%%%%%%%%%%%%%%%%%%


\newslide{区分求積法と定積分(段階的表示の例)}

\vspace*{18mm}

\slidepage
\textinit[120]
\enminit

\begin{layer}{120}{0}
\addtext{8}{\pnp}{$\displaystyle\int_a^b f(x)\,dx=\lim_{\varDelta x_k \to 0}\sum_{k}f(x_k)\varDelta x_k$}\adde
\end{layer}

%%%%%%%%%%%%%%%%

%%%%%%%%%%%%%%%%%%%%


\sameslide

\vspace*{18mm}

\slidepage
\textinit[120]
\enminit

\begin{layer}{120}{0}
\addtext{8}{\pnp}{$\displaystyle\int_a^b f(x)\,dx=\lim_{\varDelta x_k \to 0}\sum_{k}f(x_k)\varDelta x_k$}\adde
\putnotese{75}{25}{\scalebox{1.25}{%%% /Users/takatoosetsuo/polytech23.git/205-0904/fig/kubunkyuuseki8.tex 
%%% Generator=presen23206.cdy 
{\unitlength=1cm%
\begin{picture}%
(4.5,3)(-0.5,-0.5)%
\linethickness{0.008in}%%
\Large\bf\boldmath%
\small%
\linethickness{0.012in}%%
\polyline(0.5,2.043)(0.618,1.946)(0.732,1.859)(0.842,1.782)(0.948,1.713)(1.05,1.654)%
(1.148,1.605)(1.242,1.565)(1.332,1.534)(1.418,1.512)(1.5,1.5)(1.619,1.5)(1.734,1.521)%
(1.847,1.558)(1.958,1.605)(2.068,1.658)(2.177,1.712)(2.287,1.762)(2.396,1.804)(2.507,1.833)%
(2.619,1.844)(2.696,1.84)(2.776,1.83)(2.858,1.813)(2.942,1.789)(3.029,1.758)(3.118,1.72)%
(3.21,1.675)(3.304,1.624)(3.401,1.565)(3.5,1.5)%
%
\linethickness{0.008in}%%
\linethickness{0.004in}%%
\polyline(0.5,0)(0.5,2.043)%
%
\linethickness{0.008in}%%
\linethickness{0.004in}%%
\polyline(3.5,0)(3.5,1.5)%
%
\linethickness{0.008in}%%
\settowidth{\Width}{$y=f(x)$}\setlength{\Width}{-1\Width}%
\settoheight{\Height}{$y=f(x)$}\settodepth{\Depth}{$y=f(x)$}\setlength{\Height}{-0.5\Height}\setlength{\Depth}{0.5\Depth}\addtolength{\Height}{\Depth}%
\put(  3.600,  2.120){\hspace*{\Width}\raisebox{\Height}{$y=f(x)$}}%
%
\polyline(0.5,-0.05)(0.5,0.05)%
%
\settowidth{\Width}{$a$}\setlength{\Width}{-0.5\Width}%
\settoheight{\Height}{$a$}\settodepth{\Depth}{$a$}\setlength{\Height}{-\Height}%
\put(  0.500, -0.100){\hspace*{\Width}\raisebox{\Height}{$a$}}%
%
\polyline(3.5,-0.05)(3.5,0.05)%
%
\settowidth{\Width}{$b$}\setlength{\Width}{-0.5\Width}%
\settoheight{\Height}{$b$}\settodepth{\Depth}{$b$}\setlength{\Height}{-\Height}%
\put(  3.500, -0.100){\hspace*{\Width}\raisebox{\Height}{$b$}}%
%
\linethickness{0.004in}%%
\polyline(0.5,0)(0.5,2.043)(0.875,2.043)(0.875,0)%
%
\linethickness{0.008in}%%
\linethickness{0.004in}%%
\polyline(0.875,0)(0.875,1.76)(1.25,1.76)(1.25,0)%
%
\linethickness{0.008in}%%
\linethickness{0.004in}%%
\polyline(1.25,0)(1.25,1.562)(1.625,1.562)(1.625,0)%
%
\linethickness{0.008in}%%
\linethickness{0.004in}%%
\polyline(1.625,0)(1.625,1.502)(2,1.502)(2,0)%
%
\linethickness{0.008in}%%
\linethickness{0.004in}%%
\polyline(2,0)(2,1.625)(2.375,1.625)(2.375,0)%
%
\linethickness{0.008in}%%
\linethickness{0.004in}%%
\polyline(2.375,0)(2.375,1.796)(2.75,1.796)(2.75,0)%
%
\linethickness{0.008in}%%
\linethickness{0.004in}%%
\polyline(2.75,0)(2.75,1.833)(3.125,1.833)(3.125,0)%
%
\linethickness{0.008in}%%
\linethickness{0.004in}%%
\polyline(3.125,0)(3.125,1.716)(3.5,1.716)(3.5,0)%
%
\linethickness{0.008in}%%
\polyline(-0.5,0)(4,0)%
%
\polyline(0,-0.5)(0,2.5)%
%
\settowidth{\Width}{$x$}\setlength{\Width}{0\Width}%
\settoheight{\Height}{$x$}\settodepth{\Depth}{$x$}\setlength{\Height}{-0.5\Height}\setlength{\Depth}{0.5\Depth}\addtolength{\Height}{\Depth}%
\put(  4.050,  0.000){\hspace*{\Width}\raisebox{\Height}{$x$}}%
%
\settowidth{\Width}{$y$}\setlength{\Width}{-0.5\Width}%
\settoheight{\Height}{$y$}\settodepth{\Depth}{$y$}\setlength{\Height}{\Depth}%
\put(  0.000,  2.550){\hspace*{\Width}\raisebox{\Height}{$y$}}%
%
\settowidth{\Width}{O}\setlength{\Width}{-1\Width}%
\settoheight{\Height}{O}\settodepth{\Depth}{O}\setlength{\Height}{-\Height}%
\put( -0.050, -0.050){\hspace*{\Width}\raisebox{\Height}{O}}%
%
\end{picture}}%}}
\end{layer}


\sameslide

\vspace*{18mm}

\slidepage
\textinit[120]
\enminit

\begin{layer}{120}{0}
\addtext{8}{\pnp}{$\displaystyle\int_a^b f(x)\,dx=\lim_{\varDelta x_k \to 0}\sum_{k}f(x_k)\varDelta x_k$}\adde
\setwidth{60}
\addtext[8]{8}{\pnp}{$f(x)\varDelta x$を合計して極限をとればよい}\adde
\putnotese{75}{25}{\scalebox{1.25}{%%% /Users/takatoosetsuo/polytech23.git/205-0904/fig/kubunkyuuseki8b.tex 
%%% Generator=presen23206.cdy 
{\unitlength=1cm%
\begin{picture}%
(4.5,3)(-0.5,-0.5)%
\linethickness{0.008in}%%
\Large\bf\boldmath%
\small%
\linethickness{0.012in}%%
\polyline(0.5,2.043)(0.618,1.946)(0.732,1.859)(0.842,1.782)(0.948,1.713)(1.05,1.654)%
(1.148,1.605)(1.242,1.565)(1.332,1.534)(1.418,1.512)(1.5,1.5)(1.619,1.5)(1.734,1.521)%
(1.847,1.558)(1.958,1.605)(2.068,1.658)(2.177,1.712)(2.287,1.762)(2.396,1.804)(2.507,1.833)%
(2.619,1.844)(2.696,1.84)(2.776,1.83)(2.858,1.813)(2.942,1.789)(3.029,1.758)(3.118,1.72)%
(3.21,1.675)(3.304,1.624)(3.401,1.565)(3.5,1.5)%
%
\linethickness{0.008in}%%
\linethickness{0.004in}%%
\polyline(0.5,0)(0.5,2.043)%
%
\linethickness{0.008in}%%
\linethickness{0.004in}%%
\polyline(3.5,0)(3.5,1.5)%
%
\linethickness{0.008in}%%
\settowidth{\Width}{$y=f(x)$}\setlength{\Width}{-1\Width}%
\settoheight{\Height}{$y=f(x)$}\settodepth{\Depth}{$y=f(x)$}\setlength{\Height}{-0.5\Height}\setlength{\Depth}{0.5\Depth}\addtolength{\Height}{\Depth}%
\put(  3.600,  2.120){\hspace*{\Width}\raisebox{\Height}{$y=f(x)$}}%
%
\polyline(0.5,-0.05)(0.5,0.05)%
%
\settowidth{\Width}{$a$}\setlength{\Width}{-0.5\Width}%
\settoheight{\Height}{$a$}\settodepth{\Depth}{$a$}\setlength{\Height}{-\Height}%
\put(  0.500, -0.100){\hspace*{\Width}\raisebox{\Height}{$a$}}%
%
\polyline(3.5,-0.05)(3.5,0.05)%
%
\settowidth{\Width}{$b$}\setlength{\Width}{-0.5\Width}%
\settoheight{\Height}{$b$}\settodepth{\Depth}{$b$}\setlength{\Height}{-\Height}%
\put(  3.500, -0.100){\hspace*{\Width}\raisebox{\Height}{$b$}}%
%
\linethickness{0.016in}%%
\polyline(2,0)(2,1.625)(2.375,1.625)(2.375,0)%
%
\linethickness{0.008in}%%
\settowidth{\Width}{$x$}\setlength{\Width}{-1\Width}%
\settoheight{\Height}{$x$}\settodepth{\Depth}{$x$}\setlength{\Height}{-\Height}%
\put(  2.000, -0.150){\hspace*{\Width}\raisebox{\Height}{$x$}}%
%
\polyline(2,0)(2.006,-0.006)(2.012,-0.011)(2.019,-0.016)(2.025,-0.021)(2.032,-0.026)%
(2.039,-0.031)(2.046,-0.035)(2.053,-0.039)(2.06,-0.043)(2.068,-0.047)(2.075,-0.051)%
(2.083,-0.054)(2.09,-0.057)(2.098,-0.06)(2.106,-0.063)(2.114,-0.065)(2.122,-0.067)%
(2.13,-0.069)(2.138,-0.07)(2.146,-0.072)(2.154,-0.073)(2.163,-0.074)(2.171,-0.074)%
(2.179,-0.075)(2.187,-0.075)(2.196,-0.075)(2.204,-0.074)(2.212,-0.074)(2.221,-0.073)%
(2.229,-0.072)(2.237,-0.07)(2.245,-0.069)(2.253,-0.067)(2.261,-0.065)(2.269,-0.063)%
(2.277,-0.06)(2.285,-0.057)(2.292,-0.054)(2.3,-0.051)(2.307,-0.047)(2.315,-0.043)%
(2.322,-0.039)(2.329,-0.035)(2.336,-0.031)(2.343,-0.026)(2.35,-0.021)(2.356,-0.016)%
(2.363,-0.011)(2.369,-0.006)(2.375,0)%
%
\polyline(2,1.625)(1.99,1.616)(1.98,1.607)(1.97,1.598)(1.96,1.589)(1.95,1.579)(1.941,1.57)%
(1.931,1.56)(1.922,1.551)(1.913,1.541)(1.904,1.531)(1.895,1.521)(1.886,1.511)(1.877,1.501)%
(1.869,1.49)(1.86,1.48)(1.852,1.469)(1.844,1.459)(1.835,1.448)(1.828,1.437)(1.82,1.426)%
(1.812,1.415)(1.805,1.404)(1.797,1.393)(1.79,1.381)(1.783,1.37)(1.776,1.358)(1.769,1.347)%
(1.763,1.335)(1.756,1.323)(1.75,1.311)(1.744,1.299)(1.737,1.287)(1.732,1.275)(1.726,1.263)%
(1.72,1.251)(1.715,1.239)(1.71,1.226)(1.704,1.214)(1.699,1.201)(1.695,1.189)(1.69,1.176)%
(1.686,1.163)(1.681,1.151)(1.677,1.138)(1.673,1.125)(1.669,1.112)(1.665,1.099)(1.662,1.086)%
(1.659,1.073)(1.655,1.06)%
%
\polyline(1.655,0.565)(1.659,0.552)(1.662,0.539)(1.665,0.526)(1.669,0.513)(1.673,0.5)%
(1.677,0.487)(1.681,0.474)(1.686,0.461)(1.69,0.449)(1.695,0.436)(1.699,0.424)(1.704,0.411)%
(1.71,0.399)(1.715,0.386)(1.72,0.374)(1.726,0.362)(1.732,0.35)(1.737,0.337)(1.744,0.325)%
(1.75,0.313)(1.756,0.302)(1.763,0.29)(1.769,0.278)(1.776,0.267)(1.783,0.255)(1.79,0.244)%
(1.797,0.232)(1.805,0.221)(1.812,0.21)(1.82,0.199)(1.828,0.188)(1.835,0.177)(1.844,0.166)%
(1.852,0.155)(1.86,0.145)(1.869,0.134)(1.877,0.124)(1.886,0.114)(1.895,0.104)(1.904,0.094)%
(1.913,0.084)(1.922,0.074)(1.931,0.064)(1.941,0.055)(1.95,0.045)(1.96,0.036)(1.97,0.027)%
(1.98,0.018)(1.99,0.009)(2,0)%
%
\settowidth{\Width}{$f(x)$}\setlength{\Width}{-0.5\Width}%
\settoheight{\Height}{$f(x)$}\settodepth{\Depth}{$f(x)$}\setlength{\Height}{-0.5\Height}\setlength{\Depth}{0.5\Depth}\addtolength{\Height}{\Depth}%
\put(  1.630,  0.810){\hspace*{\Width}\raisebox{\Height}{$f(x)$}}%
%
\settowidth{\Width}{$\varDelta x$}\setlength{\Width}{0\Width}%
\settoheight{\Height}{$\varDelta x$}\settodepth{\Depth}{$\varDelta x$}\setlength{\Height}{-\Height}%
\put(  2.140, -0.150){\hspace*{\Width}\raisebox{\Height}{$\varDelta x$}}%
%
\polyline(-0.5,0)(4,0)%
%
\polyline(0,-0.5)(0,2.5)%
%
\settowidth{\Width}{$x$}\setlength{\Width}{0\Width}%
\settoheight{\Height}{$x$}\settodepth{\Depth}{$x$}\setlength{\Height}{-0.5\Height}\setlength{\Depth}{0.5\Depth}\addtolength{\Height}{\Depth}%
\put(  4.050,  0.000){\hspace*{\Width}\raisebox{\Height}{$x$}}%
%
\settowidth{\Width}{$y$}\setlength{\Width}{-0.5\Width}%
\settoheight{\Height}{$y$}\settodepth{\Depth}{$y$}\setlength{\Height}{\Depth}%
\put(  0.000,  2.550){\hspace*{\Width}\raisebox{\Height}{$y$}}%
%
\settowidth{\Width}{O}\setlength{\Width}{-1\Width}%
\settoheight{\Height}{O}\settodepth{\Depth}{O}\setlength{\Height}{-\Height}%
\put( -0.050, -0.050){\hspace*{\Width}\raisebox{\Height}{O}}%
%
\end{picture}}%}}
\end{layer}


\sameslide

\vspace*{18mm}

\slidepage
\textinit[120]
\enminit

\begin{layer}{120}{0}
\addtext{8}{\pnp}{$\displaystyle\int_a^b f(x)\,dx=\lim_{\varDelta x_k \to 0}\sum_{k}f(x_k)\varDelta x_k$}\adde
\setwidth{60}
\addtext[8]{8}{\pnp}{$f(x)\varDelta x$を合計して極限をとればよい}\adde
\addtext[8]{8}{\pnp}{面積でなくてもよい}\adde
\putnotese{75}{25}{\scalebox{1.25}{%%% /Users/takatoosetsuo/polytech23.git/205-0904/fig/kubunkyuuseki8b.tex 
%%% Generator=presen23206.cdy 
{\unitlength=1cm%
\begin{picture}%
(4.5,3)(-0.5,-0.5)%
\linethickness{0.008in}%%
\Large\bf\boldmath%
\small%
\linethickness{0.012in}%%
\polyline(0.5,2.043)(0.618,1.946)(0.732,1.859)(0.842,1.782)(0.948,1.713)(1.05,1.654)%
(1.148,1.605)(1.242,1.565)(1.332,1.534)(1.418,1.512)(1.5,1.5)(1.619,1.5)(1.734,1.521)%
(1.847,1.558)(1.958,1.605)(2.068,1.658)(2.177,1.712)(2.287,1.762)(2.396,1.804)(2.507,1.833)%
(2.619,1.844)(2.696,1.84)(2.776,1.83)(2.858,1.813)(2.942,1.789)(3.029,1.758)(3.118,1.72)%
(3.21,1.675)(3.304,1.624)(3.401,1.565)(3.5,1.5)%
%
\linethickness{0.008in}%%
\linethickness{0.004in}%%
\polyline(0.5,0)(0.5,2.043)%
%
\linethickness{0.008in}%%
\linethickness{0.004in}%%
\polyline(3.5,0)(3.5,1.5)%
%
\linethickness{0.008in}%%
\settowidth{\Width}{$y=f(x)$}\setlength{\Width}{-1\Width}%
\settoheight{\Height}{$y=f(x)$}\settodepth{\Depth}{$y=f(x)$}\setlength{\Height}{-0.5\Height}\setlength{\Depth}{0.5\Depth}\addtolength{\Height}{\Depth}%
\put(  3.600,  2.120){\hspace*{\Width}\raisebox{\Height}{$y=f(x)$}}%
%
\polyline(0.5,-0.05)(0.5,0.05)%
%
\settowidth{\Width}{$a$}\setlength{\Width}{-0.5\Width}%
\settoheight{\Height}{$a$}\settodepth{\Depth}{$a$}\setlength{\Height}{-\Height}%
\put(  0.500, -0.100){\hspace*{\Width}\raisebox{\Height}{$a$}}%
%
\polyline(3.5,-0.05)(3.5,0.05)%
%
\settowidth{\Width}{$b$}\setlength{\Width}{-0.5\Width}%
\settoheight{\Height}{$b$}\settodepth{\Depth}{$b$}\setlength{\Height}{-\Height}%
\put(  3.500, -0.100){\hspace*{\Width}\raisebox{\Height}{$b$}}%
%
\linethickness{0.016in}%%
\polyline(2,0)(2,1.625)(2.375,1.625)(2.375,0)%
%
\linethickness{0.008in}%%
\settowidth{\Width}{$x$}\setlength{\Width}{-1\Width}%
\settoheight{\Height}{$x$}\settodepth{\Depth}{$x$}\setlength{\Height}{-\Height}%
\put(  2.000, -0.150){\hspace*{\Width}\raisebox{\Height}{$x$}}%
%
\polyline(2,0)(2.006,-0.006)(2.012,-0.011)(2.019,-0.016)(2.025,-0.021)(2.032,-0.026)%
(2.039,-0.031)(2.046,-0.035)(2.053,-0.039)(2.06,-0.043)(2.068,-0.047)(2.075,-0.051)%
(2.083,-0.054)(2.09,-0.057)(2.098,-0.06)(2.106,-0.063)(2.114,-0.065)(2.122,-0.067)%
(2.13,-0.069)(2.138,-0.07)(2.146,-0.072)(2.154,-0.073)(2.163,-0.074)(2.171,-0.074)%
(2.179,-0.075)(2.187,-0.075)(2.196,-0.075)(2.204,-0.074)(2.212,-0.074)(2.221,-0.073)%
(2.229,-0.072)(2.237,-0.07)(2.245,-0.069)(2.253,-0.067)(2.261,-0.065)(2.269,-0.063)%
(2.277,-0.06)(2.285,-0.057)(2.292,-0.054)(2.3,-0.051)(2.307,-0.047)(2.315,-0.043)%
(2.322,-0.039)(2.329,-0.035)(2.336,-0.031)(2.343,-0.026)(2.35,-0.021)(2.356,-0.016)%
(2.363,-0.011)(2.369,-0.006)(2.375,0)%
%
\polyline(2,1.625)(1.99,1.616)(1.98,1.607)(1.97,1.598)(1.96,1.589)(1.95,1.579)(1.941,1.57)%
(1.931,1.56)(1.922,1.551)(1.913,1.541)(1.904,1.531)(1.895,1.521)(1.886,1.511)(1.877,1.501)%
(1.869,1.49)(1.86,1.48)(1.852,1.469)(1.844,1.459)(1.835,1.448)(1.828,1.437)(1.82,1.426)%
(1.812,1.415)(1.805,1.404)(1.797,1.393)(1.79,1.381)(1.783,1.37)(1.776,1.358)(1.769,1.347)%
(1.763,1.335)(1.756,1.323)(1.75,1.311)(1.744,1.299)(1.737,1.287)(1.732,1.275)(1.726,1.263)%
(1.72,1.251)(1.715,1.239)(1.71,1.226)(1.704,1.214)(1.699,1.201)(1.695,1.189)(1.69,1.176)%
(1.686,1.163)(1.681,1.151)(1.677,1.138)(1.673,1.125)(1.669,1.112)(1.665,1.099)(1.662,1.086)%
(1.659,1.073)(1.655,1.06)%
%
\polyline(1.655,0.565)(1.659,0.552)(1.662,0.539)(1.665,0.526)(1.669,0.513)(1.673,0.5)%
(1.677,0.487)(1.681,0.474)(1.686,0.461)(1.69,0.449)(1.695,0.436)(1.699,0.424)(1.704,0.411)%
(1.71,0.399)(1.715,0.386)(1.72,0.374)(1.726,0.362)(1.732,0.35)(1.737,0.337)(1.744,0.325)%
(1.75,0.313)(1.756,0.302)(1.763,0.29)(1.769,0.278)(1.776,0.267)(1.783,0.255)(1.79,0.244)%
(1.797,0.232)(1.805,0.221)(1.812,0.21)(1.82,0.199)(1.828,0.188)(1.835,0.177)(1.844,0.166)%
(1.852,0.155)(1.86,0.145)(1.869,0.134)(1.877,0.124)(1.886,0.114)(1.895,0.104)(1.904,0.094)%
(1.913,0.084)(1.922,0.074)(1.931,0.064)(1.941,0.055)(1.95,0.045)(1.96,0.036)(1.97,0.027)%
(1.98,0.018)(1.99,0.009)(2,0)%
%
\settowidth{\Width}{$f(x)$}\setlength{\Width}{-0.5\Width}%
\settoheight{\Height}{$f(x)$}\settodepth{\Depth}{$f(x)$}\setlength{\Height}{-0.5\Height}\setlength{\Depth}{0.5\Depth}\addtolength{\Height}{\Depth}%
\put(  1.630,  0.810){\hspace*{\Width}\raisebox{\Height}{$f(x)$}}%
%
\settowidth{\Width}{$\varDelta x$}\setlength{\Width}{0\Width}%
\settoheight{\Height}{$\varDelta x$}\settodepth{\Depth}{$\varDelta x$}\setlength{\Height}{-\Height}%
\put(  2.140, -0.150){\hspace*{\Width}\raisebox{\Height}{$\varDelta x$}}%
%
\polyline(-0.5,0)(4,0)%
%
\polyline(0,-0.5)(0,2.5)%
%
\settowidth{\Width}{$x$}\setlength{\Width}{0\Width}%
\settoheight{\Height}{$x$}\settodepth{\Depth}{$x$}\setlength{\Height}{-0.5\Height}\setlength{\Depth}{0.5\Depth}\addtolength{\Height}{\Depth}%
\put(  4.050,  0.000){\hspace*{\Width}\raisebox{\Height}{$x$}}%
%
\settowidth{\Width}{$y$}\setlength{\Width}{-0.5\Width}%
\settoheight{\Height}{$y$}\settodepth{\Depth}{$y$}\setlength{\Height}{\Depth}%
\put(  0.000,  2.550){\hspace*{\Width}\raisebox{\Height}{$y$}}%
%
\settowidth{\Width}{O}\setlength{\Width}{-1\Width}%
\settoheight{\Height}{O}\settodepth{\Depth}{O}\setlength{\Height}{-\Height}%
\put( -0.050, -0.050){\hspace*{\Width}\raisebox{\Height}{O}}%
%
\end{picture}}%}}
\end{layer}


\newslide{itemizeとenumerate環境の使い方}

\vspace*{18mm}

\slidepage
\textinit[120]

\begin{layer}{120}{0}
\addtext{8}{\ten}{段階的表示では,後ろの行の位置がずれることがある}
\addtext{8}{\ten}{itemize〜end}
\addtext{16}{}{itemize}
\addtext{16}{}{item:\hspace{0mm}:内容}
\addtext{16}{}{item:\hspace{0mm}:結論}
\addtext{16}{}{end}
\textinit[120]
\addtext[8]{72}{\ten}{enumerate:\hspace{0mm}:[]〜end}
\addtext{80}{}{enumerate:\hspace{0mm}:[(1)]}
\addtext{80}{}{item:\hspace{0mm}:内容}
\addtext{80}{}{item:\hspace{0mm}:結論}
\addtext{80}{}{end}
\end{layer}

\label{pageend}\mbox{}

\end{document}
