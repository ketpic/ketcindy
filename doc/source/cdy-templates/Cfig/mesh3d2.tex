%%% /Users/hannya/Desktop/fig/table.tex 
%%% Generator=table.cdy 
{\unitlength=1cm%
\begin{picture}%
(13,4.8)(0,0)%
\special{pn 8}%
%
\special{pa     0 -1890}\special{pa     0    -0}%
\special{fp}%
\special{pa  1181 -1890}\special{pa  1181    -0}%
\special{fp}%
\special{pa  5118 -1890}\special{pa  5118    -0}%
\special{fp}%
\special{pa     0 -1890}\special{pa  5118 -1890}%
\special{fp}%
\special{pa     0 -1575}\special{pa  5118 -1575}%
\special{fp}%
\special{pa     0  -945}\special{pa  5118  -945}%
\special{fp}%
\special{pa     0    -0}\special{pa  5118    -0}%
\special{fp}%
\settowidth{\Width}{nomaltype}\setlength{\Width}{-0.5\Width}%
\settoheight{\Height}{nomaltype}\settodepth{\Depth}{nomaltype}\setlength{\Height}{-0.5\Height}\setlength{\Depth}{0.5\Depth}\addtolength{\Height}{\Depth}%
\put(  1.500,  4.400){\hspace*{\Width}\raisebox{\Height}{nomaltype}}%
%
\settowidth{\Width}{説明}\setlength{\Width}{-0.5\Width}%
\settoheight{\Height}{説明}\settodepth{\Depth}{説明}\setlength{\Height}{-0.5\Height}\setlength{\Depth}{0.5\Depth}\addtolength{\Height}{\Depth}%
\put(  8.000,  4.400){\hspace*{\Width}\raisebox{\Height}{説明}}%
%
\settowidth{\Width}{perface}\setlength{\Width}{-0.5\Width}%
\settoheight{\Height}{perface}\settodepth{\Depth}{perface}\setlength{\Height}{-0.5\Height}\setlength{\Depth}{0.5\Depth}\addtolength{\Height}{\Depth}%
\put(  1.500,  3.200){\hspace*{\Width}\raisebox{\Height}{perface}}%
%
\settowidth{\Width}{ \begin{minipage}{95mm}各面の法線はその面上の三角形の法線。その結果、面の端で光の当たり方が不連続になり、格子構造が明らかになる。 \end{minipage}}\setlength{\Width}{0\Width}%
\settoheight{\Height}{ \begin{minipage}{95mm}各面の法線はその面上の三角形の法線。その結果、面の端で光の当たり方が不連続になり、格子構造が明らかになる。 \end{minipage}}\settodepth{\Depth}{ \begin{minipage}{95mm}各面の法線はその面上の三角形の法線。その結果、面の端で光の当たり方が不連続になり、格子構造が明らかになる。 \end{minipage}}\setlength{\Height}{-0.5\Height}\setlength{\Depth}{0.5\Depth}\addtolength{\Height}{\Depth}%
\put(  3.050,  3.200){\hspace*{\Width}\raisebox{\Height}{ \begin{minipage}{95mm}各面の法線はその面上の三角形の法線。その結果、面の端で光の当たり方が不連続になり、格子構造が明らかになる。 \end{minipage}}}%
%
\settowidth{\Width}{pervertex}\setlength{\Width}{-0.5\Width}%
\settoheight{\Height}{pervertex}\settodepth{\Depth}{pervertex}\setlength{\Height}{-0.5\Height}\setlength{\Depth}{0.5\Depth}\addtolength{\Height}{\Depth}%
\put(  1.500,  1.200){\hspace*{\Width}\raisebox{\Height}{pervertex}}%
%
\settowidth{\Width}{ \begin{minipage}{95mm}各面の法線はその面上の三角形の頂点の3本の法線をとり、その一次結合によって計算される。その結果、面の端での光の当たり方が連続的になり、格子構造が見えなくなる。 \end{minipage}}\setlength{\Width}{0\Width}%
\settoheight{\Height}{ \begin{minipage}{95mm}各面の法線はその面上の三角形の頂点の3本の法線をとり、その一次結合によって計算される。その結果、面の端での光の当たり方が連続的になり、格子構造が見えなくなる。 \end{minipage}}\settodepth{\Depth}{ \begin{minipage}{95mm}各面の法線はその面上の三角形の頂点の3本の法線をとり、その一次結合によって計算される。その結果、面の端での光の当たり方が連続的になり、格子構造が見えなくなる。 \end{minipage}}\setlength{\Height}{-0.5\Height}\setlength{\Depth}{0.5\Depth}\addtolength{\Height}{\Depth}%
\put(  3.050,  1.200){\hspace*{\Width}\raisebox{\Height}{ \begin{minipage}{95mm}各面の法線はその面上の三角形の頂点の3本の法線をとり、その一次結合によって計算される。その結果、面の端での光の当たり方が連続的になり、格子構造が見えなくなる。 \end{minipage}}}%
%
\end{picture}}%