\documentclass{ujarticle}
\usepackage{ketpic,ketlayer}
\usepackage{amsmath,amssymb}
\usepackage{graphicx}
\usepackage{xcolor}
\usepackage{bm,enumerate}
\usepackage[dvipdfmx,colorlinks=true,urlcolor=blue]{hyperref}

\setmargin{20}{20}{20}{20}

\西暦

\renewcommand{\labelitemi}{・}
\renewcommand{\labelitemii}{・}

\begin{document}

\begin{center}
KeTCindyについての補足 (Windows)
\end{center}

\hfill 修正日:\today

\begin{enumerate}[\bf\large 1.]

\item Cinderellaのインストール
\begin{itemize}
\item \url{https://beta.cinderella.de}から「保存」でダウンロードする.
\item インストーラを右クリック「管理者として実行」を選ぶ.
\item インストール先をProgram Files(または (x86))を選ぶ.
\end{itemize}

\item Sumatraのインストール
\begin{itemize}
\item \url{https://www.sumatrapdfreader.org/download-free-pdf-viewer.html} からダウンロードする.
\item \verb|Option|を選び,インストール先をProgram Files(または (x86))にする.
\end{itemize}

\item KeTTeXのインストール

\begin{enumerate}[(1)]
\item 以下から\verb|kettex.exe|をダウンロードする.\\
    \hspace*{10mm}\url{https://www.dropbox.com/s/fthw4btjqqs33tc/kettex.exe?dl=0}
\item 自己解凍ファイルになっているので,クリック(必要なら右クリック)して解凍する.
\item 解凍してできるkettexフォルダを\verb|C:\|(\verb|\|は\yen の意味) に入れる.
\end{enumerate}

\item KeTCindyのインストール
\begin{itemize}
\item ketcindysettings.cdyを利用する.
\begin{enumerate}[(1)]
\item 左方にあるボタンで,言語,TeXの種類,描画コードを選ぶ.\\
\hspace*{10mm}ボタンを押すと順に項目が変わる.
\item 中央にあるボタンでTeXシステムを選ぶ.\\
\hspace*{10mm}KeTTeX,TeXLive以外の場合は,CindyScriptでパスを設定してから.\verb|Other|を選ぶ.
\item 右側にあるボタンを順に押す.\\
\hspace*{10mm}\verb|Mkinit|:初期化ファイルketcindy.iniをユーザホームに作成\\
\hspace*{10mm}\verb|Update|:TeXに入っているketlib関連のファイルを更新(コピー)\\
\hspace*{10mm}\verb|Work|:作業フォルダketcindy.iniをユーザホームに作成
\end{enumerate}
\item ketcindysettings.cdyのUpdateでエラーが出た場合\\
\hspace*{10mm}updaterフォルダにあるupdateketcindy.batを右クリックして管理者として実行する.
\end{itemize}

\item KeTCindyのテストラン
    \begin{enumerate}[(1)]
    \item ketcindysettings.cdyを終了してから,作業ディレクトリketcindyを開く.
    \item ketcindyの中のtemplate1basic.cdyをダブルクリック.\\
      \hspace*{10mm}画面に白い枠が出れば,ライブラリの読み込みは成功.
    \item スクリーンの左上部にあるFigureボタンを押して,PDFが表示されれば成功.
  \end{enumerate}

\item TeXWorksの設定(kettexの場合)
  \begin{itemize}
 \item KeTTeXでは,\verb|C:\kettex\texlive\bin\win32|にすでに入っている.\\
 \hspace*{5mm}注)TeXLiveの場合は\verb|C:\texlive\(西暦年)\bin\win32|
  \item TeXworksを立ち上げ,「TeXworks \verb|>| 環境設定 \verb|>| タイプセット」
  \item 上の欄(パス)に以下を追加\\
  \hspace*{5mm}\verb|C:\kettex\texlive\bin\win32|\\
  \hspace*{10mm}注) 上の行を上の欄の先頭になるように移動する.
  \item 下の欄の横にある + をクリック
    \begin{itemize}
    \item 名前:uplatex(ptex2pdf)またはplatex(ptex2pdf)
    \item プログラム : ptex2pdf
    \item 引数:\\
    \hspace*{10mm} \verb|-u|(uplatexの場合のみ)\\
    \hspace*{10mm} \verb|-l|\\
    \hspace*{10mm} \verb|-ot|\\
    \hspace*{10mm}  \verb|$synctexoption|\\
    \hspace*{10mm}  \verb|$fullname|
    \item[]OKボタンを押し,デフォルトを変更してOKボタンを押す.
    \end{itemize}
  \end{itemize}

\item gccのインストール
  \begin{itemize}
    \item 曲面描画のためには, gccが必要である.
    \item minGWのホームページ\url{http://www.mingw.org}から\\
    \hspace*{10mm}download \verb|>| Install \verb|>| mingw-get-setup.exe\\
    をダウンロードして実行\\
    \hspace*{10mm}注) パッケージは,mingw32-base, mingw32-gcc-g++だけでよい.
  \end{itemize}

\item 手動でインストールする場合(KeTTeX)\\
\hspace*{1zw}注)他のTeXの場合は,適宜パスを置き換える.\\
\hspace*{3zw}\verb|C:\kettex\texlive| $=>$ \verb|C:\texlive\2020| など
  \begin{enumerate}[(1)]
  \item \verb|ketcindy(-master)\ketcindyfolder|を開いておく.
  \item scriptsフォルダの中身を以下にコピーする.\\
 \verb|C:\kettex\texlive\texmf-dist\scripts\ketcindy|
  \item styleフォルダの中身を以下にコピーする.\\
 \verb|C:\kettex\texlive\texmf-dist\tex\latex\ketcindy|
  \item docフォルダの中身を以下にコピーする.\\
 \verb|C:\kettex\texlive\texmf-dist\doc\supports\ketcindy||
  \item コマンドプロンプトで以下を実行する\\
  \hspace*{1zw}\verb|C:\kettex\texlive\bin\win32\mktexlsr|
  \item \verb|C:\Program files\Cinderella2(.exe)|をダブルクリック
  \end{enumerate}
\end{enumerate}

\end{document}