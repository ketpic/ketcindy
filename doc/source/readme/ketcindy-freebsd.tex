\documentclass[a4paper,12pt]{jarticle}
\begin{document}
\begin{center}
 K\raisebox{-0.3zw}{E}TCindy インストール手順 (FreeBSD)
\end{center}
\begin{flushright}
 修正日:2021年9月3日
\end{flushright}

この文書では,
root 権限で実行するコマンドはプロンプトに\#を使用して,
通常のユーザーで実行するコマンドはプロンプトに\%を使用しています.

root権限で実行するには,
root でログインする,
su コマンドで root になる,
sudo を使う,
等の方法があります.

\$HOME はユーザーのホームディレクトリを表す.ユーザー名が hoge の場合,
通常は,/home/hoge である.

\begin{enumerate}
 \item 準備

FreeBSD上でK\raisebox{-0.3zw}{E}TCindyを動作させるには,
以下の Ports (package) が必要である.

\noindent
x11/xorg
\\
\noindent
emulators/linux\_base-c7
\\
\noindent
x11/linux-c7-xorg-libs
\\
\noindent
lang/gcc**
\\
\noindent
math/R
\\
\noindent
math/maxima

この他に,適当なpdfビューワーが必要である.
ここでは,graphic/evince を使用するものとして話を進めていく.
違うpdfビューワーを使用するときは適当に読み替えること.

インストールされていないときは,
以下のように,root権限でインストールする.
(注意:これだけで,必要なものは,依存関係でインストールできる.)
\\[0.5zw]
\noindent
\# pkg install xorg
\\
\noindent
\# kldload linux
\\
\noindent
\# kldload linux64
\\
\noindent
\# pkg install linux-c7-xorg-libs
\\
\noindent
\# pkg install R
\\
\noindent
\# pkg install maxima
\\
\noindent
\# pkg install evince
\\[0.5zw]
次に,limux emulator を起動時から動作するように,
/etc/rc.conf に linux\_enable="YES" を追加する.
例えば,
\\[0.5zw]
\noindent
\# echo 'linux\_enable="YES"' $>>$  /etc/rc.conf
\\[0.5zw]
とする.
そして,linux emulator を動作させるために再起動する.
\\[0.5zw]
\noindent
\# shutdown -r now

\item \TeX \  Live のインストール

Ports の \TeX \  Live では 
K\raisebox{-0.3zw}{E}TCindy 
は動作しない.
そのため,\TeX \  Live のサイトから 
\TeX \  Live をインストールしなくてはならない.
例えば,
\\[0.5zw]
http://mirror.ctan.org/systems/texlive/Images/
\\[0.5zw]
からDVD イメージ texlive2021.iso をダウンロードして,
\\[0.5zw]
\noindent
\# mdconfig -a -f texlive2021.iso -u 0
\\
\noindent
\# mount\_cd9660 /dev/md0 /mnt
\\
\noindent
\# cd /mnt
\\
\noindent
\# ./install-tl
\\[0.5zw]
※ 選択肢が表示されたら I + Enter を入力
\\[0.5zw]
その他のtex環境(\TeX Works,\TeX studio等)
をPortsからインストールすると,
Portsのtexliveもインストールされてしまう.
そのため,
「tlmgr path add」を行わず,path を適切に設定して,
こちらの\TeX を優先させたほうがよいようである.
ports の \TeX \  Live よりこちらを優先させるためには,
他のPathよりPathを先に通せばよい.
例えば,csh,tcsh を使用しているときは,
\$HOME/.cshrc に,
\\[0.5zw]
\noindent
set path = (/usr/local/texlive/2021/bin/amd64-freebsd\  /sbin\  /bin \  /usr/sbin \ /usr/bin \ /usr/games \ /usr/local/sbin \  /usr/local/bin \    \$HOME/bin)
\\[0.5zw]
等を足せばよい.
なお,必要なpathを通すのを忘れないようにすること.
.cshrc を編集する前に
\\[0.5zw]
\% echo \$PATH
\\[0.5zw]
をして,必要なPathを確認しておくことをお勧めする.

\item Cinderellaのインストール

FreeBSDでのCinderellaの動作はバージョンによって,
動作しないものがある.
現時点で,
FreeBSD 12.2-RELEASE 上では Cinderella-3.0b.2017 の動作を,
FreeBSD 13.0-RELEASE 上では Cinderella-3.0b.2028 の動作を確認している.
\\[0.5zw]
https://beta.cinderella.de/Cinderella-3.0b.2017.tar.gz
\\
https://beta.cinderella.de/Cinderella-3.0b.2028.tar.gz
\\[0.5zw]
をダウンロードして,
\\[0.5zw]
\$ tar xvzf  Cinderella-3.0b.2028.tar.gz
\\[0.5zw]
とする.
そして,
\\[0.5zw]
\$ ./cinderella/Cinderella 
として,Cinderella を起動する.
解凍する前には,/etc/rc.conf に
linux\_enable="YES" を追加して,
再起動しておかないと動かいないので注意する.

\item ketcindy の設定

github から以下のように
\\[0.5zw]
\% git clone https://github.com/ketpic/ketcindy.git
\\[0.5zw]
最新版を clone するか,
\\[0.5zw]
https://github.com/ketpic/ketcindy
\\[0.5zw]
から ketcindy-master.zip 
をダウンロードして解凍して,
\\[0.5zw]
\% unzip ketcindy-master.zip
\\
\% mv ketcindy-master ketcindy
\\[0.5zw]
とする.
そして,
\\[0.5zw]
\# cp -pr ketcindy/style/*  /usr/local/texlive/2021/texmf-dist/tex/latex/ketcindy/
\# cp -pr ketcindy/scripts/*  /usr/local/texlive/2021/texmf-dist/scripts/
\# /usr/local/texlive/2021/bin/amd64-freebsd/mktexlsr 
\\[0.5zw]
とする.
そして,以下の内容の ketcindy.ini を \$HOME の下に作成する.

PathThead="/usr/local/texlive/2021/bin/amd64-freebsd";
\\
Dirhead="/usr/local/texlive/2021/texmf-dist/scripts/ketcindy";
\\
setdirectory(Dirhead);
\\
import("setketcindy.txt");
\\
PathT=PathThead+"/platex";
\\
GPACK="tpic";
\\
Pathpdf="/usr/local/bin/evince";
\\
PathR="/usr/local/bin/R";
\\
PathM="/usr/local/bin/maxima";
\\
PathC="/usr/local/bin/gcc10";
\\
PathV3="Meshlab";
\\
PathAd="acroread";
\\
PathA="asir";
\\
//PathW="";
\\
PathF="fricas";
\\
Mackc="bash";
\\
Helplist("read",[],"helpJ");
\\
setdirectory(Dircdy);
\\[1zw]
ketcindy.ini は Cinderella で 
ketcindy/doc/ketcindysettings.cdy 
を開いて,
「Mkinit」のボタンを押して,雛形を作成し,
テキストエディターで編集すると作成が楽である.
編集するとき,texlive のインストールディレクトリ,
gcc,pdfビューワーは環境によって,適宜,書き帰ること.
gccは何が入っているかわからないときは,
\\[0.5zw]
\% ls /usr/local/bin/gcc*
\\[0.5zw]
等としてみるとよい.
\end{enumerate}

以上で,FreeBSDでもketcindyが動作する.

\end{document}