\documentclass[papersize,a4paper,12pt,uplatex]{jsarticle}
\usepackage{ketpic,ketlayer}
\usepackage{amsmath}
% \usepackage{amsmath,newtxmath}
\usepackage[dvipdfmx]{graphicx,color}
\usepackage{wrapfig}
\usepackage[dvipdfmx,bookmarks=false,colorlinks=true,linkcolor=blue]{hyperref}
\setmargin{20}{20}{15}{25}
\usepackage{setspace}
\usepackage{comment}
\setcounter{tocdepth}{3}

\begin{document}
\title{\ketcindy の概要}
\author{\ketcindy\ Project Team}
\maketitle

\begin{center}  - 第3.2版 -\end{center}
\hypertarget{index}{}
\tableofcontents

\newpage

\section{\ketcindy について}
\subsection{システムの構成}
\ketcindy は,Cinderella の作図機能を利用して,作図データの\LaTeX ファイルを作成するためのスクリプトライブラリである。\ketcindy はCinderellaのプログラミング言語Cindyscriptで記述されており,ユーザーはCinderella によるインタラクティブな作図機能と,CindyScript によるプログラミングを用いて,\LaTeX 文書の挿入図を効率よく作成することができる。また,各種数式処理ソフトと連携して計算を行うことができる。

\begin{center}
\scalebox{0.9}{ %%% fig.tex 2016-4-30 21:58
%%% fig.sce 2016-4-30 21:58
{\unitlength=3.5mm%
\begin{picture}%
(  38.24000,  21.88000)( -24.72000, -11.80000)%
\special{pn 8}%
%
\special{pa 933 -397}\special{pa 929 -446}\special{pa 920 -494}\special{pa 905 -541}%
\special{pa 884 -586}\special{pa 858 -628}\special{pa 826 -665}\special{pa 790 -699}%
\special{pa 750 -728}\special{pa 707 -752}\special{pa 661 -770}\special{pa 614 -782}%
\special{pa 565 -788}\special{pa 516 -788}\special{pa 467 -782}\special{pa 419 -770}%
\special{pa 373 -752}\special{pa 330 -728}\special{pa 290 -699}\special{pa 254 -665}%
\special{pa 223 -628}\special{pa 196 -586}\special{pa 175 -541}\special{pa 160 -494}%
\special{pa 151 -446}\special{pa 148 -397}\special{pa 151 -348}\special{pa 160 -299}%
\special{pa 175 -252}\special{pa 196 -208}\special{pa 223 -166}\special{pa 254 -128}%
\special{pa 290 -94}\special{pa 330 -66}\special{pa 373 -42}\special{pa 419 -24}\special{pa 467 -11}%
\special{pa 516 -5}\special{pa 565 -5}\special{pa 614 -11}\special{pa 661 -24}\special{pa 707 -42}%
\special{pa 750 -66}\special{pa 790 -94}\special{pa 826 -128}\special{pa 858 -166}%
\special{pa 884 -208}\special{pa 905 -252}\special{pa 920 -299}\special{pa 929 -348}%
\special{pa 933 -397}%
\special{fp}%
\special{pa 1624 -628}\special{pa 1622 -665}\special{pa 1615 -700}\special{pa 1604 -735}%
\special{pa 1588 -768}\special{pa 1569 -799}\special{pa 1545 -827}\special{pa 1519 -852}%
\special{pa 1489 -873}\special{pa 1457 -891}\special{pa 1424 -904}\special{pa 1388 -913}%
\special{pa 1352 -918}\special{pa 1316 -918}\special{pa 1280 -913}\special{pa 1244 -904}%
\special{pa 1210 -891}\special{pa 1178 -873}\special{pa 1149 -852}\special{pa 1122 -827}%
\special{pa 1099 -799}\special{pa 1080 -768}\special{pa 1064 -735}\special{pa 1053 -700}%
\special{pa 1046 -665}\special{pa 1044 -628}\special{pa 1046 -592}\special{pa 1053 -556}%
\special{pa 1064 -522}\special{pa 1080 -489}\special{pa 1099 -458}\special{pa 1122 -430}%
\special{pa 1149 -405}\special{pa 1178 -383}\special{pa 1210 -366}\special{pa 1244 -352}%
\special{pa 1280 -343}\special{pa 1316 -339}\special{pa 1352 -339}\special{pa 1388 -343}%
\special{pa 1424 -352}\special{pa 1457 -366}\special{pa 1489 -383}\special{pa 1519 -405}%
\special{pa 1545 -430}\special{pa 1569 -458}\special{pa 1588 -489}\special{pa 1604 -522}%
\special{pa 1615 -556}\special{pa 1622 -592}\special{pa 1624 -628}%
\special{fp}%
\special{pa 1580 176}\special{pa 1577 128}\special{pa 1568 79}\special{pa 1553 33}%
\special{pa 1532 -11}\special{pa 1506 -53}\special{pa 1475 -90}\special{pa 1439 -124}%
\special{pa 1399 -153}\special{pa 1357 -176}\special{pa 1311 -194}\special{pa 1264 -207}%
\special{pa 1215 -213}\special{pa 1166 -213}\special{pa 1118 -207}\special{pa 1070 -194}%
\special{pa 1025 -176}\special{pa 982 -153}\special{pa 942 -124}\special{pa 906 -90}%
\special{pa 875 -53}\special{pa 849 -11}\special{pa 828 33}\special{pa 813 79}\special{pa 804 128}%
\special{pa 801 176}\special{pa 804 225}\special{pa 813 273}\special{pa 828 320}\special{pa 849 364}%
\special{pa 875 406}\special{pa 906 443}\special{pa 942 477}\special{pa 982 506}\special{pa 1025 529}%
\special{pa 1070 547}\special{pa 1118 559}\special{pa 1166 565}\special{pa 1215 565}%
\special{pa 1264 559}\special{pa 1311 547}\special{pa 1357 529}\special{pa 1399 506}%
\special{pa 1439 477}\special{pa 1475 443}\special{pa 1506 406}\special{pa 1532 364}%
\special{pa 1553 320}\special{pa 1568 273}\special{pa 1577 225}\special{pa 1580 176}%
\special{fp}%
\special{pn 32}%
\special{pa 784 -90}\special{pa 865 -39}%
\special{fp}%
\special{pn 8}%
\special{pn 32}%
\special{pa 931 -435}\special{pa 1058 -540}%
\special{fp}%
\special{pn 8}%
\special{pn 24}%
\special{pa 1225 -360}\special{pa 1204 -227}%
\special{fp}%
\special{pn 8}%
\special{pa 1189 -291}\special{pa 1202 -213}\special{pa 1237 -284}\special{pa 1213 -287}%
\special{pa 1189 -291}\special{sh 1}\special{ip}%
\special{pn 1}%
\special{pa 1189 -291}\special{pa 1202 -213}\special{pa 1237 -284}\special{pa 1213 -287}%
\special{pa 1189 -291}\special{pa 1202 -213}%
\special{fp}%
\special{pn 8}%
\special{pn 24}%
\special{pa 1271 -205}\special{pa 1305 -326}%
\special{fp}%
\special{pn 8}%
\special{pa 1312 -261}\special{pa 1309 -339}\special{pa 1265 -274}\special{pa 1288 -267}%
\special{pa 1312 -261}\special{sh 1}\special{ip}%
\special{pn 1}%
\special{pa 1312 -261}\special{pa 1309 -339}\special{pa 1265 -274}\special{pa 1288 -267}%
\special{pa 1312 -261}\special{pa 1309 -339}%
\special{fp}%
\special{pn 8}%
\special{pa 89 155}\special{pa 85 93}\special{pa 72 32}\special{pa 50 -27}\special{pa 19 -83}%
\special{pa -21 -136}\special{pa -68 -184}\special{pa -123 -228}\special{pa -183 -265}%
\special{pa -250 -297}\special{pa -320 -321}\special{pa -393 -338}\special{pa -469 -347}%
\special{pa -545 -349}\special{pa -621 -343}\special{pa -695 -330}\special{pa -766 -308}%
\special{pa -834 -280}\special{pa -896 -246}\special{pa -953 -205}\special{pa -1002 -159}%
\special{pa -1044 -108}\special{pa -1078 -53}\special{pa -1103 5}\special{pa -1119 65}%
\special{pa -1125 127}\special{pa -1122 188}\special{pa -1109 249}\special{pa -1086 308}%
\special{pa -1055 364}\special{pa -1016 417}\special{pa -968 466}\special{pa -914 509}%
\special{pa -853 547}\special{pa -787 578}\special{pa -716 602}\special{pa -643 619}%
\special{pa -567 628}\special{pa -491 630}\special{pa -416 624}\special{pa -341 611}%
\special{pa -270 590}\special{pa -203 562}\special{pa -140 527}\special{pa -84 486}%
\special{pa -34 440}\special{pa 8 389}\special{pa 42 334}\special{pa 67 276}\special{pa 82 216}%
\special{pa 89 155}%
\special{fp}%
\special{pa -2437 142}\special{pa -2441 96}\special{pa -2453 50}\special{pa -2471 7}%
\special{pa -2496 -35}\special{pa -2527 -73}\special{pa -2565 -108}\special{pa -2607 -139}%
\special{pa -2655 -166}\special{pa -2706 -188}\special{pa -2760 -204}\special{pa -2816 -215}%
\special{pa -2874 -220}\special{pa -2932 -220}\special{pa -2989 -213}\special{pa -3045 -201}%
\special{pa -3099 -184}\special{pa -3150 -161}\special{pa -3197 -134}\special{pa -3239 -102}%
\special{pa -3275 -66}\special{pa -3306 -27}\special{pa -3330 15}\special{pa -3348 59}%
\special{pa -3358 105}\special{pa -3361 151}\special{pa -3357 197}\special{pa -3346 242}%
\special{pa -3328 285}\special{pa -3302 327}\special{pa -3271 365}\special{pa -3234 400}%
\special{pa -3191 432}\special{pa -3144 458}\special{pa -3093 480}\special{pa -3039 496}%
\special{pa -2982 507}\special{pa -2925 512}\special{pa -2867 512}\special{pa -2809 506}%
\special{pa -2753 494}\special{pa -2699 476}\special{pa -2649 453}\special{pa -2602 426}%
\special{pa -2560 394}\special{pa -2523 358}\special{pa -2493 319}\special{pa -2468 277}%
\special{pa -2451 233}\special{pa -2440 188}\special{pa -2437 142}%
\special{fp}%
\special{pa -1461 901}\special{pa -1463 870}\special{pa -1472 839}\special{pa -1486 808}%
\special{pa -1505 779}\special{pa -1530 752}\special{pa -1560 727}\special{pa -1594 705}%
\special{pa -1633 685}\special{pa -1674 669}\special{pa -1719 656}\special{pa -1765 647}%
\special{pa -1813 641}\special{pa -1861 640}\special{pa -1908 643}\special{pa -1955 649}%
\special{pa -2000 659}\special{pa -2043 673}\special{pa -2082 691}\special{pa -2118 711}%
\special{pa -2149 734}\special{pa -2175 760}\special{pa -2197 788}\special{pa -2212 817}%
\special{pa -2222 848}\special{pa -2226 879}\special{pa -2224 911}\special{pa -2216 942}%
\special{pa -2202 972}\special{pa -2182 1001}\special{pa -2157 1028}\special{pa -2127 1053}%
\special{pa -2093 1076}\special{pa -2055 1095}\special{pa -2013 1112}\special{pa -1969 1124}%
\special{pa -1922 1134}\special{pa -1875 1139}\special{pa -1827 1140}\special{pa -1779 1138}%
\special{pa -1732 1131}\special{pa -1687 1121}\special{pa -1645 1107}\special{pa -1605 1090}%
\special{pa -1570 1069}\special{pa -1539 1046}\special{pa -1512 1020}\special{pa -1491 992}%
\special{pa -1475 963}\special{pa -1465 932}\special{pa -1461 901}%
\special{fp}%
\special{pa 201 1172}\special{pa 200 1142}\special{pa 191 1113}\special{pa 177 1084}%
\special{pa 156 1056}\special{pa 130 1030}\special{pa 98 1005}\special{pa 61 983}%
\special{pa 21 963}\special{pa -24 947}\special{pa -71 934}\special{pa -121 924}\special{pa -172 918}%
\special{pa -224 915}\special{pa -275 916}\special{pa -325 921}\special{pa -374 930}%
\special{pa -420 942}\special{pa -462 957}\special{pa -501 976}\special{pa -535 997}%
\special{pa -564 1020}\special{pa -587 1046}\special{pa -605 1074}\special{pa -616 1102}%
\special{pa -620 1132}\special{pa -619 1162}\special{pa -610 1191}\special{pa -596 1220}%
\special{pa -575 1248}\special{pa -549 1274}\special{pa -517 1299}\special{pa -480 1321}%
\special{pa -439 1341}\special{pa -395 1357}\special{pa -347 1370}\special{pa -298 1380}%
\special{pa -247 1386}\special{pa -195 1389}\special{pa -144 1388}\special{pa -94 1383}%
\special{pa -45 1374}\special{pa 1 1362}\special{pa 43 1347}\special{pa 82 1328}\special{pa 116 1307}%
\special{pa 145 1283}\special{pa 168 1258}\special{pa 186 1230}\special{pa 197 1202}%
\special{pa 201 1172}%
\special{fp}%
\special{pa 1001 926}\special{pa 999 888}\special{pa 992 849}\special{pa 978 812}%
\special{pa 959 776}\special{pa 935 742}\special{pa 905 711}\special{pa 872 684}\special{pa 834 659}%
\special{pa 792 639}\special{pa 749 623}\special{pa 703 612}\special{pa 655 605}\special{pa 608 604}%
\special{pa 560 607}\special{pa 513 615}\special{pa 468 627}\special{pa 426 644}\special{pa 386 666}%
\special{pa 350 691}\special{pa 319 720}\special{pa 292 751}\special{pa 270 786}\special{pa 254 822}%
\special{pa 244 860}\special{pa 239 898}\special{pa 241 937}\special{pa 249 975}\special{pa 262 1013}%
\special{pa 281 1048}\special{pa 305 1082}\special{pa 335 1113}\special{pa 369 1141}%
\special{pa 406 1165}\special{pa 448 1185}\special{pa 492 1201}\special{pa 538 1212}%
\special{pa 585 1219}\special{pa 633 1221}\special{pa 680 1218}\special{pa 727 1210}%
\special{pa 772 1197}\special{pa 814 1180}\special{pa 854 1159}\special{pa 890 1134}%
\special{pa 921 1105}\special{pa 948 1073}\special{pa 970 1039}\special{pa 986 1003}%
\special{pa 996 965}\special{pa 1001 926}%
\special{fp}%
\special{pa -768 1296}\special{pa -770 1262}\special{pa -780 1228}\special{pa -795 1195}%
\special{pa -816 1163}\special{pa -843 1133}\special{pa -875 1107}\special{pa -912 1082}%
\special{pa -953 1062}\special{pa -997 1045}\special{pa -1044 1031}\special{pa -1093 1022}%
\special{pa -1143 1018}\special{pa -1194 1017}\special{pa -1244 1021}\special{pa -1294 1029}%
\special{pa -1341 1041}\special{pa -1385 1058}\special{pa -1427 1078}\special{pa -1464 1101}%
\special{pa -1496 1127}\special{pa -1524 1156}\special{pa -1546 1188}\special{pa -1562 1220}%
\special{pa -1571 1254}\special{pa -1575 1289}\special{pa -1572 1323}\special{pa -1563 1357}%
\special{pa -1548 1391}\special{pa -1526 1422}\special{pa -1499 1452}\special{pa -1467 1479}%
\special{pa -1431 1503}\special{pa -1390 1523}\special{pa -1345 1540}\special{pa -1298 1554}%
\special{pa -1249 1563}\special{pa -1199 1567}\special{pa -1148 1568}\special{pa -1098 1564}%
\special{pa -1049 1556}\special{pa -1002 1544}\special{pa -957 1527}\special{pa -916 1507}%
\special{pa -879 1484}\special{pa -846 1458}\special{pa -819 1429}\special{pa -797 1398}%
\special{pa -781 1365}\special{pa -771 1331}\special{pa -768 1296}%
\special{fp}%
\special{pa -925 -621}\special{pa -930 -652}\special{pa -942 -682}\special{pa -960 -711}%
\special{pa -984 -738}\special{pa -1015 -763}\special{pa -1051 -785}\special{pa -1091 -805}%
\special{pa -1136 -821}\special{pa -1185 -834}\special{pa -1236 -843}\special{pa -1289 -848}%
\special{pa -1343 -849}\special{pa -1397 -847}\special{pa -1451 -840}\special{pa -1504 -830}%
\special{pa -1554 -816}\special{pa -1601 -799}\special{pa -1644 -779}\special{pa -1682 -756}%
\special{pa -1716 -730}\special{pa -1744 -702}\special{pa -1766 -673}\special{pa -1781 -643}%
\special{pa -1790 -612}\special{pa -1792 -581}\special{pa -1787 -550}\special{pa -1776 -520}%
\special{pa -1757 -491}\special{pa -1733 -464}\special{pa -1702 -439}\special{pa -1667 -416}%
\special{pa -1626 -397}\special{pa -1581 -381}\special{pa -1533 -368}\special{pa -1482 -359}%
\special{pa -1429 -354}\special{pa -1374 -352}\special{pa -1320 -355}\special{pa -1266 -361}%
\special{pa -1214 -371}\special{pa -1164 -385}\special{pa -1117 -402}\special{pa -1074 -423}%
\special{pa -1035 -446}\special{pa -1001 -471}\special{pa -973 -499}\special{pa -952 -528}%
\special{pa -936 -559}\special{pa -927 -590}\special{pa -925 -621}%
\special{fp}%
\special{pa -1457 142}\special{pa -1460 103}\special{pa -1470 65}\special{pa -1486 28}%
\special{pa -1508 -7}\special{pa -1535 -40}\special{pa -1568 -69}\special{pa -1605 -96}%
\special{pa -1646 -118}\special{pa -1690 -137}\special{pa -1737 -150}\special{pa -1786 -160}%
\special{pa -1836 -164}\special{pa -1886 -163}\special{pa -1936 -158}\special{pa -1985 -148}%
\special{pa -2031 -133}\special{pa -2075 -114}\special{pa -2115 -90}\special{pa -2152 -63}%
\special{pa -2184 -33}\special{pa -2210 0}\special{pa -2231 36}\special{pa -2246 73}%
\special{pa -2255 112}\special{pa -2258 150}\special{pa -2255 189}\special{pa -2245 227}%
\special{pa -2229 264}\special{pa -2207 299}\special{pa -2180 332}\special{pa -2147 362}%
\special{pa -2110 388}\special{pa -2069 410}\special{pa -2025 429}\special{pa -1978 442}%
\special{pa -1929 452}\special{pa -1879 456}\special{pa -1829 455}\special{pa -1779 450}%
\special{pa -1730 440}\special{pa -1684 425}\special{pa -1640 406}\special{pa -1599 382}%
\special{pa -1563 355}\special{pa -1531 325}\special{pa -1505 292}\special{pa -1484 256}%
\special{pa -1469 219}\special{pa -1460 181}\special{pa -1457 142}%
\special{fp}%
\special{pn 16}%
\special{pa 11 -94}\special{pa 183 -234}%
\special{fp}%
\special{pn 8}%
\special{pn 16}%
\special{pa 89 141}\special{pa 804 127}%
\special{fp}%
\special{pn 8}%
\special{pn 16}%
\special{pa -1291 143}\special{pa -1139 141}%
\special{fp}%
\special{pn 8}%
\special{pa -1199 166}\special{pa -1125 141}\special{pa -1200 117}\special{pa -1200 142}%
\special{pa -1199 166}\special{sh 1}\special{ip}%
\special{pn 1}%
\special{pa -1199 166}\special{pa -1125 141}\special{pa -1200 117}\special{pa -1200 142}%
\special{pa -1199 166}\special{pa -1125 141}%
\special{fp}%
\special{pn 8}%
\special{pn 16}%
\special{pa -1291 143}\special{pa -1443 146}%
\special{fp}%
\special{pn 8}%
\special{pa -1382 120}\special{pa -1457 146}\special{pa -1382 169}\special{pa -1382 145}%
\special{pa -1382 120}\special{sh 1}\special{ip}%
\special{pn 1}%
\special{pa -1382 120}\special{pa -1457 146}\special{pa -1382 169}\special{pa -1382 145}%
\special{pa -1382 120}\special{pa -1457 146}%
\special{fp}%
\special{pn 8}%
\special{pn 16}%
\special{pa -1282 585}\special{pa -983 469}%
\special{fp}%
\special{pn 8}%
\special{pa -1031 513}\special{pa -970 464}\special{pa -1049 468}\special{pa -1040 491}%
\special{pa -1031 513}\special{sh 1}\special{ip}%
\special{pn 1}%
\special{pa -1031 513}\special{pa -970 464}\special{pa -1049 468}\special{pa -1040 491}%
\special{pa -1031 513}\special{pa -970 464}%
\special{fp}%
\special{pn 8}%
\special{pn 16}%
\special{pa -1282 585}\special{pa -1580 701}%
\special{fp}%
\special{pn 8}%
\special{pa -1532 656}\special{pa -1593 706}\special{pa -1514 701}\special{pa -1523 678}%
\special{pa -1532 656}\special{sh 1}\special{ip}%
\special{pn 1}%
\special{pa -1532 656}\special{pa -1593 706}\special{pa -1514 701}\special{pa -1523 678}%
\special{pa -1532 656}\special{pa -1593 706}%
\special{fp}%
\special{pn 8}%
\special{pn 16}%
\special{pa -266 757}\special{pa -298 613}%
\special{fp}%
\special{pn 8}%
\special{pa -261 668}\special{pa -301 600}\special{pa -308 678}\special{pa -285 673}%
\special{pa -261 668}\special{sh 1}\special{ip}%
\special{pn 1}%
\special{pa -261 668}\special{pa -301 600}\special{pa -308 678}\special{pa -285 673}%
\special{pa -261 668}\special{pa -301 600}%
\special{fp}%
\special{pn 8}%
\special{pn 16}%
\special{pa -266 757}\special{pa -234 902}%
\special{fp}%
\special{pn 8}%
\special{pa -271 847}\special{pa -231 915}\special{pa -224 837}\special{pa -248 842}%
\special{pa -271 847}\special{sh 1}\special{ip}%
\special{pn 1}%
\special{pa -271 847}\special{pa -231 915}\special{pa -224 837}\special{pa -248 842}%
\special{pa -271 847}\special{pa -231 915}%
\special{fp}%
\special{pn 8}%
\special{pn 16}%
\special{pa 173 549}\special{pa 342 681}%
\special{fp}%
\special{pn 8}%
\special{pa 279 662}\special{pa 353 689}\special{pa 309 624}\special{pa 294 643}\special{pa 279 662}%
\special{sh 1}\special{ip}%
\special{pn 1}%
\special{pa 279 662}\special{pa 353 689}\special{pa 309 624}\special{pa 294 643}\special{pa 279 662}%
\special{pa 353 689}%
\special{fp}%
\special{pn 8}%
\special{pn 16}%
\special{pa 173 549}\special{pa 4 417}%
\special{fp}%
\special{pn 8}%
\special{pa 67 436}\special{pa -7 409}\special{pa 37 474}\special{pa 52 455}\special{pa 67 436}%
\special{sh 1}\special{ip}%
\special{pn 1}%
\special{pa 67 436}\special{pa -7 409}\special{pa 37 474}\special{pa 52 455}\special{pa 67 436}%
\special{pa -7 409}%
\special{fp}%
\special{pn 8}%
\special{pn 24}%
\special{pa -2258 143}\special{pa -2423 143}%
\special{fp}%
\special{pn 8}%
\special{pa -2362 119}\special{pa -2437 143}\special{pa -2362 168}\special{pa -2362 143}%
\special{pa -2362 119}\special{sh 1}\special{ip}%
\special{pn 1}%
\special{pa -2362 119}\special{pa -2437 143}\special{pa -2362 168}\special{pa -2362 143}%
\special{pa -2362 119}\special{pa -2437 143}%
\special{fp}%
\special{pn 8}%
\special{pn 16}%
\special{pa -909 -237}\special{pa -1052 -419}%
\special{fp}%
\special{pn 8}%
\special{pa -995 -386}\special{pa -1061 -430}\special{pa -1034 -356}\special{pa -1015 -371}%
\special{pa -995 -386}\special{sh 1}\special{ip}%
\special{pn 1}%
\special{pa -995 -386}\special{pa -1061 -430}\special{pa -1034 -356}\special{pa -1015 -371}%
\special{pa -995 -386}\special{pa -1061 -430}%
\special{fp}%
\special{pn 8}%
\special{pn 16}%
\special{pa -907 808}\special{pa -775 597}%
\special{fp}%
\special{pn 8}%
\special{pa -787 662}\special{pa -767 585}\special{pa -828 636}\special{pa -807 649}%
\special{pa -787 662}\special{sh 1}\special{ip}%
\special{pn 1}%
\special{pa -787 662}\special{pa -767 585}\special{pa -828 636}\special{pa -807 649}%
\special{pa -787 662}\special{pa -767 585}%
\special{fp}%
\special{pn 8}%
\special{pn 16}%
\special{pa -907 808}\special{pa -1040 1019}%
\special{fp}%
\special{pn 8}%
\special{pa -1028 954}\special{pa -1047 1031}\special{pa -987 980}\special{pa -1007 967}%
\special{pa -1028 954}\special{sh 1}\special{ip}%
\special{pn 1}%
\special{pa -1028 954}\special{pa -1047 1031}\special{pa -987 980}\special{pa -1007 967}%
\special{pa -1028 954}\special{pa -1047 1031}%
\special{fp}%
\special{pn 8}%
\Large%
\settowidth{\Width}{KeTCindy概念図}\setlength{\Width}{-0.5\Width}%
\settoheight{\Height}{KeTCindy概念図}\settodepth{\Depth}{KeTCindy概念図}\setlength{\Height}{-0.5\Height}\setlength{\Depth}{0.5\Depth}\addtolength{\Height}{\Depth}%
\put(-19.0000,6.0000){\hspace*{\Width}\raisebox{\Height}{KeTCindy概念図}}%
%
%
\normalsize%
\settowidth{\Width}{CindyScript}\setlength{\Width}{-0.5\Width}%
\settoheight{\Height}{CindyScript}\settodepth{\Depth}{CindyScript}\setlength{\Height}{-0.5\Height}\setlength{\Depth}{0.5\Depth}\addtolength{\Height}{\Depth}%
\put(3.9200,2.8800){\hspace*{\Width}\raisebox{\Height}{CindyScript}}%
%
%
\settowidth{\Width}{CindyLab}\setlength{\Width}{-0.5\Width}%
\settoheight{\Height}{CindyLab}\settodepth{\Depth}{CindyLab}\setlength{\Height}{-0.5\Height}\setlength{\Depth}{0.5\Depth}\addtolength{\Height}{\Depth}%
\put(9.6800,4.5600){\hspace*{\Width}\raisebox{\Height}{CindyLab}}%
%
%
\settowidth{\Width}{作図}\setlength{\Width}{-0.5\Width}%
\settoheight{\Height}{作図}\settodepth{\Depth}{作図}\setlength{\Height}{-0.5\Height}\setlength{\Depth}{0.5\Depth}\addtolength{\Height}{\Depth}%
\put(8.6400,-1.2800){\hspace*{\Width}\raisebox{\Height}{作図}}%
%
%
\settowidth{\Width}{KeTpic}\setlength{\Width}{0\Width}%
\settoheight{\Height}{KeTpic}\settodepth{\Depth}{KeTpic}\setlength{\Height}{-\Height}%
\put(-18.6300,0.7900){\hspace*{\Width}\raisebox{\Height}{KeTpic}}%
%
%
\settowidth{\Width}{R}\setlength{\Width}{-0.5\Width}%
\settoheight{\Height}{R}\settodepth{\Depth}{R}\setlength{\Height}{-0.5\Height}\setlength{\Depth}{0.5\Depth}\addtolength{\Height}{\Depth}%
\put(-13.3800,-6.4600){\hspace*{\Width}\raisebox{\Height}{R}}%
%
%
\settowidth{\Width}{Maxima}\setlength{\Width}{-0.5\Width}%
\settoheight{\Height}{Maxima}\settodepth{\Depth}{Maxima}\setlength{\Height}{-0.5\Height}\setlength{\Depth}{0.5\Depth}\addtolength{\Height}{\Depth}%
\put(-1.5200,-8.3600){\hspace*{\Width}\raisebox{\Height}{Maxima}}%
%
%
\settowidth{\Width}{Risa/Asir}\setlength{\Width}{-0.5\Width}%
\settoheight{\Height}{Risa/Asir}\settodepth{\Depth}{Risa/Asir}\setlength{\Height}{-0.5\Height}\setlength{\Depth}{0.5\Depth}\addtolength{\Height}{\Depth}%
\put(4.5000,-6.6200){\hspace*{\Width}\raisebox{\Height}{Risa/Asir}}%
%
%
\settowidth{\Width}{FriCAS}\setlength{\Width}{0\Width}%
\settoheight{\Height}{FriCAS}\settodepth{\Depth}{FriCAS}\setlength{\Height}{-0.5\Height}\setlength{\Depth}{0.5\Depth}\addtolength{\Height}{\Depth}%
\put(-10.0900,-9.3600){\hspace*{\Width}\raisebox{\Height}{FriCAS}}%
%
%
\settowidth{\Width}{MeshLab}\setlength{\Width}{0\Width}%
\settoheight{\Height}{MeshLab}\settodepth{\Depth}{MeshLab}\setlength{\Height}{-0.5\Height}\setlength{\Depth}{0.5\Depth}\addtolength{\Height}{\Depth}%
\put(-11.6900,4.2400){\hspace*{\Width}\raisebox{\Height}{MeshLab}}%
%
%
\Large%
\settowidth{\Width}{KeTCindy}\setlength{\Width}{-0.5\Width}%
\settoheight{\Height}{KeTCindy}\settodepth{\Depth}{KeTCindy}\setlength{\Height}{-0.5\Height}\setlength{\Depth}{0.5\Depth}\addtolength{\Height}{\Depth}%
\put(-3.7600,-1.0200){\hspace*{\Width}\raisebox{\Height}{KeTCindy}}%
%
%
\settowidth{\Width}{TeX}\setlength{\Width}{-0.5\Width}%
\settoheight{\Height}{TeX}\settodepth{\Depth}{TeX}\setlength{\Height}{-0.5\Height}\setlength{\Depth}{0.5\Depth}\addtolength{\Height}{\Depth}%
\put(-21.0400,-1.0600){\hspace*{\Width}\raisebox{\Height}{TeX}}%
%
%
\settowidth{\Width}{Cinderella}\setlength{\Width}{0\Width}%
\settoheight{\Height}{Cinderella}\settodepth{\Depth}{Cinderella}\setlength{\Height}{-0.5\Height}\setlength{\Depth}{0.5\Depth}\addtolength{\Height}{\Depth}%
\put(2.0500,7.0000){\hspace*{\Width}\raisebox{\Height}{Cinderella}}%
%
%
\settowidth{\Width}{Scilab}\setlength{\Width}{0\Width}%
\settoheight{\Height}{Scilab}\settodepth{\Depth}{Scilab}\setlength{\Height}{-0.5\Height}\setlength{\Depth}{0.5\Depth}\addtolength{\Height}{\Depth}%
\put(-15.2700,-1.0800){\hspace*{\Width}\raisebox{\Height}{Scilab}}%
%
%
\end{picture}}%}
\end{center}

Cinderellaで作図した図のデータは,\ketcindy により,いったんRのファイルに書き出される。これをRで処理して\TeX ファイルを作成する。できた\TeX ファイルを,本文中に inputコマンド で挿入すれば図が表示される。(\ketcindy の初期の版ではこのデータ処理にScilabを用いていた。)

CinderellaとRやその他のソフトウェアとの連携には,バッチファイル(Macではシェルファイル)を用いている。(概念図の両方向矢印)  バッチファイルは kc.bat,シェルファイルは kc.sh で,\ketcindy  が目的に応じてこれらのファイルを書き出して実行するようになっている。

したがって,KeTCindyでの図ファイルの作成手順は次のようになる。

(1) 必要に応じてCinderellaの作図ツールで,点や線を作図しておく。

(2) Cindyscript エディタでプログラムを書く。

(3) 出力する。

\newpage
%--------- ketcindy による作図手順---------------- 
\subsection{\ketcindy による作図手順}

\ketcindy で作図し,TeXのファイルを作図する手順をチュートリアル形式で示す。

\subsubsection{平面幾何}

\ketcindy のシステムに同梱されている,template1basic.cdy をひな形として用いる。適当な場所に複製を作り,名前を変えておこう。ここでは,単に template.cdy とする。

template.cdy をダブルクリックして開き,図が表示されたら,Ctrl+9 ( Windows )  / ⌘+9 ( Mac ) でスクリプトエディタを開く。2つの画面はマルチウィンドウにしておくのがよい。

\vspace{\baselineskip}
\begin{center}\includegraphics[bb=0.00 0.00 500.03 306.52,width=10cm]{Fig/start01.pdf}\end{center}

この三角形で Cinderellaの作図機能を用いて内心を求め,内接円を描く。

まず,スクリプトエディタの \verb|Listplot([A,B,C,A]);|  の行頭にスラッシュを2本書き入れ,Shift+Enter で実行する。こうすると,この行はコメント行となり実行されない行になる。その結果,三角形が消えて点だけ残る。

画面上部の作図ツールから「線分を加える」\includegraphics[bb=0.00 0.00 6.48 5.04,width=8mm]{Fig/segment.pdf} をクリックして選択し,点Aから点Bへドラッグすると線分が描かれる。同様にして,BC, CAを引く。

\begin{center}\includegraphics[bb=0.00 0.00 414.02 329.02,height=3.5cm]{Fig/start02.pdf} \end{center}

次に,角の二等分線を引く。「角の二等分線を加える」ツール \includegraphics[bb=0.00 0.00 6.48 5.04,width=8mm]{Fig/bisector.pdf} を選択し,辺BA,BCを順にクリックすると角ABCの二等分線が引かれる。 このとき,次図左のように,該当する角を表すガイドが出る。

\hspace{20mm}\includegraphics[bb=0.00 0.00 402.02 325.02,height=3cm]{Fig/start03.pdf} \hspace{5mm}\includegraphics[bb=0.00 0.00 413.02 320.02,height=3cm]{Fig/start04.pdf}

同様にして,角Aの二等分線を引き,「交点を求める」ツール\includegraphics[bb=0.00 0.00 6.48 5.04,width=8mm]{Fig/intersection.pdf}をクリックして,2本の二等分線を順にクリックすると交点が求められる。(角Aの二等分線を引いた直後はこれが選択状態にあるので,角Bの二等分線をクリックすればよい)

\begin{center}\includegraphics[bb=0.00 0.00 397.02 319.02,height=3cm]{Fig/start05.pdf}\end{center}

内接円の半径を決めるために,辺ACに垂直で点Dを通る直線を描く。「垂線を加える」ツール \includegraphics[bb=0.00 0.00 6.48 5.04,width=8mm]{Fig/multi-add-perp.pdf} を選択し,辺AC上でマウスボタンを押し,そのまま点Dへドラッグすると垂線が引ける。

\hspace{20mm}\includegraphics[bb=0.00 0.00 403.02 319.02,height=3cm]{Fig/start06.pdf} \hspace{5mm}\includegraphics[bb=0.00 0.00 394.02 315.02,height=3cm]{Fig/start07.pdf} 

最後に,垂線と辺ACの交点を求める。

\begin{center}\includegraphics[bb=0.00 0.00 409.02 312.02,height=3cm]{Fig/start08.pdf}\end{center}

図が描かれたら,「要素を動かす」ツール \includegraphics[bb=0.00 0.00 6.48 5.04,width=8mm]{Fig/move.pdf} を選択して,始めの状態(動かすモード)に戻しておく。

これで内心の作図と半径の作図ができた。内心円は描かなくてよい。

スクリプトエディタに戻り,先ほど書いた // を消して \verb|Listplot([A,B,C,A]);|  を有効にし,次のスクリプトを追加し,Shift+Enterで実行すると,内接円が描かれる。なお,2行目を \verb|Setfiles("innercircle")| として,ファイル名も設定しておく。

\begin{verbatim}
    Circledata([D,E]);
    Pointdata("1",[D],["size=3"]);
    Letter([A,"sw","A",B,"ne","B",C,"se","C",D,"ne2","I"]);
\end{verbatim}

\begin{center}\includegraphics[bb=0.00 0.00 396.02 322.02,height=4cm]{Fig/start09.pdf} \end{center}


画面左上の「Figure」ボタンをクリックすると,プレビュー用のPDFができて表示される(下図左)。
描画面で点Bをドラッグして三角形の形を変えれば,それに応じて出力する図も変えることができる(下図右)。

\begin{center} \includegraphics[bb=0.00 0.00 396.02 331.02,height=4cm]{Fig/start10.pdf}  \includegraphics[bb=0.00 0.00 348.02 284.51,height=4cm]{Fig/incenter03.pdf} \end{center}

\vspace{\baselineskip}
\ketcindy がTikZなどの作図支援ツールと異なるのは,Cinderellaの作図機能を用いてインタラクティブに図の調整ができる点である。簡単な図であれば,座標の計算は不要で,Cinderellaの作図画面を見ながら\ketcindy の作図関数でプログラムを書くだけでよい。

なお,Cinderellaの作図機能については,付録の \hyperlink{geometrytool}{作図ツール} を参照されたい。

%---------------- 関数のグラフ--------------------------
\subsubsection{関数のグラフ}

例として,$y=\sin x$ と $y=x$  のグラフを描く。

template.cdy をダブルクリックして開き,Ctrl+9 ( Windows ) /  ⌘+9 ( Mac ) でスクリプトエディタを開く。

座標軸を描くので,\verb|Addax(0);| を \verb|Addax(1);| に変え,\verb|Listplot([A,B,C,A]);| は使わないので削除し,かわりに次のスクリプトを書く。

\begin{layer}{150}{0}
\putnotese{80}{5}{\includegraphics[bb=0.00 0.00 552.03 409.02,height=30mm]{Fig/xsinx01.pdf} }
\end{layer}
\begin{verbatim}
    Plotdata("1","y=sin(x)","x");
    Plotdata("2","y=x","x");
\end{verbatim}

これだけで右のようにグラフが描かれる。

\vspace{15mm}
%\begin{center}\includegraphics[bb=0.00 0.00 552.03 409.02,height=30mm]{Fig/xsinx01.pdf} \end{center}

描画範囲は点NEとSWをドラッグして適当に決めよう。また,点A,B,Cが残ったままだが,これを関数名を表示する場所として利用するために適当な位置にドラッグして移動する。

\begin{center}\includegraphics[bb=0.00 0.00 556.03 238.01,height=30mm]{Fig/xsinx02.pdf} \end{center}

関数名と$x$軸上の交点を表示するために,関数 \verb|Expr()| を使って次のように書く。
\begin{verbatim}
    Expr([A,"e","y=\sin x",B,"e","y=x",[-pi,0],"s2","-\pi",[pi,0],"s2","\pi",
    [2*pi,0],"s2","2\pi"]);
\end{verbatim}

注)改行せず1行に書いてよい。

\begin{center}\includegraphics[bb=0.00 0.00 546.03 236.01,height=30mm]{Fig/xsinx03.pdf} \end{center}

\verb|Figure| ボタンをクリックすると,次の図が描かれる。

\vspace{\baselineskip}
\begin{center}%%% /Users/Hannya/Desktop/KeTCindy/fig/xsinx.tex 
%%% Generator=template1basic.cdy 
{\unitlength=7mm%
\begin{picture}%
(11.04,4.53)(-4,-2)%
\special{pn 8}%
%
{%
\color[rgb]{0,0,0}%
\special{pa -1102  -209}\special{pa -1042  -164}\special{pa  -981  -112}\special{pa  -920   -54}%
\special{pa  -859     7}\special{pa  -798    67}\special{pa  -737   124}\special{pa  -676   175}%
\special{pa  -616   217}\special{pa  -555   249}\special{pa  -494   269}\special{pa  -433   276}%
\special{pa  -372   269}\special{pa  -311   249}\special{pa  -250   217}\special{pa  -190   175}%
\special{pa  -129   124}\special{pa   -68    67}\special{pa    -7     7}\special{pa    54   -53}%
\special{pa   115  -111}\special{pa   175  -164}\special{pa   236  -208}\special{pa   297  -243}%
\special{pa   358  -265}\special{pa   419  -275}\special{pa   480  -272}\special{pa   541  -255}%
\special{pa   601  -226}\special{pa   662  -186}\special{pa   723  -136}\special{pa   784   -81}%
\special{pa   845   -21}\special{pa   906    40}\special{pa   967    99}\special{pa  1027   153}%
\special{pa  1088   199}\special{pa  1149   236}\special{pa  1210   261}\special{pa  1271   274}%
\special{pa  1332   274}\special{pa  1393   260}\special{pa  1453   233}\special{pa  1514   196}%
\special{pa  1575   148}\special{pa  1636    94}\special{pa  1697    35}\special{pa  1758   -26}%
\special{pa  1818   -85}\special{pa  1879  -141}\special{pa  1940  -189}%
\special{fp}%
}%
{%
\color[rgb]{0,0,0}%
\special{pa  -551   551}\special{pa  -494   494}\special{pa  -433   433}\special{pa  -372   372}%
\special{pa  -311   311}\special{pa  -250   250}\special{pa  -190   190}\special{pa  -129   129}%
\special{pa   -68    68}\special{pa    -7     7}\special{pa    54   -54}\special{pa   115  -115}%
\special{pa   175  -175}\special{pa   236  -236}\special{pa   297  -297}\special{pa   358  -358}%
\special{pa   419  -419}\special{pa   480  -480}\special{pa   541  -541}\special{pa   601  -601}%
\special{pa   662  -662}\special{pa   697  -697}%
\special{fp}%
}%
{%
\color[rgb]{0,0,0}%
\settowidth{\Width}{$y=\sin x$}\setlength{\Width}{0\Width}%
\settoheight{\Height}{$y=\sin x$}\settodepth{\Depth}{$y=\sin x$}\setlength{\Height}{-0.5\Height}\setlength{\Depth}{0.5\Depth}\addtolength{\Height}{\Depth}%
\put(2.9614286,0.6800000){\hspace*{\Width}\raisebox{\Height}{$y=\sin x$}}%
%
}%
{%
\color[rgb]{0,0,0}%
\settowidth{\Width}{$y=x$}\setlength{\Width}{0\Width}%
\settoheight{\Height}{$y=x$}\settodepth{\Depth}{$y=x$}\setlength{\Height}{-0.5\Height}\setlength{\Depth}{0.5\Depth}\addtolength{\Height}{\Depth}%
\put(2.5514286,2.0100000){\hspace*{\Width}\raisebox{\Height}{$y=x$}}%
%
}%
{%
\color[rgb]{0,0,0}%
\settowidth{\Width}{$-\pi$}\setlength{\Width}{-0.5\Width}%
\settoheight{\Height}{$-\pi$}\settodepth{\Depth}{$-\pi$}\setlength{\Height}{-\Height}%
\put(-3.1400000,-0.2142857){\hspace*{\Width}\raisebox{\Height}{$-\pi$}}%
%
}%
{%
\color[rgb]{0,0,0}%
\settowidth{\Width}{$\pi$}\setlength{\Width}{-0.5\Width}%
\settoheight{\Height}{$\pi$}\settodepth{\Depth}{$\pi$}\setlength{\Height}{-\Height}%
\put(3.1400000,-0.2142857){\hspace*{\Width}\raisebox{\Height}{$\pi$}}%
%
}%
{%
\color[rgb]{0,0,0}%
\settowidth{\Width}{$2\pi$}\setlength{\Width}{-0.5\Width}%
\settoheight{\Height}{$2\pi$}\settodepth{\Depth}{$2\pi$}\setlength{\Height}{-\Height}%
\put(6.2800000,-0.2142857){\hspace*{\Width}\raisebox{\Height}{$2\pi$}}%
%
}%
\special{pa -1102    -0}\special{pa  1940    -0}%
\special{fp}%
\special{pa     0   551}\special{pa     0  -697}%
\special{fp}%
\settowidth{\Width}{$x$}\setlength{\Width}{0\Width}%
\settoheight{\Height}{$x$}\settodepth{\Depth}{$x$}\setlength{\Height}{-0.5\Height}\setlength{\Depth}{0.5\Depth}\addtolength{\Height}{\Depth}%
\put(7.1114286,0.0000000){\hspace*{\Width}\raisebox{\Height}{$x$}}%
%
\settowidth{\Width}{$y$}\setlength{\Width}{-0.5\Width}%
\settoheight{\Height}{$y$}\settodepth{\Depth}{$y$}\setlength{\Height}{\Depth}%
\put(0.0000000,2.6014286){\hspace*{\Width}\raisebox{\Height}{$y$}}%
%
\settowidth{\Width}{O}\setlength{\Width}{-1\Width}%
\settoheight{\Height}{O}\settodepth{\Depth}{O}\setlength{\Height}{-\Height}%
\put(-0.0714286,-0.0714286){\hspace*{\Width}\raisebox{\Height}{O}}%
%
\end{picture}}% \end{center}


%----------------  空間図形 --------------------------
\subsubsection{空間図形}

\ketcindy のシステムに同梱されている samples フォルダから,s05spacefigure フォルダを開き,s0501basic.cdy をひな形として使う。適当な場所に複製を作り,名前を変えておくとよい。ここでは,3Dbasic.cdy として進める。

3Dbasic.cdy を開くと次のような画面になる。

\vspace{\baselineskip}
\begin{center}
 \includegraphics[bb=0 0 879.05 447.02 , width=8cm]{Fig/3dstart.pdf}
\end{center}

下のスライダで点TH,Fl をドラッグすると,空間での視点の位置が変わる。(座標軸が回転する)

ここでは,正四面体と,高さを求めるのによく使われる補助線を描いてみよう。スクリプトエディタを開き,次の3行を消す。

\begin{verbatim}
    Ch=[1];
    if(contains(Ch,1),
     Skeletonparadata("1",[1.5]);
    );
\end{verbatim}

かわりに次のスクリプトを書いて Shift+Enter で実行する。

\begin{layer}{150}{0}
\putnotese{90}{0}{ %%% /Users/Hannya/Desktop/KeTCindy/fig/tetrahedron.tex 
%%% Generator=3Dbasic.cdy 
{\unitlength=1cm%
\begin{picture}%
(5,4.65)(-2.54,-0.95)%
\special{pn 8}%
%
{%
\color[rgb]{0,0,0}%
}%
{%
\color[rgb]{0,0,0}%
}%
{%
\color[rgb]{0,0,0}%
}%
{%
\color[rgb]{0,0,0}%
}%
{%
\color[rgb]{0,0,0}%
}%
{%
\color[rgb]{0,0,0}%
}%
{%
\color[rgb]{0,0,0}%
}%
{%
\color[rgb]{0,0,0}%
}%
{%
\color[rgb]{0,0,0}%
}%
{%
\color[rgb]{0,0,0}%
}%
{%
\color[rgb]{0,0,0}%
\special{pn 8}%
\special{pa 767 -155}\special{pa 759 -155}\special{fp}\special{pa 728 -154}\special{pa 720 -154}\special{fp}%
\special{pa 688 -154}\special{pa 680 -153}\special{fp}\special{pa 649 -153}\special{pa 641 -153}\special{fp}%
\special{pa 610 -152}\special{pa 602 -152}\special{fp}\special{pa 570 -152}\special{pa 562 -152}\special{fp}%
\special{pa 531 -151}\special{pa 523 -151}\special{fp}\special{pa 492 -150}\special{pa 484 -150}\special{fp}%
\special{pa 452 -150}\special{pa 444 -150}\special{fp}\special{pa 413 -149}\special{pa 405 -149}\special{fp}%
\special{pa 374 -148}\special{pa 366 -148}\special{fp}\special{pa 334 -148}\special{pa 326 -148}\special{fp}%
\special{pa 295 -147}\special{pa 287 -147}\special{fp}\special{pa 256 -147}\special{pa 248 -146}\special{fp}%
\special{pa 217 -146}\special{pa 209 -146}\special{fp}\special{pa 177 -145}\special{pa 169 -145}\special{fp}%
\special{pa 138 -145}\special{pa 130 -145}\special{fp}\special{pa 99 -144}\special{pa 91 -144}\special{fp}%
\special{pa 59 -143}\special{pa 51 -143}\special{fp}\special{pa 20 -143}\special{pa 12 -143}\special{fp}%
\special{pa -19 -142}\special{pa -27 -142}\special{fp}\special{pa -59 -142}\special{pa -67 -141}\special{fp}%
\special{pa -98 -141}\special{pa -106 -141}\special{fp}\special{pa -137 -140}\special{pa -145 -140}\special{fp}%
\special{pa -177 -140}\special{pa -185 -139}\special{fp}\special{pa -216 -139}\special{pa -224 -139}\special{fp}%
\special{pa -255 -138}\special{pa -263 -138}\special{fp}\special{pa -295 -138}\special{pa -303 -138}\special{fp}%
\special{pa -334 -137}\special{pa -342 -137}\special{fp}\special{pa -373 -136}\special{pa -381 -136}\special{fp}%
\special{pa -413 -136}\special{pa -421 -136}\special{fp}\special{pa -452 -135}\special{pa -460 -135}\special{fp}%
\special{pa -491 -135}\special{pa -499 -134}\special{fp}\special{pa -531 -134}\special{pa -539 -134}\special{fp}%
\special{pa -570 -133}\special{pa -578 -133}\special{fp}\special{pa -609 -133}\special{pa -617 -133}\special{fp}%
\special{pa -648 -132}\special{pa -656 -132}\special{fp}\special{pa -688 -131}\special{pa -696 -131}\special{fp}%
\special{pa -727 -131}\special{pa -735 -131}\special{fp}\special{pa -766 -130}\special{pa -774 -130}\special{fp}%
\special{pa -806 -129}\special{pa -814 -129}\special{fp}\special{pn 8}%
}%
{%
\color[rgb]{0,0,0}%
\special{pa     0 -1294}\special{pa  -381    77}%
\special{fp}%
}%
{%
\color[rgb]{0,0,0}%
\special{pn 8}%
\special{pa -385 78}\special{pa -378 77}\special{fp}\special{pa -347 71}\special{pa -339 69}\special{fp}%
\special{pa -309 63}\special{pa -301 61}\special{fp}\special{pa -271 55}\special{pa -263 54}\special{fp}%
\special{pa -233 48}\special{pa -225 46}\special{fp}\special{pa -195 40}\special{pa -187 38}\special{fp}%
\special{pa -156 32}\special{pa -149 31}\special{fp}\special{pa -118 24}\special{pa -110 23}\special{fp}%
\special{pa -80 17}\special{pa -72 15}\special{fp}\special{pa -42 9}\special{pa -34 7}\special{fp}%
\special{pa -4 1}\special{pa 4 0}\special{fp}\special{pa 34 -7}\special{pa 42 -8}\special{fp}%
\special{pa 72 -14}\special{pa 80 -16}\special{fp}\special{pa 111 -22}\special{pa 118 -24}\special{fp}%
\special{pa 149 -30}\special{pa 157 -31}\special{fp}\special{pa 187 -37}\special{pa 195 -39}\special{fp}%
\special{pa 225 -45}\special{pa 233 -47}\special{fp}\special{pa 263 -53}\special{pa 271 -54}\special{fp}%
\special{pa 301 -61}\special{pa 309 -62}\special{fp}\special{pa 339 -68}\special{pa 347 -70}\special{fp}%
\special{pa 378 -76}\special{pa 385 -78}\special{fp}\special{pa 416 -84}\special{pa 424 -85}\special{fp}%
\special{pa 454 -92}\special{pa 462 -93}\special{fp}\special{pa 492 -99}\special{pa 500 -101}\special{fp}%
\special{pa 530 -107}\special{pa 538 -109}\special{fp}\special{pa 568 -115}\special{pa 576 -116}\special{fp}%
\special{pa 607 -122}\special{pa 614 -124}\special{fp}\special{pa 645 -130}\special{pa 653 -132}\special{fp}%
\special{pa 683 -138}\special{pa 691 -140}\special{fp}\special{pa 721 -146}\special{pa 729 -147}\special{fp}%
\special{pa 759 -153}\special{pa 767 -155}\special{fp}\special{pn 8}%
}%
{%
\color[rgb]{0,0,0}%
\special{pn 8}%
\special{pa 767 -155}\special{pa 759 -155}\special{fp}\special{pa 728 -154}\special{pa 720 -154}\special{fp}%
\special{pa 688 -154}\special{pa 680 -153}\special{fp}\special{pa 649 -153}\special{pa 641 -153}\special{fp}%
\special{pa 610 -152}\special{pa 602 -152}\special{fp}\special{pa 570 -152}\special{pa 562 -152}\special{fp}%
\special{pa 531 -151}\special{pa 523 -151}\special{fp}\special{pa 492 -150}\special{pa 484 -150}\special{fp}%
\special{pa 452 -150}\special{pa 444 -150}\special{fp}\special{pa 413 -149}\special{pa 405 -149}\special{fp}%
\special{pa 374 -148}\special{pa 366 -148}\special{fp}\special{pa 334 -148}\special{pa 326 -148}\special{fp}%
\special{pa 295 -147}\special{pa 287 -147}\special{fp}\special{pa 256 -147}\special{pa 248 -146}\special{fp}%
\special{pa 217 -146}\special{pa 209 -146}\special{fp}\special{pa 177 -145}\special{pa 169 -145}\special{fp}%
\special{pa 138 -145}\special{pa 130 -145}\special{fp}\special{pa 99 -144}\special{pa 91 -144}\special{fp}%
\special{pa 59 -143}\special{pa 51 -143}\special{fp}\special{pa 20 -143}\special{pa 12 -143}\special{fp}%
\special{pa -19 -142}\special{pa -27 -142}\special{fp}\special{pa -59 -142}\special{pa -67 -141}\special{fp}%
\special{pa -98 -141}\special{pa -106 -141}\special{fp}\special{pa -137 -140}\special{pa -145 -140}\special{fp}%
\special{pa -177 -140}\special{pa -185 -139}\special{fp}\special{pa -216 -139}\special{pa -224 -139}\special{fp}%
\special{pa -255 -138}\special{pa -263 -138}\special{fp}\special{pa -295 -138}\special{pa -303 -138}\special{fp}%
\special{pa -334 -137}\special{pa -342 -137}\special{fp}\special{pa -373 -136}\special{pa -381 -136}\special{fp}%
\special{pa -413 -136}\special{pa -421 -136}\special{fp}\special{pa -452 -135}\special{pa -460 -135}\special{fp}%
\special{pa -491 -135}\special{pa -499 -134}\special{fp}\special{pa -531 -134}\special{pa -539 -134}\special{fp}%
\special{pa -570 -133}\special{pa -578 -133}\special{fp}\special{pa -609 -133}\special{pa -617 -133}\special{fp}%
\special{pa -648 -132}\special{pa -656 -132}\special{fp}\special{pa -688 -131}\special{pa -696 -131}\special{fp}%
\special{pa -727 -131}\special{pa -735 -131}\special{fp}\special{pa -766 -130}\special{pa -774 -130}\special{fp}%
\special{pa -806 -129}\special{pa -814 -129}\special{fp}\special{pn 8}%
}%
{%
\color[rgb]{0,0,0}%
\special{pa  -810  -129}\special{pa    47   284}%
\special{fp}%
}%
{%
\color[rgb]{0,0,0}%
\special{pa   763  -155}\special{pa    47   284}%
\special{fp}%
}%
{%
\color[rgb]{0,0,0}%
\special{pa  -810  -129}\special{pa     0 -1295}%
\special{fp}%
}%
{%
\color[rgb]{0,0,0}%
\special{pa   763  -155}\special{pa     0 -1295}%
\special{fp}%
}%
{%
\color[rgb]{0,0,0}%
\special{pa    47   284}\special{pa     0 -1295}%
\special{fp}%
}%
\end{picture}}%}
\end{layer}
\begin{verbatim}
   Putpoint3d("A",2*[-1,-1/sqrt(3),0],"fix");
   Putpoint3d("B",2*[1,-1/sqrt(3),0],"fix");
   Putpoint3d("C",2*[0,sqrt(3)-1/sqrt(3),0],"fix");
   Putpoint3d("D",2*[0,0,sqrt(3)],"fix");
   Putpoint3d("M",(B3d+C3d)/2,"fix");
   phd=Concatobj([[A,B,C],[A,B,D],[A,C,D],[B,C,D]]);
   Spaceline("1",[D,M,A]);
   VertexEdgeFace("1",phd,["Edg=nogeo"]);
   Nohiddenbyfaces("1","phf3d1");
\end{verbatim}


%----------------  作表 --------------------------
\subsubsection{作表}

\TeX\ で表を作るのはかなり面倒だが,\ketcindy\ を使えば比較的簡単に作表ができる。次の図は関数の増減・凹凸表である。以下は紹介にとどめる。関数リファレンスに例を掲載しているので参照されたい。

\begin{center} %%% /Users/Hannya/ketcindy/ketwork/zogen3.tex 2016-12-8 15:42
%%% zogen3.sce 2016-12-8 15:42
{\unitlength=1cm%
\begin{picture}%
(   8.00000,   4.00000)(  -0.00000,  -0.00000)%
\special{pn 8}%
%
\special{pa 0 -1575}\special{pa 0 0}%
\special{fp}%
\special{pa 394 -1575}\special{pa 394 0}%
\special{fp}%
\special{pa 787 -1575}\special{pa 787 0}%
\special{fp}%
\special{pa 1181 -1575}\special{pa 1181 0}%
\special{fp}%
\special{pa 1575 -1575}\special{pa 1575 0}%
\special{fp}%
\special{pa 1969 -1575}\special{pa 1969 0}%
\special{fp}%
\special{pa 2362 -1575}\special{pa 2362 0}%
\special{fp}%
\special{pa 2756 -1575}\special{pa 2756 0}%
\special{fp}%
\special{pa 3150 -1575}\special{pa 3150 0}%
\special{fp}%
\special{pa 0 -1575}\special{pa 3150 -1575}%
\special{fp}%
\special{pa 0 -1181}\special{pa 3150 -1181}%
\special{fp}%
\special{pa 0 -787}\special{pa 3150 -787}%
\special{fp}%
\special{pa 0 -394}\special{pa 3150 -394}%
\special{fp}%
\special{pa 0 0}\special{pa 3150 0}%
\special{fp}%
\settowidth{\Width}{$x$}\setlength{\Width}{-0.5\Width}%
\settoheight{\Height}{$x$}\settodepth{\Depth}{$x$}\setlength{\Height}{-0.5\Height}\setlength{\Depth}{0.5\Depth}\addtolength{\Height}{\Depth}%
\put(0.5000,3.5000){\hspace*{\Width}\raisebox{\Height}{$x$}}%
%
%
\settowidth{\Width}{$\cdots$}\setlength{\Width}{-0.5\Width}%
\settoheight{\Height}{$\cdots$}\settodepth{\Depth}{$\cdots$}\setlength{\Height}{-0.5\Height}\setlength{\Depth}{0.5\Depth}\addtolength{\Height}{\Depth}%
\put(1.5000,3.5000){\hspace*{\Width}\raisebox{\Height}{$\cdots$}}%
%
%
\settowidth{\Width}{$-1$}\setlength{\Width}{-0.5\Width}%
\settoheight{\Height}{$-1$}\settodepth{\Depth}{$-1$}\setlength{\Height}{-0.5\Height}\setlength{\Depth}{0.5\Depth}\addtolength{\Height}{\Depth}%
\put(2.5000,3.5000){\hspace*{\Width}\raisebox{\Height}{$-1$}}%
%
%
\settowidth{\Width}{$\cdots$}\setlength{\Width}{-0.5\Width}%
\settoheight{\Height}{$\cdots$}\settodepth{\Depth}{$\cdots$}\setlength{\Height}{-0.5\Height}\setlength{\Depth}{0.5\Depth}\addtolength{\Height}{\Depth}%
\put(3.5000,3.5000){\hspace*{\Width}\raisebox{\Height}{$\cdots$}}%
%
%
\settowidth{\Width}{$0$}\setlength{\Width}{-0.5\Width}%
\settoheight{\Height}{$0$}\settodepth{\Depth}{$0$}\setlength{\Height}{-0.5\Height}\setlength{\Depth}{0.5\Depth}\addtolength{\Height}{\Depth}%
\put(4.5000,3.5000){\hspace*{\Width}\raisebox{\Height}{$0$}}%
%
%
\settowidth{\Width}{$\cdots$}\setlength{\Width}{-0.5\Width}%
\settoheight{\Height}{$\cdots$}\settodepth{\Depth}{$\cdots$}\setlength{\Height}{-0.5\Height}\setlength{\Depth}{0.5\Depth}\addtolength{\Height}{\Depth}%
\put(5.5000,3.5000){\hspace*{\Width}\raisebox{\Height}{$\cdots$}}%
%
%
\settowidth{\Width}{$1$}\setlength{\Width}{-0.5\Width}%
\settoheight{\Height}{$1$}\settodepth{\Depth}{$1$}\setlength{\Height}{-0.5\Height}\setlength{\Depth}{0.5\Depth}\addtolength{\Height}{\Depth}%
\put(6.5000,3.5000){\hspace*{\Width}\raisebox{\Height}{$1$}}%
%
%
\settowidth{\Width}{$\cdots$}\setlength{\Width}{-0.5\Width}%
\settoheight{\Height}{$\cdots$}\settodepth{\Depth}{$\cdots$}\setlength{\Height}{-0.5\Height}\setlength{\Depth}{0.5\Depth}\addtolength{\Height}{\Depth}%
\put(7.5000,3.5000){\hspace*{\Width}\raisebox{\Height}{$\cdots$}}%
%
%
\settowidth{\Width}{$y'$}\setlength{\Width}{-0.5\Width}%
\settoheight{\Height}{$y'$}\settodepth{\Depth}{$y'$}\setlength{\Height}{-0.5\Height}\setlength{\Depth}{0.5\Depth}\addtolength{\Height}{\Depth}%
\put(0.5000,2.5000){\hspace*{\Width}\raisebox{\Height}{$y'$}}%
%
%
\settowidth{\Width}{$+$}\setlength{\Width}{-0.5\Width}%
\settoheight{\Height}{$+$}\settodepth{\Depth}{$+$}\setlength{\Height}{-0.5\Height}\setlength{\Depth}{0.5\Depth}\addtolength{\Height}{\Depth}%
\put(1.5000,2.5000){\hspace*{\Width}\raisebox{\Height}{$+$}}%
%
%
\settowidth{\Width}{$+$}\setlength{\Width}{-0.5\Width}%
\settoheight{\Height}{$+$}\settodepth{\Depth}{$+$}\setlength{\Height}{-0.5\Height}\setlength{\Depth}{0.5\Depth}\addtolength{\Height}{\Depth}%
\put(2.5000,2.5000){\hspace*{\Width}\raisebox{\Height}{$+$}}%
%
%
\settowidth{\Width}{$+$}\setlength{\Width}{-0.5\Width}%
\settoheight{\Height}{$+$}\settodepth{\Depth}{$+$}\setlength{\Height}{-0.5\Height}\setlength{\Depth}{0.5\Depth}\addtolength{\Height}{\Depth}%
\put(3.5000,2.5000){\hspace*{\Width}\raisebox{\Height}{$+$}}%
%
%
\settowidth{\Width}{$0$}\setlength{\Width}{-0.5\Width}%
\settoheight{\Height}{$0$}\settodepth{\Depth}{$0$}\setlength{\Height}{-0.5\Height}\setlength{\Depth}{0.5\Depth}\addtolength{\Height}{\Depth}%
\put(4.5000,2.5000){\hspace*{\Width}\raisebox{\Height}{$0$}}%
%
%
\settowidth{\Width}{$-$}\setlength{\Width}{-0.5\Width}%
\settoheight{\Height}{$-$}\settodepth{\Depth}{$-$}\setlength{\Height}{-0.5\Height}\setlength{\Depth}{0.5\Depth}\addtolength{\Height}{\Depth}%
\put(5.5000,2.5000){\hspace*{\Width}\raisebox{\Height}{$-$}}%
%
%
\settowidth{\Width}{$-$}\setlength{\Width}{-0.5\Width}%
\settoheight{\Height}{$-$}\settodepth{\Depth}{$-$}\setlength{\Height}{-0.5\Height}\setlength{\Depth}{0.5\Depth}\addtolength{\Height}{\Depth}%
\put(6.5000,2.5000){\hspace*{\Width}\raisebox{\Height}{$-$}}%
%
%
\settowidth{\Width}{$-$}\setlength{\Width}{-0.5\Width}%
\settoheight{\Height}{$-$}\settodepth{\Depth}{$-$}\setlength{\Height}{-0.5\Height}\setlength{\Depth}{0.5\Depth}\addtolength{\Height}{\Depth}%
\put(7.5000,2.5000){\hspace*{\Width}\raisebox{\Height}{$-$}}%
%
%
\settowidth{\Width}{$y''$}\setlength{\Width}{-0.5\Width}%
\settoheight{\Height}{$y''$}\settodepth{\Depth}{$y''$}\setlength{\Height}{-0.5\Height}\setlength{\Depth}{0.5\Depth}\addtolength{\Height}{\Depth}%
\put(0.5000,1.5000){\hspace*{\Width}\raisebox{\Height}{$y''$}}%
%
%
\settowidth{\Width}{$+$}\setlength{\Width}{-0.5\Width}%
\settoheight{\Height}{$+$}\settodepth{\Depth}{$+$}\setlength{\Height}{-0.5\Height}\setlength{\Depth}{0.5\Depth}\addtolength{\Height}{\Depth}%
\put(1.5000,1.5000){\hspace*{\Width}\raisebox{\Height}{$+$}}%
%
%
\settowidth{\Width}{$0$}\setlength{\Width}{-0.5\Width}%
\settoheight{\Height}{$0$}\settodepth{\Depth}{$0$}\setlength{\Height}{-0.5\Height}\setlength{\Depth}{0.5\Depth}\addtolength{\Height}{\Depth}%
\put(2.5000,1.5000){\hspace*{\Width}\raisebox{\Height}{$0$}}%
%
%
\settowidth{\Width}{$-$}\setlength{\Width}{-0.5\Width}%
\settoheight{\Height}{$-$}\settodepth{\Depth}{$-$}\setlength{\Height}{-0.5\Height}\setlength{\Depth}{0.5\Depth}\addtolength{\Height}{\Depth}%
\put(3.5000,1.5000){\hspace*{\Width}\raisebox{\Height}{$-$}}%
%
%
\settowidth{\Width}{$-$}\setlength{\Width}{-0.5\Width}%
\settoheight{\Height}{$-$}\settodepth{\Depth}{$-$}\setlength{\Height}{-0.5\Height}\setlength{\Depth}{0.5\Depth}\addtolength{\Height}{\Depth}%
\put(4.5000,1.5000){\hspace*{\Width}\raisebox{\Height}{$-$}}%
%
%
\settowidth{\Width}{$-$}\setlength{\Width}{-0.5\Width}%
\settoheight{\Height}{$-$}\settodepth{\Depth}{$-$}\setlength{\Height}{-0.5\Height}\setlength{\Depth}{0.5\Depth}\addtolength{\Height}{\Depth}%
\put(5.5000,1.5000){\hspace*{\Width}\raisebox{\Height}{$-$}}%
%
%
\settowidth{\Width}{$0$}\setlength{\Width}{-0.5\Width}%
\settoheight{\Height}{$0$}\settodepth{\Depth}{$0$}\setlength{\Height}{-0.5\Height}\setlength{\Depth}{0.5\Depth}\addtolength{\Height}{\Depth}%
\put(6.5000,1.5000){\hspace*{\Width}\raisebox{\Height}{$0$}}%
%
%
\settowidth{\Width}{$+$}\setlength{\Width}{-0.5\Width}%
\settoheight{\Height}{$+$}\settodepth{\Depth}{$+$}\setlength{\Height}{-0.5\Height}\setlength{\Depth}{0.5\Depth}\addtolength{\Height}{\Depth}%
\put(7.5000,1.5000){\hspace*{\Width}\raisebox{\Height}{$+$}}%
%
%
\settowidth{\Width}{$y$}\setlength{\Width}{-0.5\Width}%
\settoheight{\Height}{$y$}\settodepth{\Depth}{$y$}\setlength{\Height}{-0.5\Height}\setlength{\Depth}{0.5\Depth}\addtolength{\Height}{\Depth}%
\put(0.5000,0.5000){\hspace*{\Width}\raisebox{\Height}{$y$}}%
%
%
\settowidth{\Width}{$\nelarrow$}\setlength{\Width}{-0.5\Width}%
\settoheight{\Height}{$\nelarrow$}\settodepth{\Depth}{$\nelarrow$}\setlength{\Height}{-0.5\Height}\setlength{\Depth}{0.5\Depth}\addtolength{\Height}{\Depth}%
\put(1.5000,0.5000){\hspace*{\Width}\raisebox{\Height}{$\nelarrow$}}%
%
%
\settowidth{\Width}{$\frac{1}{\sqrt{e}}$}\setlength{\Width}{-0.5\Width}%
\settoheight{\Height}{$\frac{1}{\sqrt{e}}$}\settodepth{\Depth}{$\frac{1}{\sqrt{e}}$}\setlength{\Height}{-0.5\Height}\setlength{\Depth}{0.5\Depth}\addtolength{\Height}{\Depth}%
\put(2.5000,0.5000){\hspace*{\Width}\raisebox{\Height}{$\frac{1}{\sqrt{e}}$}}%
%
%
\settowidth{\Width}{$\nerarrow$}\setlength{\Width}{-0.5\Width}%
\settoheight{\Height}{$\nerarrow$}\settodepth{\Depth}{$\nerarrow$}\setlength{\Height}{-0.5\Height}\setlength{\Depth}{0.5\Depth}\addtolength{\Height}{\Depth}%
\put(3.5000,0.5000){\hspace*{\Width}\raisebox{\Height}{$\nerarrow$}}%
%
%
\settowidth{\Width}{$1$}\setlength{\Width}{-0.5\Width}%
\settoheight{\Height}{$1$}\settodepth{\Depth}{$1$}\setlength{\Height}{-0.5\Height}\setlength{\Depth}{0.5\Depth}\addtolength{\Height}{\Depth}%
\put(4.5000,0.5000){\hspace*{\Width}\raisebox{\Height}{$1$}}%
%
%
\settowidth{\Width}{$\serarrow$}\setlength{\Width}{-0.5\Width}%
\settoheight{\Height}{$\serarrow$}\settodepth{\Depth}{$\serarrow$}\setlength{\Height}{-0.5\Height}\setlength{\Depth}{0.5\Depth}\addtolength{\Height}{\Depth}%
\put(5.5000,0.5000){\hspace*{\Width}\raisebox{\Height}{$\serarrow$}}%
%
%
\settowidth{\Width}{$\frac{1}{\sqrt{e}}$}\setlength{\Width}{-0.5\Width}%
\settoheight{\Height}{$\frac{1}{\sqrt{e}}$}\settodepth{\Depth}{$\frac{1}{\sqrt{e}}$}\setlength{\Height}{-0.5\Height}\setlength{\Depth}{0.5\Depth}\addtolength{\Height}{\Depth}%
\put(6.5000,0.5000){\hspace*{\Width}\raisebox{\Height}{$\frac{1}{\sqrt{e}}$}}%
%
%
\settowidth{\Width}{$\selarrow$}\setlength{\Width}{-0.5\Width}%
\settoheight{\Height}{$\selarrow$}\settodepth{\Depth}{$\selarrow$}\setlength{\Height}{-0.5\Height}\setlength{\Depth}{0.5\Depth}\addtolength{\Height}{\Depth}%
\put(7.5000,0.5000){\hspace*{\Width}\raisebox{\Height}{$\selarrow$}}%
%
%
\end{picture}}% \end{center}

% ====== 他のソフトとの連携 ===============

\subsubsection{他のソフトとの連携}

\ketcindy\ はCindyscriptで記述されているが,Cindyscriptはプログラミング言語であり,数式処理ソフトではない。そこで,R や Maxima などと連携することにより,機能を拡張することができるようになっている。統計計算はR,数式処理を用いた計算はMaximaを利用すると便利である。

【例】Rを用いて箱ひげ図を描く

 \begin{center}\scalebox{0.8}{ %%% t4boxplot2.tex 2016-4-23 9:27
%%% t4boxplot2.sce 2016-4-23 9:27
{\unitlength=1cm%
\begin{picture}%
(   8.00000,   6.00000)(  -1.00000,  -1.00000)%
\special{pn 8}%
%
\special{pa 394 -295}\special{pa 394 -492}%
\special{fp}%
\special{pa 1181 -197}\special{pa 1181 -591}\special{pa 1713 -591}\special{pa 1713 -197}%
\special{pa 1181 -197}%
\special{fp}%
\special{pa 1949 -295}\special{pa 1949 -492}%
\special{fp}%
\special{pn 16}%
\special{pa 1378 -197}\special{pa 1378 -591}%
\special{fp}%
\special{pn 8}%
\special{pa 394 -394}\special{pa 431 -394}\special{fp}\special{pa 469 -394}\special{pa 506 -394}\special{fp}%
\special{pa 544 -394}\special{pa 581 -394}\special{fp}\special{pa 619 -394}\special{pa 656 -394}\special{fp}%
\special{pa 694 -394}\special{pa 731 -394}\special{fp}\special{pa 769 -394}\special{pa 806 -394}\special{fp}%
\special{pa 844 -394}\special{pa 881 -394}\special{fp}\special{pa 919 -394}\special{pa 956 -394}\special{fp}%
\special{pa 994 -394}\special{pa 1031 -394}\special{fp}\special{pa 1069 -394}\special{pa 1106 -394}\special{fp}%
\special{pa 1144 -394}\special{pa 1181 -394}\special{fp}%
%
\special{pa 1713 -394}\special{pa 1746 -394}\special{fp}\special{pa 1780 -394}\special{pa 1814 -394}\special{fp}%
\special{pa 1848 -394}\special{pa 1881 -394}\special{fp}\special{pa 1915 -394}\special{pa 1949 -394}\special{fp}%
%
%
\special{pa 787 -1083}\special{pa 787 -1280}%
\special{fp}%
\special{pa 1299 -984}\special{pa 1299 -1378}\special{pa 1673 -1378}\special{pa 1673 -984}%
\special{pa 1299 -984}%
\special{fp}%
\special{pa 1969 -1083}\special{pa 1969 -1280}%
\special{fp}%
\special{pn 16}%
\special{pa 1378 -984}\special{pa 1378 -1378}%
\special{fp}%
\special{pn 8}%
\special{pa 787 -1181}\special{pa 827 -1181}\special{fp}\special{pa 866 -1181}\special{pa 906 -1181}\special{fp}%
\special{pa 945 -1181}\special{pa 984 -1181}\special{fp}\special{pa 1024 -1181}\special{pa 1063 -1181}\special{fp}%
\special{pa 1102 -1181}\special{pa 1142 -1181}\special{fp}\special{pa 1181 -1181}\special{pa 1220 -1181}\special{fp}%
\special{pa 1260 -1181}\special{pa 1299 -1181}\special{fp}%
%
\special{pa 1673 -1181}\special{pa 1706 -1181}\special{fp}\special{pa 1739 -1181}\special{pa 1772 -1181}\special{fp}%
\special{pa 1804 -1181}\special{pa 1837 -1181}\special{fp}\special{pa 1870 -1181}\special{pa 1903 -1181}\special{fp}%
\special{pa 1936 -1181}\special{pa 1969 -1181}\special{fp}%
%
\special{pa 0 0}\special{pa 2386 0}\special{pa 2386 -1709}\special{pa 0 -1709}\special{pa 0 0}%
\special{fp}%
\settowidth{\Width}{$0$}\setlength{\Width}{-0.5\Width}%
\settoheight{\Height}{$0$}\settodepth{\Depth}{$0$}\setlength{\Height}{-\Height}%
\put(0.0000,-0.1500){\hspace*{\Width}\raisebox{\Height}{$0$}}%
%
%
\special{pa 0 0}\special{pa 0 39}%
\special{fp}%
\settowidth{\Width}{$1$}\setlength{\Width}{-0.5\Width}%
\settoheight{\Height}{$1$}\settodepth{\Depth}{$1$}\setlength{\Height}{-\Height}%
\put(1.0000,-0.1500){\hspace*{\Width}\raisebox{\Height}{$1$}}%
%
%
\special{pa 394 0}\special{pa 394 39}%
\special{fp}%
\settowidth{\Width}{$2$}\setlength{\Width}{-0.5\Width}%
\settoheight{\Height}{$2$}\settodepth{\Depth}{$2$}\setlength{\Height}{-\Height}%
\put(2.0000,-0.1500){\hspace*{\Width}\raisebox{\Height}{$2$}}%
%
%
\special{pa 787 0}\special{pa 787 39}%
\special{fp}%
\settowidth{\Width}{$3$}\setlength{\Width}{-0.5\Width}%
\settoheight{\Height}{$3$}\settodepth{\Depth}{$3$}\setlength{\Height}{-\Height}%
\put(3.0000,-0.1500){\hspace*{\Width}\raisebox{\Height}{$3$}}%
%
%
\special{pa 1181 0}\special{pa 1181 39}%
\special{fp}%
\settowidth{\Width}{$4$}\setlength{\Width}{-0.5\Width}%
\settoheight{\Height}{$4$}\settodepth{\Depth}{$4$}\setlength{\Height}{-\Height}%
\put(4.0000,-0.1500){\hspace*{\Width}\raisebox{\Height}{$4$}}%
%
%
\special{pa 1575 0}\special{pa 1575 39}%
\special{fp}%
\settowidth{\Width}{$5$}\setlength{\Width}{-0.5\Width}%
\settoheight{\Height}{$5$}\settodepth{\Depth}{$5$}\setlength{\Height}{-\Height}%
\put(5.0000,-0.1500){\hspace*{\Width}\raisebox{\Height}{$5$}}%
%
%
\special{pa 1969 0}\special{pa 1969 39}%
\special{fp}%
\settowidth{\Width}{$6$}\setlength{\Width}{-0.5\Width}%
\settoheight{\Height}{$6$}\settodepth{\Depth}{$6$}\setlength{\Height}{-\Height}%
\put(6.0000,-0.1500){\hspace*{\Width}\raisebox{\Height}{$6$}}%
%
%
\special{pa 2362 0}\special{pa 2362 39}%
\special{fp}%
\settowidth{\Width}{$\mbox{dt1}$}\setlength{\Width}{-1\Width}%
\settoheight{\Height}{$\mbox{dt1}$}\settodepth{\Depth}{$\mbox{dt1}$}\setlength{\Height}{-0.5\Height}\setlength{\Depth}{0.5\Depth}\addtolength{\Height}{\Depth}%
\put(-0.1500,1.0000){\hspace*{\Width}\raisebox{\Height}{$\mbox{dt1}$}}%
%
%
\special{pa 0 -394}\special{pa -39 -394}%
\special{fp}%
\settowidth{\Width}{$\mbox{dt2}$}\setlength{\Width}{-1\Width}%
\settoheight{\Height}{$\mbox{dt2}$}\settodepth{\Depth}{$\mbox{dt2}$}\setlength{\Height}{-0.5\Height}\setlength{\Depth}{0.5\Depth}\addtolength{\Height}{\Depth}%
\put(-0.1500,3.0000){\hspace*{\Width}\raisebox{\Height}{$\mbox{dt2}$}}%
%
%
\special{pa 0 -1181}\special{pa -39 -1181}%
\special{fp}%
\end{picture}}% } \end{center}
 
          
【例】Maxima を用いて $ \sin x $の7次のテイラー展開を行う。
 
 \begin{center}\scalebox{0.8}{ %%% taylor.tex 2016-4-29 16:11
%%% taylor.sce 2016-4-29 16:5
{\unitlength=6mm%
\begin{picture}%
(  10.00000,   6.00000)(  -5.00000,  -3.00000)%
\special{pn 8}%
%
\special{pa -1181 -227}\special{pa -1143 -234}\special{fp}\special{pa -1104 -235}\special{pa -1087 -235}\special{pa -1066 -230}\special{fp}%
\special{pa -1028 -220}\special{pa -992 -206}\special{pa -992 -206}\special{fp}\special{pa -958 -186}\special{pa -945 -179}\special{pa -926 -165}\special{fp}%
\special{pa -894 -142}\special{pa -865 -117}\special{fp}\special{pa -836 -91}\special{pa -807 -64}\special{fp}%
\special{pa -779 -37}\special{pa -756 -14}\special{pa -752 -9}\special{fp}\special{pa -724 18}\special{pa -709 33}\special{pa -696 45}\special{fp}%
\special{pa -668 72}\special{pa -661 79}\special{pa -640 99}\special{fp}\special{pa -610 125}\special{pa -580 149}\special{fp}%
\special{pa -549 172}\special{pa -520 191}\special{pa -516 193}\special{fp}\special{pa -481 210}\special{pa -472 215}\special{pa -445 224}\special{fp}%
\special{pa -407 232}\special{pa -378 236}\special{pa -368 235}\special{fp}\special{pa -329 232}\special{pa -292 222}\special{fp}%
\special{pa -256 208}\special{pa -236 199}\special{pa -221 189}\special{fp}\special{pa -188 169}\special{pa -157 145}\special{fp}%
\special{pa -127 121}\special{pa -98 95}\special{fp}\special{pa -70 68}\special{pa -47 47}\special{pa -41 41}\special{fp}%
\special{pa -14 14}\special{pa 0 0}\special{pa 14 -14}\special{fp}\special{pa 41 -41}\special{pa 47 -47}\special{pa 70 -68}\special{fp}%
\special{pa 98 -95}\special{pa 127 -121}\special{fp}\special{pa 157 -145}\special{pa 188 -169}\special{fp}%
\special{pa 221 -189}\special{pa 236 -199}\special{pa 256 -208}\special{fp}\special{pa 292 -222}\special{pa 329 -232}\special{fp}%
\special{pa 368 -235}\special{pa 378 -236}\special{pa 407 -232}\special{fp}\special{pa 445 -224}\special{pa 472 -215}\special{pa 481 -210}\special{fp}%
\special{pa 516 -193}\special{pa 520 -191}\special{pa 549 -172}\special{fp}\special{pa 580 -149}\special{pa 610 -125}\special{fp}%
\special{pa 640 -99}\special{pa 661 -79}\special{pa 668 -72}\special{fp}\special{pa 696 -45}\special{pa 709 -33}\special{pa 724 -18}\special{fp}%
\special{pa 752 9}\special{pa 756 14}\special{pa 779 37}\special{fp}\special{pa 807 64}\special{pa 836 91}\special{fp}%
\special{pa 865 117}\special{pa 894 142}\special{fp}\special{pa 926 165}\special{pa 945 179}\special{pa 958 186}\special{fp}%
\special{pa 992 206}\special{pa 992 206}\special{pa 1028 220}\special{fp}\special{pa 1066 230}\special{pa 1087 235}\special{pa 1104 235}\special{fp}%
\special{pa 1143 234}\special{pa 1181 227}\special{fp}%
%
\special{pa -1080 -709}\special{pa -1039 -564}\special{pa -992 -433}\special{pa -945 -327}%
\special{pa -898 -239}\special{pa -850 -163}\special{pa -803 -96}\special{pa -756 -35}%
\special{pa -709 22}\special{pa -661 73}\special{pa -614 118}\special{pa -567 158}%
\special{pa -520 190}\special{pa -472 214}\special{pa -425 230}\special{pa -378 236}%
\special{pa -331 233}\special{pa -283 220}\special{pa -236 199}\special{pa -189 169}%
\special{pa -142 133}\special{pa -94 92}\special{pa -47 47}\special{pa 0 0}\special{pa 47 -47}%
\special{pa 94 -92}\special{pa 142 -133}\special{pa 189 -169}\special{pa 236 -199}%
\special{pa 283 -220}\special{pa 331 -233}\special{pa 378 -236}\special{pa 425 -230}%
\special{pa 472 -214}\special{pa 520 -190}\special{pa 567 -158}\special{pa 614 -118}%
\special{pa 661 -73}\special{pa 709 -22}\special{pa 756 35}\special{pa 803 96}\special{pa 850 163}%
\special{pa 898 239}\special{pa 945 327}\special{pa 992 433}\special{pa 1039 564}%
\special{pa 1080 709}%
\special{fp}%
\special{pa -1181 0}\special{pa 1181 0}%
\special{fp}%
\special{pa 0 709}\special{pa 0 -709}%
\special{fp}%
\settowidth{\Width}{$x$}\setlength{\Width}{0\Width}%
\settoheight{\Height}{$x$}\settodepth{\Depth}{$x$}\setlength{\Height}{-0.5\Height}\setlength{\Depth}{0.5\Depth}\addtolength{\Height}{\Depth}%
\put(5.0500,0.0000){\hspace*{\Width}\raisebox{\Height}{$x$}}%
%
%
\settowidth{\Width}{$y$}\setlength{\Width}{-0.5\Width}%
\settoheight{\Height}{$y$}\settodepth{\Depth}{$y$}\setlength{\Height}{\Depth}%
\put(0.0000,3.0500){\hspace*{\Width}\raisebox{\Height}{$y$}}%
%
%
\settowidth{\Width}{O}\setlength{\Width}{0\Width}%
\settoheight{\Height}{O}\settodepth{\Depth}{O}\setlength{\Height}{-\Height}%
\put(0.0500,-0.0500){\hspace*{\Width}\raisebox{\Height}{O}}%
%
%
\end{picture}}%} \end{center}

% ====== プロットデータ ===============
\newpage
\section{プロットデータ} 
プロットデータ(Plot Data) とは,関数のグラフや幾何要素を描くデータのことである。\ketcindy\ では PD と略すことがある。

たとえば,線分は端点の座標2つからなるリストで表現できる。曲線は,描画範囲を分割して線分の集まりとして描画しており,このときのプロットデータはそれらの線分の端点のリストである。

プロットデータの名称は\ketcindy が次の規則により命名する。

\vspace{\baselineskip}
・名称の頭部は,プロットデータを作成する関数ごとに決まっている。

・第1引数に name が与えられる場合,name を頭部に付加する。

\hspace{10mm} 【例】\verb|Listplot("1",[[0,0],[1,2]]);|  のとき,sg1
      
・第1引数の name を省略できる場合,引数で用いられた点の名前を頭部に付加する。

\hspace{10mm} 【例】\verb|Listplot([A,B,C]);| のとき,sgABC


\vspace{\baselineskip}
プロットデータを生成したときは,Cindyscriptエディタのコンソールにその名称を表示する。たとえば,\verb|Listplot([A,B,C,A])| で三角形ABCを描くと,コンソールに

\hspace{10mm}  \verb|generate Listplot sgABCA|

と表示される。プロットデータを操作する関数では,この名称を用いる。

\begin{center}\includegraphics[bb=0.00 0.00 298.02 115.01,width=6cm]{Fig/pdtoconsole.pdf} \end{center}

プロットデータの内容は,Cindyscriptの println() 関数を用いてコンソールに表示することができる。たとえば上記の場合,

\hspace{10mm}  \verb|println(sgABCA)|
        
とすると,

\hspace{10mm}  \verb| [[1,3],[-1,0],[3,0],[1,3]] |

と表示される。A,B,C,A のそれぞれの座標からなるリストである。

プロットデータは,Cindyscriptによるプログラムで作成してそれを\ketcindy で利用することもできる。( \hyperlink{listplot}{Listplot()の例}  を参照)ただし,要素の数が大きいとエラーとなるので,1つのプロットデータの要素は200程度とするのがよい。これより多い場合は分割する。

\newpage
%======= Cindyscript ===========

\section{Cindyscript}
\subsection{Cindyscriptエディタ}
CindyscriptはCinderellaのプログラミング言語で,Cinderella上のスクリプトエディタで記述する。スクリプトエディタは,「スクリプト」メニューの「Cindyscript」を選択するか,Ctrl+9 (Windows) / ⌘+9 (Mac) で開く。\\
\vspace{110mm}
\begin{layer}{150}{0}
\putnotese{5.5}{15}{\includegraphics[bb=0.00 0.00 687.84 451.68,width=12.5cm]{Fig/slot.pdf}}
\arrowlineseg[16]{20}{20}{10}{90}
\putnotese{15}{5}{スロット}
\arrowlineseg[16]{40}{20}{10}{100}
\putnotese{32}{5}{ページ名}
\arrowlineseg[16]{80}{20}{15}{140}
\putnotese{50}{5}{フォントサイズ}
\arrowlineseg[16]{107}{20}{15}{140}
\putnotese{80}{5}{描画面を前面に}
\arrowlineseg[16]{120}{20}{10}{110}
\putnotese{110}{5}{実行}
\arrowlineseg[16]{125}{20}{10}{90}
\putnotese{120}{5}{ヘルプ}
\putnotese{90}{35}{メインウィンドウ}
\putnotese{90}{65}{コンソール}
\end{layer}

\noindent
{\bf  スロット}

スロットはそれぞれの実行タイミングでスクリプトを実行するものであり,他のプログラミング言語にはない特徴である。(スロットが隠れているときは境界線をドラッグする)

よく使うスロットは次の通り。

\begin{tabbing}
123456789012345678\=\kill
Draw \>描画面になにか変化が起きる(点を動かすなど)と実行される。\\
 \>通常はここにスクリプトを書く。ひな形の templatebasic1.cdy では,\\
 \>Ketinit() などが記述された figures ページが用意されている。\\
Initialization \>スクリプトを実行すると、始めに1度 だけ実行される。\\
 \>したがって,関数定義や変数の初期設定などを書く。\\
 \>ひな形の templatebasic1.cdy ではここに KETlib というページがあり,\\
 \>\ketcindy の初期設定に関する記述がある。\\
Key Typed   \>キーボードが押されたとき実行される。\\
   \> KeTCindyでは,ボタンによらずキーボードで出力を行うための\\
   \>スクリプトが書かれている。
\end{tabbing}

1つのスロットに複数のページを作ることができる。たとえば,KETlib以外に初期設定のスクリプトを書く場合は,Initialzation スロットのフォルダアイコンをクリックすることで新しいページができる。

KeTCindyの描画コマンドは Draw スロットに書く。

\vspace{\baselineskip}\noindent
{\bf  ページ名}

各スロットでは,ページを分けて記述することができる。各ページの名前はスクリプトエディタの上の欄に書くことができる。

%\vspace{\baselineskip}
\noindent
{\bf  フォントサイズ}

編集エリアのフォントサイズを変更する。

%\vspace{\baselineskip}
\noindent
{\bf  実行ボタン}

プログラムを実行する。プログラムの実行は Shift+Enter でもできる。

%\vspace{\baselineskip}
\noindent
{\bf  ヘルプボタン}

ブラウザを開いてマニュアルを表示する。

%\vspace{\baselineskip}
\noindent
{\bf  コンソール}

print() 関数の結果やエラーメッセージが表示する。エラーメッセージは,「WARNING:」または「syntax error」に続いてその内容と該当する行番号が示される。これを読んでスクリプトの書き間違いをチェックする。

\subsection{スクリプトの記述}
編集エリアにプログラムを書くと,文字が色分けされて表示される。組み込み関数は青,ユーザー定義関数は紫,定義されていない関数は赤,文字列は緑で表示される。KeTCindyの関数はユーザー定義関数なので紫色で表示される。

\begin{layer}{150}{0}
\putnotese{35}{5}{\includegraphics[bb=0.00 0.00 350.02 184.01,width=7cm]{Fig/coloring.pdf} }
\putnotee{5}{27}{組み込み関数}
\arrowlineseg[20]{35}{27}{10}{180}
\putnotee{77}{9}{ユーザー定義関数}
\arrowlineseg[10]{67}{9}{8}{0}
\putnotee{110}{17}{文字列}
\arrowlineseg[15]{80}{17}{28}{0}
\end{layer}
\vspace{45mm}

編集エリアでは,Ctrl+C と Ctrl+V によるコピーアンドペースト,Ctrl+X と Ctrl+V によるカットアンドぺーストができる。他のテキストエディタなどとの間でのコピーも同様にできる。

文字列の選択はマウスドラッグまたは Shift+カーソルキーでおこなえる。

Ctrl+F による検索はできない。

スクリプトを記述するときの基本的なルールは次の通り。

\vspace{\baselineskip}
・基本的に小文字で書く。大文字と小文字は区別される。

\hspace{5mm}\ketcindy では,Cindyscriptに組み込みの変数名・関数名と区別しやすいように,

\hspace{5mm}次の規則により名前を付けている。

\hspace{5mm}・グローバルな変数はすべて大文字か,大文字で始まるものとする。

\hspace{5mm}・局所変数は小文字で,関数定義の冒頭で regional() により局所変数として宣言する。

\hspace{5mm}・関数名は大文字で始まる。

・複数の半角スペースは無視され,一つの半角スペースと見なされる。

・行末にはセミコロンを書く。改行だけでは命令文の終わりにならない。

\subsection{変数と定数}
\vspace{\baselineskip}\noindent
{\bf  変数}

Cindyscriptでは,変数の型の宣言は不要。使用されたときに何が代入されたかで自動的に型が決まり,さらに,宣言し直さなくても異なる型の値を代入することができる。

\vspace{\baselineskip}

【例】
\begin{verbatim}
    a=10;
    b=2;
    c=a+sqrt(b);
    a="の平方根";
    println("10に "+b+a+" を加えると"+c);
\end{verbatim}

この例では,始めにaは整数型であるが,4行目で文字列に変わる。

文字列はダブルクウォートでくくる。異なる型の演算には注意を要するが,例外的に,5行目のように,文字列と数を+演算子で結ぶと,数は文字列化されて結合される。

\vspace{\baselineskip}\noindent
{\bf  予約定数}

Cindyscriptでは,円周率 (pi) と虚数単位(i) が定数として予約されている。i は,変数として使用することもでき,そのような場合,虚数単位に戻すには  \verb|i=complex(0,1)| を実行する。

\vspace{\baselineskip}\noindent
{\bf KeTCindyの予約変数}

 \ketcindy  が内部的に使用する予約変数がある。そのうち次のものはユーザーが値を変更または設定することができる。
\begin{tabbing}
1234\=567890123\=45678989012345678901234567890123\=\kill
  \>Fhead  \>書き出されるファイル名の頭部。Setfiles() によって設定できる。\\
  \>Texparent  \>親プロセスのファイル名。Setparent()によって設定できる。\\
  \>Dirhead  \>パスの頭部\\
  \>Dirlib  \>ライブラリ ketlib のパス\\
  \>Dirbin  \>ketbin のパス\\
  \>Dirwork  \>作業ディレクトリのパス。Changework()によって設定できる。\\
  \>Shellfile  \>シェルファイル名
\end{tabbing}

以下の予約変数は,ライブラリが使用するグローバル変数であるので,ユーザーはこれらの変数に値を代入してはいけない。なお,変数は大文字小文字を区別するので,すべて小文字で書く分には支障はない。ユーザーが作るプログラムでは,すべて小文字か,先頭だけが大文字の変数を使うことを勧める。

\vspace{\baselineskip}
ADDAXES, ArrowlineNumber, ArrowheadNumber, BezierNumber, COM0thlist, COM1stlist, COM2ndlist, Dq, FUNLIST, Fnamesc ,Fnamescibody,Fnameout,Fnametex, GDATALIST, GLIST, GCLIST, GOUTLIST, KCOLOR, KETPICCOUNT,KETPICLAYER, LETTERlist, LFmark, MilliIn, PenThick,PenThickInit,  POUTLIST, SCALEX, SCALEY, SCIRELIST, SCIWRLIST, TenSize, TenSizeInit, ULEN, XMAX, XMIN, YaSize, YaThick,   YMAX, YMIN, VLIST


\vspace{\baselineskip}\noindent
{\bf  リスト}

リストとは,数や文字などを集めたもので,それぞれのものを「要素」といい,\verb|[ ]|の中にコンマで区切って記述する。要素は型を問わない。\ketcindy\ では,曲線を描くプロットデータが座標のリストであり,リスト処理をうまく使えば \ketcindy\ で効率的に作図ができる。

リストのn番目の要素にアクセスするのに,アンダーバー\_ を使う。

\begin{verbatim}
  list=[1,2,3,4,5];
  println(list_2);
\end{verbatim}
とすると,list の中の2番目の要素 2 が表示される。
\begin{verbatim}
  list=[1,2,3,4,5];
  list_2="a";
\end{verbatim}
とすると,list の中の2番目の要素が文字 a に変わる。

たとえば,曲線の交点を求める \hyperlink{intersectcrvs}{Intersectcrvs()} の戻り値から交点の座標を求めるにはアンダーバーを使う。使用例は,Intersectcrvs() の例を参照されたい。

\subsection{よく使うCindyscriptのコマンド}
\vspace{\baselineskip}\noindent
{\bf 値の表示}

print(値)   :コンソールに値を表示する。改行しない。

println(値) :コンソールに値を表示する。改行する。

\vspace{\baselineskip}
【例】関数Intersectcrvs() の戻り値を表示する。
\begin{verbatim}
    tmp=intersectcrvs("sgAB","crCD");
    println(tmp);
\end{verbatim}

\vspace{\baselineskip}\noindent
{\bf 条件判断}

if(A,B,C)  : もしAが真なら(成り立てば)Bを,偽ならCを実行する。

Aの条件判断には次のものがよく使われる。
 \begin{tabbing}
1234\=56789012345678989012\=3456789012\=34567890123\=\kill
 \>  \verb|a|が\verb|b|より大きい \> \verb|if(a>b|,$\cdots$\\
 \>  \verb|a|が\verb|b|より小さい \> \verb|if(a<b|,$\cdots$\\
 \>  \verb|a|が\verb|b|以上  \> \verb|if(a>=b|,$\cdots$ (\verb|>|と=の順序に注意)\\
 \>  \verb|a|が\verb|b|以下  \> \verb|if(a<=b|,$\cdots$ (\verb|<|と=の順序に注意)\\
 \>  \verb|a|と\verb|b|が等しい \>  \verb|if(a==b|,$\cdots$ (等号を2つ)\\
 \>  \verb|a|と\verb|b|が異なる \> \verb|if(a!=b|,$\cdots$\\
\end{tabbing}

if 文はネストして使うことができる。

\vspace{\baselineskip}
  【例】n が正,負,ゼロのいずれかを判断して,コンソールに表示する。
\begin{verbatim}
    if(n>0,print("正"),if(n==0,print("0"),print("負")));
\end{verbatim}

\vspace{\baselineskip}\noindent
{\bf 繰り返し}

repeat(n,操作)    :操作をn回繰り返す。

repeat(n,s,操作)  :操作をn回繰り返す。カウンタとしてsを使う。(文字は他でも可)

\vspace{\baselineskip}
  【例】Aを4個並べて描画面に表示する。
\begin{verbatim}
    repeat(4,s,drawtext([s,0],4));
\end{verbatim}
  ここで,sの値は4回繰り返すうちに,1,2,3,4と変化する。

\vspace{\baselineskip}\noindent
{\bf リストによる繰り返し}

  forall(list,処理)  :リストの要素すべてに渡るように繰り返す。
  
\vspace{\baselineskip}
  【例】点のペアをリストとし,線分を描く。
\begin{verbatim}
    sglist=[[A,B],[C,D],[E,F]];
    forall(sglist,Listplot(#));
\end{verbatim}
  これは,
\begin{verbatim}
    Listplot([A,B]);
    Listplot([C,D]);
    Listplot([E,F]);
\end{verbatim}
とするのと同じ。ここで,\verb|#|は実行変数と呼ばれ,リストのそれぞれの要素を表す。

\vspace{\baselineskip}\noindent
{\bf ユーザー定義関数}

ユーザー定義関数は次の書式で定義する。

\hspace{10mm}関数名(引数):=(処理)

\vspace{\baselineskip}
【例】  引数の値の正負を表示する関数 sign(n) を定義する。
\begin{verbatim}
  sign(n):=(
    if(n>0,print("正"),print("0または負"));
   );
\end{verbatim}
定義した関数は
\begin{verbatim}
  n=3;
  println(sign(n));
\end{verbatim}
のようにして使う。

KeTCindyでは,アニメーションPDFを作成するときに,フレームを定義するのに使う。

\vspace{\baselineskip}\noindent
{\bf 幾何要素へのアクセス}

Cinderellaでは,点の座標は同次座標で表されており,点の名称でそのまま座標を取得できることが多い。そのため,たとえば Listplot() 関数では,点を指定するのに,座標 \verb|[a,b]| の代わりに点名を使うことができる。

\vspace{\baselineskip}
Listplot() の書式1  \verb|Listplot("1",[[1,1],[4,5]])|

Listplot() の書式2  \verb|Listplot("1",[A,B])|

\vspace{\baselineskip}

しかし,明確に直交座標で取得したい場合は  \verb|A.xy|(x,y 座標)  \verb|A.x|(x 座標 )   \verb|A.y|(y 座標)  として取得する。

\vspace{\baselineskip}\noindent
{\bf リスト処理}

Cindyscriptのリスト処理のうち,よく使うものを挙げる。

aからbまでの整数のリストは \verb|a..b| (ドット2つ)で生成できる。このリストの各要素を番号代わりに使って,\verb|apply(list,expr)| を用いると座標のリストを作ることができる。\verb|apply(list,expr)| は,\verb|list| の各要素に,処理 \verb|expr| を施したリストを作る関数である。

\vspace{\baselineskip}
【例】星形五角形を描く
\begin{verbatim}
    r=2;
    pt=apply(0..5,r*[cos(pi/2+#*4*pi/5),sin(pi/2+#*4*pi/5)]);
    repeat(5,s,Listplot(text(s),[pt_s,pt_(s+1)]));
\end{verbatim}

ここで,\verb|text(s)| は,数を文字列に変換するCindyscriptの組み込み関数。

\vspace{\baselineskip}\noindent
{\bf よくあるエラーメッセージ}
\begin{tabbing}
1234\=5678901234567890123456789\=\kill
 \>Index out of range \>リストの要素の個数外の値を指定した。\\
 \>String Index out of range \>文字列のインデックスが範囲外。\\
 \>Potential type mismatch \>変数の型が合わない。文字と実数をかけ算したときなど。\\
 \>unexpected ) \>括弧の種類が前後で合っていない。\\
 \>close  without open \>閉じ括弧に対応する開き括弧がない。\\
 \>open  without close \>開き括弧に対応する閉じ括弧がない。\\
 \>Unknown function \>関数が定義されていない。
\end{tabbing}


\vspace{\baselineskip}
その他,Cindyscriptについては,スクリプトエディタからヘルプを参照されたい。

\end{document}