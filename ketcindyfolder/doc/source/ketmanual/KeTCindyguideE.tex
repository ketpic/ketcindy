\documentclass[papersize,a4paper,12pt]{article}
\usepackage{ketpic,ketlayer}
\usepackage{amsmath,amssymb}
% \usepackage{amsmath,newtxmath}
%\usepackage[dvipdfmx]{graphicx,color}
\usepackage{graphicx,color}
\usepackage{wrapfig}
%\usepackage[dvipdfmx,bookmarks=false,colorlinks=true,linkcolor=blue]{hyperref}
\usepackage[bookmarks=false,colorlinks=true,linkcolor=blue]{hyperref}
\setmargin{20}{20}{15}{25}
\usepackage{setspace}
\usepackage{comment}
\usepackage{bm,enumerate}
\usepackage{pict2e}

%\newcommand{\cmd}[1]{
%\begin{center}{\bf\large #1}\end{center}
%\hypertarget{#1}{}
%}

\newenvironment{cmd}[2]{
\hypertarget{#2}{}
\begin{center}{\bf\large #1}\end{center}
\begin{description}
}{
\end{description}
\begin{flushright} \hyperlink{functionlist}{$\Rightarrow$Command List}\end{flushright}
}

% item command for this documentation
\newcommand{\itemket}[1]{
\item[\Ltab{27mm}{#1}]
}


\begin{document}
\title{Guide to \ketcindy}
\author{\ketcindy\ Project Team}
\maketitle

\begin{center}  - ver.3.2 -\end{center}

\hypertarget{index}{}
\tableofcontents

\newpage

\section{About \ketcindy}

\subsection{Overview}

\ketcindy\ is a library of Cindyscript 
which is a programming language of Cinderella. 
It converts the data computed 
for generating dynamic graphics on Cinderella 
into \TeX\ graphical codes. 
Synchronized use of 
interactive graphics capabilities of Cinderella 
and well-structured programming capabilities of Cindyscript 
enables ordinary \TeX\ users to efficiently embed 
high-quality graphics into \TeX\ documents. 
Moreover, the collaborative use of \ketcindy\ 
and other software such as R, Maxima and C 
has been enabled.

\begin{center}
%%% fig.tex 2016-4-30 21:58
%%% fig.sce 2016-4-30 21:58
{\unitlength=3.5mm%
\begin{picture}%
(  38.24000,  21.88000)( -24.72000, -11.80000)%
\special{pn 8}%
%
\special{pa 933 -397}\special{pa 929 -446}\special{pa 920 -494}\special{pa 905 -541}%
\special{pa 884 -586}\special{pa 858 -628}\special{pa 826 -665}\special{pa 790 -699}%
\special{pa 750 -728}\special{pa 707 -752}\special{pa 661 -770}\special{pa 614 -782}%
\special{pa 565 -788}\special{pa 516 -788}\special{pa 467 -782}\special{pa 419 -770}%
\special{pa 373 -752}\special{pa 330 -728}\special{pa 290 -699}\special{pa 254 -665}%
\special{pa 223 -628}\special{pa 196 -586}\special{pa 175 -541}\special{pa 160 -494}%
\special{pa 151 -446}\special{pa 148 -397}\special{pa 151 -348}\special{pa 160 -299}%
\special{pa 175 -252}\special{pa 196 -208}\special{pa 223 -166}\special{pa 254 -128}%
\special{pa 290 -94}\special{pa 330 -66}\special{pa 373 -42}\special{pa 419 -24}\special{pa 467 -11}%
\special{pa 516 -5}\special{pa 565 -5}\special{pa 614 -11}\special{pa 661 -24}\special{pa 707 -42}%
\special{pa 750 -66}\special{pa 790 -94}\special{pa 826 -128}\special{pa 858 -166}%
\special{pa 884 -208}\special{pa 905 -252}\special{pa 920 -299}\special{pa 929 -348}%
\special{pa 933 -397}%
\special{fp}%
\special{pa 1624 -628}\special{pa 1622 -665}\special{pa 1615 -700}\special{pa 1604 -735}%
\special{pa 1588 -768}\special{pa 1569 -799}\special{pa 1545 -827}\special{pa 1519 -852}%
\special{pa 1489 -873}\special{pa 1457 -891}\special{pa 1424 -904}\special{pa 1388 -913}%
\special{pa 1352 -918}\special{pa 1316 -918}\special{pa 1280 -913}\special{pa 1244 -904}%
\special{pa 1210 -891}\special{pa 1178 -873}\special{pa 1149 -852}\special{pa 1122 -827}%
\special{pa 1099 -799}\special{pa 1080 -768}\special{pa 1064 -735}\special{pa 1053 -700}%
\special{pa 1046 -665}\special{pa 1044 -628}\special{pa 1046 -592}\special{pa 1053 -556}%
\special{pa 1064 -522}\special{pa 1080 -489}\special{pa 1099 -458}\special{pa 1122 -430}%
\special{pa 1149 -405}\special{pa 1178 -383}\special{pa 1210 -366}\special{pa 1244 -352}%
\special{pa 1280 -343}\special{pa 1316 -339}\special{pa 1352 -339}\special{pa 1388 -343}%
\special{pa 1424 -352}\special{pa 1457 -366}\special{pa 1489 -383}\special{pa 1519 -405}%
\special{pa 1545 -430}\special{pa 1569 -458}\special{pa 1588 -489}\special{pa 1604 -522}%
\special{pa 1615 -556}\special{pa 1622 -592}\special{pa 1624 -628}%
\special{fp}%
\special{pa 1580 176}\special{pa 1577 128}\special{pa 1568 79}\special{pa 1553 33}%
\special{pa 1532 -11}\special{pa 1506 -53}\special{pa 1475 -90}\special{pa 1439 -124}%
\special{pa 1399 -153}\special{pa 1357 -176}\special{pa 1311 -194}\special{pa 1264 -207}%
\special{pa 1215 -213}\special{pa 1166 -213}\special{pa 1118 -207}\special{pa 1070 -194}%
\special{pa 1025 -176}\special{pa 982 -153}\special{pa 942 -124}\special{pa 906 -90}%
\special{pa 875 -53}\special{pa 849 -11}\special{pa 828 33}\special{pa 813 79}\special{pa 804 128}%
\special{pa 801 176}\special{pa 804 225}\special{pa 813 273}\special{pa 828 320}\special{pa 849 364}%
\special{pa 875 406}\special{pa 906 443}\special{pa 942 477}\special{pa 982 506}\special{pa 1025 529}%
\special{pa 1070 547}\special{pa 1118 559}\special{pa 1166 565}\special{pa 1215 565}%
\special{pa 1264 559}\special{pa 1311 547}\special{pa 1357 529}\special{pa 1399 506}%
\special{pa 1439 477}\special{pa 1475 443}\special{pa 1506 406}\special{pa 1532 364}%
\special{pa 1553 320}\special{pa 1568 273}\special{pa 1577 225}\special{pa 1580 176}%
\special{fp}%
\special{pn 32}%
\special{pa 784 -90}\special{pa 865 -39}%
\special{fp}%
\special{pn 8}%
\special{pn 32}%
\special{pa 931 -435}\special{pa 1058 -540}%
\special{fp}%
\special{pn 8}%
\special{pn 24}%
\special{pa 1225 -360}\special{pa 1204 -227}%
\special{fp}%
\special{pn 8}%
\special{pa 1189 -291}\special{pa 1202 -213}\special{pa 1237 -284}\special{pa 1213 -287}%
\special{pa 1189 -291}\special{sh 1}\special{ip}%
\special{pn 1}%
\special{pa 1189 -291}\special{pa 1202 -213}\special{pa 1237 -284}\special{pa 1213 -287}%
\special{pa 1189 -291}\special{pa 1202 -213}%
\special{fp}%
\special{pn 8}%
\special{pn 24}%
\special{pa 1271 -205}\special{pa 1305 -326}%
\special{fp}%
\special{pn 8}%
\special{pa 1312 -261}\special{pa 1309 -339}\special{pa 1265 -274}\special{pa 1288 -267}%
\special{pa 1312 -261}\special{sh 1}\special{ip}%
\special{pn 1}%
\special{pa 1312 -261}\special{pa 1309 -339}\special{pa 1265 -274}\special{pa 1288 -267}%
\special{pa 1312 -261}\special{pa 1309 -339}%
\special{fp}%
\special{pn 8}%
\special{pa 89 155}\special{pa 85 93}\special{pa 72 32}\special{pa 50 -27}\special{pa 19 -83}%
\special{pa -21 -136}\special{pa -68 -184}\special{pa -123 -228}\special{pa -183 -265}%
\special{pa -250 -297}\special{pa -320 -321}\special{pa -393 -338}\special{pa -469 -347}%
\special{pa -545 -349}\special{pa -621 -343}\special{pa -695 -330}\special{pa -766 -308}%
\special{pa -834 -280}\special{pa -896 -246}\special{pa -953 -205}\special{pa -1002 -159}%
\special{pa -1044 -108}\special{pa -1078 -53}\special{pa -1103 5}\special{pa -1119 65}%
\special{pa -1125 127}\special{pa -1122 188}\special{pa -1109 249}\special{pa -1086 308}%
\special{pa -1055 364}\special{pa -1016 417}\special{pa -968 466}\special{pa -914 509}%
\special{pa -853 547}\special{pa -787 578}\special{pa -716 602}\special{pa -643 619}%
\special{pa -567 628}\special{pa -491 630}\special{pa -416 624}\special{pa -341 611}%
\special{pa -270 590}\special{pa -203 562}\special{pa -140 527}\special{pa -84 486}%
\special{pa -34 440}\special{pa 8 389}\special{pa 42 334}\special{pa 67 276}\special{pa 82 216}%
\special{pa 89 155}%
\special{fp}%
\special{pa -2437 142}\special{pa -2441 96}\special{pa -2453 50}\special{pa -2471 7}%
\special{pa -2496 -35}\special{pa -2527 -73}\special{pa -2565 -108}\special{pa -2607 -139}%
\special{pa -2655 -166}\special{pa -2706 -188}\special{pa -2760 -204}\special{pa -2816 -215}%
\special{pa -2874 -220}\special{pa -2932 -220}\special{pa -2989 -213}\special{pa -3045 -201}%
\special{pa -3099 -184}\special{pa -3150 -161}\special{pa -3197 -134}\special{pa -3239 -102}%
\special{pa -3275 -66}\special{pa -3306 -27}\special{pa -3330 15}\special{pa -3348 59}%
\special{pa -3358 105}\special{pa -3361 151}\special{pa -3357 197}\special{pa -3346 242}%
\special{pa -3328 285}\special{pa -3302 327}\special{pa -3271 365}\special{pa -3234 400}%
\special{pa -3191 432}\special{pa -3144 458}\special{pa -3093 480}\special{pa -3039 496}%
\special{pa -2982 507}\special{pa -2925 512}\special{pa -2867 512}\special{pa -2809 506}%
\special{pa -2753 494}\special{pa -2699 476}\special{pa -2649 453}\special{pa -2602 426}%
\special{pa -2560 394}\special{pa -2523 358}\special{pa -2493 319}\special{pa -2468 277}%
\special{pa -2451 233}\special{pa -2440 188}\special{pa -2437 142}%
\special{fp}%
\special{pa -1461 901}\special{pa -1463 870}\special{pa -1472 839}\special{pa -1486 808}%
\special{pa -1505 779}\special{pa -1530 752}\special{pa -1560 727}\special{pa -1594 705}%
\special{pa -1633 685}\special{pa -1674 669}\special{pa -1719 656}\special{pa -1765 647}%
\special{pa -1813 641}\special{pa -1861 640}\special{pa -1908 643}\special{pa -1955 649}%
\special{pa -2000 659}\special{pa -2043 673}\special{pa -2082 691}\special{pa -2118 711}%
\special{pa -2149 734}\special{pa -2175 760}\special{pa -2197 788}\special{pa -2212 817}%
\special{pa -2222 848}\special{pa -2226 879}\special{pa -2224 911}\special{pa -2216 942}%
\special{pa -2202 972}\special{pa -2182 1001}\special{pa -2157 1028}\special{pa -2127 1053}%
\special{pa -2093 1076}\special{pa -2055 1095}\special{pa -2013 1112}\special{pa -1969 1124}%
\special{pa -1922 1134}\special{pa -1875 1139}\special{pa -1827 1140}\special{pa -1779 1138}%
\special{pa -1732 1131}\special{pa -1687 1121}\special{pa -1645 1107}\special{pa -1605 1090}%
\special{pa -1570 1069}\special{pa -1539 1046}\special{pa -1512 1020}\special{pa -1491 992}%
\special{pa -1475 963}\special{pa -1465 932}\special{pa -1461 901}%
\special{fp}%
\special{pa 201 1172}\special{pa 200 1142}\special{pa 191 1113}\special{pa 177 1084}%
\special{pa 156 1056}\special{pa 130 1030}\special{pa 98 1005}\special{pa 61 983}%
\special{pa 21 963}\special{pa -24 947}\special{pa -71 934}\special{pa -121 924}\special{pa -172 918}%
\special{pa -224 915}\special{pa -275 916}\special{pa -325 921}\special{pa -374 930}%
\special{pa -420 942}\special{pa -462 957}\special{pa -501 976}\special{pa -535 997}%
\special{pa -564 1020}\special{pa -587 1046}\special{pa -605 1074}\special{pa -616 1102}%
\special{pa -620 1132}\special{pa -619 1162}\special{pa -610 1191}\special{pa -596 1220}%
\special{pa -575 1248}\special{pa -549 1274}\special{pa -517 1299}\special{pa -480 1321}%
\special{pa -439 1341}\special{pa -395 1357}\special{pa -347 1370}\special{pa -298 1380}%
\special{pa -247 1386}\special{pa -195 1389}\special{pa -144 1388}\special{pa -94 1383}%
\special{pa -45 1374}\special{pa 1 1362}\special{pa 43 1347}\special{pa 82 1328}\special{pa 116 1307}%
\special{pa 145 1283}\special{pa 168 1258}\special{pa 186 1230}\special{pa 197 1202}%
\special{pa 201 1172}%
\special{fp}%
\special{pa 1001 926}\special{pa 999 888}\special{pa 992 849}\special{pa 978 812}%
\special{pa 959 776}\special{pa 935 742}\special{pa 905 711}\special{pa 872 684}\special{pa 834 659}%
\special{pa 792 639}\special{pa 749 623}\special{pa 703 612}\special{pa 655 605}\special{pa 608 604}%
\special{pa 560 607}\special{pa 513 615}\special{pa 468 627}\special{pa 426 644}\special{pa 386 666}%
\special{pa 350 691}\special{pa 319 720}\special{pa 292 751}\special{pa 270 786}\special{pa 254 822}%
\special{pa 244 860}\special{pa 239 898}\special{pa 241 937}\special{pa 249 975}\special{pa 262 1013}%
\special{pa 281 1048}\special{pa 305 1082}\special{pa 335 1113}\special{pa 369 1141}%
\special{pa 406 1165}\special{pa 448 1185}\special{pa 492 1201}\special{pa 538 1212}%
\special{pa 585 1219}\special{pa 633 1221}\special{pa 680 1218}\special{pa 727 1210}%
\special{pa 772 1197}\special{pa 814 1180}\special{pa 854 1159}\special{pa 890 1134}%
\special{pa 921 1105}\special{pa 948 1073}\special{pa 970 1039}\special{pa 986 1003}%
\special{pa 996 965}\special{pa 1001 926}%
\special{fp}%
\special{pa -768 1296}\special{pa -770 1262}\special{pa -780 1228}\special{pa -795 1195}%
\special{pa -816 1163}\special{pa -843 1133}\special{pa -875 1107}\special{pa -912 1082}%
\special{pa -953 1062}\special{pa -997 1045}\special{pa -1044 1031}\special{pa -1093 1022}%
\special{pa -1143 1018}\special{pa -1194 1017}\special{pa -1244 1021}\special{pa -1294 1029}%
\special{pa -1341 1041}\special{pa -1385 1058}\special{pa -1427 1078}\special{pa -1464 1101}%
\special{pa -1496 1127}\special{pa -1524 1156}\special{pa -1546 1188}\special{pa -1562 1220}%
\special{pa -1571 1254}\special{pa -1575 1289}\special{pa -1572 1323}\special{pa -1563 1357}%
\special{pa -1548 1391}\special{pa -1526 1422}\special{pa -1499 1452}\special{pa -1467 1479}%
\special{pa -1431 1503}\special{pa -1390 1523}\special{pa -1345 1540}\special{pa -1298 1554}%
\special{pa -1249 1563}\special{pa -1199 1567}\special{pa -1148 1568}\special{pa -1098 1564}%
\special{pa -1049 1556}\special{pa -1002 1544}\special{pa -957 1527}\special{pa -916 1507}%
\special{pa -879 1484}\special{pa -846 1458}\special{pa -819 1429}\special{pa -797 1398}%
\special{pa -781 1365}\special{pa -771 1331}\special{pa -768 1296}%
\special{fp}%
\special{pa -925 -621}\special{pa -930 -652}\special{pa -942 -682}\special{pa -960 -711}%
\special{pa -984 -738}\special{pa -1015 -763}\special{pa -1051 -785}\special{pa -1091 -805}%
\special{pa -1136 -821}\special{pa -1185 -834}\special{pa -1236 -843}\special{pa -1289 -848}%
\special{pa -1343 -849}\special{pa -1397 -847}\special{pa -1451 -840}\special{pa -1504 -830}%
\special{pa -1554 -816}\special{pa -1601 -799}\special{pa -1644 -779}\special{pa -1682 -756}%
\special{pa -1716 -730}\special{pa -1744 -702}\special{pa -1766 -673}\special{pa -1781 -643}%
\special{pa -1790 -612}\special{pa -1792 -581}\special{pa -1787 -550}\special{pa -1776 -520}%
\special{pa -1757 -491}\special{pa -1733 -464}\special{pa -1702 -439}\special{pa -1667 -416}%
\special{pa -1626 -397}\special{pa -1581 -381}\special{pa -1533 -368}\special{pa -1482 -359}%
\special{pa -1429 -354}\special{pa -1374 -352}\special{pa -1320 -355}\special{pa -1266 -361}%
\special{pa -1214 -371}\special{pa -1164 -385}\special{pa -1117 -402}\special{pa -1074 -423}%
\special{pa -1035 -446}\special{pa -1001 -471}\special{pa -973 -499}\special{pa -952 -528}%
\special{pa -936 -559}\special{pa -927 -590}\special{pa -925 -621}%
\special{fp}%
\special{pa -1457 142}\special{pa -1460 103}\special{pa -1470 65}\special{pa -1486 28}%
\special{pa -1508 -7}\special{pa -1535 -40}\special{pa -1568 -69}\special{pa -1605 -96}%
\special{pa -1646 -118}\special{pa -1690 -137}\special{pa -1737 -150}\special{pa -1786 -160}%
\special{pa -1836 -164}\special{pa -1886 -163}\special{pa -1936 -158}\special{pa -1985 -148}%
\special{pa -2031 -133}\special{pa -2075 -114}\special{pa -2115 -90}\special{pa -2152 -63}%
\special{pa -2184 -33}\special{pa -2210 0}\special{pa -2231 36}\special{pa -2246 73}%
\special{pa -2255 112}\special{pa -2258 150}\special{pa -2255 189}\special{pa -2245 227}%
\special{pa -2229 264}\special{pa -2207 299}\special{pa -2180 332}\special{pa -2147 362}%
\special{pa -2110 388}\special{pa -2069 410}\special{pa -2025 429}\special{pa -1978 442}%
\special{pa -1929 452}\special{pa -1879 456}\special{pa -1829 455}\special{pa -1779 450}%
\special{pa -1730 440}\special{pa -1684 425}\special{pa -1640 406}\special{pa -1599 382}%
\special{pa -1563 355}\special{pa -1531 325}\special{pa -1505 292}\special{pa -1484 256}%
\special{pa -1469 219}\special{pa -1460 181}\special{pa -1457 142}%
\special{fp}%
\special{pn 16}%
\special{pa 11 -94}\special{pa 183 -234}%
\special{fp}%
\special{pn 8}%
\special{pn 16}%
\special{pa 89 141}\special{pa 804 127}%
\special{fp}%
\special{pn 8}%
\special{pn 16}%
\special{pa -1291 143}\special{pa -1139 141}%
\special{fp}%
\special{pn 8}%
\special{pa -1199 166}\special{pa -1125 141}\special{pa -1200 117}\special{pa -1200 142}%
\special{pa -1199 166}\special{sh 1}\special{ip}%
\special{pn 1}%
\special{pa -1199 166}\special{pa -1125 141}\special{pa -1200 117}\special{pa -1200 142}%
\special{pa -1199 166}\special{pa -1125 141}%
\special{fp}%
\special{pn 8}%
\special{pn 16}%
\special{pa -1291 143}\special{pa -1443 146}%
\special{fp}%
\special{pn 8}%
\special{pa -1382 120}\special{pa -1457 146}\special{pa -1382 169}\special{pa -1382 145}%
\special{pa -1382 120}\special{sh 1}\special{ip}%
\special{pn 1}%
\special{pa -1382 120}\special{pa -1457 146}\special{pa -1382 169}\special{pa -1382 145}%
\special{pa -1382 120}\special{pa -1457 146}%
\special{fp}%
\special{pn 8}%
\special{pn 16}%
\special{pa -1282 585}\special{pa -983 469}%
\special{fp}%
\special{pn 8}%
\special{pa -1031 513}\special{pa -970 464}\special{pa -1049 468}\special{pa -1040 491}%
\special{pa -1031 513}\special{sh 1}\special{ip}%
\special{pn 1}%
\special{pa -1031 513}\special{pa -970 464}\special{pa -1049 468}\special{pa -1040 491}%
\special{pa -1031 513}\special{pa -970 464}%
\special{fp}%
\special{pn 8}%
\special{pn 16}%
\special{pa -1282 585}\special{pa -1580 701}%
\special{fp}%
\special{pn 8}%
\special{pa -1532 656}\special{pa -1593 706}\special{pa -1514 701}\special{pa -1523 678}%
\special{pa -1532 656}\special{sh 1}\special{ip}%
\special{pn 1}%
\special{pa -1532 656}\special{pa -1593 706}\special{pa -1514 701}\special{pa -1523 678}%
\special{pa -1532 656}\special{pa -1593 706}%
\special{fp}%
\special{pn 8}%
\special{pn 16}%
\special{pa -266 757}\special{pa -298 613}%
\special{fp}%
\special{pn 8}%
\special{pa -261 668}\special{pa -301 600}\special{pa -308 678}\special{pa -285 673}%
\special{pa -261 668}\special{sh 1}\special{ip}%
\special{pn 1}%
\special{pa -261 668}\special{pa -301 600}\special{pa -308 678}\special{pa -285 673}%
\special{pa -261 668}\special{pa -301 600}%
\special{fp}%
\special{pn 8}%
\special{pn 16}%
\special{pa -266 757}\special{pa -234 902}%
\special{fp}%
\special{pn 8}%
\special{pa -271 847}\special{pa -231 915}\special{pa -224 837}\special{pa -248 842}%
\special{pa -271 847}\special{sh 1}\special{ip}%
\special{pn 1}%
\special{pa -271 847}\special{pa -231 915}\special{pa -224 837}\special{pa -248 842}%
\special{pa -271 847}\special{pa -231 915}%
\special{fp}%
\special{pn 8}%
\special{pn 16}%
\special{pa 173 549}\special{pa 342 681}%
\special{fp}%
\special{pn 8}%
\special{pa 279 662}\special{pa 353 689}\special{pa 309 624}\special{pa 294 643}\special{pa 279 662}%
\special{sh 1}\special{ip}%
\special{pn 1}%
\special{pa 279 662}\special{pa 353 689}\special{pa 309 624}\special{pa 294 643}\special{pa 279 662}%
\special{pa 353 689}%
\special{fp}%
\special{pn 8}%
\special{pn 16}%
\special{pa 173 549}\special{pa 4 417}%
\special{fp}%
\special{pn 8}%
\special{pa 67 436}\special{pa -7 409}\special{pa 37 474}\special{pa 52 455}\special{pa 67 436}%
\special{sh 1}\special{ip}%
\special{pn 1}%
\special{pa 67 436}\special{pa -7 409}\special{pa 37 474}\special{pa 52 455}\special{pa 67 436}%
\special{pa -7 409}%
\special{fp}%
\special{pn 8}%
\special{pn 24}%
\special{pa -2258 143}\special{pa -2423 143}%
\special{fp}%
\special{pn 8}%
\special{pa -2362 119}\special{pa -2437 143}\special{pa -2362 168}\special{pa -2362 143}%
\special{pa -2362 119}\special{sh 1}\special{ip}%
\special{pn 1}%
\special{pa -2362 119}\special{pa -2437 143}\special{pa -2362 168}\special{pa -2362 143}%
\special{pa -2362 119}\special{pa -2437 143}%
\special{fp}%
\special{pn 8}%
\special{pn 16}%
\special{pa -909 -237}\special{pa -1052 -419}%
\special{fp}%
\special{pn 8}%
\special{pa -995 -386}\special{pa -1061 -430}\special{pa -1034 -356}\special{pa -1015 -371}%
\special{pa -995 -386}\special{sh 1}\special{ip}%
\special{pn 1}%
\special{pa -995 -386}\special{pa -1061 -430}\special{pa -1034 -356}\special{pa -1015 -371}%
\special{pa -995 -386}\special{pa -1061 -430}%
\special{fp}%
\special{pn 8}%
\special{pn 16}%
\special{pa -907 808}\special{pa -775 597}%
\special{fp}%
\special{pn 8}%
\special{pa -787 662}\special{pa -767 585}\special{pa -828 636}\special{pa -807 649}%
\special{pa -787 662}\special{sh 1}\special{ip}%
\special{pn 1}%
\special{pa -787 662}\special{pa -767 585}\special{pa -828 636}\special{pa -807 649}%
\special{pa -787 662}\special{pa -767 585}%
\special{fp}%
\special{pn 8}%
\special{pn 16}%
\special{pa -907 808}\special{pa -1040 1019}%
\special{fp}%
\special{pn 8}%
\special{pa -1028 954}\special{pa -1047 1031}\special{pa -987 980}\special{pa -1007 967}%
\special{pa -1028 954}\special{sh 1}\special{ip}%
\special{pn 1}%
\special{pa -1028 954}\special{pa -1047 1031}\special{pa -987 980}\special{pa -1007 967}%
\special{pa -1028 954}\special{pa -1047 1031}%
\special{fp}%
\special{pn 8}%
\Large%
\settowidth{\Width}{KeTCindy概念図}\setlength{\Width}{-0.5\Width}%
\settoheight{\Height}{KeTCindy概念図}\settodepth{\Depth}{KeTCindy概念図}\setlength{\Height}{-0.5\Height}\setlength{\Depth}{0.5\Depth}\addtolength{\Height}{\Depth}%
\put(-19.0000,6.0000){\hspace*{\Width}\raisebox{\Height}{KeTCindy概念図}}%
%
%
\normalsize%
\settowidth{\Width}{CindyScript}\setlength{\Width}{-0.5\Width}%
\settoheight{\Height}{CindyScript}\settodepth{\Depth}{CindyScript}\setlength{\Height}{-0.5\Height}\setlength{\Depth}{0.5\Depth}\addtolength{\Height}{\Depth}%
\put(3.9200,2.8800){\hspace*{\Width}\raisebox{\Height}{CindyScript}}%
%
%
\settowidth{\Width}{CindyLab}\setlength{\Width}{-0.5\Width}%
\settoheight{\Height}{CindyLab}\settodepth{\Depth}{CindyLab}\setlength{\Height}{-0.5\Height}\setlength{\Depth}{0.5\Depth}\addtolength{\Height}{\Depth}%
\put(9.6800,4.5600){\hspace*{\Width}\raisebox{\Height}{CindyLab}}%
%
%
\settowidth{\Width}{作図}\setlength{\Width}{-0.5\Width}%
\settoheight{\Height}{作図}\settodepth{\Depth}{作図}\setlength{\Height}{-0.5\Height}\setlength{\Depth}{0.5\Depth}\addtolength{\Height}{\Depth}%
\put(8.6400,-1.2800){\hspace*{\Width}\raisebox{\Height}{作図}}%
%
%
\settowidth{\Width}{KeTpic}\setlength{\Width}{0\Width}%
\settoheight{\Height}{KeTpic}\settodepth{\Depth}{KeTpic}\setlength{\Height}{-\Height}%
\put(-18.6300,0.7900){\hspace*{\Width}\raisebox{\Height}{KeTpic}}%
%
%
\settowidth{\Width}{R}\setlength{\Width}{-0.5\Width}%
\settoheight{\Height}{R}\settodepth{\Depth}{R}\setlength{\Height}{-0.5\Height}\setlength{\Depth}{0.5\Depth}\addtolength{\Height}{\Depth}%
\put(-13.3800,-6.4600){\hspace*{\Width}\raisebox{\Height}{R}}%
%
%
\settowidth{\Width}{Maxima}\setlength{\Width}{-0.5\Width}%
\settoheight{\Height}{Maxima}\settodepth{\Depth}{Maxima}\setlength{\Height}{-0.5\Height}\setlength{\Depth}{0.5\Depth}\addtolength{\Height}{\Depth}%
\put(-1.5200,-8.3600){\hspace*{\Width}\raisebox{\Height}{Maxima}}%
%
%
\settowidth{\Width}{Risa/Asir}\setlength{\Width}{-0.5\Width}%
\settoheight{\Height}{Risa/Asir}\settodepth{\Depth}{Risa/Asir}\setlength{\Height}{-0.5\Height}\setlength{\Depth}{0.5\Depth}\addtolength{\Height}{\Depth}%
\put(4.5000,-6.6200){\hspace*{\Width}\raisebox{\Height}{Risa/Asir}}%
%
%
\settowidth{\Width}{FriCAS}\setlength{\Width}{0\Width}%
\settoheight{\Height}{FriCAS}\settodepth{\Depth}{FriCAS}\setlength{\Height}{-0.5\Height}\setlength{\Depth}{0.5\Depth}\addtolength{\Height}{\Depth}%
\put(-10.0900,-9.3600){\hspace*{\Width}\raisebox{\Height}{FriCAS}}%
%
%
\settowidth{\Width}{MeshLab}\setlength{\Width}{0\Width}%
\settoheight{\Height}{MeshLab}\settodepth{\Depth}{MeshLab}\setlength{\Height}{-0.5\Height}\setlength{\Depth}{0.5\Depth}\addtolength{\Height}{\Depth}%
\put(-11.6900,4.2400){\hspace*{\Width}\raisebox{\Height}{MeshLab}}%
%
%
\Large%
\settowidth{\Width}{KeTCindy}\setlength{\Width}{-0.5\Width}%
\settoheight{\Height}{KeTCindy}\settodepth{\Depth}{KeTCindy}\setlength{\Height}{-0.5\Height}\setlength{\Depth}{0.5\Depth}\addtolength{\Height}{\Depth}%
\put(-3.7600,-1.0200){\hspace*{\Width}\raisebox{\Height}{KeTCindy}}%
%
%
\settowidth{\Width}{TeX}\setlength{\Width}{-0.5\Width}%
\settoheight{\Height}{TeX}\settodepth{\Depth}{TeX}\setlength{\Height}{-0.5\Height}\setlength{\Depth}{0.5\Depth}\addtolength{\Height}{\Depth}%
\put(-21.0400,-1.0600){\hspace*{\Width}\raisebox{\Height}{TeX}}%
%
%
\settowidth{\Width}{Cinderella}\setlength{\Width}{0\Width}%
\settoheight{\Height}{Cinderella}\settodepth{\Depth}{Cinderella}\setlength{\Height}{-0.5\Height}\setlength{\Depth}{0.5\Depth}\addtolength{\Height}{\Depth}%
\put(2.0500,7.0000){\hspace*{\Width}\raisebox{\Height}{Cinderella}}%
%
%
\settowidth{\Width}{Scilab}\setlength{\Width}{0\Width}%
\settoheight{\Height}{Scilab}\settodepth{\Depth}{Scilab}\setlength{\Height}{-0.5\Height}\setlength{\Depth}{0.5\Depth}\addtolength{\Height}{\Depth}%
\put(-15.2700,-1.0800){\hspace*{\Width}\raisebox{\Height}{Scilab}}%
%
%
\end{picture}}%
\end{center}

Firstly, dynamic figure is generated on Cinderella. 
Secondly, \ketcindy\ generates 
a source file of R and makes R execute it 
for the generation of \TeX\ graphical codes. 
Thirdly, those codes are formatted into 
\TeX\ file which is input in the targetting \TeX\ document 
via the command \verb|\input|. 
Finally, usual compilation procedure of \TeX\ results in 
the generation of final PDF output 
including the corresponding figure. 
A batch file \verb|kc.bat| for Windows 
or a shell file \verb|kc.sh| for Mac or Linux 
is generated via \ketcindy\ 
in order to batch-process all the steps 
from the second to the last. 
Also by using these files, 
collaboration of Cinderella and other software 
as shown in the schematic diagram above 
is processed. 

Summarizingly, specific steps to generate a \TeX\ figure 
are listed as follows.

\begin{enumerate}[(1)]
\item 
Generate the needed geometric elements 
on the Euclidean view of Cinderella 
using its drawing tools. 
These elements can be moved interactively. 

\hspace{30mm}\includegraphics[bb=0.00 0.00 408.02 347.02,width=6cm]{Fig/incenter01.pdf}

\item 
Input the \ketcindy\ codes into Cindyscript editor 
to specify the graphical elements to be displayed 
in \TeX\ final output. 
Also \ketcindy\ codes are used 
to generate supplementary graphical elements 
and handle them. 

\hspace{10mm}\includegraphics[bb=0.00 0.00 811.04 257.01,width=12cm]{Fig/incenter02E.pdf}

In this stage, the programming capabilities 
inherently implemented to Cindyscript can be used simultaneously. 
Execute the whole program by clicking the "Run" button. 
For more details, see section 3. 

\item 
Click the button named \verb|Figures| in Euclidean view 
to automatically generate the following files 
in the folder named "fig". 
Here, "incenter" is the name specified 
via the command \verb|Setfiles("incenter")| 
in step (2). 

\begin{tabbing}
12\=1234567897890123456\=\kill

 \> \verb|kc|.sh or \verb|kc.bat| \> shell script file(Mac) or batch file(Windows) \\
 \> \verb|incenter.r| \> \\
 \> \verb|incenter.tex| \> \TeX\ file composed of graphical codes\\
 \> \verb|incentermain.aux| \> \\
 \> \verb|incentermain.log| \> \\
 \> \verb|incentermain.pdf| \> PDF file to display the resulting graphical image\\
 \> \verb|incentermain.tex| \> \TeX\ file temporarily used to generate 
the file \verb|incentermain.pdf|
\end{tabbing}

Subsequently, the file \verb|incentermain.pdf| 
is automatically displayed as shown below. 

\hspace{30mm}\includegraphics[bb=0.00 0.00 348.02 284.51,width=6cm]{Fig/incenter03.pdf}

We can manipulate this final output 
by modifying the inputs in steps (1) and (2) 
before processing the step (3) again. 

\vspace{\baselineskip}
\item  
Using \ketpic\ package of \TeX , 
\verb|incenter.tex| can be read 
into the targetting \TeX\ document 
via the command 
\begin{center}
\verb|\input{incenter}|
\end{center} 
Then the same figure is embedded in the targetting PDF output. 

\end{enumerate}


\newpage

\subsection{Geometric Figures}

Producing geometric figures in the plane is easy. Moreover, we can add hatchings  in some areas, which is better than shading 
for monochrome printing.
The following are the main parts of the script.

\begin{verbatim}
    Listplot([A,B,C,A]);
    Circledata([D,E]);
    Bowdata([B,A],[1,0.5,"Expr=c","da"]); 
    Bowdata([C,B],[1,0.5,"Expr=a","da"]);
    Bowdata([A,C],[1,0.5,"Expr=b","da"]); 
    Hatchdata("2",["oi"],[["crDE"],["sgABCA"]],["dr,0.7",""]); 
    Pointdata("I",D,["size=4"]);
    Letter([A,"sw","A",B,"ne","B",C,"se","C",D,"se","I"]);
\end{verbatim}

\begin{center}
\includegraphics[bb=0.00 0.00 416.00 347.00,height=42mm]{Fig/hatch.pdf}
\hspace{2mm}
%%% /Users/takatoosetsuo/Dropbox/2016ketpic/0801ACA/ACAedutakato/fig/s106bowhatch.tex 2016-11-22 14:11
%%% s106bowhatch.sce 2016-11-22 14:11
{\unitlength=5mm%
\begin{picture}%
(   9.00000,   7.50000)(  -4.50000,  -4.00000)%
\linethickness{0.008in}%
%
\polyline(-4.00000,-3.00000)(1.00000,3.00000)(4.00000,-3.00000)(-4.00000,-3.00000)%
%
\linethickness{0.008in}%
\polyline(2.68261,-0.86841)(2.66580,-0.60125)(2.61564,-0.33831)(2.53292,-0.08372)%
(2.41895,0.15849)(2.27551,0.38451)(2.10488,0.59076)(1.90975,0.77401)(1.69318,0.93135)%
(1.45861,1.06031)(1.20972,1.15885)(0.95044,1.22542)(0.68486,1.25897)(0.41718,1.25897)%
(0.15160,1.22542)(-0.10768,1.15885)(-0.35657,1.06031)(-0.59114,0.93135)(-0.80771,0.77401)%
(-1.00284,0.59076)(-1.17347,0.38451)(-1.31691,0.15849)(-1.43088,-0.08372)(-1.51360,-0.33831)%
(-1.56376,-0.60125)(-1.58057,-0.86841)(-1.56376,-1.13557)(-1.51360,-1.39851)(-1.43088,-1.65310)%
(-1.31691,-1.89531)(-1.17347,-2.12133)(-1.00284,-2.32758)(-0.80771,-2.51083)(-0.59114,-2.66817)%
(-0.35657,-2.79713)(-0.10768,-2.89567)(0.15160,-2.96224)(0.41718,-2.99579)(0.68486,-2.99579)%
(0.95044,-2.96224)(1.20972,-2.89567)(1.45861,-2.79713)(1.69318,-2.66817)(1.90975,-2.51083)%
(2.10488,-2.32758)(2.27551,-2.12133)(2.41895,-1.89531)(2.53292,-1.65310)(2.61564,-1.39851)%
(2.66580,-1.13557)(2.68261,-0.86841)%
%
\linethickness{0.008in}%
\settowidth{\Width}{$c$}\setlength{\Width}{-0.5\Width}%
\settoheight{\Height}{$c$}\settodepth{\Depth}{$c$}\setlength{\Height}{-0.5\Height}\setlength{\Depth}{0.5\Depth}\addtolength{\Height}{\Depth}%
\put(-2.1000,0.5000){\hspace*{\Width}\raisebox{\Height}{$c$}}%
%
%
\polyline(1.00000,3.00000)(0.93350,2.96496)(0.86726,2.92942)(0.82559,2.90667)\polyline(0.65300,2.81003)(0.60499,2.78242)(0.54011,2.74445)(0.48231,2.71006)%
\polyline(0.31364,2.60672)(0.28345,2.58784)(0.22002,2.54751)(0.15689,2.50670)(0.14697,2.50018)%
\polyline(-0.01756,2.39038)(-0.03066,2.38150)(-0.09255,2.33884)(-0.15413,2.29573)(-0.17992,2.27739)%
\polyline(-0.34009,2.16131)(-0.39719,2.11876)(-0.45713,2.07340)(-0.49791,2.04206)%
\polyline(-0.65340,1.91979)(-0.69348,1.88755)(-0.75170,1.84000)(-0.80652,1.79455)%
\polyline(-0.95709,1.66627)(-0.98100,1.64553)(-1.03742,1.59586)(-1.09347,1.54578)(-1.10519,1.53514)%
\polyline(-1.25070,1.40115)(-1.25938,1.39304)(-1.31392,1.34132)(-1.36808,1.28919)(-1.39355,1.26431)%
\polyline(-1.53375,1.12477)(-1.58080,1.07672)(-1.63300,1.02262)(-1.67116,0.98248)%
\polyline(-1.80578,0.83754)(-1.83771,0.80242)(-1.88787,0.74643)(-1.93761,0.69007)%
\polyline(-2.25767,0.30599)(-2.30413,0.24691)(-2.35016,0.18748)(-2.37895,0.14973)%
\polyline(-2.49723,-0.00882)(-2.52984,-0.05359)(-2.57364,-0.11468)(-2.61242,-0.16963)%
\polyline(-2.72439,-0.33270)(-2.74427,-0.36224)(-2.78577,-0.42491)(-2.82682,-0.48788)(-2.83322,-0.49788)%
\polyline(-2.93878,-0.66516)(-2.94713,-0.67861)(-2.98629,-0.74277)(-3.02497,-0.80722)(-3.04105,-0.83448)%
\polyline(-3.14007,-1.00572)(-3.17490,-1.06784)(-3.21117,-1.13368)(-3.23565,-1.17891)%
\polyline(-3.32788,-1.35390)(-3.35135,-1.39966)(-3.38516,-1.46680)(-3.41672,-1.53064)%
\polyline(-3.50201,-1.70912)(-3.51540,-1.73779)(-3.54670,-1.80613)(-3.57750,-1.87470)(-3.58387,-1.88919)%
\polyline(-3.66220,-2.07083)(-3.66682,-2.08176)(-3.69557,-2.15121)(-3.72381,-2.22088)(-3.73694,-2.25397)%
\polyline(-3.80817,-2.43851)(-3.83158,-2.50156)(-3.85721,-2.57222)(-3.87572,-2.62443)%
\polyline(-3.93965,-2.81162)(-3.95452,-2.85671)(-3.97752,-2.92827)(-4.00000,-3.00000)%
%
%
\settowidth{\Width}{$a$}\setlength{\Width}{-0.5\Width}%
\settoheight{\Height}{$a$}\settodepth{\Depth}{$a$}\setlength{\Height}{-0.5\Height}\setlength{\Depth}{0.5\Depth}\addtolength{\Height}{\Depth}%
\put(3.1000,0.3000){\hspace*{\Width}\raisebox{\Height}{$a$}}%
%
%
\polyline(4.00000,-3.00000)(3.99537,-2.93631)(3.99028,-2.87266)(3.98505,-2.81279)%
\polyline(3.96607,-2.62594)(3.96525,-2.61847)(3.95783,-2.55504)(3.94995,-2.49168)(3.94307,-2.43954)%
\polyline(3.91607,-2.25368)(3.91378,-2.23883)(3.90358,-2.17579)(3.89292,-2.11283)(3.88507,-2.06845)%
\polyline(3.85010,-1.88392)(3.84569,-1.86182)(3.83273,-1.79929)(3.81931,-1.73686)(3.81115,-1.70020)%
\polyline(3.76827,-1.51735)(3.76110,-1.48817)(3.74541,-1.42627)(3.72926,-1.36449)(3.72145,-1.33547)%
\polyline(3.67074,-1.15464)(3.66018,-1.11859)(3.64178,-1.05744)(3.62294,-0.99642)(3.61614,-0.97494)%
\polyline(3.55768,-0.79646)(3.54313,-0.75380)(3.52206,-0.69351)(3.50056,-0.63339)(3.49540,-0.61928)%
\polyline(3.42932,-0.44348)(3.41016,-0.39450)(3.38648,-0.33520)(3.36235,-0.27608)(3.35946,-0.26914)%
\polyline(3.28587,-0.09635)(3.26155,-0.04139)(3.23528,0.01681)(3.20858,0.07482)\polyline(2.98501,0.52197)(2.95462,0.57814)(2.92382,0.63407)(2.89445,0.68651)\polyline(2.80037,0.84905)(2.79655,0.85553)(2.76373,0.91030)(2.73050,0.96483)(2.70281,1.00954)%
\polyline(2.60182,1.16788)(2.59363,1.18048)(2.55843,1.23376)(2.52284,1.28678)(2.49744,1.32402)%
\polyline(2.38973,1.47787)(2.37663,1.49621)(2.33912,1.54789)(2.30124,1.59930)(2.27873,1.62937)%
\polyline(2.16450,1.77844)(2.14597,1.80210)(2.10623,1.85209)(2.06612,1.90178)(2.04708,1.92502)%
\polyline(1.92654,2.06904)(1.90210,2.09757)(1.86020,2.14576)(1.81795,2.19364)(1.80292,2.21043)%
\polyline(1.67629,2.34913)(1.64549,2.38204)(1.60151,2.42834)(1.55720,2.47432)(1.54670,2.48507)%
\polyline(1.41421,2.61818)(1.37663,2.65497)(1.33066,2.69929)(1.28438,2.74328)(1.27889,2.74842)%
\polyline(1.14080,2.87571)(1.09604,2.91582)(1.04818,2.95809)(1.00000,3.00000)%
%
\settowidth{\Width}{$b$}\setlength{\Width}{-0.5\Width}%
\settoheight{\Height}{$b$}\settodepth{\Depth}{$b$}\setlength{\Height}{-0.5\Height}\setlength{\Depth}{0.5\Depth}\addtolength{\Height}{\Depth}%
\put(0.0000,-3.8000){\hspace*{\Width}\raisebox{\Height}{$b$}}%
%
%
\polyline(-4.00000,-3.00000)(-3.92871,-3.02940)(-3.85720,-3.05826)(-3.82987,-3.06906)%
\polyline(-3.65858,-3.13518)(-3.64141,-3.14167)(-3.56908,-3.16840)(-3.49654,-3.19460)(-3.48616,-3.19826)%
\polyline(-3.31262,-3.25825)(-3.27782,-3.26995)(-3.20454,-3.29398)(-3.13808,-3.31524)%
\polyline(-2.96254,-3.36906)(-2.90972,-3.38467)(-2.83560,-3.40597)(-2.78608,-3.41980)%
\polyline(-2.60877,-3.46747)(-2.53762,-3.48566)(-2.46276,-3.50420)(-2.43062,-3.51190)%
\polyline(-2.25173,-3.55326)(-2.23740,-3.55648)(-2.16203,-3.57279)(-2.08654,-3.58854)(-2.07214,-3.59143)%
\polyline(-1.89189,-3.62640)(-1.85940,-3.63243)(-1.78347,-3.64594)(-1.71107,-3.65826)%
\polyline(-1.52970,-3.68682)(-1.47884,-3.69432)(-1.40246,-3.70500)(-1.34786,-3.71223)%
\polyline(-1.16560,-3.73445)(-1.09624,-3.74206)(-1.01953,-3.74991)(-0.98297,-3.75337)%
\polyline(-0.80004,-3.76915)(-0.78906,-3.77002)(-0.71214,-3.77559)(-0.63519,-3.78058)(-0.61686,-3.78164)%
\polyline(-0.43348,-3.79089)(-0.40413,-3.79215)(-0.32706,-3.79486)(-0.24998,-3.79700)%
\polyline(0.24998,-3.79700)(0.32706,-3.79486)(0.40413,-3.79215)(0.43348,-3.79089)%
\polyline(0.61686,-3.78164)(0.63519,-3.78058)(0.71214,-3.77559)(0.78906,-3.77002)(0.80004,-3.76915)%
\polyline(0.98297,-3.75337)(1.01953,-3.74991)(1.09624,-3.74206)(1.16560,-3.73445)%
\polyline(1.34786,-3.71223)(1.40246,-3.70500)(1.47884,-3.69432)(1.52970,-3.68682)%
\polyline(1.71107,-3.65826)(1.78347,-3.64594)(1.85940,-3.63243)(1.89189,-3.62640)%
\polyline(2.07214,-3.59143)(2.08654,-3.58854)(2.16203,-3.57279)(2.23740,-3.55648)(2.25173,-3.55326)%
\polyline(2.43062,-3.51190)(2.46276,-3.50420)(2.53762,-3.48566)(2.60877,-3.46747)%
\polyline(2.78608,-3.41980)(2.83560,-3.40597)(2.90972,-3.38467)(2.96254,-3.36906)%
\polyline(3.13808,-3.31524)(3.20454,-3.29398)(3.27782,-3.26995)(3.31262,-3.25825)%
\polyline(3.48616,-3.19826)(3.49655,-3.19460)(3.56908,-3.16840)(3.64141,-3.14167)(3.65858,-3.13518)%
\polyline(3.82987,-3.06906)(3.85720,-3.05826)(3.92871,-3.02940)(4.00000,-3.00000)%
%
%
\linethickness{0.006in}%
\polyline(3.96750,-3.00000)(3.98920,-2.97830)%
\polyline(3.61400,-3.00000)(3.87130,-2.74260)%
\polyline(3.26040,-3.00000)(3.75350,-2.50690)%
\polyline(2.90690,-3.00000)(3.63560,-2.27120)%
\polyline(2.55330,-3.00000)(3.51780,-2.03550)%
\polyline(2.19970,-3.00000)(3.39990,-1.79980)%
\polyline(1.84620,-3.00000)(3.28210,-1.56410)%
\polyline(1.49260,-3.00000)(3.16420,-1.32840)%
\polyline(1.13910,-3.00000)(1.26550,-2.87360)%
\polyline(2.55550,-1.58360)(3.04640,-1.09270)%
\polyline(0.78550,-3.00000)(0.80490,-2.98060)%
\polyline(2.66690,-1.11870)(2.92850,-0.85700)%
\polyline(0.43200,-3.00000)(0.43620,-2.99580)%
\polyline(2.67560,-0.75640)(2.81070,-0.62130)%
\polyline(0.07840,-3.00000)(0.12340,-2.95500)%
\polyline(2.63560,-0.44280)(2.69280,-0.38560)%
\polyline(-0.27510,-3.00000)(-0.15290,-2.87780)%
\polyline(2.55950,-0.16540)(2.57500,-0.14990)%
\polyline(-0.62870,-3.00000)(-0.40120,-2.77260)%
\polyline(2.45440,0.08310)(2.45710,0.08580)%
\polyline(-0.98220,-3.00000)(-0.62550,-2.64320)%
\polyline(2.32470,0.30700)(2.33930,0.32150)%
\polyline(-1.33580,-3.00000)(-0.82780,-2.49200)%
\polyline(2.17280,0.50860)(2.22140,0.55720)%
\polyline(-1.68930,-3.00000)(-1.00920,-2.31990)%
\polyline(2.00000,0.68930)(2.10360,0.79290)%
\polyline(-2.04290,-3.00000)(-1.16930,-2.12640)%
\polyline(1.80630,0.84920)(1.98570,1.02860)%
\polyline(-2.39640,-3.00000)(-1.30710,-1.91070)%
\polyline(1.59110,0.98750)(1.86790,1.26430)%
\polyline(-2.75000,-3.00000)(-1.42200,-1.67200)%
\polyline(1.35240,1.10240)(1.75000,1.50000)%
\polyline(-3.10360,-3.00000)(-1.51080,-1.40720)%
\polyline(1.08680,1.19040)(1.63210,1.73570)%
\polyline(-3.45710,-3.00000)(-1.56550,-1.10840)%
\polyline(0.78870,1.24590)(1.51430,1.97140)%
\polyline(-3.81070,-3.00000)(-1.57400,-0.76330)%
\polyline(0.44830,1.25900)(1.39640,2.20710)%
\polyline(-3.17890,-2.01470)(-1.51620,-0.35200)%
\polyline(0.03000,1.19420)(1.27860,2.44280)%
\polyline(-1.41120,0.10660)(-1.24330,0.27450)%
\polyline(-0.58060,0.93710)(1.16070,2.67850)%
\polyline(0.35660,2.22790)(1.04290,2.91420)%
%
\linethickness{0.008in}%
\linethickness{0.004in}%
\put(0.55102,-0.86841){\circle*{0.160000}}

\put(0.55102,-0.86841){\circle{0.160000}}
%
\linethickness{0.008in}%
\settowidth{\Width}{A}\setlength{\Width}{-1\Width}%
\settoheight{\Height}{A}\settodepth{\Depth}{A}\setlength{\Height}{-\Height}%
\put(-4.0500,-3.0500){\hspace*{\Width}\raisebox{\Height}{A}}%
%
%
\settowidth{\Width}{B}\setlength{\Width}{0\Width}%
\settoheight{\Height}{B}\settodepth{\Depth}{B}\setlength{\Height}{\Depth}%
\put(1.0500,3.0500){\hspace*{\Width}\raisebox{\Height}{B}}%
%
%
\settowidth{\Width}{C}\setlength{\Width}{0\Width}%
\settoheight{\Height}{C}\settodepth{\Depth}{C}\setlength{\Height}{-\Height}%
\put(4.0500,-3.0500){\hspace*{\Width}\raisebox{\Height}{C}}%
%
%
\settowidth{\Width}{I}\setlength{\Width}{0\Width}%
\settoheight{\Height}{I}\settodepth{\Depth}{I}\setlength{\Height}{-\Height}%
\put(0.6000,-0.9200){\hspace*{\Width}\raisebox{\Height}{I}}%
%
%
\end{picture}}%
\end{center}


\subsection{Graphs of Functions}
\ketcindy\ can produce graphs of functions with
\begin{verbatim}
    Plotdata("1","x^2","x");
\end{verbatim}
or parametrically with
\begin{verbatim}
    Paramplot("1","[2*cos(t),sin(t)]", "t=[0,2*pi]");
\end{verbatim}
\noindent Here we give an example of the solution curve of a differential equation. The script is:
%\verb|        // data are assigned to the variable de1.|\\
%\verb|        // [0,XMAX] is the range of t.|\\
\begin{verbatim}
    Deqplot("1","y``=-L.x*y`-G.x*y","t=[0,XMAX]",0,[C.y,0]);
    // the equation is y''=-ay'-by (a=L.x, b=G.x).
    // C.y,0 are initial values of y and y' at t=0.
    Expr(M,"e","\displaystyle\frac{d^2 x}{dt^2}+"
           +"+L.x+"\frac{dx}{dt}+"+G.x+"x=0");
\end{verbatim}

Note that points C, G, L  on segments AB, EF, HK are movable, and 
are used to decide the coefficients and the initial value as you can see in the above scripts.
\vspace{3mm}

\begin{center}
\includegraphics[bb=0.00 0.00 385.00 398.00,height=60mm]{Fig/diffeq1.pdf}
\hspace{5mm}
%%% /Users/takatoosetsuo/Dropbox/2016ketpic/0801ACA/ACAedutakato/fig/s210diffeq1.tex 2016-11-22 12:24
%%% s210diffeq1.sce 2016-11-22 12:24
{\unitlength=4.5mm%
\begin{picture}%
(  12.61000,  13.04000)(  -0.94000,  -7.54000)%
\linethickness{0.008in}%
%
\polyline(0.00000,2.59000)(0.05864,2.55661)(0.11729,2.45745)(0.17593,2.29535)(0.23457,2.07473)%
(0.29322,1.80149)(0.35186,1.48287)(0.41050,1.12727)(0.46915,0.74400)(0.52779,0.34303)%
(0.58643,-0.06524)(0.64508,-0.47024)(0.70372,-0.86157)(0.76236,-1.22918)(0.82101,-1.56370)%
(0.87965,-1.85662)(0.93829,-2.10057)(0.99693,-2.28945)(1.05558,-2.41859)(1.11422,-2.48491)%
(1.17286,-2.48693)(1.23151,-2.42486)(1.29015,-2.30053)(1.34879,-2.11740)(1.40744,-1.88041)%
(1.46608,-1.59587)(1.52472,-1.27131)(1.58337,-0.91523)(1.64201,-0.53694)(1.70065,-0.14628)%
(1.75930,0.24665)(1.81794,0.63170)(1.87658,0.99899)(1.93523,1.33912)(1.99387,1.64342)%
(2.05251,1.90420)(2.11116,2.11490)(2.16980,2.27028)(2.22844,2.36655)(2.28709,2.40147)%
(2.34573,2.37436)(2.40437,2.28617)(2.46302,2.13941)(2.52166,1.93807)(2.58030,1.68757)%
(2.63894,1.39454)(2.69759,1.06671)(2.75623,0.71265)(2.81487,0.34159)(2.87352,-0.03684)%
(2.93216,-0.41285)(2.99080,-0.77678)(3.04945,-1.11929)(3.10809,-1.43163)(3.16673,-1.70586)%
(3.22538,-1.93508)(3.28402,-2.11355)(3.34266,-2.23686)(3.40131,-2.30205)(3.45995,-2.30766)%
(3.51859,-2.25378)(3.57724,-2.14202)(3.63588,-1.97550)(3.69452,-1.75870)(3.75317,-1.49742)%
(3.81181,-1.19856)(3.87045,-0.86997)(3.92910,-0.52022)(3.98774,-0.15842)(4.04638,0.20607)%
(4.10503,0.56384)(4.16367,0.90570)(4.22231,1.22290)(4.28095,1.50736)(4.33960,1.75188)%
(4.39824,1.95031)(4.45688,2.09770)(4.51553,2.19048)(4.57417,2.22644)(4.63281,2.20489)%
(4.69146,2.12660)(4.75010,1.99380)(4.80874,1.81014)(4.86739,1.58054)(4.92603,1.31109)%
(4.98467,1.00890)(5.04332,0.68187)(5.10196,0.33853)(5.16060,-0.01221)(5.21925,-0.36128)%
(5.27789,-0.69969)(5.33653,-1.01877)(5.39518,-1.31036)(5.45382,-1.56705)(5.51246,-1.78237)%
(5.57111,-1.95093)(5.62975,-2.06855)(5.68839,-2.13242)(5.74704,-2.14109)(5.80568,-2.09454)%
(5.86432,-1.99420)(5.92296,-1.84285)(5.98161,-1.64459)(6.04025,-1.40472)(6.09889,-1.12957)%
(6.15754,-0.82637)(6.21618,-0.50305)(6.27482,-0.16801)(6.33347,0.17006)(6.39211,0.50245)%
(6.45075,0.82060)(6.50940,1.11639)(6.56804,1.38225)(6.62668,1.61148)(6.68533,1.79828)%
(6.74397,1.93803)(6.80261,2.02729)(6.86126,2.06396)(6.91990,2.04730)(6.97854,1.97794)%
(7.03719,1.85788)(7.09583,1.69040)(7.15447,1.48001)(7.21312,1.23228)(7.27176,0.95376)%
(7.33040,0.65173)(7.38905,0.33407)(7.44769,0.00903)(7.50633,-0.31499)(7.56497,-0.62964)%
(7.62362,-0.92685)(7.68226,-1.19903)(7.74090,-1.43926)(7.79955,-1.64147)(7.85819,-1.80059)%
(7.91683,-1.91270)(7.97548,-1.97509)(8.03412,-1.98634)(8.09276,-1.94635)(8.15141,-1.85636)%
(8.21005,-1.71887)(8.26869,-1.53762)(8.32734,-1.31745)(8.38598,-1.06417)(8.44462,-0.78444)%
(8.50327,-0.48558)(8.56191,-0.17536)(8.62055,0.13818)(8.67920,0.44695)(8.73784,0.74302)%
(8.79648,1.01879)(8.85513,1.26724)(8.91377,1.48208)(8.97241,1.65789)(9.03106,1.79030)%
(9.08970,1.87606)(9.14834,1.91314)(9.20698,1.90077)(9.26563,1.83947)(9.32427,1.73100)%
(9.38291,1.57834)(9.44156,1.38560)(9.50020,1.15789)(9.55884,0.90122)(9.61749,0.62231)%
(9.67613,0.32845)(9.73477,0.02725)(9.79342,-0.27348)(9.85206,-0.56601)(9.91070,-0.84282)%
(9.96935,-1.09685)(10.02799,-1.32162)(10.08663,-1.51147)(10.14528,-1.66164)(10.20392,-1.76840)%
(10.26256,-1.82918)(10.32121,-1.84259)(10.37985,-1.80845)(10.43849,-1.72785)(10.49714,-1.60303)%
(10.55578,-1.43737)(10.61442,-1.23532)(10.67307,-1.00221)(10.73171,-0.74417)(10.79035,-0.46795)%
(10.84899,-0.18074)(10.90764,0.11002)(10.96628,0.39683)(11.02492,0.67230)(11.08357,0.92938)%
(11.14221,1.16152)(11.20085,1.36283)(11.25950,1.52824)(11.31814,1.65363)(11.37678,1.73592)%
(11.43543,1.77316)(11.49407,1.76456)(11.55271,1.71052)(11.61136,1.61260)(11.67000,1.47350)%
%
\linethickness{0.008in}%
\settowidth{\Width}{$\displaystyle\frac{d^2 x}{dt^2}+0.07\frac{dx}{dt}+7.52x=0$}\setlength{\Width}{0\Width}%
\settoheight{\Height}{$\displaystyle\frac{d^2 x}{dt^2}+0.07\frac{dx}{dt}+7.52x=0$}\settodepth{\Depth}{$\displaystyle\frac{d^2 x}{dt^2}+0.07\frac{dx}{dt}+7.52x=0$}\setlength{\Height}{-0.5\Height}\setlength{\Depth}{0.5\Depth}\addtolength{\Height}{\Depth}%
\put(2.0500,4.0000){\hspace*{\Width}\raisebox{\Height}{$\displaystyle\frac{d^2 x}{dt^2}+0.07\frac{dx}{dt}+7.52x=0$}}%
%
%
\polyline(-0.94000,0.00000)(11.67000,0.00000)%
%
\linethickness{0.008in}%
\polyline(0.00000,-7.54000)(0.00000,5.50000)%
%
\linethickness{0.008in}%
\settowidth{\Width}{$x$}\setlength{\Width}{0\Width}%
\settoheight{\Height}{$x$}\settodepth{\Depth}{$x$}\setlength{\Height}{-0.5\Height}\setlength{\Depth}{0.5\Depth}\addtolength{\Height}{\Depth}%
\put(11.7200,0.0000){\hspace*{\Width}\raisebox{\Height}{$x$}}%
%
%
\settowidth{\Width}{$y$}\setlength{\Width}{-0.5\Width}%
\settoheight{\Height}{$y$}\settodepth{\Depth}{$y$}\setlength{\Height}{\Depth}%
\put(0.0000,5.5500){\hspace*{\Width}\raisebox{\Height}{$y$}}%
%
%
\settowidth{\Width}{O}\setlength{\Width}{-1\Width}%
\settoheight{\Height}{O}\settodepth{\Depth}{O}\setlength{\Height}{-\Height}%
\put(-0.0500,-0.0500){\hspace*{\Width}\raisebox{\Height}{O}}%
%
%
\end{picture}}%

\end{center}

%\putnotes{35}{53}{%%% /Users/takatoosetsuo/Dropbox/2016ketpic/0801ACA/ACAedutakato/fig/s305incanddec.tex 2016-11-29 18:14
%%% 
{\unitlength=6mm%
\begin{picture}%
(  10.50000,  11.00000)(  -0.00000,  -0.00000)%
\linethickness{0.008in}%
%
\small%
\polyline(0.00000,11.00000)(0.00000,0.00000)%
%
\linethickness{0.008in}%
\polyline(10.50000,11.00000)(10.50000,0.00000)%
%
\linethickness{0.008in}%
\polyline(0.00000,11.00000)(10.50000,11.00000)%
%
\linethickness{0.008in}%
\polyline(0.00000,9.80000)(10.50000,9.80000)%
%
\linethickness{0.008in}%
\polyline(0.00000,6.00000)(10.50000,6.00000)%
%
\linethickness{0.008in}%
\polyline(0.00000,0.00000)(10.50000,0.00000)%
%
\linethickness{0.008in}%
\polyline(1.50000,11.00000)(1.50000,6.00000)%
%
\linethickness{0.008in}%
\polyline(3.00000,11.00000)(3.00000,6.00000)%
%
\linethickness{0.008in}%
\polyline(4.50000,11.00000)(4.50000,6.00000)%
%
\linethickness{0.008in}%
\polyline(6.00000,11.00000)(6.00000,6.00000)%
%
\linethickness{0.008in}%
\polyline(7.50000,11.00000)(7.50000,6.00000)%
%
\linethickness{0.008in}%
\polyline(9.00000,11.00000)(9.00000,6.00000)%
%
\linethickness{0.008in}%
\polyline(0.00000,8.60000)(1.50000,8.60000)%
%
\linethickness{0.008in}%
\polyline(3.00000,8.60000)(10.50000,8.60000)%
%
\linethickness{0.008in}%
\polyline(0.00000,7.40000)(1.50000,7.40000)%
%
\linethickness{0.008in}%
\polyline(3.00000,7.40000)(10.50000,7.40000)%
%
\linethickness{0.008in}%
\polyline(1.50000,9.80000)(3.00000,6.00000)%
%
\linethickness{0.008in}%
\polyline(3.00000,9.80000)(1.50000,6.00000)%
%
\linethickness{0.008in}%
\settowidth{\Width}{$x$}\setlength{\Width}{-0.5\Width}%
\settoheight{\Height}{$x$}\settodepth{\Depth}{$x$}\setlength{\Height}{-0.5\Height}\setlength{\Depth}{0.5\Depth}\addtolength{\Height}{\Depth}%
\put(0.7500,10.4000){\hspace*{\Width}\raisebox{\Height}{$x$}}%
%
%
\settowidth{\Width}{$0$}\setlength{\Width}{-0.5\Width}%
\settoheight{\Height}{$0$}\settodepth{\Depth}{$0$}\setlength{\Height}{-0.5\Height}\setlength{\Depth}{0.5\Depth}\addtolength{\Height}{\Depth}%
\put(2.2500,10.4000){\hspace*{\Width}\raisebox{\Height}{$0$}}%
%
%
\settowidth{\Width}{$\cdots$}\setlength{\Width}{-0.5\Width}%
\settoheight{\Height}{$\cdots$}\settodepth{\Depth}{$\cdots$}\setlength{\Height}{-0.5\Height}\setlength{\Depth}{0.5\Depth}\addtolength{\Height}{\Depth}%
\put(3.7500,10.4000){\hspace*{\Width}\raisebox{\Height}{$\cdots$}}%
%
%
\settowidth{\Width}{$e$}\setlength{\Width}{-0.5\Width}%
\settoheight{\Height}{$e$}\settodepth{\Depth}{$e$}\setlength{\Height}{-0.5\Height}\setlength{\Depth}{0.5\Depth}\addtolength{\Height}{\Depth}%
\put(5.2500,10.4000){\hspace*{\Width}\raisebox{\Height}{$e$}}%
%
%
\settowidth{\Width}{$\cdots$}\setlength{\Width}{-0.5\Width}%
\settoheight{\Height}{$\cdots$}\settodepth{\Depth}{$\cdots$}\setlength{\Height}{-0.5\Height}\setlength{\Depth}{0.5\Depth}\addtolength{\Height}{\Depth}%
\put(6.7500,10.4000){\hspace*{\Width}\raisebox{\Height}{$\cdots$}}%
%
%
\settowidth{\Width}{$e\sqrt{e}$}\setlength{\Width}{-0.5\Width}%
\settoheight{\Height}{$e\sqrt{e}$}\settodepth{\Depth}{$e\sqrt{e}$}\setlength{\Height}{-0.5\Height}\setlength{\Depth}{0.5\Depth}\addtolength{\Height}{\Depth}%
\put(8.2500,10.4000){\hspace*{\Width}\raisebox{\Height}{$e\sqrt{e}$}}%
%
%
\settowidth{\Width}{$\cdots$}\setlength{\Width}{-0.5\Width}%
\settoheight{\Height}{$\cdots$}\settodepth{\Depth}{$\cdots$}\setlength{\Height}{-0.5\Height}\setlength{\Depth}{0.5\Depth}\addtolength{\Height}{\Depth}%
\put(9.7500,10.4000){\hspace*{\Width}\raisebox{\Height}{$\cdots$}}%
%
%
\settowidth{\Width}{$y'$}\setlength{\Width}{-0.5\Width}%
\settoheight{\Height}{$y'$}\settodepth{\Depth}{$y'$}\setlength{\Height}{-0.5\Height}\setlength{\Depth}{0.5\Depth}\addtolength{\Height}{\Depth}%
\put(0.7500,9.2000){\hspace*{\Width}\raisebox{\Height}{$y'$}}%
%
%
\settowidth{\Width}{$$}\setlength{\Width}{-0.5\Width}%
\settoheight{\Height}{$$}\settodepth{\Depth}{$$}\setlength{\Height}{-0.5\Height}\setlength{\Depth}{0.5\Depth}\addtolength{\Height}{\Depth}%
\put(2.2500,9.2000){\hspace*{\Width}\raisebox{\Height}{$$}}%
%
%
\settowidth{\Width}{$+$}\setlength{\Width}{-0.5\Width}%
\settoheight{\Height}{$+$}\settodepth{\Depth}{$+$}\setlength{\Height}{-0.5\Height}\setlength{\Depth}{0.5\Depth}\addtolength{\Height}{\Depth}%
\put(3.7500,9.2000){\hspace*{\Width}\raisebox{\Height}{$+$}}%
%
%
\settowidth{\Width}{$0$}\setlength{\Width}{-0.5\Width}%
\settoheight{\Height}{$0$}\settodepth{\Depth}{$0$}\setlength{\Height}{-0.5\Height}\setlength{\Depth}{0.5\Depth}\addtolength{\Height}{\Depth}%
\put(5.2500,9.2000){\hspace*{\Width}\raisebox{\Height}{$0$}}%
%
%
\settowidth{\Width}{$-$}\setlength{\Width}{-0.5\Width}%
\settoheight{\Height}{$-$}\settodepth{\Depth}{$-$}\setlength{\Height}{-0.5\Height}\setlength{\Depth}{0.5\Depth}\addtolength{\Height}{\Depth}%
\put(6.7500,9.2000){\hspace*{\Width}\raisebox{\Height}{$-$}}%
%
%
\settowidth{\Width}{$-$}\setlength{\Width}{-0.5\Width}%
\settoheight{\Height}{$-$}\settodepth{\Depth}{$-$}\setlength{\Height}{-0.5\Height}\setlength{\Depth}{0.5\Depth}\addtolength{\Height}{\Depth}%
\put(8.2500,9.2000){\hspace*{\Width}\raisebox{\Height}{$-$}}%
%
%
\settowidth{\Width}{$-$}\setlength{\Width}{-0.5\Width}%
\settoheight{\Height}{$-$}\settodepth{\Depth}{$-$}\setlength{\Height}{-0.5\Height}\setlength{\Depth}{0.5\Depth}\addtolength{\Height}{\Depth}%
\put(9.7500,9.2000){\hspace*{\Width}\raisebox{\Height}{$-$}}%
%
%
\settowidth{\Width}{$y''$}\setlength{\Width}{-0.5\Width}%
\settoheight{\Height}{$y''$}\settodepth{\Depth}{$y''$}\setlength{\Height}{-0.5\Height}\setlength{\Depth}{0.5\Depth}\addtolength{\Height}{\Depth}%
\put(0.7500,8.0000){\hspace*{\Width}\raisebox{\Height}{$y''$}}%
%
%
\settowidth{\Width}{$$}\setlength{\Width}{-0.5\Width}%
\settoheight{\Height}{$$}\settodepth{\Depth}{$$}\setlength{\Height}{-0.5\Height}\setlength{\Depth}{0.5\Depth}\addtolength{\Height}{\Depth}%
\put(2.2500,8.0000){\hspace*{\Width}\raisebox{\Height}{$$}}%
%
%
\settowidth{\Width}{$-$}\setlength{\Width}{-0.5\Width}%
\settoheight{\Height}{$-$}\settodepth{\Depth}{$-$}\setlength{\Height}{-0.5\Height}\setlength{\Depth}{0.5\Depth}\addtolength{\Height}{\Depth}%
\put(3.7500,8.0000){\hspace*{\Width}\raisebox{\Height}{$-$}}%
%
%
\settowidth{\Width}{$-$}\setlength{\Width}{-0.5\Width}%
\settoheight{\Height}{$-$}\settodepth{\Depth}{$-$}\setlength{\Height}{-0.5\Height}\setlength{\Depth}{0.5\Depth}\addtolength{\Height}{\Depth}%
\put(5.2500,8.0000){\hspace*{\Width}\raisebox{\Height}{$-$}}%
%
%
\settowidth{\Width}{$-$}\setlength{\Width}{-0.5\Width}%
\settoheight{\Height}{$-$}\settodepth{\Depth}{$-$}\setlength{\Height}{-0.5\Height}\setlength{\Depth}{0.5\Depth}\addtolength{\Height}{\Depth}%
\put(6.7500,8.0000){\hspace*{\Width}\raisebox{\Height}{$-$}}%
%
%
\settowidth{\Width}{$0$}\setlength{\Width}{-0.5\Width}%
\settoheight{\Height}{$0$}\settodepth{\Depth}{$0$}\setlength{\Height}{-0.5\Height}\setlength{\Depth}{0.5\Depth}\addtolength{\Height}{\Depth}%
\put(8.2500,8.0000){\hspace*{\Width}\raisebox{\Height}{$0$}}%
%
%
\settowidth{\Width}{$+$}\setlength{\Width}{-0.5\Width}%
\settoheight{\Height}{$+$}\settodepth{\Depth}{$+$}\setlength{\Height}{-0.5\Height}\setlength{\Depth}{0.5\Depth}\addtolength{\Height}{\Depth}%
\put(9.7500,8.0000){\hspace*{\Width}\raisebox{\Height}{$+$}}%
%
%
\settowidth{\Width}{$y$}\setlength{\Width}{-0.5\Width}%
\settoheight{\Height}{$y$}\settodepth{\Depth}{$y$}\setlength{\Height}{-0.5\Height}\setlength{\Depth}{0.5\Depth}\addtolength{\Height}{\Depth}%
\put(0.7500,6.7000){\hspace*{\Width}\raisebox{\Height}{$y$}}%
%
%
\settowidth{\Width}{$$}\setlength{\Width}{-0.5\Width}%
\settoheight{\Height}{$$}\settodepth{\Depth}{$$}\setlength{\Height}{-0.5\Height}\setlength{\Depth}{0.5\Depth}\addtolength{\Height}{\Depth}%
\put(2.2500,6.7000){\hspace*{\Width}\raisebox{\Height}{$$}}%
%
%
\settowidth{\Width}{$$}\setlength{\Width}{-0.5\Width}%
\settoheight{\Height}{$$}\settodepth{\Depth}{$$}\setlength{\Height}{-0.5\Height}\setlength{\Depth}{0.5\Depth}\addtolength{\Height}{\Depth}%
\put(3.7500,6.7000){\hspace*{\Width}\raisebox{\Height}{$$}}%
%
%
\settowidth{\Width}{$\dfrac{10}{e}$}\setlength{\Width}{-0.5\Width}%
\settoheight{\Height}{$\dfrac{10}{e}$}\settodepth{\Depth}{$\dfrac{10}{e}$}\setlength{\Height}{-0.5\Height}\setlength{\Depth}{0.5\Depth}\addtolength{\Height}{\Depth}%
\put(5.2500,6.7000){\hspace*{\Width}\raisebox{\Height}{$\dfrac{10}{e}$}}%
%
%
\settowidth{\Width}{$$}\setlength{\Width}{-0.5\Width}%
\settoheight{\Height}{$$}\settodepth{\Depth}{$$}\setlength{\Height}{-0.5\Height}\setlength{\Depth}{0.5\Depth}\addtolength{\Height}{\Depth}%
\put(6.7500,6.7000){\hspace*{\Width}\raisebox{\Height}{$$}}%
%
%
\settowidth{\Width}{$\dfrac{15}{e\sqrt{e}}$}\setlength{\Width}{-0.5\Width}%
\settoheight{\Height}{$\dfrac{15}{e\sqrt{e}}$}\settodepth{\Depth}{$\dfrac{15}{e\sqrt{e}}$}\setlength{\Height}{-0.5\Height}\setlength{\Depth}{0.5\Depth}\addtolength{\Height}{\Depth}%
\put(8.2500,6.7000){\hspace*{\Width}\raisebox{\Height}{$\dfrac{15}{e\sqrt{e}}$}}%
%
%
\settowidth{\Width}{$$}\setlength{\Width}{-0.5\Width}%
\settoheight{\Height}{$$}\settodepth{\Depth}{$$}\setlength{\Height}{-0.5\Height}\setlength{\Depth}{0.5\Depth}\addtolength{\Height}{\Depth}%
\put(9.7500,6.7000){\hspace*{\Width}\raisebox{\Height}{$$}}%
%
%
\settowidth{\Width}{%%% /Users/takatoosetsuo/Dropbox/2016ketpic/0801ACA/ACAedutakato/fig/s305graphforincanddec.tex 2016-11-22 12:32
%%% s305graphforincanddec.sce 2016-11-22 12:32
{\unitlength=5mm%
\begin{picture}%
(   9.96000,   5.66000)(  -0.65000,  -1.22000)%
\linethickness{0.008in}%
%
\polyline(0.89996,-1.22000)(0.94040,-0.65340)(1.03444,0.32737)(1.12848,1.07113)(1.22253,1.64347)%
(1.31657,2.08897)(1.41061,2.43881)(1.50465,2.71531)(1.59869,2.93480)(1.69273,3.10943)%
(1.78677,3.24837)(1.88081,3.35867)(1.97485,3.44579)(2.06889,3.51402)(2.16293,3.56675)%
(2.25697,3.60671)(2.35101,3.63608)(2.44505,3.65664)(2.53909,3.66984)(2.63313,3.67689)%
(2.72717,3.67877)(2.82121,3.67632)(2.91525,3.67020)(3.00929,3.66101)(3.10333,3.64923)%
(3.19737,3.63526)(3.29141,3.61947)(3.38545,3.60214)(3.47949,3.58353)(3.57354,3.56385)%
(3.66758,3.54330)(3.76162,3.52202)(3.85566,3.50016)(3.94970,3.47783)(4.04374,3.45514)%
(4.13778,3.43218)(4.23182,3.40901)(4.32586,3.38571)(4.41990,3.36233)(4.51394,3.33892)%
(4.60798,3.31553)(4.70202,3.29219)(4.79606,3.26892)(4.89010,3.24577)(4.98414,3.22274)%
(5.07818,3.19987)(5.17222,3.17717)(5.26626,3.15465)(5.36030,3.13232)(5.45434,3.11020)%
(5.54838,3.08830)(5.64242,3.06661)(5.73646,3.04516)(5.83051,3.02393)(5.92455,3.00294)%
(6.01859,2.98218)(6.11263,2.96167)(6.20667,2.94139)(6.30071,2.92136)(6.39475,2.90156)%
(6.48879,2.88201)(6.58283,2.86270)(6.67687,2.84362)(6.77091,2.82478)(6.86495,2.80618)%
(6.95899,2.78781)(7.05303,2.76967)(7.14707,2.75176)(7.24111,2.73408)(7.33515,2.71661)%
(7.42919,2.69937)(7.52323,2.68235)(7.61727,2.66555)(7.71131,2.64895)(7.80535,2.63256)%
(7.89939,2.61639)(7.99343,2.60041)(8.08747,2.58463)(8.18152,2.56906)(8.27556,2.55367)%
(8.36960,2.53848)(8.46364,2.52348)(8.55768,2.50866)(8.65172,2.49402)(8.74576,2.47957)%
(8.83980,2.46529)(8.93384,2.45118)(9.02788,2.43725)(9.12192,2.42348)(9.21596,2.40988)%
(9.31000,2.39644)%
%
\linethickness{0.008in}%
\polyline(-0.65000,0.00000)(9.31000,0.00000)%
%
\linethickness{0.008in}%
\polyline(0.00000,-1.22000)(0.00000,4.44000)%
%
\linethickness{0.008in}%
\settowidth{\Width}{$x$}\setlength{\Width}{0\Width}%
\settoheight{\Height}{$x$}\settodepth{\Depth}{$x$}\setlength{\Height}{-0.5\Height}\setlength{\Depth}{0.5\Depth}\addtolength{\Height}{\Depth}%
\put(9.3600,0.0000){\hspace*{\Width}\raisebox{\Height}{$x$}}%
%
%
\settowidth{\Width}{$y$}\setlength{\Width}{-0.5\Width}%
\settoheight{\Height}{$y$}\settodepth{\Depth}{$y$}\setlength{\Height}{\Depth}%
\put(0.0000,4.4900){\hspace*{\Width}\raisebox{\Height}{$y$}}%
%
%
\settowidth{\Width}{O}\setlength{\Width}{-1\Width}%
\settoheight{\Height}{O}\settodepth{\Depth}{O}\setlength{\Height}{-\Height}%
\put(-0.0500,-0.0500){\hspace*{\Width}\raisebox{\Height}{O}}%
%
%
\end{picture}}%}\setlength{\Width}{-0.5\Width}%
\settoheight{\Height}{%%% /Users/takatoosetsuo/Dropbox/2016ketpic/0801ACA/ACAedutakato/fig/s305graphforincanddec.tex 2016-11-22 12:32
%%% s305graphforincanddec.sce 2016-11-22 12:32
{\unitlength=5mm%
\begin{picture}%
(   9.96000,   5.66000)(  -0.65000,  -1.22000)%
\linethickness{0.008in}%
%
\polyline(0.89996,-1.22000)(0.94040,-0.65340)(1.03444,0.32737)(1.12848,1.07113)(1.22253,1.64347)%
(1.31657,2.08897)(1.41061,2.43881)(1.50465,2.71531)(1.59869,2.93480)(1.69273,3.10943)%
(1.78677,3.24837)(1.88081,3.35867)(1.97485,3.44579)(2.06889,3.51402)(2.16293,3.56675)%
(2.25697,3.60671)(2.35101,3.63608)(2.44505,3.65664)(2.53909,3.66984)(2.63313,3.67689)%
(2.72717,3.67877)(2.82121,3.67632)(2.91525,3.67020)(3.00929,3.66101)(3.10333,3.64923)%
(3.19737,3.63526)(3.29141,3.61947)(3.38545,3.60214)(3.47949,3.58353)(3.57354,3.56385)%
(3.66758,3.54330)(3.76162,3.52202)(3.85566,3.50016)(3.94970,3.47783)(4.04374,3.45514)%
(4.13778,3.43218)(4.23182,3.40901)(4.32586,3.38571)(4.41990,3.36233)(4.51394,3.33892)%
(4.60798,3.31553)(4.70202,3.29219)(4.79606,3.26892)(4.89010,3.24577)(4.98414,3.22274)%
(5.07818,3.19987)(5.17222,3.17717)(5.26626,3.15465)(5.36030,3.13232)(5.45434,3.11020)%
(5.54838,3.08830)(5.64242,3.06661)(5.73646,3.04516)(5.83051,3.02393)(5.92455,3.00294)%
(6.01859,2.98218)(6.11263,2.96167)(6.20667,2.94139)(6.30071,2.92136)(6.39475,2.90156)%
(6.48879,2.88201)(6.58283,2.86270)(6.67687,2.84362)(6.77091,2.82478)(6.86495,2.80618)%
(6.95899,2.78781)(7.05303,2.76967)(7.14707,2.75176)(7.24111,2.73408)(7.33515,2.71661)%
(7.42919,2.69937)(7.52323,2.68235)(7.61727,2.66555)(7.71131,2.64895)(7.80535,2.63256)%
(7.89939,2.61639)(7.99343,2.60041)(8.08747,2.58463)(8.18152,2.56906)(8.27556,2.55367)%
(8.36960,2.53848)(8.46364,2.52348)(8.55768,2.50866)(8.65172,2.49402)(8.74576,2.47957)%
(8.83980,2.46529)(8.93384,2.45118)(9.02788,2.43725)(9.12192,2.42348)(9.21596,2.40988)%
(9.31000,2.39644)%
%
\linethickness{0.008in}%
\polyline(-0.65000,0.00000)(9.31000,0.00000)%
%
\linethickness{0.008in}%
\polyline(0.00000,-1.22000)(0.00000,4.44000)%
%
\linethickness{0.008in}%
\settowidth{\Width}{$x$}\setlength{\Width}{0\Width}%
\settoheight{\Height}{$x$}\settodepth{\Depth}{$x$}\setlength{\Height}{-0.5\Height}\setlength{\Depth}{0.5\Depth}\addtolength{\Height}{\Depth}%
\put(9.3600,0.0000){\hspace*{\Width}\raisebox{\Height}{$x$}}%
%
%
\settowidth{\Width}{$y$}\setlength{\Width}{-0.5\Width}%
\settoheight{\Height}{$y$}\settodepth{\Depth}{$y$}\setlength{\Height}{\Depth}%
\put(0.0000,4.4900){\hspace*{\Width}\raisebox{\Height}{$y$}}%
%
%
\settowidth{\Width}{O}\setlength{\Width}{-1\Width}%
\settoheight{\Height}{O}\settodepth{\Depth}{O}\setlength{\Height}{-\Height}%
\put(-0.0500,-0.0500){\hspace*{\Width}\raisebox{\Height}{O}}%
%
%
\end{picture}}%}\settodepth{\Depth}{%%% /Users/takatoosetsuo/Dropbox/2016ketpic/0801ACA/ACAedutakato/fig/s305graphforincanddec.tex 2016-11-22 12:32
%%% s305graphforincanddec.sce 2016-11-22 12:32
{\unitlength=5mm%
\begin{picture}%
(   9.96000,   5.66000)(  -0.65000,  -1.22000)%
\linethickness{0.008in}%
%
\polyline(0.89996,-1.22000)(0.94040,-0.65340)(1.03444,0.32737)(1.12848,1.07113)(1.22253,1.64347)%
(1.31657,2.08897)(1.41061,2.43881)(1.50465,2.71531)(1.59869,2.93480)(1.69273,3.10943)%
(1.78677,3.24837)(1.88081,3.35867)(1.97485,3.44579)(2.06889,3.51402)(2.16293,3.56675)%
(2.25697,3.60671)(2.35101,3.63608)(2.44505,3.65664)(2.53909,3.66984)(2.63313,3.67689)%
(2.72717,3.67877)(2.82121,3.67632)(2.91525,3.67020)(3.00929,3.66101)(3.10333,3.64923)%
(3.19737,3.63526)(3.29141,3.61947)(3.38545,3.60214)(3.47949,3.58353)(3.57354,3.56385)%
(3.66758,3.54330)(3.76162,3.52202)(3.85566,3.50016)(3.94970,3.47783)(4.04374,3.45514)%
(4.13778,3.43218)(4.23182,3.40901)(4.32586,3.38571)(4.41990,3.36233)(4.51394,3.33892)%
(4.60798,3.31553)(4.70202,3.29219)(4.79606,3.26892)(4.89010,3.24577)(4.98414,3.22274)%
(5.07818,3.19987)(5.17222,3.17717)(5.26626,3.15465)(5.36030,3.13232)(5.45434,3.11020)%
(5.54838,3.08830)(5.64242,3.06661)(5.73646,3.04516)(5.83051,3.02393)(5.92455,3.00294)%
(6.01859,2.98218)(6.11263,2.96167)(6.20667,2.94139)(6.30071,2.92136)(6.39475,2.90156)%
(6.48879,2.88201)(6.58283,2.86270)(6.67687,2.84362)(6.77091,2.82478)(6.86495,2.80618)%
(6.95899,2.78781)(7.05303,2.76967)(7.14707,2.75176)(7.24111,2.73408)(7.33515,2.71661)%
(7.42919,2.69937)(7.52323,2.68235)(7.61727,2.66555)(7.71131,2.64895)(7.80535,2.63256)%
(7.89939,2.61639)(7.99343,2.60041)(8.08747,2.58463)(8.18152,2.56906)(8.27556,2.55367)%
(8.36960,2.53848)(8.46364,2.52348)(8.55768,2.50866)(8.65172,2.49402)(8.74576,2.47957)%
(8.83980,2.46529)(8.93384,2.45118)(9.02788,2.43725)(9.12192,2.42348)(9.21596,2.40988)%
(9.31000,2.39644)%
%
\linethickness{0.008in}%
\polyline(-0.65000,0.00000)(9.31000,0.00000)%
%
\linethickness{0.008in}%
\polyline(0.00000,-1.22000)(0.00000,4.44000)%
%
\linethickness{0.008in}%
\settowidth{\Width}{$x$}\setlength{\Width}{0\Width}%
\settoheight{\Height}{$x$}\settodepth{\Depth}{$x$}\setlength{\Height}{-0.5\Height}\setlength{\Depth}{0.5\Depth}\addtolength{\Height}{\Depth}%
\put(9.3600,0.0000){\hspace*{\Width}\raisebox{\Height}{$x$}}%
%
%
\settowidth{\Width}{$y$}\setlength{\Width}{-0.5\Width}%
\settoheight{\Height}{$y$}\settodepth{\Depth}{$y$}\setlength{\Height}{\Depth}%
\put(0.0000,4.4900){\hspace*{\Width}\raisebox{\Height}{$y$}}%
%
%
\settowidth{\Width}{O}\setlength{\Width}{-1\Width}%
\settoheight{\Height}{O}\settodepth{\Depth}{O}\setlength{\Height}{-\Height}%
\put(-0.0500,-0.0500){\hspace*{\Width}\raisebox{\Height}{O}}%
%
%
\end{picture}}%}\setlength{\Height}{-0.5\Height}\setlength{\Depth}{0.5\Depth}\addtolength{\Height}{\Depth}%
\put(5.2500,3.0000){\hspace*{\Width}\raisebox{\Height}{%%% /Users/takatoosetsuo/Dropbox/2016ketpic/0801ACA/ACAedutakato/fig/s305graphforincanddec.tex 2016-11-22 12:32
%%% s305graphforincanddec.sce 2016-11-22 12:32
{\unitlength=5mm%
\begin{picture}%
(   9.96000,   5.66000)(  -0.65000,  -1.22000)%
\linethickness{0.008in}%
%
\polyline(0.89996,-1.22000)(0.94040,-0.65340)(1.03444,0.32737)(1.12848,1.07113)(1.22253,1.64347)%
(1.31657,2.08897)(1.41061,2.43881)(1.50465,2.71531)(1.59869,2.93480)(1.69273,3.10943)%
(1.78677,3.24837)(1.88081,3.35867)(1.97485,3.44579)(2.06889,3.51402)(2.16293,3.56675)%
(2.25697,3.60671)(2.35101,3.63608)(2.44505,3.65664)(2.53909,3.66984)(2.63313,3.67689)%
(2.72717,3.67877)(2.82121,3.67632)(2.91525,3.67020)(3.00929,3.66101)(3.10333,3.64923)%
(3.19737,3.63526)(3.29141,3.61947)(3.38545,3.60214)(3.47949,3.58353)(3.57354,3.56385)%
(3.66758,3.54330)(3.76162,3.52202)(3.85566,3.50016)(3.94970,3.47783)(4.04374,3.45514)%
(4.13778,3.43218)(4.23182,3.40901)(4.32586,3.38571)(4.41990,3.36233)(4.51394,3.33892)%
(4.60798,3.31553)(4.70202,3.29219)(4.79606,3.26892)(4.89010,3.24577)(4.98414,3.22274)%
(5.07818,3.19987)(5.17222,3.17717)(5.26626,3.15465)(5.36030,3.13232)(5.45434,3.11020)%
(5.54838,3.08830)(5.64242,3.06661)(5.73646,3.04516)(5.83051,3.02393)(5.92455,3.00294)%
(6.01859,2.98218)(6.11263,2.96167)(6.20667,2.94139)(6.30071,2.92136)(6.39475,2.90156)%
(6.48879,2.88201)(6.58283,2.86270)(6.67687,2.84362)(6.77091,2.82478)(6.86495,2.80618)%
(6.95899,2.78781)(7.05303,2.76967)(7.14707,2.75176)(7.24111,2.73408)(7.33515,2.71661)%
(7.42919,2.69937)(7.52323,2.68235)(7.61727,2.66555)(7.71131,2.64895)(7.80535,2.63256)%
(7.89939,2.61639)(7.99343,2.60041)(8.08747,2.58463)(8.18152,2.56906)(8.27556,2.55367)%
(8.36960,2.53848)(8.46364,2.52348)(8.55768,2.50866)(8.65172,2.49402)(8.74576,2.47957)%
(8.83980,2.46529)(8.93384,2.45118)(9.02788,2.43725)(9.12192,2.42348)(9.21596,2.40988)%
(9.31000,2.39644)%
%
\linethickness{0.008in}%
\polyline(-0.65000,0.00000)(9.31000,0.00000)%
%
\linethickness{0.008in}%
\polyline(0.00000,-1.22000)(0.00000,4.44000)%
%
\linethickness{0.008in}%
\settowidth{\Width}{$x$}\setlength{\Width}{0\Width}%
\settoheight{\Height}{$x$}\settodepth{\Depth}{$x$}\setlength{\Height}{-0.5\Height}\setlength{\Depth}{0.5\Depth}\addtolength{\Height}{\Depth}%
\put(9.3600,0.0000){\hspace*{\Width}\raisebox{\Height}{$x$}}%
%
%
\settowidth{\Width}{$y$}\setlength{\Width}{-0.5\Width}%
\settoheight{\Height}{$y$}\settodepth{\Depth}{$y$}\setlength{\Height}{\Depth}%
\put(0.0000,4.4900){\hspace*{\Width}\raisebox{\Height}{$y$}}%
%
%
\settowidth{\Width}{O}\setlength{\Width}{-1\Width}%
\settoheight{\Height}{O}\settodepth{\Depth}{O}\setlength{\Height}{-\Height}%
\put(-0.0500,-0.0500){\hspace*{\Width}\raisebox{\Height}{O}}%
%
%
\end{picture}}%}}%
%
%
\end{picture}}%}
%\putnotes{35}{110}{Figure \thefigno\ \ Table}\addtocounter{figno}{1}%
%\putnotes{97}{65}{\input{fig/kumamonthin.tex}}
%\putnotes{97}{59}{%
%\includegraphics[bb=0.00 0.00 183.63 142.07]{fig/figkumamon.pdf}}
%\putnotene{100}{72}{\input{fig/kumamoto.tex}}
%\putnotes{97}{110}{Figure \thefigno\ \ B\'ezier Curve}\addtocounter{figno}{1}%

\subsection{Drawing Tables}
Writing the code for tables to be inserted into the \TeX\ documents is sometimes troublesome.
However, it is not a hard job for \ketcindy\ (see the output in Figure.

\begin{verbatim}
      xLst=apply(1..7,15);
      yLst=[10,10,10,10,80];
      rmvL=apply(1..6,"c"+text(#)+"r4r5");
      rmvL=concat(rmvL,["r2c1c2","r3c1c2"]);
      Tabledata("",xLst,yLst,rmvL);
      Tlistplot(["c1r1","c2r4"]);
      Tlistplot(["c2r1","c1r4"]);
      Putrowexpr(1,"c",
          ["x","0","\cdots","e","\cdots","e\sqrt{e}","\cdots"]);
      Putrowexpr(2,"c",["y`","","+","0","-","-","-"]);
      Putrowexpr(3,"c",["y``","","-","-","-","0","+"]);
      Putrowexpr(4,"c",["y","","","10/e","","15/e\sqrt{e}",""]);
      Putcell("c0r4","c7r5","c","\input{fig/graph}");
\end{verbatim}

\vspace{2mm}

\begin{center}
%%% /Users/takatoosetsuo/Dropbox/2016ketpic/0801ACA/ACAedutakato/fig/s305incanddec.tex 2016-11-29 18:14
%%% 
{\unitlength=6mm%
\begin{picture}%
(  10.50000,  11.00000)(  -0.00000,  -0.00000)%
\linethickness{0.008in}%
%
\small%
\polyline(0.00000,11.00000)(0.00000,0.00000)%
%
\linethickness{0.008in}%
\polyline(10.50000,11.00000)(10.50000,0.00000)%
%
\linethickness{0.008in}%
\polyline(0.00000,11.00000)(10.50000,11.00000)%
%
\linethickness{0.008in}%
\polyline(0.00000,9.80000)(10.50000,9.80000)%
%
\linethickness{0.008in}%
\polyline(0.00000,6.00000)(10.50000,6.00000)%
%
\linethickness{0.008in}%
\polyline(0.00000,0.00000)(10.50000,0.00000)%
%
\linethickness{0.008in}%
\polyline(1.50000,11.00000)(1.50000,6.00000)%
%
\linethickness{0.008in}%
\polyline(3.00000,11.00000)(3.00000,6.00000)%
%
\linethickness{0.008in}%
\polyline(4.50000,11.00000)(4.50000,6.00000)%
%
\linethickness{0.008in}%
\polyline(6.00000,11.00000)(6.00000,6.00000)%
%
\linethickness{0.008in}%
\polyline(7.50000,11.00000)(7.50000,6.00000)%
%
\linethickness{0.008in}%
\polyline(9.00000,11.00000)(9.00000,6.00000)%
%
\linethickness{0.008in}%
\polyline(0.00000,8.60000)(1.50000,8.60000)%
%
\linethickness{0.008in}%
\polyline(3.00000,8.60000)(10.50000,8.60000)%
%
\linethickness{0.008in}%
\polyline(0.00000,7.40000)(1.50000,7.40000)%
%
\linethickness{0.008in}%
\polyline(3.00000,7.40000)(10.50000,7.40000)%
%
\linethickness{0.008in}%
\polyline(1.50000,9.80000)(3.00000,6.00000)%
%
\linethickness{0.008in}%
\polyline(3.00000,9.80000)(1.50000,6.00000)%
%
\linethickness{0.008in}%
\settowidth{\Width}{$x$}\setlength{\Width}{-0.5\Width}%
\settoheight{\Height}{$x$}\settodepth{\Depth}{$x$}\setlength{\Height}{-0.5\Height}\setlength{\Depth}{0.5\Depth}\addtolength{\Height}{\Depth}%
\put(0.7500,10.4000){\hspace*{\Width}\raisebox{\Height}{$x$}}%
%
%
\settowidth{\Width}{$0$}\setlength{\Width}{-0.5\Width}%
\settoheight{\Height}{$0$}\settodepth{\Depth}{$0$}\setlength{\Height}{-0.5\Height}\setlength{\Depth}{0.5\Depth}\addtolength{\Height}{\Depth}%
\put(2.2500,10.4000){\hspace*{\Width}\raisebox{\Height}{$0$}}%
%
%
\settowidth{\Width}{$\cdots$}\setlength{\Width}{-0.5\Width}%
\settoheight{\Height}{$\cdots$}\settodepth{\Depth}{$\cdots$}\setlength{\Height}{-0.5\Height}\setlength{\Depth}{0.5\Depth}\addtolength{\Height}{\Depth}%
\put(3.7500,10.4000){\hspace*{\Width}\raisebox{\Height}{$\cdots$}}%
%
%
\settowidth{\Width}{$e$}\setlength{\Width}{-0.5\Width}%
\settoheight{\Height}{$e$}\settodepth{\Depth}{$e$}\setlength{\Height}{-0.5\Height}\setlength{\Depth}{0.5\Depth}\addtolength{\Height}{\Depth}%
\put(5.2500,10.4000){\hspace*{\Width}\raisebox{\Height}{$e$}}%
%
%
\settowidth{\Width}{$\cdots$}\setlength{\Width}{-0.5\Width}%
\settoheight{\Height}{$\cdots$}\settodepth{\Depth}{$\cdots$}\setlength{\Height}{-0.5\Height}\setlength{\Depth}{0.5\Depth}\addtolength{\Height}{\Depth}%
\put(6.7500,10.4000){\hspace*{\Width}\raisebox{\Height}{$\cdots$}}%
%
%
\settowidth{\Width}{$e\sqrt{e}$}\setlength{\Width}{-0.5\Width}%
\settoheight{\Height}{$e\sqrt{e}$}\settodepth{\Depth}{$e\sqrt{e}$}\setlength{\Height}{-0.5\Height}\setlength{\Depth}{0.5\Depth}\addtolength{\Height}{\Depth}%
\put(8.2500,10.4000){\hspace*{\Width}\raisebox{\Height}{$e\sqrt{e}$}}%
%
%
\settowidth{\Width}{$\cdots$}\setlength{\Width}{-0.5\Width}%
\settoheight{\Height}{$\cdots$}\settodepth{\Depth}{$\cdots$}\setlength{\Height}{-0.5\Height}\setlength{\Depth}{0.5\Depth}\addtolength{\Height}{\Depth}%
\put(9.7500,10.4000){\hspace*{\Width}\raisebox{\Height}{$\cdots$}}%
%
%
\settowidth{\Width}{$y'$}\setlength{\Width}{-0.5\Width}%
\settoheight{\Height}{$y'$}\settodepth{\Depth}{$y'$}\setlength{\Height}{-0.5\Height}\setlength{\Depth}{0.5\Depth}\addtolength{\Height}{\Depth}%
\put(0.7500,9.2000){\hspace*{\Width}\raisebox{\Height}{$y'$}}%
%
%
\settowidth{\Width}{$$}\setlength{\Width}{-0.5\Width}%
\settoheight{\Height}{$$}\settodepth{\Depth}{$$}\setlength{\Height}{-0.5\Height}\setlength{\Depth}{0.5\Depth}\addtolength{\Height}{\Depth}%
\put(2.2500,9.2000){\hspace*{\Width}\raisebox{\Height}{$$}}%
%
%
\settowidth{\Width}{$+$}\setlength{\Width}{-0.5\Width}%
\settoheight{\Height}{$+$}\settodepth{\Depth}{$+$}\setlength{\Height}{-0.5\Height}\setlength{\Depth}{0.5\Depth}\addtolength{\Height}{\Depth}%
\put(3.7500,9.2000){\hspace*{\Width}\raisebox{\Height}{$+$}}%
%
%
\settowidth{\Width}{$0$}\setlength{\Width}{-0.5\Width}%
\settoheight{\Height}{$0$}\settodepth{\Depth}{$0$}\setlength{\Height}{-0.5\Height}\setlength{\Depth}{0.5\Depth}\addtolength{\Height}{\Depth}%
\put(5.2500,9.2000){\hspace*{\Width}\raisebox{\Height}{$0$}}%
%
%
\settowidth{\Width}{$-$}\setlength{\Width}{-0.5\Width}%
\settoheight{\Height}{$-$}\settodepth{\Depth}{$-$}\setlength{\Height}{-0.5\Height}\setlength{\Depth}{0.5\Depth}\addtolength{\Height}{\Depth}%
\put(6.7500,9.2000){\hspace*{\Width}\raisebox{\Height}{$-$}}%
%
%
\settowidth{\Width}{$-$}\setlength{\Width}{-0.5\Width}%
\settoheight{\Height}{$-$}\settodepth{\Depth}{$-$}\setlength{\Height}{-0.5\Height}\setlength{\Depth}{0.5\Depth}\addtolength{\Height}{\Depth}%
\put(8.2500,9.2000){\hspace*{\Width}\raisebox{\Height}{$-$}}%
%
%
\settowidth{\Width}{$-$}\setlength{\Width}{-0.5\Width}%
\settoheight{\Height}{$-$}\settodepth{\Depth}{$-$}\setlength{\Height}{-0.5\Height}\setlength{\Depth}{0.5\Depth}\addtolength{\Height}{\Depth}%
\put(9.7500,9.2000){\hspace*{\Width}\raisebox{\Height}{$-$}}%
%
%
\settowidth{\Width}{$y''$}\setlength{\Width}{-0.5\Width}%
\settoheight{\Height}{$y''$}\settodepth{\Depth}{$y''$}\setlength{\Height}{-0.5\Height}\setlength{\Depth}{0.5\Depth}\addtolength{\Height}{\Depth}%
\put(0.7500,8.0000){\hspace*{\Width}\raisebox{\Height}{$y''$}}%
%
%
\settowidth{\Width}{$$}\setlength{\Width}{-0.5\Width}%
\settoheight{\Height}{$$}\settodepth{\Depth}{$$}\setlength{\Height}{-0.5\Height}\setlength{\Depth}{0.5\Depth}\addtolength{\Height}{\Depth}%
\put(2.2500,8.0000){\hspace*{\Width}\raisebox{\Height}{$$}}%
%
%
\settowidth{\Width}{$-$}\setlength{\Width}{-0.5\Width}%
\settoheight{\Height}{$-$}\settodepth{\Depth}{$-$}\setlength{\Height}{-0.5\Height}\setlength{\Depth}{0.5\Depth}\addtolength{\Height}{\Depth}%
\put(3.7500,8.0000){\hspace*{\Width}\raisebox{\Height}{$-$}}%
%
%
\settowidth{\Width}{$-$}\setlength{\Width}{-0.5\Width}%
\settoheight{\Height}{$-$}\settodepth{\Depth}{$-$}\setlength{\Height}{-0.5\Height}\setlength{\Depth}{0.5\Depth}\addtolength{\Height}{\Depth}%
\put(5.2500,8.0000){\hspace*{\Width}\raisebox{\Height}{$-$}}%
%
%
\settowidth{\Width}{$-$}\setlength{\Width}{-0.5\Width}%
\settoheight{\Height}{$-$}\settodepth{\Depth}{$-$}\setlength{\Height}{-0.5\Height}\setlength{\Depth}{0.5\Depth}\addtolength{\Height}{\Depth}%
\put(6.7500,8.0000){\hspace*{\Width}\raisebox{\Height}{$-$}}%
%
%
\settowidth{\Width}{$0$}\setlength{\Width}{-0.5\Width}%
\settoheight{\Height}{$0$}\settodepth{\Depth}{$0$}\setlength{\Height}{-0.5\Height}\setlength{\Depth}{0.5\Depth}\addtolength{\Height}{\Depth}%
\put(8.2500,8.0000){\hspace*{\Width}\raisebox{\Height}{$0$}}%
%
%
\settowidth{\Width}{$+$}\setlength{\Width}{-0.5\Width}%
\settoheight{\Height}{$+$}\settodepth{\Depth}{$+$}\setlength{\Height}{-0.5\Height}\setlength{\Depth}{0.5\Depth}\addtolength{\Height}{\Depth}%
\put(9.7500,8.0000){\hspace*{\Width}\raisebox{\Height}{$+$}}%
%
%
\settowidth{\Width}{$y$}\setlength{\Width}{-0.5\Width}%
\settoheight{\Height}{$y$}\settodepth{\Depth}{$y$}\setlength{\Height}{-0.5\Height}\setlength{\Depth}{0.5\Depth}\addtolength{\Height}{\Depth}%
\put(0.7500,6.7000){\hspace*{\Width}\raisebox{\Height}{$y$}}%
%
%
\settowidth{\Width}{$$}\setlength{\Width}{-0.5\Width}%
\settoheight{\Height}{$$}\settodepth{\Depth}{$$}\setlength{\Height}{-0.5\Height}\setlength{\Depth}{0.5\Depth}\addtolength{\Height}{\Depth}%
\put(2.2500,6.7000){\hspace*{\Width}\raisebox{\Height}{$$}}%
%
%
\settowidth{\Width}{$$}\setlength{\Width}{-0.5\Width}%
\settoheight{\Height}{$$}\settodepth{\Depth}{$$}\setlength{\Height}{-0.5\Height}\setlength{\Depth}{0.5\Depth}\addtolength{\Height}{\Depth}%
\put(3.7500,6.7000){\hspace*{\Width}\raisebox{\Height}{$$}}%
%
%
\settowidth{\Width}{$\dfrac{10}{e}$}\setlength{\Width}{-0.5\Width}%
\settoheight{\Height}{$\dfrac{10}{e}$}\settodepth{\Depth}{$\dfrac{10}{e}$}\setlength{\Height}{-0.5\Height}\setlength{\Depth}{0.5\Depth}\addtolength{\Height}{\Depth}%
\put(5.2500,6.7000){\hspace*{\Width}\raisebox{\Height}{$\dfrac{10}{e}$}}%
%
%
\settowidth{\Width}{$$}\setlength{\Width}{-0.5\Width}%
\settoheight{\Height}{$$}\settodepth{\Depth}{$$}\setlength{\Height}{-0.5\Height}\setlength{\Depth}{0.5\Depth}\addtolength{\Height}{\Depth}%
\put(6.7500,6.7000){\hspace*{\Width}\raisebox{\Height}{$$}}%
%
%
\settowidth{\Width}{$\dfrac{15}{e\sqrt{e}}$}\setlength{\Width}{-0.5\Width}%
\settoheight{\Height}{$\dfrac{15}{e\sqrt{e}}$}\settodepth{\Depth}{$\dfrac{15}{e\sqrt{e}}$}\setlength{\Height}{-0.5\Height}\setlength{\Depth}{0.5\Depth}\addtolength{\Height}{\Depth}%
\put(8.2500,6.7000){\hspace*{\Width}\raisebox{\Height}{$\dfrac{15}{e\sqrt{e}}$}}%
%
%
\settowidth{\Width}{$$}\setlength{\Width}{-0.5\Width}%
\settoheight{\Height}{$$}\settodepth{\Depth}{$$}\setlength{\Height}{-0.5\Height}\setlength{\Depth}{0.5\Depth}\addtolength{\Height}{\Depth}%
\put(9.7500,6.7000){\hspace*{\Width}\raisebox{\Height}{$$}}%
%
%
\settowidth{\Width}{%%% /Users/takatoosetsuo/Dropbox/2016ketpic/0801ACA/ACAedutakato/fig/s305graphforincanddec.tex 2016-11-22 12:32
%%% s305graphforincanddec.sce 2016-11-22 12:32
{\unitlength=5mm%
\begin{picture}%
(   9.96000,   5.66000)(  -0.65000,  -1.22000)%
\linethickness{0.008in}%
%
\polyline(0.89996,-1.22000)(0.94040,-0.65340)(1.03444,0.32737)(1.12848,1.07113)(1.22253,1.64347)%
(1.31657,2.08897)(1.41061,2.43881)(1.50465,2.71531)(1.59869,2.93480)(1.69273,3.10943)%
(1.78677,3.24837)(1.88081,3.35867)(1.97485,3.44579)(2.06889,3.51402)(2.16293,3.56675)%
(2.25697,3.60671)(2.35101,3.63608)(2.44505,3.65664)(2.53909,3.66984)(2.63313,3.67689)%
(2.72717,3.67877)(2.82121,3.67632)(2.91525,3.67020)(3.00929,3.66101)(3.10333,3.64923)%
(3.19737,3.63526)(3.29141,3.61947)(3.38545,3.60214)(3.47949,3.58353)(3.57354,3.56385)%
(3.66758,3.54330)(3.76162,3.52202)(3.85566,3.50016)(3.94970,3.47783)(4.04374,3.45514)%
(4.13778,3.43218)(4.23182,3.40901)(4.32586,3.38571)(4.41990,3.36233)(4.51394,3.33892)%
(4.60798,3.31553)(4.70202,3.29219)(4.79606,3.26892)(4.89010,3.24577)(4.98414,3.22274)%
(5.07818,3.19987)(5.17222,3.17717)(5.26626,3.15465)(5.36030,3.13232)(5.45434,3.11020)%
(5.54838,3.08830)(5.64242,3.06661)(5.73646,3.04516)(5.83051,3.02393)(5.92455,3.00294)%
(6.01859,2.98218)(6.11263,2.96167)(6.20667,2.94139)(6.30071,2.92136)(6.39475,2.90156)%
(6.48879,2.88201)(6.58283,2.86270)(6.67687,2.84362)(6.77091,2.82478)(6.86495,2.80618)%
(6.95899,2.78781)(7.05303,2.76967)(7.14707,2.75176)(7.24111,2.73408)(7.33515,2.71661)%
(7.42919,2.69937)(7.52323,2.68235)(7.61727,2.66555)(7.71131,2.64895)(7.80535,2.63256)%
(7.89939,2.61639)(7.99343,2.60041)(8.08747,2.58463)(8.18152,2.56906)(8.27556,2.55367)%
(8.36960,2.53848)(8.46364,2.52348)(8.55768,2.50866)(8.65172,2.49402)(8.74576,2.47957)%
(8.83980,2.46529)(8.93384,2.45118)(9.02788,2.43725)(9.12192,2.42348)(9.21596,2.40988)%
(9.31000,2.39644)%
%
\linethickness{0.008in}%
\polyline(-0.65000,0.00000)(9.31000,0.00000)%
%
\linethickness{0.008in}%
\polyline(0.00000,-1.22000)(0.00000,4.44000)%
%
\linethickness{0.008in}%
\settowidth{\Width}{$x$}\setlength{\Width}{0\Width}%
\settoheight{\Height}{$x$}\settodepth{\Depth}{$x$}\setlength{\Height}{-0.5\Height}\setlength{\Depth}{0.5\Depth}\addtolength{\Height}{\Depth}%
\put(9.3600,0.0000){\hspace*{\Width}\raisebox{\Height}{$x$}}%
%
%
\settowidth{\Width}{$y$}\setlength{\Width}{-0.5\Width}%
\settoheight{\Height}{$y$}\settodepth{\Depth}{$y$}\setlength{\Height}{\Depth}%
\put(0.0000,4.4900){\hspace*{\Width}\raisebox{\Height}{$y$}}%
%
%
\settowidth{\Width}{O}\setlength{\Width}{-1\Width}%
\settoheight{\Height}{O}\settodepth{\Depth}{O}\setlength{\Height}{-\Height}%
\put(-0.0500,-0.0500){\hspace*{\Width}\raisebox{\Height}{O}}%
%
%
\end{picture}}%}\setlength{\Width}{-0.5\Width}%
\settoheight{\Height}{%%% /Users/takatoosetsuo/Dropbox/2016ketpic/0801ACA/ACAedutakato/fig/s305graphforincanddec.tex 2016-11-22 12:32
%%% s305graphforincanddec.sce 2016-11-22 12:32
{\unitlength=5mm%
\begin{picture}%
(   9.96000,   5.66000)(  -0.65000,  -1.22000)%
\linethickness{0.008in}%
%
\polyline(0.89996,-1.22000)(0.94040,-0.65340)(1.03444,0.32737)(1.12848,1.07113)(1.22253,1.64347)%
(1.31657,2.08897)(1.41061,2.43881)(1.50465,2.71531)(1.59869,2.93480)(1.69273,3.10943)%
(1.78677,3.24837)(1.88081,3.35867)(1.97485,3.44579)(2.06889,3.51402)(2.16293,3.56675)%
(2.25697,3.60671)(2.35101,3.63608)(2.44505,3.65664)(2.53909,3.66984)(2.63313,3.67689)%
(2.72717,3.67877)(2.82121,3.67632)(2.91525,3.67020)(3.00929,3.66101)(3.10333,3.64923)%
(3.19737,3.63526)(3.29141,3.61947)(3.38545,3.60214)(3.47949,3.58353)(3.57354,3.56385)%
(3.66758,3.54330)(3.76162,3.52202)(3.85566,3.50016)(3.94970,3.47783)(4.04374,3.45514)%
(4.13778,3.43218)(4.23182,3.40901)(4.32586,3.38571)(4.41990,3.36233)(4.51394,3.33892)%
(4.60798,3.31553)(4.70202,3.29219)(4.79606,3.26892)(4.89010,3.24577)(4.98414,3.22274)%
(5.07818,3.19987)(5.17222,3.17717)(5.26626,3.15465)(5.36030,3.13232)(5.45434,3.11020)%
(5.54838,3.08830)(5.64242,3.06661)(5.73646,3.04516)(5.83051,3.02393)(5.92455,3.00294)%
(6.01859,2.98218)(6.11263,2.96167)(6.20667,2.94139)(6.30071,2.92136)(6.39475,2.90156)%
(6.48879,2.88201)(6.58283,2.86270)(6.67687,2.84362)(6.77091,2.82478)(6.86495,2.80618)%
(6.95899,2.78781)(7.05303,2.76967)(7.14707,2.75176)(7.24111,2.73408)(7.33515,2.71661)%
(7.42919,2.69937)(7.52323,2.68235)(7.61727,2.66555)(7.71131,2.64895)(7.80535,2.63256)%
(7.89939,2.61639)(7.99343,2.60041)(8.08747,2.58463)(8.18152,2.56906)(8.27556,2.55367)%
(8.36960,2.53848)(8.46364,2.52348)(8.55768,2.50866)(8.65172,2.49402)(8.74576,2.47957)%
(8.83980,2.46529)(8.93384,2.45118)(9.02788,2.43725)(9.12192,2.42348)(9.21596,2.40988)%
(9.31000,2.39644)%
%
\linethickness{0.008in}%
\polyline(-0.65000,0.00000)(9.31000,0.00000)%
%
\linethickness{0.008in}%
\polyline(0.00000,-1.22000)(0.00000,4.44000)%
%
\linethickness{0.008in}%
\settowidth{\Width}{$x$}\setlength{\Width}{0\Width}%
\settoheight{\Height}{$x$}\settodepth{\Depth}{$x$}\setlength{\Height}{-0.5\Height}\setlength{\Depth}{0.5\Depth}\addtolength{\Height}{\Depth}%
\put(9.3600,0.0000){\hspace*{\Width}\raisebox{\Height}{$x$}}%
%
%
\settowidth{\Width}{$y$}\setlength{\Width}{-0.5\Width}%
\settoheight{\Height}{$y$}\settodepth{\Depth}{$y$}\setlength{\Height}{\Depth}%
\put(0.0000,4.4900){\hspace*{\Width}\raisebox{\Height}{$y$}}%
%
%
\settowidth{\Width}{O}\setlength{\Width}{-1\Width}%
\settoheight{\Height}{O}\settodepth{\Depth}{O}\setlength{\Height}{-\Height}%
\put(-0.0500,-0.0500){\hspace*{\Width}\raisebox{\Height}{O}}%
%
%
\end{picture}}%}\settodepth{\Depth}{%%% /Users/takatoosetsuo/Dropbox/2016ketpic/0801ACA/ACAedutakato/fig/s305graphforincanddec.tex 2016-11-22 12:32
%%% s305graphforincanddec.sce 2016-11-22 12:32
{\unitlength=5mm%
\begin{picture}%
(   9.96000,   5.66000)(  -0.65000,  -1.22000)%
\linethickness{0.008in}%
%
\polyline(0.89996,-1.22000)(0.94040,-0.65340)(1.03444,0.32737)(1.12848,1.07113)(1.22253,1.64347)%
(1.31657,2.08897)(1.41061,2.43881)(1.50465,2.71531)(1.59869,2.93480)(1.69273,3.10943)%
(1.78677,3.24837)(1.88081,3.35867)(1.97485,3.44579)(2.06889,3.51402)(2.16293,3.56675)%
(2.25697,3.60671)(2.35101,3.63608)(2.44505,3.65664)(2.53909,3.66984)(2.63313,3.67689)%
(2.72717,3.67877)(2.82121,3.67632)(2.91525,3.67020)(3.00929,3.66101)(3.10333,3.64923)%
(3.19737,3.63526)(3.29141,3.61947)(3.38545,3.60214)(3.47949,3.58353)(3.57354,3.56385)%
(3.66758,3.54330)(3.76162,3.52202)(3.85566,3.50016)(3.94970,3.47783)(4.04374,3.45514)%
(4.13778,3.43218)(4.23182,3.40901)(4.32586,3.38571)(4.41990,3.36233)(4.51394,3.33892)%
(4.60798,3.31553)(4.70202,3.29219)(4.79606,3.26892)(4.89010,3.24577)(4.98414,3.22274)%
(5.07818,3.19987)(5.17222,3.17717)(5.26626,3.15465)(5.36030,3.13232)(5.45434,3.11020)%
(5.54838,3.08830)(5.64242,3.06661)(5.73646,3.04516)(5.83051,3.02393)(5.92455,3.00294)%
(6.01859,2.98218)(6.11263,2.96167)(6.20667,2.94139)(6.30071,2.92136)(6.39475,2.90156)%
(6.48879,2.88201)(6.58283,2.86270)(6.67687,2.84362)(6.77091,2.82478)(6.86495,2.80618)%
(6.95899,2.78781)(7.05303,2.76967)(7.14707,2.75176)(7.24111,2.73408)(7.33515,2.71661)%
(7.42919,2.69937)(7.52323,2.68235)(7.61727,2.66555)(7.71131,2.64895)(7.80535,2.63256)%
(7.89939,2.61639)(7.99343,2.60041)(8.08747,2.58463)(8.18152,2.56906)(8.27556,2.55367)%
(8.36960,2.53848)(8.46364,2.52348)(8.55768,2.50866)(8.65172,2.49402)(8.74576,2.47957)%
(8.83980,2.46529)(8.93384,2.45118)(9.02788,2.43725)(9.12192,2.42348)(9.21596,2.40988)%
(9.31000,2.39644)%
%
\linethickness{0.008in}%
\polyline(-0.65000,0.00000)(9.31000,0.00000)%
%
\linethickness{0.008in}%
\polyline(0.00000,-1.22000)(0.00000,4.44000)%
%
\linethickness{0.008in}%
\settowidth{\Width}{$x$}\setlength{\Width}{0\Width}%
\settoheight{\Height}{$x$}\settodepth{\Depth}{$x$}\setlength{\Height}{-0.5\Height}\setlength{\Depth}{0.5\Depth}\addtolength{\Height}{\Depth}%
\put(9.3600,0.0000){\hspace*{\Width}\raisebox{\Height}{$x$}}%
%
%
\settowidth{\Width}{$y$}\setlength{\Width}{-0.5\Width}%
\settoheight{\Height}{$y$}\settodepth{\Depth}{$y$}\setlength{\Height}{\Depth}%
\put(0.0000,4.4900){\hspace*{\Width}\raisebox{\Height}{$y$}}%
%
%
\settowidth{\Width}{O}\setlength{\Width}{-1\Width}%
\settoheight{\Height}{O}\settodepth{\Depth}{O}\setlength{\Height}{-\Height}%
\put(-0.0500,-0.0500){\hspace*{\Width}\raisebox{\Height}{O}}%
%
%
\end{picture}}%}\setlength{\Height}{-0.5\Height}\setlength{\Depth}{0.5\Depth}\addtolength{\Height}{\Depth}%
\put(5.2500,3.0000){\hspace*{\Width}\raisebox{\Height}{%%% /Users/takatoosetsuo/Dropbox/2016ketpic/0801ACA/ACAedutakato/fig/s305graphforincanddec.tex 2016-11-22 12:32
%%% s305graphforincanddec.sce 2016-11-22 12:32
{\unitlength=5mm%
\begin{picture}%
(   9.96000,   5.66000)(  -0.65000,  -1.22000)%
\linethickness{0.008in}%
%
\polyline(0.89996,-1.22000)(0.94040,-0.65340)(1.03444,0.32737)(1.12848,1.07113)(1.22253,1.64347)%
(1.31657,2.08897)(1.41061,2.43881)(1.50465,2.71531)(1.59869,2.93480)(1.69273,3.10943)%
(1.78677,3.24837)(1.88081,3.35867)(1.97485,3.44579)(2.06889,3.51402)(2.16293,3.56675)%
(2.25697,3.60671)(2.35101,3.63608)(2.44505,3.65664)(2.53909,3.66984)(2.63313,3.67689)%
(2.72717,3.67877)(2.82121,3.67632)(2.91525,3.67020)(3.00929,3.66101)(3.10333,3.64923)%
(3.19737,3.63526)(3.29141,3.61947)(3.38545,3.60214)(3.47949,3.58353)(3.57354,3.56385)%
(3.66758,3.54330)(3.76162,3.52202)(3.85566,3.50016)(3.94970,3.47783)(4.04374,3.45514)%
(4.13778,3.43218)(4.23182,3.40901)(4.32586,3.38571)(4.41990,3.36233)(4.51394,3.33892)%
(4.60798,3.31553)(4.70202,3.29219)(4.79606,3.26892)(4.89010,3.24577)(4.98414,3.22274)%
(5.07818,3.19987)(5.17222,3.17717)(5.26626,3.15465)(5.36030,3.13232)(5.45434,3.11020)%
(5.54838,3.08830)(5.64242,3.06661)(5.73646,3.04516)(5.83051,3.02393)(5.92455,3.00294)%
(6.01859,2.98218)(6.11263,2.96167)(6.20667,2.94139)(6.30071,2.92136)(6.39475,2.90156)%
(6.48879,2.88201)(6.58283,2.86270)(6.67687,2.84362)(6.77091,2.82478)(6.86495,2.80618)%
(6.95899,2.78781)(7.05303,2.76967)(7.14707,2.75176)(7.24111,2.73408)(7.33515,2.71661)%
(7.42919,2.69937)(7.52323,2.68235)(7.61727,2.66555)(7.71131,2.64895)(7.80535,2.63256)%
(7.89939,2.61639)(7.99343,2.60041)(8.08747,2.58463)(8.18152,2.56906)(8.27556,2.55367)%
(8.36960,2.53848)(8.46364,2.52348)(8.55768,2.50866)(8.65172,2.49402)(8.74576,2.47957)%
(8.83980,2.46529)(8.93384,2.45118)(9.02788,2.43725)(9.12192,2.42348)(9.21596,2.40988)%
(9.31000,2.39644)%
%
\linethickness{0.008in}%
\polyline(-0.65000,0.00000)(9.31000,0.00000)%
%
\linethickness{0.008in}%
\polyline(0.00000,-1.22000)(0.00000,4.44000)%
%
\linethickness{0.008in}%
\settowidth{\Width}{$x$}\setlength{\Width}{0\Width}%
\settoheight{\Height}{$x$}\settodepth{\Depth}{$x$}\setlength{\Height}{-0.5\Height}\setlength{\Depth}{0.5\Depth}\addtolength{\Height}{\Depth}%
\put(9.3600,0.0000){\hspace*{\Width}\raisebox{\Height}{$x$}}%
%
%
\settowidth{\Width}{$y$}\setlength{\Width}{-0.5\Width}%
\settoheight{\Height}{$y$}\settodepth{\Depth}{$y$}\setlength{\Height}{\Depth}%
\put(0.0000,4.4900){\hspace*{\Width}\raisebox{\Height}{$y$}}%
%
%
\settowidth{\Width}{O}\setlength{\Width}{-1\Width}%
\settoheight{\Height}{O}\settodepth{\Depth}{O}\setlength{\Height}{-\Height}%
\put(-0.0500,-0.0500){\hspace*{\Width}\raisebox{\Height}{O}}%
%
%
\end{picture}}%}}%
%
%
\end{picture}}%
\end{center}


\newpage

\subsection{Plotting data}
Here we call the data computed 
to generate the graphs of functions and geometric elements 
"Plotting data" which is abbreviated as PD. 
The PD to draw segment is the list of coordinates 
of its two endpoints. 
For example, 
when the coordinates of the points A and B 
are (1, 1) and (3, 2) respectively, 
PD of the segment AB named \verb|Listplot ([A,B])| 
is stored in the form \verb|[[1,1],[3,2]]|. 
Also the PD to draw a curve is the collection of 
those for drawing small segments 
which connect contiguous dividing points of the curve. 
PD are automatically given names via \ketcindy\ 
following the rules below.

\begin{itemize}
\item 
The beginning part of the PD's name 
depends on the kind of the corresponding graphical element. 
For instance, 
\verb|sg| is associated to segments and 
\verb|cr| is associated to circles. 

\item 
When some extra name is specified 
as the first argument in the definition of PD, 
it is added to the beginning part given above. 
For instance, the PD defined below 
is given the name \verb|sg1|. 
\begin{center}
\verb|Listplot("1",[[0,0],[1,2]]);| 
\end{center}

\item 
When the extra name is not needed, 
the names of the points are added 
to the beginning part given above. 
For instance, the PD defined below  
is given the name \verb|sgABC|. 
\begin{center}
\verb|Listplot([A,B,C]);|
\end{center}

\end{itemize}

\noindent 
Once PD are generated, 
their names are displayed on the console view of Cinderella. 
For instance, when the PD named \verb|sgABCA| is generated, 
the corresponding message is displayed as shown below. 

\begin{center}
\includegraphics[bb=0.00 0.00 298.02 115.01,width=6cm]{Fig/pdtoconsole.pdf}
\end{center}
Also the content of PD is displayed 
via the function \verb|println()| of Cindyscript. 
For instance, inputting the command 
\verb|println(sgABCA)| makes the following list displayed. 
\begin{center}
\verb| [[1,3],[-1,0],[3,0],[1,3]] |
\end{center}
This list is composed of the coordinates of the points A, B, and C. 

These names of PD are used 
when the corresponding PD need to be transformed. 
For instance, 
PD to draw the parallel transport of the segment AB 
is generated via the \ketcindy\ command 
\begin{center}
\verb|Translatedata("1","sgAB",[2,3]);|
\end{center}

PD can be generated also 
by using the programming capability of Cindyscript 
which can be subsequently used in \ketcindy .  
For more details, 
see the example of \verb|Listplot()| 
in the command reference. 
Inclusion of too much elements into a single PD 
may cause some error. 
To prevent such error, 
PD should be divided into several PD 
each of which is composed of 200 elements or so. 


\newpage

\section{Cindyscript}

\subsection{Cindyscript editor}

Choose "Cindyscript" in the "Scripting" menu 
or push keybuttons Ctrl+9 (Windows) / Command+9 (Mac), 
then Cindyscript editor opens as shown below. 

\begin{layer}{150}{0}
\putnotese{7}{15}{\includegraphics[bb=0.00 0.00 703.04 425.02,width=14cm]{Fig/slotE.pdf}}
\arrowlineseg[16]{30}{20}{10}{90}
\putnotese{25}{5}{Slots}
\arrowlineseg[16]{50}{20}{10}{100}
\putnotese{42}{5}{Page name}
\arrowlineseg[16]{90}{20}{10}{110}
\putnotese{80}{5}{Font size}
%\arrowlineseg[16]{107}{20}{15}{140}
%\putnotese{80}{5}{描画面を前面に}
\arrowlineseg[16]{135}{20}{10}{110}
\putnotese{125}{5}{Run}
\arrowlineseg[16]{142}{20}{10}{100}
\putnotese{135}{5}{Help}
\putnotese{100}{35}{Text field}
\putnotese{100}{80}{Console}
\end{layer}

\vspace{105mm}

Commands can be input into preferred "slot". 
Specific timing for execution of commands 
is assigned to each slot. 
The slot for current work can be chosen 
only by clicking the corresponding tab in the menu. 
Users can add extra pages to each slot. 
For instance, 
when some initialization other than 
those included in \verb|KETlib| is needed, 
clicking the folder icon of "Initialization" makes a new page open 
in which extra commands can be input. 
The name of each page can be given 
by directly inputting it into the "Page name" column. 
The font size of the scripts can be tuned 
by changing the number in the "Font size" column. 
Frequently used slots are listed below. 

\begin{itemize}

\item 
Draw

The commnds in this slot are executed 
when some change, like movement of point, 
occurs in the Euclidean view. 
In \verb|templatebasic1.cdy|, 
the protoype page named \verb|figure| 
including the \ketcindy\ commnads 
like \verb|Ketinit();| and \verb|Windispg();| 
which are unconditionally necessary 
has been prepared.  
The \ketcindy\ commands for drawing 
should be input into this slot. 

\item 
Initialization 

The definitions of functions 
and the initial values of variables 
are input here. 
The commands in this slot are exected 
only once just after the "Run" button is clicked. 
Thus, the initial data in this slot is changed 
when some modifications are made in other slots. 
In \verb|templatebasic1.cdy|, 
the protoype page named \verb|KETlib| 
including the default setting of \ketcindy\ 
has been prepared. 

\item 
Key Typed

The commnds in this slot are executed 
when some key is pushed. 

\end{itemize}

Clicking "Run" button or pushing the keybuttons Shift+Enter 
makes the whole program be executed. 
The results derived from executing the function \verb|print()| 
and error messages are displayed on the console view 
which is put at the bottom part of Cindyscript editor. 
Each error and its location 
is displayed together with the message 
"WARNING" or "syntax error". 
The outputs displayed on the console 
can be copied to other usual text editors. 

Click the "Help" button, 
then reference manual of Cinderella opens 
as shown below. \\

\includegraphics[bb=0.00 0.00 712.04 577.03,width=14cm]{Fig/CindyhelpE.pdf}


\subsection{Input}

The attribute of each input into Cindyscript 
is specified via the color of the corresponding letters 
as listed below. 

\begin{itemize}
\item 
The functions which are inherently implemented to Cinderella 
are displayed via blue color. 
\item 
The functions which are defined by user, 
including those of \ketcindy , 
are displayed via purple color. 
\item 
The functions which are not yet defined 
are displayed via red color. 
\item 
Strings are displayed via green color. 
\end{itemize}
As in the console view, 
copying and pasting to the other usual editing software 
via pushing the keybuttons Ctrl+C and Ctrl+V 
is accessible. 
Cutting and pasting via Ctrl+X and Ctrl+V 
is also possible. 
Also as in the other editing software, 
preferred strings can be specified 
via dragging mouse 
or pushing the keybutton Shift and moving the sursor. 
Serching for words via pushing Ctrl+F 
has not been enabled. 

The fundamental rule of describing scripts on Cindyscript editor 
are listed below. 
\begin{itemize}
\item 
Upper- and lowercase letters are distinguished. 
Using lowercase letters is preferable. 
\item 
As in \TeX , 
several blanks are regarded as a single blank. 
\item 
A semicolon should be located at the end of each row. 
Starting a new paragraph 
does not result in the ending of commnds. 
\end{itemize}
Particularly, in case of \ketcindy , 
the input of commands are controlled 
by the following rules. 
\begin{itemize}
\item 
The names of global variables 
begin with uppercase letters. 
\item 
The names of local variables 
begin with lowercase letters. 
Local variables are declared at the beginning part 
of the definitions of functions 
along with the Cinderella command \verb|regional()|. 
\item 
The names of functions 
begin with uppercase letters. 
\end{itemize}


\subsection{Variables and constants}

The declaration of the attribute of each variable 
is not needed in Cindyscript 
since it is automatically decided 
according to the input. 
Moreover, 
the different kind of value 
can be input without any declaration.  \\

\noindent 
{\bf Example}

\verb|          a=10;|

\verb|          b=2;|

\verb|          c=a+sqrt(b);|

\verb|          a="the square root of"|

\verb|          println("The sum of"+a+b+''and 10 is''+c);|\\

In this example, 
the attribute of variable \verb|a| was firstly integer, 
and then changed to string at the fourth row. 

The strings should be input with double quotation marks. 
The mathematical operations 
which involve several kind of variables 
must be taken much care. 
Exceptionally, 
connecting string and number with \verb|+| 
results in the generation of one single string. 

The variable \verb|pi| is reserved in Cindyscript 
as the ratio of the circumference of a circle to its diameter. 
Also the variable \verb|i| is reserved 
as the imaginary unit. 
When \verb|i| is used as variable once, 
it is changed to the imaginary unit via the command 
\begin{center}
\verb|i=complex(0,1);|
\end{center}

There are also some reserved variables in \ketcindy . 
Among them, the following ones can be changed by users. 
\begin{tabbing}
12\=3456789012345\=678989012345678901234567890123\=\kill
\>\verb|Fhead|  \>the beginning part of the file name 
which can be set by  \verb|Setfiles()|\\
\>\verb|Texparent|  \>the name of parent file 
which can be set by \verb|Setparent()|\\
\>\verb|Dirhead|  \>the beginning part of the path\\
\>\verb|Dirlib|  \>the path to the library ketlib\\
\>\verb|Dirbin|  \>the path to ketbin\\
\>\verb|Dirwork|  \>the path to the working directory 
which can be set by \verb|Changework()|\\
\>\verb|Shellfile|  \>the name of shell file
\end{tabbing}
Contrarily, the reserved variables listed below are the global variables 
usend in the library of \ketcindy , whence cannot be changed 
by users. 

\vspace{\baselineskip}
\noindent 
ADDAXES, ArrowlineNumber, ArrowheadNumber, BezierNumber, COM0thlist, COM1stlist, COM2ndlist, Dq, FUNLIST, Fnamesc ,Fnamescibody,Fnameout,Fnametex, GDATALIST, GLIST, GCLIST, GOUTLIST, KCOLOR, KETPICCOUNT,KETPICLAYER, LETTERlist, LFmark, MilliIn, PenThick,PenThickInit,  POUTLIST, SCALEX, SCALEY, SCIRELIST, SCIWRLIST, TenSize, TenSizeInit, ULEN, XMAX, XMIN, YaSize, YaThick,   YMAX, YMIN, VLIST

\subsection{Frequently used commands}

% -------------- Calling other softwares --------------

\newpage

\section{Collaboration with other softwares}

\subsection{Overview}

\ketcindy\ has functionalies to call other softwares such as Maxima, Risa/Asir, R and C.
Here, we introduce how to call Maxima.\vspace{1mm}

The steps are as follows.\vspace{-2mm}

\begin{enumerate}
\item Generate the shell file to call a CAS.\vspace{-2mm}
\item Execute the file.\vspace{-2mm}
\item Return the result as text.\vspace{-2mm}
\item Use the result in \ketcindy .\vspace{-2mm}
\item Produce the PDF file.\vspace{-2mm}
\end{enumerate}

And the flowchart is  as follows:
\begin{center}
{\scalebox{0.9}{%%% /Users/takatoosetsuo/Dropbox/2016ketpic/0801ACA/ACAedutakato/fig/calling.tex 2016-11-24 19:57
%%% calling.sce 2016-11-24 19:57
{\unitlength=1cm%
\begin{picture}%
(   9.78000,   5.50000)(  -4.28000,  -2.50000)%
\linethickness{0.008in}%
%
\linethickness{0.024in}%
\polyline(1.50000,0.00000)(1.50000,0.10000)(1.49508,0.16257)(1.48042,0.22361)(1.45640,0.28160)%
(1.42361,0.33511)(1.38284,0.38284)(1.33511,0.42361)(1.28160,0.45640)(1.22361,0.48042)%
(1.16257,0.49508)(1.10000,0.50000)(0.00000,0.50000)(-1.10000,0.50000)(-1.16257,0.49508)%
(-1.22361,0.48042)(-1.28160,0.45640)(-1.33511,0.42361)(-1.38284,0.38284)(-1.42361,0.33511)%
(-1.45640,0.28160)(-1.48042,0.22361)(-1.49508,0.16257)(-1.50000,0.10000)(-1.50000,0.00000)%
(-1.50000,-0.10000)(-1.49508,-0.16257)(-1.48042,-0.22361)(-1.45640,-0.28160)(-1.42361,-0.33511)%
(-1.38284,-0.38284)(-1.33511,-0.42361)(-1.28160,-0.45640)(-1.22361,-0.48042)(-1.16257,-0.49508)%
(-1.10000,-0.50000)(0.00000,-0.50000)(1.10000,-0.50000)(1.16257,-0.49508)(1.22361,-0.48042)%
(1.28160,-0.45640)(1.33511,-0.42361)(1.38284,-0.38284)(1.42361,-0.33511)(1.45640,-0.28160)%
(1.48042,-0.22361)(1.49508,-0.16257)(1.50000,-0.10000)(1.50000,0.00000)%
%
\linethickness{0.008in}%
\polyline(4.00000,-1.75000)(4.00000,-1.65121)\polyline(3.98695,-1.55357)(3.98042,-1.52639)(3.95640,-1.46840)(3.95218,-1.46152)%
\polyline(3.89496,-1.38135)(3.88284,-1.36716)(3.83511,-1.32639)(3.82031,-1.31732)%
\polyline(3.73227,-1.27317)(3.72361,-1.26958)(3.66257,-1.25492)(3.63601,-1.25283)%
\polyline(3.53733,-1.25000)(3.43853,-1.25000)\polyline(3.33974,-1.25000)(3.24095,-1.25000)%
\polyline(3.14216,-1.25000)(3.04336,-1.25000)\polyline(2.94457,-1.25000)(2.84578,-1.25000)%
\polyline(2.74698,-1.25000)(2.64819,-1.25000)\polyline(2.54940,-1.25000)(2.50000,-1.25000)(2.45060,-1.25000)%
\polyline(2.35181,-1.25000)(2.25302,-1.25000)\polyline(2.15422,-1.25000)(2.05543,-1.25000)%
\polyline(1.95664,-1.25000)(1.85784,-1.25000)\polyline(1.75905,-1.25000)(1.66026,-1.25000)%
\polyline(1.56147,-1.25000)(1.46267,-1.25000)\polyline(1.36399,-1.25283)(1.33743,-1.25492)(1.27639,-1.26958)(1.26773,-1.27317)%
\polyline(1.17969,-1.31732)(1.16489,-1.32639)(1.11716,-1.36716)(1.10504,-1.38135)%
\polyline(1.04782,-1.46152)(1.04360,-1.46840)(1.01958,-1.52639)(1.01305,-1.55357)%
\polyline(1.00000,-1.65121)(1.00000,-1.75000)(1.00000,-1.75000)\polyline(1.00000,-1.84879)(1.00000,-1.85000)(1.00492,-1.91257)(1.01305,-1.94643)%
\polyline(1.04782,-2.03848)(1.07639,-2.08511)(1.10504,-2.11865)\polyline(1.17969,-2.18268)(1.21840,-2.20640)(1.26773,-2.22683)%
\polyline(1.36399,-2.24717)(1.40000,-2.25000)(1.46267,-2.25000)\polyline(1.56147,-2.25000)(1.66026,-2.25000)%
\polyline(1.75905,-2.25000)(1.85784,-2.25000)\polyline(1.95664,-2.25000)(2.05543,-2.25000)%
\polyline(2.15422,-2.25000)(2.25302,-2.25000)\polyline(2.35181,-2.25000)(2.45060,-2.25000)%
\polyline(2.54940,-2.25000)(2.64819,-2.25000)\polyline(2.74698,-2.25000)(2.84578,-2.25000)%
\polyline(2.94457,-2.25000)(3.04336,-2.25000)\polyline(3.14216,-2.25000)(3.24095,-2.25000)%
\polyline(3.33974,-2.25000)(3.43853,-2.25000)\polyline(3.53733,-2.25000)(3.60000,-2.25000)(3.63601,-2.24717)%
\polyline(3.73227,-2.22683)(3.78160,-2.20640)(3.82031,-2.18268)\polyline(3.89496,-2.11865)(3.92361,-2.08511)(3.95218,-2.03848)%
\polyline(3.98695,-1.94643)(3.99508,-1.91257)(4.00000,-1.85000)(4.00000,-1.84879)%
%
%
\polyline(-1.00000,-1.75000)(-1.00000,-1.65121)\polyline(-1.01305,-1.55357)(-1.01958,-1.52639)(-1.04360,-1.46840)(-1.04782,-1.46152)%
\polyline(-1.10504,-1.38135)(-1.11716,-1.36716)(-1.16489,-1.32639)(-1.17969,-1.31732)%
\polyline(-1.26773,-1.27317)(-1.27639,-1.26958)(-1.33743,-1.25492)(-1.36399,-1.25283)%
\polyline(-1.46267,-1.25000)(-1.56147,-1.25000)\polyline(-1.66026,-1.25000)(-1.75905,-1.25000)%
\polyline(-1.85784,-1.25000)(-1.95664,-1.25000)\polyline(-2.05543,-1.25000)(-2.15422,-1.25000)%
\polyline(-2.25302,-1.25000)(-2.35181,-1.25000)\polyline(-2.45060,-1.25000)(-2.50000,-1.25000)(-2.54940,-1.25000)%
\polyline(-2.64819,-1.25000)(-2.74698,-1.25000)\polyline(-2.84578,-1.25000)(-2.94457,-1.25000)%
\polyline(-3.04336,-1.25000)(-3.14216,-1.25000)\polyline(-3.24095,-1.25000)(-3.33974,-1.25000)%
\polyline(-3.43853,-1.25000)(-3.53733,-1.25000)\polyline(-3.63601,-1.25283)(-3.66257,-1.25492)(-3.72361,-1.26958)(-3.73227,-1.27317)%
\polyline(-3.82031,-1.31732)(-3.83511,-1.32639)(-3.88284,-1.36716)(-3.89496,-1.38135)%
\polyline(-3.95218,-1.46152)(-3.95640,-1.46840)(-3.98042,-1.52639)(-3.98695,-1.55357)%
\polyline(-4.00000,-1.65121)(-4.00000,-1.75000)(-4.00000,-1.75000)\polyline(-4.00000,-1.84879)(-4.00000,-1.85000)(-3.99508,-1.91257)(-3.98695,-1.94643)%
\polyline(-3.95218,-2.03848)(-3.92361,-2.08511)(-3.89496,-2.11865)\polyline(-3.82031,-2.18268)(-3.78160,-2.20640)(-3.73227,-2.22683)%
\polyline(-3.63601,-2.24717)(-3.60000,-2.25000)(-3.53733,-2.25000)\polyline(-3.43853,-2.25000)(-3.33974,-2.25000)%
\polyline(-3.24095,-2.25000)(-3.14216,-2.25000)\polyline(-3.04336,-2.25000)(-2.94457,-2.25000)%
\polyline(-2.84578,-2.25000)(-2.74698,-2.25000)\polyline(-2.64819,-2.25000)(-2.54940,-2.25000)%
\polyline(-2.45060,-2.25000)(-2.35181,-2.25000)\polyline(-2.25302,-2.25000)(-2.15422,-2.25000)%
\polyline(-2.05543,-2.25000)(-1.95664,-2.25000)\polyline(-1.85784,-2.25000)(-1.75905,-2.25000)%
\polyline(-1.66026,-2.25000)(-1.56147,-2.25000)\polyline(-1.46267,-2.25000)(-1.40000,-2.25000)(-1.36399,-2.24717)%
\polyline(-1.26773,-2.22683)(-1.21840,-2.20640)(-1.17969,-2.18268)\polyline(-1.10504,-2.11865)(-1.07639,-2.08511)(-1.04782,-2.03848)%
\polyline(-1.01305,-1.94643)(-1.00492,-1.91257)(-1.00000,-1.85000)(-1.00000,-1.84879)%
%
%
\linethickness{0.012in}%
\polyline(1.52000,2.25000)(1.52000,2.29000)(1.51508,2.35257)(1.50042,2.41361)(1.47640,2.47160)%
(1.44361,2.52511)(1.40284,2.57284)(1.35511,2.61361)(1.30160,2.64640)(1.24361,2.67042)%
(1.18257,2.68508)(1.12000,2.69000)(0.00000,2.69000)(-1.12000,2.69000)(-1.18257,2.68508)%
(-1.24361,2.67042)(-1.30160,2.64640)(-1.35511,2.61361)(-1.40284,2.57284)(-1.44361,2.52511)%
(-1.47640,2.47160)(-1.50042,2.41361)(-1.51508,2.35257)(-1.52000,2.29000)(-1.52000,2.25000)%
(-1.52000,2.21000)(-1.51508,2.14743)(-1.50042,2.08639)(-1.47640,2.02840)(-1.44361,1.97489)%
(-1.40284,1.92716)(-1.35511,1.88639)(-1.30160,1.85360)(-1.24361,1.82958)(-1.18257,1.81492)%
(-1.12000,1.81000)(0.00000,1.81000)(1.12000,1.81000)(1.18257,1.81492)(1.24361,1.82958)%
(1.30160,1.85360)(1.35511,1.88639)(1.40284,1.92716)(1.44361,1.97489)(1.47640,2.02840)%
(1.50042,2.08639)(1.51508,2.14743)(1.52000,2.21000)(1.52000,2.25000)%
%
\linethickness{0.008in}%
\settowidth{\Width}{\ketcindy}\setlength{\Width}{-0.5\Width}%
\settoheight{\Height}{\ketcindy}\settodepth{\Depth}{\ketcindy}\setlength{\Height}{-0.5\Height}\setlength{\Depth}{0.5\Depth}\addtolength{\Height}{\Depth}%
\put(0.0000,0.0000){\hspace*{\Width}\raisebox{\Height}{\ketcindy}}%
%
%
\settowidth{\Width}{Scilab}\setlength{\Width}{-0.5\Width}%
\settoheight{\Height}{Scilab}\settodepth{\Depth}{Scilab}\setlength{\Height}{-0.5\Height}\setlength{\Depth}{0.5\Depth}\addtolength{\Height}{\Depth}%
\put(-2.5000,-1.7500){\hspace*{\Width}\raisebox{\Height}{Scilab}}%
%
%
\settowidth{\Width}{\LaTeX}\setlength{\Width}{-0.5\Width}%
\settoheight{\Height}{\LaTeX}\settodepth{\Depth}{\LaTeX}\setlength{\Height}{-0.5\Height}\setlength{\Depth}{0.5\Depth}\addtolength{\Height}{\Depth}%
\put(2.5000,-1.7500){\hspace*{\Width}\raisebox{\Height}{\LaTeX}}%
%
%
\settowidth{\Width}{Maxima}\setlength{\Width}{-0.5\Width}%
\settoheight{\Height}{Maxima}\settodepth{\Depth}{Maxima}\setlength{\Height}{-0.5\Height}\setlength{\Depth}{0.5\Depth}\addtolength{\Height}{\Depth}%
\put(0.0000,2.2500){\hspace*{\Width}\raisebox{\Height}{Maxima}}%
%
%
\settowidth{\Width}{\begin{minipage}{40mm} Source File\\ Batch File\end{minipage}}\setlength{\Width}{0\Width}%
\settoheight{\Height}{\begin{minipage}{40mm} Source File\\ Batch File\end{minipage}}\settodepth{\Depth}{\begin{minipage}{40mm} Source File\\ Batch File\end{minipage}}\setlength{\Height}{-\Height}%
\put(-3.1800,1.3900){\hspace*{\Width}\raisebox{\Height}{\begin{minipage}{40mm} Source File\\ Batch File\end{minipage}}}%
%
%
\settowidth{\Width}{Returned Results (textfile)}\setlength{\Width}{0\Width}%
\settoheight{\Height}{Returned Results (textfile)}\settodepth{\Depth}{Returned Results (textfile)}\setlength{\Height}{-0.5\Height}\setlength{\Depth}{0.5\Depth}\addtolength{\Height}{\Depth}%
\put(0.7200,1.1100){\hspace*{\Width}\raisebox{\Height}{Returned Results (textfile)}}%
%
%
\settowidth{\Width}{Further Use in \ketcindy}\setlength{\Width}{0\Width}%
\settoheight{\Height}{Further Use in \ketcindy}\settodepth{\Depth}{Further Use in \ketcindy}\setlength{\Height}{-0.5\Height}\setlength{\Depth}{0.5\Depth}\addtolength{\Height}{\Depth}%
\put(1.6600,0.1000){\hspace*{\Width}\raisebox{\Height}{Further Use in \ketcindy}}%
%
%
\linethickness{0.024in}%
\polyline(-0.50000,-0.75000)(-0.92929,-1.17929)%
%
\linethickness{0.008in}%
\polygon*(-0.90920,-1.07180)(-1.00000,-1.25000)(-0.82180,-1.15920)(-0.86550,-1.11550)%
(-0.90920,-1.07180)\linethickness{0.001in}%
\polyline(-0.90920,-1.07180)(-1.00000,-1.25000)(-0.82180,-1.15920)(-0.86550,-1.11550)%
(-0.90920,-1.07180)(-1.00000,-1.25000)%
%
\linethickness{0.008in}%
\linethickness{0.024in}%
\polyline(1.05150,-1.21856)(0.57621,-0.81475)%
%
\linethickness{0.008in}%
\polygon*(0.68497,-0.82606)(0.50000,-0.75000)(0.60494,-0.92026)(0.64496,-0.87316)%
(0.68497,-0.82606)\linethickness{0.001in}%
\polyline(0.68497,-0.82606)(0.50000,-0.75000)(0.60494,-0.92026)(0.64496,-0.87316)%
(0.68497,-0.82606)(0.50000,-0.75000)%
%
\linethickness{0.008in}%
\linethickness{0.024in}%
\polyline(-0.75000,-1.75000)(0.58426,-1.76113)%
%
\linethickness{0.008in}%
\polygon*(0.49354,-1.82218)(0.68426,-1.76197)(0.49457,-1.69858)(0.49406,-1.76038)%
(0.49354,-1.82218)\linethickness{0.001in}%
\polyline(0.49354,-1.82218)(0.68426,-1.76197)(0.49457,-1.69858)(0.49406,-1.76038)%
(0.49354,-1.82218)(0.68426,-1.76197)%
%
\linethickness{0.008in}%
\linethickness{0.024in}%
\polyline(-0.20528,0.77111)(-0.19842,1.37117)%
%
\linethickness{0.008in}%
\polygon*(-0.13765,1.28025)(-0.19728,1.47116)(-0.26125,1.28167)(-0.19945,1.28096)%
(-0.13765,1.28025)\linethickness{0.001in}%
\polyline(-0.13765,1.28025)(-0.19728,1.47116)(-0.26125,1.28167)(-0.19945,1.28096)%
(-0.13765,1.28025)(-0.19728,1.47116)%
%
\linethickness{0.008in}%
\linethickness{0.024in}%
\polyline(0.30272,1.47116)(0.29586,0.87110)%
%
\linethickness{0.008in}%
\polygon*(0.23509,0.96202)(0.29472,0.77111)(0.35869,0.96060)(0.29689,0.96131)(0.23509,0.96202)%
\linethickness{0.001in}%
\polyline(0.23509,0.96202)(0.29472,0.77111)(0.35869,0.96060)(0.29689,0.96131)(0.23509,0.96202)%
(0.29472,0.77111)%
%
\linethickness{0.008in}%
\end{picture}}%}}

\end{center}

When interfacing with Maxima, commands \texttt{Mxfun}, \texttt{CalcbyM} and \texttt{Mxtex} are all we need to complete the task. \texttt{Mxfun} and \texttt{CalcbyM} are for calling single command and multi commands of Maxima respectively. \texttt{Mxtex} is used for code conversion to \LaTeX. The output of Maxima  is returned to \ketcindy\ as a string or a list of strings for further processing. 

The options of these commands are:\\
\hspace*{10mm}\Ltab{25mm}{{\tt "m/r"}}To decide whether the result file will be  made again or not. \\
\hspace*{10mm}\Ltab{25mm}{}If these options are not given, \ketcindy\ decides automatically.\\
\hspace*{10mm}\Ltab{25mm}{{\tt "Disp=y/n"}}To decide whether the result will be displayed in the console or not. \\
\hspace*{10mm}\Ltab{25mm}{}It is only availabe for \texttt{Mxfun} and \texttt{Mxtex}.
The default is "y".


\subsection{Commands related to Maxima}

\subsubsection*{Mxfun}

The arguments are name of variable in \ketcindy, name of a function of Maxima, and a list of arguments of the function.\\
\hspace*{10mm}\verb|Mxfun("1","diff",["sin(x)","x"]);|  // The return is "cos(x)", assgined to mx1.\\
The above is equivallent to\\
\hspace*{10mm}\verb|Mxfun("1","diff(sin(x),x)",[]);|


\subsubsection*{Mxtex}

The arguments are name of variable in \ketcindy, an expression in Maxima format.\\
\hspace*{10mm}\verb|Mxtex("1",mx1);|  // The return is \verb|"\cos x"|, assgined to tx1.\\
\hspace*{10mm}\verb|Expr([0,1],"e",tx1]);|

\begin{center}
%%% /Users/takatoosetsuo/Dropbox/kettoday/0826/fig/maxima.tex 
%%% Generator=maxima.cdy 
{\unitlength=1cm%
\begin{picture}%
(10,2)(-5,-0.5)%
\special{pn 8}%
%
{%
\color[rgb]{0,0,0}%
\settowidth{\Width}{$\cos x$}\setlength{\Width}{0\Width}%
\settoheight{\Height}{$\cos x$}\settodepth{\Depth}{$\cos x$}\setlength{\Height}{-0.5\Height}\setlength{\Depth}{0.5\Depth}\addtolength{\Height}{\Depth}%
\put(1.0500000,1.0000000){\hspace*{\Width}\raisebox{\Height}{$\cos x$}}%
%
}%
{%
\color[rgb]{0,0,0}%
\special{pa   394   -20}\special{pa   394    20}%
\special{fp}%
}%
{%
\color[rgb]{0,0,0}%
\settowidth{\Width}{$1$}\setlength{\Width}{-0.5\Width}%
\settoheight{\Height}{$1$}\settodepth{\Depth}{$1$}\setlength{\Height}{-\Height}%
\put(1.0000000,-0.1000000){\hspace*{\Width}\raisebox{\Height}{$1$}}%
%
}%
{%
\color[rgb]{0,0,0}%
\special{pa    20  -394}\special{pa   -20  -394}%
\special{fp}%
}%
{%
\color[rgb]{0,0,0}%
\settowidth{\Width}{$1$}\setlength{\Width}{-1\Width}%
\settoheight{\Height}{$1$}\settodepth{\Depth}{$1$}\setlength{\Height}{-0.5\Height}\setlength{\Depth}{0.5\Depth}\addtolength{\Height}{\Depth}%
\put(-0.1000000,1.0000000){\hspace*{\Width}\raisebox{\Height}{$1$}}%
%
}%
\special{pa -1969    -0}\special{pa  1969    -0}%
\special{fp}%
\special{pa     0   197}\special{pa     0  -591}%
\special{fp}%
\settowidth{\Width}{$x$}\setlength{\Width}{0\Width}%
\settoheight{\Height}{$x$}\settodepth{\Depth}{$x$}\setlength{\Height}{-0.5\Height}\setlength{\Depth}{0.5\Depth}\addtolength{\Height}{\Depth}%
\put(5.0500000,0.0000000){\hspace*{\Width}\raisebox{\Height}{$x$}}%
%
\settowidth{\Width}{$y$}\setlength{\Width}{-0.5\Width}%
\settoheight{\Height}{$y$}\settodepth{\Depth}{$y$}\setlength{\Height}{\Depth}%
\put(0.0000000,1.5500000){\hspace*{\Width}\raisebox{\Height}{$y$}}%
%
\settowidth{\Width}{O}\setlength{\Width}{-1\Width}%
\settoheight{\Height}{O}\settodepth{\Depth}{O}\setlength{\Height}{-\Height}%
\put(-0.0500000,-0.0500000){\hspace*{\Width}\raisebox{\Height}{O}}%
%
\end{picture}}%
\end{center}

\subsubsection*{CalcbyM}

The arguments are name of variable in \ketcindy, a list of commands and the arguments of Maxima.\\
\hspace*{10mm}\verb|fn="sin(x)^4";|\\
\hspace*{10mm}\verb|cmdL=[|\\
\hspace*{10mm}\verb|  "df:diff",[fn,"x"],|\\
\hspace*{10mm}\verb|  "df:ratsimp",["df"],|\\
\hspace*{10mm}\verb|  "F:integrate",[fn,"x"],|\\
\hspace*{10mm}\verb|  "F","ratsimp",["F"],|\\
\hspace*{10mm}\verb|  "df::F",[]|\\
\hspace*{10mm}\verb|];|\\
\hspace*{10mm}\verb|CalcbyM("ans",cmdL);|\vspace{2mm}\\
The returned value is a list of df and F as strings, though these are not displayed in the console. They can be used, for example,\vspace{2mm}\\
\hspace*{10mm}\verb|Plotdata("1",fn,"x",["Num=200","do"]);|\\
\hspace*{10mm}\verb|Plotdata("2",ans_1,"x",["Num=200","dr"]);|\\
\hspace*{10mm}\verb|Plotdata("3",ans_2,"x",["Num=200","da"]);|\\
\hspace*{10mm}\verb|Mxtex("1",fn);|\\
\hspace*{10mm}\verb|Mxtex("1",ans_1);|\\
\hspace*{10mm}\verb|Mxtex("2",ans_2);|\\
\hspace*{10mm}\verb|Expr([A,"e",tx1,B,"e",tx2,C,"w",tx3]);|

\vspace{2mm}

\begin{center}
%%% /Users/takatoosetsuo/Dropbox/2016ketpic/0801ACA/ACAedutakato/fig/s10diffint.tex 2016-11-22 14:39
%%% s10diffint.sce 2016-11-22 14:39
{\unitlength=6mm%
\begin{picture}%
(  14.00000,   6.00000)(  -7.00000,  -3.00000)%
\linethickness{0.008in}%
%
\put(-7.00000,0.18631){\circle*{0.033867}}\put(-6.86435,0.09120){\circle*{0.033867}}%
\put(-6.71067,0.02991){\circle*{0.033867}}\put(-6.54691,0.00506){\circle*{0.033867}}%
\put(-6.38112,0.00015){\circle*{0.033867}}\put(-6.21521,0.00005){\circle*{0.033867}}%
\put(-6.04937,0.00325){\circle*{0.033867}}\put(-5.88486,0.02306){\circle*{0.033867}}%
\put(-5.72855,0.07727){\circle*{0.033867}}\put(-5.58956,0.16732){\circle*{0.033867}}%
\put(-5.46869,0.28078){\circle*{0.033867}}\put(-5.35958,0.40566){\circle*{0.033867}}%
\put(-5.25668,0.53579){\circle*{0.033867}}\put(-5.15473,0.66667){\circle*{0.033867}}%
\put(-5.04816,0.79376){\circle*{0.033867}}\put(-4.92850,0.90824){\circle*{0.033867}}%
\put(-4.78443,0.98726){\circle*{0.033867}}\put(-4.62201,0.98224){\circle*{0.033867}}%
\put(-4.48126,0.89658){\circle*{0.033867}}\put(-4.36410,0.77946){\circle*{0.033867}}%
\put(-4.25810,0.65189){\circle*{0.033867}}\put(-4.15645,0.52077){\circle*{0.033867}}%
\put(-4.05329,0.39085){\circle*{0.033867}}\put(-3.94277,0.26722){\circle*{0.033867}}%
\put(-3.81998,0.15598){\circle*{0.033867}}\put(-3.67893,0.06945){\circle*{0.033867}}%
\put(-3.52122,0.01967){\circle*{0.033867}}\put(-3.35643,0.00230){\circle*{0.033867}}%
\put(-3.19056,0.00001){\circle*{0.033867}}\put(-3.02466,0.00029){\circle*{0.033867}}%
\put(-2.85891,0.00627){\circle*{0.033867}}\put(-2.69587,0.03518){\circle*{0.033867}}%
\put(-2.54430,0.10126){\circle*{0.033867}}\put(-2.41092,0.19931){\circle*{0.033867}}%
\put(-2.29373,0.31660){\circle*{0.033867}}\put(-2.18723,0.44376){\circle*{0.033867}}%
\put(-2.08515,0.57453){\circle*{0.033867}}\put(-1.98241,0.70478){\circle*{0.033867}}%
\put(-1.87335,0.82973){\circle*{0.033867}}\put(-1.74755,0.93726){\circle*{0.033867}}%
\put(-1.59510,0.99741){\circle*{0.033867}}\put(-1.43551,0.96320){\circle*{0.033867}}%
\put(-1.30272,0.86449){\circle*{0.033867}}\put(-1.19048,0.74243){\circle*{0.033867}}%
\put(-1.08639,0.61328){\circle*{0.033867}}\put(-0.98474,0.48216){\circle*{0.033867}}%
\put(-0.88030,0.35328){\circle*{0.033867}}\put(-0.76650,0.23267){\circle*{0.033867}}%
\put(-0.63855,0.12756){\circle*{0.033867}}\put(-0.49203,0.05097){\circle*{0.033867}}%
\put(-0.33133,0.01168){\circle*{0.033867}}\put(-0.16590,0.00082){\circle*{0.033867}}%
\put(-0.00000,0.00000){\circle*{0.033867}}\put(0.16590,0.00082){\circle*{0.033867}}%
\put(0.33133,0.01168){\circle*{0.033867}}\put(0.49203,0.05097){\circle*{0.033867}}%
\put(0.63855,0.12756){\circle*{0.033867}}\put(0.76650,0.23267){\circle*{0.033867}}%
\put(0.88030,0.35328){\circle*{0.033867}}\put(0.98474,0.48216){\circle*{0.033867}}%
\put(1.08639,0.61328){\circle*{0.033867}}\put(1.19048,0.74243){\circle*{0.033867}}%
\put(1.30272,0.86449){\circle*{0.033867}}\put(1.43551,0.96320){\circle*{0.033867}}%
\put(1.59510,0.99741){\circle*{0.033867}}\put(1.74755,0.93726){\circle*{0.033867}}%
\put(1.87335,0.82973){\circle*{0.033867}}\put(1.98241,0.70478){\circle*{0.033867}}%
\put(2.08515,0.57453){\circle*{0.033867}}\put(2.18723,0.44376){\circle*{0.033867}}%
\put(2.29373,0.31660){\circle*{0.033867}}\put(2.41092,0.19931){\circle*{0.033867}}%
\put(2.54430,0.10126){\circle*{0.033867}}\put(2.69587,0.03518){\circle*{0.033867}}%
\put(2.85891,0.00627){\circle*{0.033867}}\put(3.02466,0.00029){\circle*{0.033867}}%
\put(3.19056,0.00001){\circle*{0.033867}}\put(3.35643,0.00230){\circle*{0.033867}}%
\put(3.52122,0.01967){\circle*{0.033867}}\put(3.67893,0.06945){\circle*{0.033867}}%
\put(3.81998,0.15598){\circle*{0.033867}}\put(3.94277,0.26722){\circle*{0.033867}}%
\put(4.05329,0.39085){\circle*{0.033867}}\put(4.15645,0.52077){\circle*{0.033867}}%
\put(4.25810,0.65189){\circle*{0.033867}}\put(4.36410,0.77946){\circle*{0.033867}}%
\put(4.48126,0.89658){\circle*{0.033867}}\put(4.62201,0.98224){\circle*{0.033867}}%
\put(4.78443,0.98726){\circle*{0.033867}}\put(4.92850,0.90824){\circle*{0.033867}}%
\put(5.04816,0.79376){\circle*{0.033867}}\put(5.15473,0.66667){\circle*{0.033867}}%
\put(5.25668,0.53579){\circle*{0.033867}}\put(5.35958,0.40566){\circle*{0.033867}}%
\put(5.46869,0.28078){\circle*{0.033867}}\put(5.58956,0.16732){\circle*{0.033867}}%
\put(5.72855,0.07727){\circle*{0.033867}}\put(5.88486,0.02306){\circle*{0.033867}}%
\put(6.04937,0.00325){\circle*{0.033867}}\put(6.21521,0.00005){\circle*{0.033867}}%
\put(6.38112,0.00015){\circle*{0.033867}}\put(6.54691,0.00506){\circle*{0.033867}}%
\put(6.71067,0.02991){\circle*{0.033867}}\put(6.86435,0.09120){\circle*{0.033867}}%
\put(7.00000,0.18631){\circle*{0.033867}}%
\polyline(-7.00000,-0.85515)(-6.92965,-0.69786)(-6.85930,-0.54230)(-6.78894,-0.39790)%
(-6.71859,-0.27213)(-6.64824,-0.17001)(-6.57789,-0.09379)(-6.50754,-0.04294)(-6.43719,-0.01427)%
(-6.36683,-0.00232)(-6.29648,-0.00001)(-6.22613,0.00074)(-6.15578,0.00814)(-6.08543,0.02974)%
(-6.01508,0.07171)(-5.94472,0.13812)(-5.87437,0.23056)(-5.80402,0.34784)(-5.73367,0.48595)%
(-5.66332,0.63830)(-5.59296,0.79615)(-5.52261,0.94920)(-5.45226,1.08641)(-5.38191,1.19681)%
(-5.31156,1.27039)(-5.24121,1.29890)(-5.17085,1.27654)(-5.10050,1.20051)(-5.03015,1.07130)%
(-4.95980,0.89279)(-4.88945,0.67201)(-4.81910,0.41879)(-4.74874,0.14510)(-4.67839,-0.13573)%
(-4.60804,-0.40988)(-4.53769,-0.66399)(-4.46734,-0.88603)(-4.39698,-1.06609)(-4.32663,-1.19703)%
(-4.25628,-1.27486)(-4.18593,-1.29900)(-4.11558,-1.27212)(-4.04523,-1.19992)(-3.97487,-1.09060)%
(-3.90452,-0.95411)(-3.83417,-0.80141)(-3.76382,-0.64356)(-3.69347,-0.49087)(-3.62312,-0.35216)%
(-3.55276,-0.23411)(-3.48241,-0.14079)(-3.41206,-0.07352)(-3.34171,-0.03079)(-3.27136,-0.00859)%
(-3.20101,-0.00084)(-3.13065,0.00001)(-3.06030,0.00213)(-2.98995,0.01363)(-2.91960,0.04165)%
(-2.84925,0.09169)(-2.77889,0.16703)(-2.70854,0.26831)(-2.63819,0.39335)(-2.56784,0.53723)%
(-2.49749,0.69256)(-2.42714,0.84997)(-2.35678,0.99882)(-2.28643,1.12800)(-2.21608,1.22678)%
(-2.14573,1.28565)(-2.07538,1.29715)(-2.00503,1.25645)(-1.93467,1.16188)(-1.86432,1.01508)%
(-1.79397,0.82106)(-1.72362,0.58787)(-1.65327,0.32616)(-1.58291,0.04846)(-1.51256,-0.23162)%
(-1.44221,-0.50033)(-1.37186,-0.74465)(-1.30151,-0.95322)(-1.23116,-1.11702)(-1.16080,-1.22994)%
(-1.09045,-1.28917)(-1.02010,-1.29527)(-0.94975,-1.25199)(-0.87940,-1.16598)(-0.80905,-1.04611)%
(-0.73869,-0.90277)(-0.66834,-0.74707)(-0.59799,-0.58989)(-0.52764,-0.44112)(-0.45729,-0.30890)%
(-0.38693,-0.19904)(-0.31658,-0.11470)(-0.24623,-0.05618)(-0.17588,-0.02110)(-0.10553,-0.00465)%
(-0.03518,-0.00017)(0.03518,0.00017)(0.10553,0.00465)(0.17588,0.02110)(0.24623,0.05618)%
(0.31658,0.11470)(0.38693,0.19904)(0.45729,0.30890)(0.52764,0.44112)(0.59799,0.58989)%
(0.66834,0.74707)(0.73869,0.90277)(0.80905,1.04611)(0.87940,1.16598)(0.94975,1.25199)%
(1.02010,1.29527)(1.09045,1.28917)(1.16080,1.22994)(1.23116,1.11702)(1.30151,0.95322)%
(1.37186,0.74465)(1.44221,0.50033)(1.51256,0.23162)(1.58291,-0.04846)(1.65327,-0.32616)%
(1.72362,-0.58787)(1.79397,-0.82106)(1.86432,-1.01508)(1.93467,-1.16188)(2.00503,-1.25645)%
(2.07538,-1.29715)(2.14573,-1.28565)(2.21608,-1.22678)(2.28643,-1.12800)(2.35678,-0.99882)%
(2.42714,-0.84997)(2.49749,-0.69256)(2.56784,-0.53723)(2.63819,-0.39335)(2.70854,-0.26831)%
(2.77889,-0.16703)(2.84925,-0.09169)(2.91960,-0.04165)(2.98995,-0.01363)(3.06030,-0.00213)%
(3.13065,-0.00001)(3.20101,0.00084)(3.27136,0.00859)(3.34171,0.03079)(3.41206,0.07352)%
(3.48241,0.14079)(3.55276,0.23411)(3.62312,0.35216)(3.69347,0.49087)(3.76382,0.64356)%
(3.83417,0.80141)(3.90452,0.95411)(3.97487,1.09060)(4.04523,1.19992)(4.11558,1.27212)%
(4.18593,1.29900)(4.25628,1.27486)(4.32663,1.19703)(4.39698,1.06609)(4.46734,0.88603)%
(4.53769,0.66399)(4.60804,0.40988)(4.67839,0.13573)(4.74874,-0.14510)(4.81910,-0.41879)%
(4.88945,-0.67201)(4.95980,-0.89279)(5.03015,-1.07130)(5.10050,-1.20051)(5.17085,-1.27654)%
(5.24121,-1.29890)(5.31156,-1.27039)(5.38191,-1.19681)(5.45226,-1.08641)(5.52261,-0.94920)%
(5.59296,-0.79615)(5.66332,-0.63830)(5.73367,-0.48595)(5.80402,-0.34784)(5.87437,-0.23056)%
(5.94472,-0.13812)(6.01508,-0.07171)(6.08543,-0.02974)(6.15578,-0.00814)(6.22613,-0.00074)%
(6.29648,0.00001)(6.36683,0.00232)(6.43719,0.01427)(6.50754,0.04294)(6.57789,0.09379)%
(6.64824,0.17001)(6.71859,0.27213)(6.78894,0.39790)(6.85930,0.54230)(6.92965,0.69786)%
(7.00000,0.85515)%
%
\linethickness{0.008in}%
\polyline(-7.00000,-2.38581)(-6.92965,-2.37469)(-6.85930,-2.36703)(-6.83473,-2.36529)%
\polyline(-6.66831,-2.35788)(-6.64824,-2.35741)(-6.57789,-2.35662)(-6.50754,-2.35631)(-6.50170,-2.35630)%
\polyline(-6.33509,-2.35619)(-6.29648,-2.35619)(-6.22613,-2.35619)(-6.16847,-2.35619)%
\polyline(-6.00186,-2.35582)(-5.94472,-2.35535)(-5.87437,-2.35409)(-5.83528,-2.35274)%
\polyline(-5.66908,-2.34149)(-5.66332,-2.34096)(-5.59296,-2.33123)(-5.52261,-2.31757)(-5.50506,-2.31299)%
\polyline(-5.34707,-2.26083)(-5.31156,-2.24588)(-5.24121,-2.20998)(-5.19912,-2.18468)%
\polyline(-5.06246,-2.08960)(-5.03015,-2.06456)(-4.95980,-2.00476)(-4.93512,-1.98224)%
\polyline(-4.81392,-1.86793)(-4.74874,-1.80347)(-4.69573,-1.75050)\polyline(-4.57694,-1.63367)(-4.53769,-1.59595)(-4.46734,-1.53162)(-4.45452,-1.52069)%
\polyline(-4.32507,-1.41586)(-4.25628,-1.36814)(-4.18593,-1.32562)(-4.18531,-1.32530)%
\polyline(-4.03403,-1.25588)(-3.97487,-1.23577)(-3.90452,-1.21725)(-3.87373,-1.21121)%
\polyline(-3.70879,-1.18840)(-3.69347,-1.18695)(-3.62312,-1.18273)(-3.55276,-1.18027)(-3.54242,-1.18007)%
\polyline(-3.37581,-1.17827)(-3.34171,-1.17816)(-3.27136,-1.17810)(-3.20920,-1.17810)%
\polyline(-3.04258,-1.17809)(-2.98995,-1.17808)(-2.91960,-1.17799)(-2.87597,-1.17780)%
\polyline(-2.70937,-1.17533)(-2.70854,-1.17531)(-2.63819,-1.17237)(-2.56784,-1.16747)(-2.54313,-1.16481)%
\polyline(-2.37875,-1.13846)(-2.35678,-1.13372)(-2.28643,-1.11359)(-2.21977,-1.08924)%
\polyline(-2.07073,-1.01510)(-2.00503,-0.97354)(-1.93467,-0.92284)(-1.93299,-0.92149)%
\polyline(-1.80512,-0.81473)(-1.79397,-0.80499)(-1.72362,-0.73952)(-1.68367,-0.70070)%
\polyline(-1.56523,-0.58351)(-1.51256,-0.53095)(-1.44677,-0.46635)\polyline(-1.32465,-0.35302)(-1.30151,-0.33232)(-1.23116,-0.27408)(-1.19576,-0.24755)%
\polyline(-1.05705,-0.15553)(-1.02010,-0.13436)(-0.94975,-0.10044)(-0.90702,-0.08359)%
\polyline(-0.74756,-0.03602)(-0.73869,-0.03390)(-0.66834,-0.02155)(-0.59799,-0.01290)(-0.58285,-0.01166)%
\polyline(-0.41652,-0.00246)(-0.38693,-0.00162)(-0.31658,-0.00061)(-0.24992,-0.00020)%
\polyline(-0.08331,-0.00000)(-0.03518,-0.00000)(0.03518,0.00000)(0.08331,0.00000)%
\polyline(0.24992,0.00020)(0.31658,0.00061)(0.38693,0.00162)(0.41652,0.00246)\polyline(0.58285,0.01166)(0.59799,0.01290)(0.66834,0.02155)(0.73869,0.03390)(0.74756,0.03602)%
\polyline(0.90702,0.08359)(0.94975,0.10044)(1.02010,0.13436)(1.05705,0.15553)\polyline(1.19576,0.24755)(1.23116,0.27408)(1.30151,0.33232)(1.32465,0.35302)%
\polyline(1.44677,0.46635)(1.51256,0.53095)(1.56523,0.58351)\polyline(1.68367,0.70070)(1.72362,0.73952)(1.79397,0.80499)(1.80512,0.81473)%
\polyline(1.93299,0.92149)(1.93467,0.92284)(2.00503,0.97354)(2.07073,1.01510)\polyline(2.21977,1.08924)(2.28643,1.11359)(2.35678,1.13372)(2.37875,1.13846)%
\polyline(2.54313,1.16481)(2.56784,1.16747)(2.63819,1.17237)(2.70854,1.17531)(2.70937,1.17533)%
\polyline(2.87597,1.17780)(2.91960,1.17799)(2.98995,1.17808)(3.04258,1.17809)\polyline(3.20920,1.17810)(3.27136,1.17810)(3.34171,1.17816)(3.37581,1.17827)%
\polyline(3.54242,1.18007)(3.55276,1.18027)(3.62312,1.18273)(3.69347,1.18695)(3.70879,1.18840)%
\polyline(3.87373,1.21121)(3.90452,1.21725)(3.97487,1.23577)(4.03403,1.25588)\polyline(4.18531,1.32530)(4.18593,1.32562)(4.25628,1.36814)(4.32507,1.41586)%
\polyline(4.45452,1.52069)(4.46734,1.53162)(4.53769,1.59595)(4.57694,1.63367)\polyline(4.69573,1.75050)(4.74874,1.80347)(4.81392,1.86793)%
\polyline(4.93512,1.98224)(4.95980,2.00476)(5.03015,2.06456)(5.06246,2.08960)\polyline(5.19912,2.18468)(5.24121,2.20998)(5.31156,2.24588)(5.34707,2.26083)%
\polyline(5.50506,2.31299)(5.52261,2.31757)(5.59296,2.33123)(5.66332,2.34096)(5.66908,2.34149)%
\polyline(5.83528,2.35274)(5.87437,2.35409)(5.94472,2.35535)(6.00186,2.35582)\polyline(6.16847,2.35619)(6.22613,2.35619)(6.29648,2.35619)(6.33509,2.35619)%
\polyline(6.50170,2.35630)(6.50754,2.35631)(6.57789,2.35662)(6.64824,2.35741)(6.66831,2.35788)%
\polyline(6.83473,2.36529)(6.85930,2.36703)(6.92965,2.37469)(7.00000,2.38581)%
%
\settowidth{\Width}{$\sin ^4x$}\setlength{\Width}{0\Width}%
\settoheight{\Height}{$\sin ^4x$}\settodepth{\Depth}{$\sin ^4x$}\setlength{\Height}{-0.5\Height}\setlength{\Depth}{0.5\Depth}\addtolength{\Height}{\Depth}%
\put(5.2100,1.2100){\hspace*{\Width}\raisebox{\Height}{$\sin ^4x$}}%
%
%
\settowidth{\Width}{$4\,\cos x\,\sin ^3x$}\setlength{\Width}{0\Width}%
\settoheight{\Height}{$4\,\cos x\,\sin ^3x$}\settodepth{\Depth}{$4\,\cos x\,\sin ^3x$}\setlength{\Height}{-0.5\Height}\setlength{\Depth}{0.5\Depth}\addtolength{\Height}{\Depth}%
\put(5.2400,-1.8300){\hspace*{\Width}\raisebox{\Height}{$4\,\cos x\,\sin ^3x$}}%
%
%
\settowidth{\Width}{$\frac{\sin \left(4\,x\right)-8\,\sin \left(2\,x\right)+12\,x}{32}$}\setlength{\Width}{-1\Width}%
\settoheight{\Height}{$\frac{\sin \left(4\,x\right)-8\,\sin \left(2\,x\right)+12\,x}{32}$}\settodepth{\Depth}{$\frac{\sin \left(4\,x\right)-8\,\sin \left(2\,x\right)+12\,x}{32}$}\setlength{\Height}{-0.5\Height}\setlength{\Depth}{0.5\Depth}\addtolength{\Height}{\Depth}%
\put(5.5000,2.7300){\hspace*{\Width}\raisebox{\Height}{$\frac{\sin \left(4\,x\right)-8\,\sin \left(2\,x\right)+12\,x}{32}$}}%
%
%
\polyline(-7.00000,0.00000)(7.00000,0.00000)%
%
\linethickness{0.008in}%
\polyline(0.00000,-3.00000)(0.00000,3.00000)%
%
\linethickness{0.008in}%
\settowidth{\Width}{$x$}\setlength{\Width}{0\Width}%
\settoheight{\Height}{$x$}\settodepth{\Depth}{$x$}\setlength{\Height}{-0.5\Height}\setlength{\Depth}{0.5\Depth}\addtolength{\Height}{\Depth}%
\put(7.0500,0.0000){\hspace*{\Width}\raisebox{\Height}{$x$}}%
%
%
\settowidth{\Width}{$y$}\setlength{\Width}{-0.5\Width}%
\settoheight{\Height}{$y$}\settodepth{\Depth}{$y$}\setlength{\Height}{\Depth}%
\put(0.0000,3.0500){\hspace*{\Width}\raisebox{\Height}{$y$}}%
%
%
\settowidth{\Width}{O}\setlength{\Width}{-1\Width}%
\settoheight{\Height}{O}\settodepth{\Depth}{O}\setlength{\Height}{-\Height}%
\put(-0.0500,-0.0500){\hspace*{\Width}\raisebox{\Height}{O}}%
%
%
\end{picture}}%
\end{center}

\noindent
{\bf Remark} See KeTCindyreferenceE.pdf for more information.

\subsection{Commands related to R}

\verb|Rfun| and \verb|CalcbyR| are simillar to \verb|Mxfun| and \verb|CalcbyM|.\\
 See \verb|KeTCindyreferenceE.pdf| or \verb|samples/s08R| for more information.

% -------------- 3d figures --------------

\newpage

\section{Three Dimentional figures of \ketcindy}

\subsection{Summary and Geometric Elements}

In KeTCindy's 3D-mode, there are two rectangular areas surrounded by a white frame on the Euclidean view.

The main area on the left side of the screen is simillar to that of two dimentional figures. Figures in this area will be drawn to the \TeX\ document. The view direction can be moved with sliders under the main area.  \verb|TH| and \verb|FI| mean angles $\theta$ and $\varphi$ respectively, which are polar cocordinates of the view direction.

Figures from the view direction $(0,\ \varphi)$ are displayed in the sub area on the right side.

 \hspace{40mm} mainarea \hspace{40mm} subarea
\begin{center}
\includegraphics[bb=0.00 0.00 863.04 378.52,width=12cm]{Fig/3dscreen.pdf}
\end{center}

With internal command \verb|Ptseg3data| which is called in \verb|Start3d|, 
a point put to the main area with the drawing tool of Cinderella is regarded as a 3D point by \ketcindy, and a correspoinding point is put in the sub area. 
Though the initial coordinate of $z$ is 0, we can change it moving the point in the sub area.

For example, if we put point \verb|A| on the main area, point \verb|Az| will be put in the sub area and the 3D coordinates calculated from \verb|A| and \verb|Az|  are assigned to varible \verb|A3d|.\vspace{-1mm}

\begin{description}
\item[\bf Remark]Note that point \verb|Az| will not be deleted automatically even if point \verb|A| is deleted. We should delete it manually.
\end{description}

Geometric segment in the main area generates the corresponding geometric segment in the sub area as well.

\subsection{Lines and Curves}

\ketcindy\ commands \verb|Spaceline| and \verb|Spacecurve| are used do draw lines and curves in the space. Additionally, \verb|Xyzax3data| is used to draw axis.

\begin{description}
\item[Examples]\mbox{}\\
\verb|Xyzax3data("","x=[-5,5]","y=[-5,5]","z=[-5,5]");|\\
\verb|Spacecurve("1","3*[cos(t),sin(t),0.1*t]","t=[0,4*pi]",["Num=200"]);|\\
\verb|pt1=[3,0,0]; pt2=[3,0,3*0.1*4*pi];|\\
\verb|Spaceline("1",[pt1,pt2]);|\\
\verb|Skeletonparadata("1");| 

\item[Remark]The last command skeleton elimination is for skeleton elimination.
Compare two figures below. The right one is with skeleton elimination.

\end{description}

\begin{center}
\input{Fig/spacelinecurve1.tex}\hspace{10mm}%%% /Users/takatoosetsuo/Dropbox/kettoday/0905/fig/spacelinecurve2.tex 
%%% Generator=spacelinecurve.cdy 
{\unitlength=6mm%
\begin{picture}%
(10.11,8.8)(-5,-4)%
\special{pn 8}%
%
\settowidth{\Width}{$x$}\setlength{\Width}{-0.5\Width}%
\settoheight{\Height}{$x$}\settodepth{\Depth}{$x$}\setlength{\Height}{-0.5\Height}\setlength{\Depth}{0.5\Depth}\addtolength{\Height}{\Depth}%
\put(-2.7100000,-3.0200000){\hspace*{\Width}\raisebox{\Height}{$x$}}%
%
\settowidth{\Width}{$y$}\setlength{\Width}{-0.5\Width}%
\settoheight{\Height}{$y$}\settodepth{\Depth}{$y$}\setlength{\Height}{-0.5\Height}\setlength{\Depth}{0.5\Depth}\addtolength{\Height}{\Depth}%
\put(4.6300000,-1.7200000){\hspace*{\Width}\raisebox{\Height}{$y$}}%
%
\settowidth{\Width}{$z$}\setlength{\Width}{-0.5\Width}%
\settoheight{\Height}{$z$}\settodepth{\Depth}{$z$}\setlength{\Height}{-0.5\Height}\setlength{\Depth}{0.5\Depth}\addtolength{\Height}{\Depth}%
\put(0.0000000,4.1500000){\hspace*{\Width}\raisebox{\Height}{$z$}}%
%
\special{pa   591  -657}\special{pa   497  -554}%
\special{fp}%
\special{pa   416  -463}\special{pa   -29    32}%
\special{fp}%
\special{pa  -131   146}\special{pa  -591   657}%
\special{fp}%
\special{pa -1023  -380}\special{pa  -770  -286}%
\special{fp}%
\special{pa  -645  -239}\special{pa  -414  -154}%
\special{fp}%
\special{pa  -295  -110}\special{pa    79    29}%
\special{fp}%
\special{pa   248    92}\special{pa   492   182}%
\special{fp}%
\special{pa   617   229}\special{pa  1023   380}%
\special{fp}%
\special{pa     0   905}\special{pa     0   486}%
\special{fp}%
\special{pa     0   367}\special{pa     0   146}%
\special{fp}%
\special{pa     0    26}\special{pa     0  -905}%
\special{fp}%
\special{pa  -354   394}\special{pa  -315   405}\special{pa  -275   413}\special{pa  -233   420}%
\special{pa  -191   425}\special{pa  -147   429}\special{pa  -104   430}\special{pa   -59   430}%
\special{pa   -15   428}\special{pa    30   424}\special{pa    74   419}\special{pa   118   412}%
\special{pa   162   403}\special{pa   205   392}\special{pa   247   379}\special{pa   288   365}%
\special{pa   328   349}\special{pa   367   332}\special{pa   404   313}\special{pa   440   292}%
\special{pa   474   270}\special{pa   506   247}\special{pa   536   223}\special{pa   564   197}%
\special{pa   590   170}\special{pa   614   142}\special{pa   635   114}\special{pa   653    84}%
\special{pa   669    54}\special{pa   683    24}\special{pa   693    -8}\special{pa   701   -39}%
\special{pa   706   -71}\special{pa   709  -103}\special{pa   708  -135}\special{pa   705  -167}%
\special{pa   699  -199}\special{pa   694  -217}%
\special{fp}%
\special{pa   631  -346}\special{pa   628  -351}\special{pa   606  -379}\special{pa   582  -407}%
\special{pa   555  -433}\special{pa   527  -458}\special{pa   496  -482}\special{pa   463  -505}%
\special{pa   428  -527}\special{pa   392  -547}\special{pa   354  -565}\special{pa   315  -582}%
\special{pa   275  -597}\special{pa   233  -611}\special{pa   191  -623}\special{pa   147  -633}%
\special{pa   104  -642}\special{pa    59  -648}\special{pa    59  -648}%
\special{fp}%
\special{pa   -59  -657}\special{pa   -74  -658}\special{pa  -118  -657}\special{pa  -162  -655}%
\special{pa  -205  -651}\special{pa  -247  -645}\special{pa  -288  -638}\special{pa  -328  -629}%
\special{pa  -367  -618}\special{pa  -404  -606}\special{pa  -440  -592}\special{pa  -474  -577}%
\special{pa  -506  -561}\special{pa  -536  -543}\special{pa  -564  -524}\special{pa  -590  -504}%
\special{pa  -598  -497}%
\special{fp}%
\special{pa  -675  -406}\special{pa  -683  -392}\special{pa  -693  -368}\special{pa  -701  -343}%
\special{pa  -706  -318}\special{pa  -709  -293}\special{pa  -708  -267}\special{pa  -705  -242}%
\special{pa  -699  -217}\special{pa  -690  -193}\special{pa  -678  -169}\special{pa  -664  -145}%
\special{pa  -647  -122}\special{pa  -628   -99}\special{pa  -606   -78}\special{pa  -582   -57}%
\special{pa  -555   -38}\special{pa  -527   -19}\special{pa  -496    -2}\special{pa  -463    14}%
\special{pa  -428    29}\special{pa  -392    42}\special{pa  -354    53}\special{pa  -315    64}%
\special{pa  -275    72}\special{pa  -233    79}\special{pa  -191    84}\special{pa  -147    87}%
\special{pa  -104    89}\special{pa   -59    89}\special{pa   -15    87}\special{pa    30    83}%
\special{pa    74    78}\special{pa   118    71}\special{pa   162    61}\special{pa   205    51}%
\special{pa   247    38}\special{pa   288    24}\special{pa   328     8}\special{pa   367    -9}%
\special{pa   404   -28}\special{pa   440   -49}\special{pa   474   -71}\special{pa   506   -94}%
\special{pa   536  -118}\special{pa   564  -144}\special{pa   590  -171}\special{pa   614  -199}%
\special{pa   635  -227}\special{pa   653  -257}\special{pa   669  -287}\special{pa   683  -318}%
\special{pa   693  -349}\special{pa   701  -380}\special{pa   706  -412}\special{pa   709  -444}%
\special{pa   708  -476}\special{pa   705  -508}\special{pa   699  -540}\special{pa   690  -571}%
\special{pa   678  -602}\special{pa   664  -633}\special{pa   647  -663}\special{pa   628  -692}%
\special{pa   606  -720}\special{pa   582  -748}\special{pa   555  -774}\special{pa   527  -799}%
\special{pa   496  -823}\special{pa   463  -846}\special{pa   428  -868}\special{pa   392  -888}%
\special{pa   354  -906}\special{pa   315  -923}\special{pa   275  -938}\special{pa   233  -952}%
\special{pa   191  -964}\special{pa   147  -974}\special{pa   104  -983}\special{pa    59  -989}%
\special{pa    15  -994}\special{pa   -30  -997}\special{pa   -74  -999}\special{pa  -118  -998}%
\special{pa  -162  -996}\special{pa  -205  -992}\special{pa  -247  -986}\special{pa  -288  -979}%
\special{pa  -328  -970}\special{pa  -367  -959}\special{pa  -404  -947}\special{pa  -440  -933}%
\special{pa  -474  -918}\special{pa  -506  -902}\special{pa  -536  -884}\special{pa  -564  -865}%
\special{pa  -590  -846}\special{pa  -614  -825}\special{pa  -635  -803}\special{pa  -653  -780}%
\special{pa  -669  -757}\special{pa  -683  -733}\special{pa  -693  -709}\special{pa  -701  -684}%
\special{pa  -706  -659}\special{pa  -709  -634}\special{pa  -708  -609}\special{pa  -705  -583}%
\special{pa  -699  -558}\special{pa  -690  -534}\special{pa  -678  -510}\special{pa  -664  -486}%
\special{pa  -647  -463}\special{pa  -628  -440}\special{pa  -606  -419}\special{pa  -582  -398}%
\special{pa  -555  -379}\special{pa  -527  -360}\special{pa  -496  -343}\special{pa  -463  -327}%
\special{pa  -428  -313}\special{pa  -392  -299}\special{pa  -354  -288}%
\special{fp}%
\special{pa  -354   394}\special{pa  -354  -288}%
\special{fp}%
\end{picture}}%
\end{center}

\subsection{Two Dimensional Figures}

Data of two dimensional figures such as polyhedra or planes are given in obj format.

\begin{description}
\item[Examples]\mbox{}\\
\verb|Start3d();|\\
\verb|vertex=[[2,2,-2],[2,-2,-2],[-2,-2,-2],[-2,2,-2]];|\\
\verb|Reflect3d1(``1'',vertex,[[0,0,0],[1,0,0],[0,1,0]];|\\
\verb|vertex=concat(vertex,ref3d1);|\\
\verb|edge=[[1,2,6,5],[1,5,8,4],[1,4,3,2],[2,3,7,6],[3,4,8,7],[5,6,7,8]];|\\
\verb|cube=[vertex,edge];|\\
\verb|plane=[[[-3,1,-3],[3,-1,-3],[-4,5,3],[2,3,3]],[[1,2,4,3]]];|\\
\verb|tmp=Concatobj([cubic,plane]);|\\
\verb|VertexEdgeFace("1",tmp,["Vtx=nogeo","Edg=nogeo"]);|\\
\verb|Nohiddenbyfaces("1","phf3d1");| // for the figure on the left

\item[Remark]\mbox{}\\
Command \verb|Concatobj| combines data in obj format.\\
Command \verb|VertexEdgeFace| assigns vertices to {\tt phv}, edges to {\tt phe} and faces to {\tt phf}.\\
Command \verb|Nohiddenbyfaces| is for hiddenline elimination.\\
Use \verb|Skeletonparadata("1")| if the figure on the right is desirable.
\end{description}

\vspace{-10mm}

\begin{center}
%%% /Users/takatoosetsuo/Dropbox/kettoday/0905spatialfigure/fig/polygonplane1.tex 
%%% Generator=2Dfigures.cdy 
{\unitlength=6mm%
\begin{picture}%
(10.11,8.52)(-5,-4)%
\special{pn 8}%
%
\special{pa  -581  -436}\special{pa  -116  -683}%
\special{fp}%
\special{pn 8}%
\special{pa -662 -394}\special{pa -655 -397}\special{fp}\special{pa -623 -414}\special{pa -616 -418}\special{fp}%
\special{pa -585 -434}\special{pa -578 -438}\special{fp}\special{pn 8}%
\special{pa  -116  -683}\special{pa   658  -482}%
\special{fp}%
\special{pa   658  -482}\special{pa   116  -194}%
\special{fp}%
\special{pn 8}%
\special{pa -662 483}\special{pa -655 480}\special{fp}\special{pa -628 465}\special{pa -621 462}\special{fp}%
\special{pa -594 447}\special{pa -587 444}\special{fp}\special{pa -560 429}\special{pa -553 426}\special{fp}%
\special{pa -526 411}\special{pa -519 408}\special{fp}\special{pa -492 393}\special{pa -485 390}\special{fp}%
\special{pa -458 376}\special{pa -451 372}\special{fp}\special{pa -424 358}\special{pa -417 354}\special{fp}%
\special{pa -390 340}\special{pa -383 336}\special{fp}\special{pa -356 322}\special{pa -349 318}\special{fp}%
\special{pa -323 304}\special{pa -315 300}\special{fp}\special{pa -289 286}\special{pa -282 282}\special{fp}%
\special{pa -255 268}\special{pa -248 264}\special{fp}\special{pa -221 250}\special{pa -214 246}\special{fp}%
\special{pa -187 232}\special{pa -180 228}\special{fp}\special{pa -153 214}\special{pa -146 210}\special{fp}%
\special{pa -119 196}\special{pa -112 192}\special{fp}\special{pn 8}%
\special{pn 8}%
\special{pa -116 198}\special{pa -116 190}\special{fp}\special{pa -116 158}\special{pa -116 150}\special{fp}%
\special{pa -116 118}\special{pa -116 110}\special{fp}\special{pa -116 78}\special{pa -116 70}\special{fp}%
\special{pa -116 38}\special{pa -116 30}\special{fp}\special{pa -116 -1}\special{pa -116 -9}\special{fp}%
\special{pa -116 -41}\special{pa -116 -49}\special{fp}\special{pa -116 -81}\special{pa -116 -89}\special{fp}%
\special{pa -116 -121}\special{pa -116 -129}\special{fp}\special{pa -116 -161}\special{pa -116 -169}\special{fp}%
\special{pa -116 -201}\special{pa -116 -209}\special{fp}\special{pa -116 -241}\special{pa -116 -249}\special{fp}%
\special{pa -116 -281}\special{pa -116 -289}\special{fp}\special{pa -116 -320}\special{pa -116 -328}\special{fp}%
\special{pa -116 -360}\special{pa -116 -368}\special{fp}\special{pa -116 -400}\special{pa -116 -408}\special{fp}%
\special{pa -116 -440}\special{pa -116 -448}\special{fp}\special{pa -116 -480}\special{pa -116 -488}\special{fp}%
\special{pa -116 -520}\special{pa -116 -528}\special{fp}\special{pa -116 -560}\special{pa -116 -568}\special{fp}%
\special{pa -116 -600}\special{pa -116 -608}\special{fp}\special{pa -116 -639}\special{pa -116 -647}\special{fp}%
\special{pa -116 -679}\special{pa -116 -687}\special{fp}\special{pn 8}%
\special{pa  -581  -436}\special{pa  -116  -683}%
\special{fp}%
\special{pn 8}%
\special{pa -662 -394}\special{pa -655 -397}\special{fp}\special{pa -623 -414}\special{pa -616 -418}\special{fp}%
\special{pa -585 -434}\special{pa -578 -438}\special{fp}\special{pn 8}%
\special{pa  -658   482}\special{pa  -658  -256}%
\special{fp}%
\special{pn 8}%
\special{pa -658 -252}\special{pa -658 -260}\special{fp}\special{pa -658 -287}\special{pa -658 -295}\special{fp}%
\special{pa -658 -322}\special{pa -658 -330}\special{fp}\special{pa -658 -357}\special{pa -658 -365}\special{fp}%
\special{pa -658 -392}\special{pa -658 -400}\special{fp}\special{pn 8}%
\special{pa  -330  -310}\special{pa   116  -194}%
\special{fp}%
\special{pn 8}%
\special{pa -662 -397}\special{pa -654 -395}\special{fp}\special{pa -625 -387}\special{pa -618 -385}\special{fp}%
\special{pa -589 -378}\special{pa -581 -376}\special{fp}\special{pa -552 -368}\special{pa -545 -366}\special{fp}%
\special{pa -516 -359}\special{pa -508 -357}\special{fp}\special{pa -480 -349}\special{pa -472 -347}\special{fp}%
\special{pa -443 -340}\special{pa -435 -337}\special{fp}\special{pa -407 -330}\special{pa -399 -328}\special{fp}%
\special{pa -370 -320}\special{pa -362 -318}\special{fp}\special{pa -334 -311}\special{pa -326 -309}\special{fp}%
\special{pn 8}%
\special{pa   116   683}\special{pa   116   424}%
\special{fp}%
\special{pa   116   -40}\special{pa   116  -194}%
\special{fp}%
\special{pn 8}%
\special{pa 116 428}\special{pa 116 420}\special{fp}\special{pa 116 389}\special{pa 116 381}\special{fp}%
\special{pa 116 351}\special{pa 116 343}\special{fp}\special{pa 116 312}\special{pa 116 304}\special{fp}%
\special{pa 116 273}\special{pa 116 265}\special{fp}\special{pa 116 235}\special{pa 116 227}\special{fp}%
\special{pa 116 196}\special{pa 116 188}\special{fp}\special{pa 116 157}\special{pa 116 149}\special{fp}%
\special{pa 116 119}\special{pa 116 111}\special{fp}\special{pa 116 80}\special{pa 116 72}\special{fp}%
\special{pa 116 41}\special{pa 116 33}\special{fp}\special{pa 116 3}\special{pa 116 -5}\special{fp}%
\special{pa 116 -36}\special{pa 116 -44}\special{fp}\special{pn 8}%
\special{pa  -658   482}\special{pa   116   683}%
\special{fp}%
\special{pa   658   396}\special{pa   333   568}%
\special{fp}%
\special{pa   213   631}\special{pa   116   683}%
\special{fp}%
\special{pn 8}%
\special{pa 336 566}\special{pa 329 570}\special{fp}\special{pa 297 587}\special{pa 289 591}\special{fp}%
\special{pa 257 608}\special{pa 250 612}\special{fp}\special{pa 217 629}\special{pa 210 633}\special{fp}%
\special{pn 8}%
\special{pn 8}%
\special{pa -119 193}\special{pa -112 195}\special{fp}\special{pa -81 203}\special{pa -73 205}\special{fp}%
\special{pa -42 213}\special{pa -34 215}\special{fp}\special{pa -3 223}\special{pa 4 225}\special{fp}%
\special{pa 35 233}\special{pa 43 235}\special{fp}\special{pa 74 243}\special{pa 82 245}\special{fp}%
\special{pa 113 253}\special{pa 120 255}\special{fp}\special{pa 151 263}\special{pa 159 265}\special{fp}%
\special{pa 190 274}\special{pa 198 276}\special{fp}\special{pa 229 284}\special{pa 236 286}\special{fp}%
\special{pa 267 294}\special{pa 275 296}\special{fp}\special{pa 306 304}\special{pa 314 306}\special{fp}%
\special{pa 345 314}\special{pa 353 316}\special{fp}\special{pa 383 324}\special{pa 391 326}\special{fp}%
\special{pa 422 334}\special{pa 430 336}\special{fp}\special{pa 461 344}\special{pa 469 346}\special{fp}%
\special{pa 499 354}\special{pa 507 356}\special{fp}\special{pa 538 364}\special{pa 546 366}\special{fp}%
\special{pa 577 374}\special{pa 585 376}\special{fp}\special{pa 616 385}\special{pa 623 387}\special{fp}%
\special{pa 654 395}\special{pa 662 397}\special{fp}\special{pn 8}%
\special{pa   658   396}\special{pa   658  -482}%
\special{fp}%
\special{pa  -116  -683}\special{pa   658  -482}%
\special{fp}%
\special{pa   658  -482}\special{pa   116  -194}%
\special{fp}%
\special{pa   445   881}\special{pa   -19   648}%
\special{fp}%
\special{pn 8}%
\special{pa -16 650}\special{pa -23 646}\special{fp}\special{pa -51 632}\special{pa -58 628}\special{fp}%
\special{pa -87 614}\special{pa -94 611}\special{fp}\special{pa -122 597}\special{pa -129 593}\special{fp}%
\special{pa -158 579}\special{pa -165 575}\special{fp}\special{pa -193 561}\special{pa -200 557}\special{fp}%
\special{pa -228 543}\special{pa -236 540}\special{fp}\special{pa -264 525}\special{pa -271 522}\special{fp}%
\special{pa -299 508}\special{pa -306 504}\special{fp}\special{pa -335 490}\special{pa -342 486}\special{fp}%
\special{pa -370 472}\special{pa -377 468}\special{fp}\special{pa -406 454}\special{pa -413 451}\special{fp}%
\special{pa -441 436}\special{pa -448 433}\special{fp}\special{pn 8}%
\special{pa  -658  -163}\special{pa  -794  -543}%
\special{fp}%
\special{pn 8}%
\special{pa -443 438}\special{pa -446 431}\special{fp}\special{pa -457 401}\special{pa -459 393}\special{fp}%
\special{pa -470 364}\special{pa -473 356}\special{fp}\special{pa -483 326}\special{pa -486 319}\special{fp}%
\special{pa -497 289}\special{pa -499 281}\special{fp}\special{pa -510 252}\special{pa -513 244}\special{fp}%
\special{pa -523 214}\special{pa -526 207}\special{fp}\special{pa -537 177}\special{pa -539 169}\special{fp}%
\special{pa -550 139}\special{pa -553 132}\special{fp}\special{pa -563 102}\special{pa -566 95}\special{fp}%
\special{pa -577 65}\special{pa -579 57}\special{fp}\special{pa -590 27}\special{pa -593 20}\special{fp}%
\special{pa -603 -10}\special{pa -606 -18}\special{fp}\special{pa -617 -47}\special{pa -619 -55}\special{fp}%
\special{pa -630 -85}\special{pa -633 -92}\special{fp}\special{pa -643 -122}\special{pa -646 -130}\special{fp}%
\special{pa -657 -159}\special{pa -659 -167}\special{fp}\special{pn 8}%
\special{pa    95   -96}\special{pa  -794  -543}%
\special{fp}%
\special{pa   445   881}\special{pa    95   -96}%
\special{fp}%
\end{picture}}%\hspace{10mm}%%% /Users/takatoosetsuo/Dropbox/kettoday/0905spatialfigure/fig/polygonplane2.tex 
%%% Generator=2Dfigures.cdy 
{\unitlength=6mm%
\begin{picture}%
(10.11,8.52)(-5,-4)%
\special{pn 8}%
%
\special{pa  -658   482}\special{pa  -116   194}%
\special{fp}%
\special{pa  -116   194}\special{pa  -116  -153}%
\special{fp}%
\special{pa  -116  -300}\special{pa  -116  -683}%
\special{fp}%
\special{pa  -658  -396}\special{pa  -628  -411}%
\special{fp}%
\special{pa  -534  -461}\special{pa  -116  -683}%
\special{fp}%
\special{pa  -658   482}\special{pa  -658  -396}%
\special{fp}%
\special{pa  -658  -396}\special{pa  -412  -331}%
\special{fp}%
\special{pa  -248  -289}\special{pa   116  -194}%
\special{fp}%
\special{pa   116   683}\special{pa   116    40}%
\special{fp}%
\special{pa   116  -120}\special{pa   116  -194}%
\special{fp}%
\special{pa  -658   482}\special{pa   116   683}%
\special{fp}%
\special{pa   658   396}\special{pa   372   547}%
\special{fp}%
\special{pa   293   589}\special{pa   116   683}%
\special{fp}%
\special{pa  -116   194}\special{pa    71   243}%
\special{fp}%
\special{pa   160   266}\special{pa   180   271}%
\special{fp}%
\special{pa   283   298}\special{pa   658   396}%
\special{fp}%
\special{pa   658   396}\special{pa   658  -482}%
\special{fp}%
\special{pa  -116  -683}\special{pa   658  -482}%
\special{fp}%
\special{pa   658  -482}\special{pa   116  -194}%
\special{fp}%
\special{pa   445   881}\special{pa    57   686}%
\special{fp}%
\special{pa   -95   610}\special{pa  -445   435}%
\special{fp}%
\special{pa  -445   435}\special{pa  -449   421}%
\special{fp}%
\special{pa  -480   337}\special{pa  -631   -88}%
\special{fp}%
\special{pa  -685  -238}\special{pa  -794  -543}%
\special{fp}%
\special{pa    95   -96}\special{pa  -794  -543}%
\special{fp}%
\special{pa   445   881}\special{pa    95   -96}%
\special{fp}%
\end{picture}}%
\end{center}

\noindent
{\bf Remark} See \verb|KeTCindyreferenceE.pdf| or \verb|samples/s05spacefigure| for more information.

\subsection{Surfaces}

Two variable function is defined as a list of one of the followings.

\begin{enumerate}[\hspace*{5mm}\bf 1.]
\item \verb|["z=f(x,y)","x=[a,b]","y=[c,d]"]|
\item \verb|["z=f(x,y)","x=x(u,v)","y(u,v)","u=[a,b]","v=[c,d]"]|
\item \verb|["p","x=x(u,v)","y=y(u,v)","z=z(u,v)","u=[a,b]","v=[c,d]"]|
\end{enumerate}

Optionally, you can add what boundaries should be drawn.
The default is "wesn". Here, for example, "w" means the boundary defined by $[a,t]\ (c\leqq t\leqq d)$.

\ketcindy\ calls C to speed up the calculation of hidden lines elimination.

\begin{description}
\item[Example]\mbox{}\\
\verb|Start3d();|\\
\verb|Xyzax3data("","x=[-5,5]","y=[-5,5]","z=[-5,5]");|\\
\verb|fd=["z=x^2-y^2","x=[-2,2]","y=[-2,2]","senw"];|\\
\verb|Startsurf();|\\
\verb|Sfbdparadata("1",fd);|\\
\verb|Crvsfparadata("1","ax3d","sfbd3d1",fd);|\\
\verb|ExeccmdC("1");|
\verb|Windispg();|

\begin{center}
%%% /Users/takatoosetsuo/Dropbox/kettoday/0905spatialfigure/fig/saddle1E.tex 
%%% Generator=saddle.cdy 
{\unitlength=6mm%
\begin{picture}%
(9.59,9.56)(-4.62,-4.42)%
\special{pn 8}%
%
\settowidth{\Width}{$x$}\setlength{\Width}{-0.5\Width}%
\settoheight{\Height}{$x$}\settodepth{\Depth}{$x$}\setlength{\Height}{-0.5\Height}\setlength{\Depth}{0.5\Depth}\addtolength{\Height}{\Depth}%
\put(-3.3400000,-1.9500000){\hspace*{\Width}\raisebox{\Height}{$x$}}%
%
\settowidth{\Width}{$y$}\setlength{\Width}{-0.5\Width}%
\settoheight{\Height}{$y$}\settodepth{\Depth}{$y$}\setlength{\Height}{-0.5\Height}\setlength{\Depth}{0.5\Depth}\addtolength{\Height}{\Depth}%
\put(4.2500000,-1.4900000){\hspace*{\Width}\raisebox{\Height}{$y$}}%
%
\settowidth{\Width}{$z$}\setlength{\Width}{-0.5\Width}%
\settoheight{\Height}{$z$}\settodepth{\Depth}{$z$}\setlength{\Height}{-0.5\Height}\setlength{\Depth}{0.5\Depth}\addtolength{\Height}{\Depth}%
\put(0.0000000,4.7700000){\hspace*{\Width}\raisebox{\Height}{$z$}}%
%
\special{pa  -196   290}\special{pa  -193   279}\special{pa  -193   278}%
\special{fp}%
\special{pa  -193   278}\special{pa  -190   265}\special{pa  -190   265}%
\special{fp}%
\special{pa  -190   265}\special{pa  -186   248}\special{pa  -185   248}%
\special{fp}%
\special{pa  -185   248}\special{pa  -184   242}\special{pa  -178   220}\special{pa  -178   220}%
\special{fp}%
\special{pa  -178   220}\special{pa  -172   199}\special{pa  -170   192}\special{pa  -166   179}%
\special{pa  -162   167}\special{pa  -160   160}\special{pa  -155   144}\special{pa  -154   142}%
\special{pa  -148   125}\special{pa  -147   122}\special{pa  -142   110}\special{pa  -139   103}%
\special{pa  -136    95}\special{pa  -132    85}\special{pa  -130    82}\special{pa  -124    69}%
\special{pa  -118    58}\special{pa  -116    54}\special{pa  -112    48}\special{pa  -108    41}%
\special{pa  -106    38}\special{pa  -101    31}\special{pa   -94    23}\special{pa   -93    22}%
\special{pa   -88    17}\special{pa   -85    14}\special{pa   -82    12}\special{pa   -78     9}%
\special{pa   -76     8}\special{pa   -70     6}\special{pa   -64     4}\special{pa   -62     4}%
\special{pa   -58     3}\special{pa   -54     4}\special{pa   -53     4}\special{pa   -47     5}%
\special{pa   -41     8}\special{pa   -39     9}\special{pa   -35    12}\special{pa   -31    14}%
\special{pa   -29    17}\special{pa   -24    21}\special{pa   -23    22}\special{pa   -17    29}%
\special{pa   -16    30}\special{pa   -11    37}\special{pa    -8    41}\special{pa    -5    47}%
\special{pa    -0    54}\special{pa     1    57}\special{pa     7    68}\special{pa    13    80}%
\special{pa    15    84}\special{pa    19    94}\special{pa    23   102}\special{pa    25   108}%
\special{pa    30   122}\special{pa    31   124}\special{pa    37   140}\special{pa    38   143}%
\special{pa    43   158}\special{pa    46   167}\special{pa    49   177}\special{pa    54   192}%
\special{pa    55   197}\special{pa    61   218}\special{pa    67   240}\special{pa    69   247}%
\special{pa    73   263}\special{pa    77   278}\special{pa    79   287}\special{pa    84   310}%
\special{fp}%
\special{pa    84   310}\special{pa    85   312}\special{pa    91   337}%
\special{fp}%
\special{pa    91   337}\special{pa    91   339}\special{pa    92   342}%
\special{fp}%
\special{pa    92   342}\special{pa    92   344}\special{pa    95   358}%
\special{fp}%
\special{pa    95   358}\special{pa    97   366}\special{pa   100   380}%
\special{fp}%
\special{pa   100   380}\special{pa   103   394}%
\special{fp}%
\special{pa  -663    38}\special{pa  -648   -23}\special{pa  -633   -81}\special{pa  -618  -136}%
\special{pa  -603  -189}\special{pa  -588  -239}\special{pa  -573  -286}\special{pa  -558  -331}%
\special{pa  -543  -373}\special{pa  -528  -412}\special{pa  -513  -448}\special{pa  -498  -482}%
\special{pa  -484  -513}\special{pa  -469  -542}\special{pa  -454  -567}\special{pa  -439  -591}%
\special{pa  -424  -611}\special{pa  -409  -629}\special{pa  -394  -644}\special{pa  -379  -656}%
\special{pa  -364  -665}\special{pa  -349  -672}\special{pa  -334  -676}\special{pa  -319  -678}%
\special{pa  -304  -677}\special{pa  -289  -673}\special{pa  -274  -666}\special{pa  -259  -657}%
\special{pa  -245  -645}\special{pa  -230  -630}\special{pa  -215  -613}\special{pa  -200  -593}%
\special{pa  -185  -570}\special{pa  -170  -545}\special{pa  -155  -516}\special{pa  -140  -486}%
\special{pa  -125  -452}\special{pa  -110  -416}\special{pa   -95  -377}\special{pa   -80  -335}%
\special{pa   -78  -327}%
\special{fp}%
\special{pa   -78  -327}\special{pa   -65  -291}\special{pa   -50  -244}\special{pa   -35  -194}%
\special{pa   -20  -142}\special{pa    -5   -87}\special{pa     9   -29}\special{pa    24    32}%
\special{pa    39    95}\special{pa    54   161}\special{pa    69   229}\special{pa    84   301}%
\special{fp}%
\special{pa   663   -38}\special{pa   651    35}\special{pa   640   105}\special{pa   628   172}%
\special{pa   617   237}\special{pa   605   299}\special{pa   593   358}\special{pa   582   415}%
\special{pa   570   469}\special{pa   559   520}\special{pa   547   569}\special{pa   536   614}%
\special{pa   524   658}\special{pa   512   698}\special{pa   501   736}\special{pa   489   771}%
\special{pa   478   803}\special{pa   466   833}\special{pa   454   860}\special{pa   443   884}%
\special{pa   431   906}\special{pa   420   925}\special{pa   408   941}\special{pa   397   954}%
\special{pa   385   965}\special{pa   373   973}\special{pa   362   979}\special{pa   350   982}%
\special{pa   339   982}\special{pa   327   979}\special{pa   316   974}\special{pa   304   965}%
\special{pa   292   955}\special{pa   281   941}\special{pa   269   925}\special{pa   258   906}%
\special{pa   246   885}\special{pa   235   861}\special{pa   223   834}\special{pa   211   804}%
\special{pa   200   772}\special{pa   188   737}\special{pa   177   699}\special{pa   165   659}%
\special{pa   154   616}\special{pa   142   570}\special{pa   130   521}\special{pa   119   470}%
\special{pa   107   416}\special{pa   103   394}%
\special{fp}%
\special{pa   103   394}\special{pa   100   380}%
\special{fp}%
\special{pa   100   380}\special{pa    97   366}%
\special{fp}%
\special{pa    97   366}\special{pa    96   360}\special{pa    93   344}%
\special{fp}%
\special{pa    93   344}\special{pa    91   337}%
\special{fp}%
\special{pa    91   337}\special{pa    86   312}%
\special{fp}%
\special{pa    86   312}\special{pa    84   301}%
\special{fp}%
\special{pa   -78  -327}\special{pa   -69  -361}\special{pa   -54  -419}\special{pa   -39  -475}%
\special{pa   -24  -527}\special{pa    -9  -577}\special{pa     5  -625}\special{pa    20  -669}%
\special{pa    35  -711}\special{pa    50  -751}\special{pa    65  -787}\special{pa    80  -821}%
\special{pa    95  -852}\special{pa   110  -880}\special{pa   125  -906}\special{pa   140  -929}%
\special{pa   155  -950}\special{pa   170  -967}\special{pa   185  -982}\special{pa   200  -994}%
\special{pa   215 -1004}\special{pa   230 -1011}\special{pa   245 -1015}\special{pa   259 -1017}%
\special{pa   274 -1015}\special{pa   289 -1012}\special{pa   304 -1005}\special{pa   319  -996}%
\special{pa   334  -984}\special{pa   349  -969}\special{pa   364  -952}\special{pa   379  -932}%
\special{pa   394  -909}\special{pa   409  -883}\special{pa   424  -855}\special{pa   439  -824}%
\special{pa   454  -791}\special{pa   469  -755}\special{pa   484  -716}\special{pa   498  -674}%
\special{pa   513  -630}\special{pa   528  -583}\special{pa   543  -533}\special{pa   558  -480}%
\special{pa   573  -425}\special{pa   588  -368}\special{pa   603  -307}\special{pa   618  -244}%
\special{pa   633  -178}\special{pa   648  -109}\special{pa   663   -38}%
\special{fp}%
\special{pa  -196   289}\special{pa  -200   306}\special{pa  -211   352}\special{pa  -223   395}%
\special{pa  -235   436}\special{pa  -246   473}\special{pa  -258   508}\special{pa  -269   541}%
\special{pa  -281   571}\special{pa  -292   598}\special{pa  -304   622}\special{pa  -316   643}%
\special{pa  -327   662}\special{pa  -339   679}\special{pa  -350   692}\special{pa  -362   703}%
\special{pa  -373   711}\special{pa  -385   716}\special{pa  -397   719}\special{pa  -408   719}%
\special{pa  -420   716}\special{pa  -431   711}\special{pa  -443   703}\special{pa  -454   692}%
\special{pa  -466   679}\special{pa  -478   663}\special{pa  -489   644}\special{pa  -501   622}%
\special{pa  -512   598}\special{pa  -524   571}\special{pa  -536   542}\special{pa  -547   509}%
\special{pa  -559   474}\special{pa  -570   437}\special{pa  -582   396}\special{pa  -593   353}%
\special{pa  -605   307}\special{pa  -617   259}\special{pa  -628   208}\special{pa  -640   154}%
\special{pa  -651    97}\special{pa  -663    38}%
\special{fp}%
\special{pa   723  -423}\special{pa   593  -347}%
\special{fp}%
\special{pa   -29    17}\special{pa  -127    74}%
\special{fp}%
\special{pa  -127    74}\special{pa  -595   348}%
\special{fp}%
\special{pa  -595   348}\special{pa  -723   423}%
\special{fp}%
\special{pa  -934  -328}\special{pa  -597  -210}%
\special{fp}%
\special{pa    18     6}\special{pa   620   218}%
\special{fp}%
\special{pa   620   218}\special{pa   934   328}%
\special{fp}%
\special{pa     0  1044}\special{pa     0    55}%
\special{fp}%
\special{pa     0   -65}\special{pa     0  -607}%
\special{fp}%
\special{pa     0  -607}\special{pa     0 -1053}%
\special{fp}%
\special{pn 8}%
\special{pa -85 -297}\special{pa -83 -304}\special{fp}\special{pa -78 -324}\special{pa -77 -331}\special{fp}%
\special{pn 8}%
\special{pa -84 -304}\special{pa -85 -297}\special{fp}\special{pa -90 -265}\special{pa -91 -257}\special{fp}%
\special{pa -96 -226}\special{pa -97 -218}\special{fp}\special{pa -103 -186}\special{pa -104 -178}\special{fp}%
\special{pa -109 -147}\special{pa -111 -139}\special{fp}\special{pa -116 -107}\special{pa -117 -99}\special{fp}%
\special{pa -123 -68}\special{pa -124 -60}\special{fp}\special{pa -130 -28}\special{pa -131 -21}\special{fp}%
\special{pa -137 11}\special{pa -139 19}\special{fp}\special{pa -145 50}\special{pa -146 58}\special{fp}%
\special{pa -152 89}\special{pa -154 97}\special{fp}\special{pa -160 129}\special{pa -162 136}\special{fp}%
\special{pa -168 168}\special{pa -170 176}\special{fp}\special{pa -177 207}\special{pa -179 215}\special{fp}%
\special{pa -186 246}\special{pa -187 254}\special{fp}\special{pa -195 285}\special{pa -197 292}\special{fp}%
\special{pn 8}%
\special{pn 8}%
\special{pa 597 -349}\special{pa 590 -345}\special{fp}\special{pa 562 -329}\special{pa 555 -325}\special{fp}%
\special{pa 528 -309}\special{pa 521 -305}\special{fp}\special{pa 493 -289}\special{pa 486 -285}\special{fp}%
\special{pa 458 -268}\special{pa 452 -264}\special{fp}\special{pa 424 -248}\special{pa 417 -244}\special{fp}%
\special{pa 389 -228}\special{pa 382 -224}\special{fp}\special{pa 355 -208}\special{pa 348 -204}\special{fp}%
\special{pa 320 -187}\special{pa 313 -183}\special{fp}\special{pa 286 -167}\special{pa 279 -163}\special{fp}%
\special{pa 251 -147}\special{pa 244 -143}\special{fp}\special{pa 217 -127}\special{pa 210 -123}\special{fp}%
\special{pa 182 -107}\special{pa 175 -103}\special{fp}\special{pa 148 -86}\special{pa 141 -82}\special{fp}%
\special{pa 113 -66}\special{pa 106 -62}\special{fp}\special{pa 79 -46}\special{pa 72 -42}\special{fp}%
\special{pa 44 -26}\special{pa 37 -22}\special{fp}\special{pa 9 -6}\special{pa 3 -2}\special{fp}%
\special{pa -25 15}\special{pa -32 19}\special{fp}\special{pn 8}%
\special{pa -601 -211}\special{pa -593 -208}\special{fp}\special{pa -564 -198}\special{pa -557 -196}\special{fp}%
\special{pa -528 -186}\special{pa -521 -183}\special{fp}\special{pa -492 -173}\special{pa -485 -170}\special{fp}%
\special{pa -456 -160}\special{pa -448 -157}\special{fp}\special{pa -420 -147}\special{pa -412 -145}\special{fp}%
\special{pa -384 -135}\special{pa -376 -132}\special{fp}\special{pa -347 -122}\special{pa -340 -119}\special{fp}%
\special{pa -311 -109}\special{pa -304 -107}\special{fp}\special{pa -275 -97}\special{pa -267 -94}\special{fp}%
\special{pa -239 -84}\special{pa -231 -81}\special{fp}\special{pa -203 -71}\special{pa -195 -69}\special{fp}%
\special{pa -166 -58}\special{pa -159 -56}\special{fp}\special{pa -130 -46}\special{pa -123 -43}\special{fp}%
\special{pa -94 -33}\special{pa -87 -30}\special{fp}\special{pa -58 -20}\special{pa -50 -18}\special{fp}%
\special{pa -22 -8}\special{pa -14 -5}\special{fp}\special{pa 14 5}\special{pa 22 8}\special{fp}%
\special{pn 8}%
\special{pa 0 59}\special{pa 0 51}\special{fp}\special{pa 0 19}\special{pa 0 11}\special{fp}%
\special{pa 0 -21}\special{pa 0 -29}\special{fp}\special{pa 0 -61}\special{pa 0 -69}\special{fp}%
\special{pn 8}%
\end{picture}}%\hspace{10mm}%%% /Users/takatoosetsuo/Dropbox/kettoday/0905spatialfigure/fig/saddle2E.tex 
%%% Generator=saddle.cdy 
{\unitlength=6mm%
\begin{picture}%
(9.59,9.56)(-4.62,-4.42)%
\special{pn 8}%
%
\settowidth{\Width}{$x$}\setlength{\Width}{-0.5\Width}%
\settoheight{\Height}{$x$}\settodepth{\Depth}{$x$}\setlength{\Height}{-0.5\Height}\setlength{\Depth}{0.5\Depth}\addtolength{\Height}{\Depth}%
\put(-3.3400000,-1.9500000){\hspace*{\Width}\raisebox{\Height}{$x$}}%
%
\settowidth{\Width}{$y$}\setlength{\Width}{-0.5\Width}%
\settoheight{\Height}{$y$}\settodepth{\Depth}{$y$}\setlength{\Height}{-0.5\Height}\setlength{\Depth}{0.5\Depth}\addtolength{\Height}{\Depth}%
\put(4.2500000,-1.4900000){\hspace*{\Width}\raisebox{\Height}{$y$}}%
%
\settowidth{\Width}{$z$}\setlength{\Width}{-0.5\Width}%
\settoheight{\Height}{$z$}\settodepth{\Depth}{$z$}\setlength{\Height}{-0.5\Height}\setlength{\Depth}{0.5\Depth}\addtolength{\Height}{\Depth}%
\put(0.0000000,4.7700000){\hspace*{\Width}\raisebox{\Height}{$z$}}%
%
\special{pa  -196   290}\special{pa  -193   279}\special{pa  -193   278}%
\special{fp}%
\special{pa  -193   278}\special{pa  -190   265}\special{pa  -190   265}%
\special{fp}%
\special{pa  -190   265}\special{pa  -186   248}\special{pa  -185   248}%
\special{fp}%
\special{pa  -185   248}\special{pa  -184   242}\special{pa  -178   220}\special{pa  -178   220}%
\special{fp}%
\special{pa  -178   220}\special{pa  -172   199}\special{pa  -170   192}\special{pa  -166   179}%
\special{pa  -162   167}\special{pa  -160   160}\special{pa  -155   144}\special{pa  -154   142}%
\special{pa  -148   125}\special{pa  -147   122}\special{pa  -142   110}\special{pa  -139   103}%
\special{pa  -136    95}\special{pa  -132    85}\special{pa  -130    82}\special{pa  -124    69}%
\special{pa  -118    58}\special{pa  -116    54}\special{pa  -112    48}\special{pa  -108    41}%
\special{pa  -106    38}\special{pa  -101    31}\special{pa   -94    23}\special{pa   -93    22}%
\special{pa   -88    17}\special{pa   -85    14}\special{pa   -82    12}\special{pa   -78     9}%
\special{pa   -76     8}\special{pa   -70     6}\special{pa   -64     4}\special{pa   -62     4}%
\special{pa   -58     3}\special{pa   -54     4}\special{pa   -53     4}\special{pa   -47     5}%
\special{pa   -41     8}\special{pa   -39     9}\special{pa   -35    12}\special{pa   -31    14}%
\special{pa   -29    17}\special{pa   -24    21}\special{pa   -23    22}\special{pa   -17    29}%
\special{pa   -16    30}\special{pa   -11    37}\special{pa    -8    41}\special{pa    -5    47}%
\special{pa    -0    54}\special{pa     1    57}\special{pa     7    68}\special{pa    13    80}%
\special{pa    15    84}\special{pa    19    94}\special{pa    23   102}\special{pa    25   108}%
\special{pa    30   122}\special{pa    31   124}\special{pa    37   140}\special{pa    38   143}%
\special{pa    43   158}\special{pa    46   167}\special{pa    49   177}\special{pa    54   192}%
\special{pa    55   197}\special{pa    61   218}\special{pa    67   240}\special{pa    69   247}%
\special{pa    73   263}\special{pa    77   278}\special{pa    79   287}\special{pa    84   310}%
\special{fp}%
\special{pa    84   310}\special{pa    85   312}\special{pa    91   337}%
\special{fp}%
\special{pa    91   337}\special{pa    91   339}\special{pa    92   342}%
\special{fp}%
\special{pa    92   342}\special{pa    92   344}\special{pa    95   358}%
\special{fp}%
\special{pa    95   358}\special{pa    97   366}\special{pa   100   380}%
\special{fp}%
\special{pa   100   380}\special{pa   103   394}%
\special{fp}%
\special{pa  -663    38}\special{pa  -648   -23}\special{pa  -633   -81}\special{pa  -618  -136}%
\special{pa  -603  -189}\special{pa  -588  -239}\special{pa  -573  -286}\special{pa  -558  -331}%
\special{pa  -543  -373}\special{pa  -528  -412}\special{pa  -513  -448}\special{pa  -498  -482}%
\special{pa  -484  -513}\special{pa  -469  -542}\special{pa  -454  -567}\special{pa  -439  -591}%
\special{pa  -424  -611}\special{pa  -409  -629}\special{pa  -394  -644}\special{pa  -379  -656}%
\special{pa  -364  -665}\special{pa  -349  -672}\special{pa  -334  -676}\special{pa  -319  -678}%
\special{pa  -304  -677}\special{pa  -289  -673}\special{pa  -274  -666}\special{pa  -259  -657}%
\special{pa  -245  -645}\special{pa  -230  -630}\special{pa  -215  -613}\special{pa  -200  -593}%
\special{pa  -185  -570}\special{pa  -170  -545}\special{pa  -155  -516}\special{pa  -140  -486}%
\special{pa  -125  -452}\special{pa  -110  -416}\special{pa   -95  -377}\special{pa   -80  -335}%
\special{pa   -78  -327}%
\special{fp}%
\special{pa   -78  -327}\special{pa   -65  -291}\special{pa   -50  -244}\special{pa   -35  -194}%
\special{pa   -20  -142}\special{pa    -5   -87}\special{pa     9   -29}\special{pa    24    32}%
\special{pa    39    95}\special{pa    54   161}\special{pa    69   229}\special{pa    84   301}%
\special{fp}%
\special{pa   663   -38}\special{pa   651    35}\special{pa   640   105}\special{pa   628   172}%
\special{pa   617   237}\special{pa   605   299}\special{pa   593   358}\special{pa   582   415}%
\special{pa   570   469}\special{pa   559   520}\special{pa   547   569}\special{pa   536   614}%
\special{pa   524   658}\special{pa   512   698}\special{pa   501   736}\special{pa   489   771}%
\special{pa   478   803}\special{pa   466   833}\special{pa   454   860}\special{pa   443   884}%
\special{pa   431   906}\special{pa   420   925}\special{pa   408   941}\special{pa   397   954}%
\special{pa   385   965}\special{pa   373   973}\special{pa   362   979}\special{pa   350   982}%
\special{pa   339   982}\special{pa   327   979}\special{pa   316   974}\special{pa   304   965}%
\special{pa   292   955}\special{pa   281   941}\special{pa   269   925}\special{pa   258   906}%
\special{pa   246   885}\special{pa   235   861}\special{pa   223   834}\special{pa   211   804}%
\special{pa   200   772}\special{pa   188   737}\special{pa   177   699}\special{pa   165   659}%
\special{pa   154   616}\special{pa   142   570}\special{pa   130   521}\special{pa   119   470}%
\special{pa   107   416}\special{pa   103   394}%
\special{fp}%
\special{pa   103   394}\special{pa   100   380}%
\special{fp}%
\special{pa   100   380}\special{pa    97   366}%
\special{fp}%
\special{pa    97   366}\special{pa    96   360}\special{pa    93   344}%
\special{fp}%
\special{pa    93   344}\special{pa    91   337}%
\special{fp}%
\special{pa    91   337}\special{pa    86   312}%
\special{fp}%
\special{pa    86   312}\special{pa    84   301}%
\special{fp}%
\special{pa   -78  -327}\special{pa   -69  -361}\special{pa   -54  -419}\special{pa   -39  -475}%
\special{pa   -24  -527}\special{pa    -9  -577}\special{pa     5  -625}\special{pa    20  -669}%
\special{pa    35  -711}\special{pa    50  -751}\special{pa    65  -787}\special{pa    80  -821}%
\special{pa    95  -852}\special{pa   110  -880}\special{pa   125  -906}\special{pa   140  -929}%
\special{pa   155  -950}\special{pa   170  -967}\special{pa   185  -982}\special{pa   200  -994}%
\special{pa   215 -1004}\special{pa   230 -1011}\special{pa   245 -1015}\special{pa   259 -1017}%
\special{pa   274 -1015}\special{pa   289 -1012}\special{pa   304 -1005}\special{pa   319  -996}%
\special{pa   334  -984}\special{pa   349  -969}\special{pa   364  -952}\special{pa   379  -932}%
\special{pa   394  -909}\special{pa   409  -883}\special{pa   424  -855}\special{pa   439  -824}%
\special{pa   454  -791}\special{pa   469  -755}\special{pa   484  -716}\special{pa   498  -674}%
\special{pa   513  -630}\special{pa   528  -583}\special{pa   543  -533}\special{pa   558  -480}%
\special{pa   573  -425}\special{pa   588  -368}\special{pa   603  -307}\special{pa   618  -244}%
\special{pa   633  -178}\special{pa   648  -109}\special{pa   663   -38}%
\special{fp}%
\special{pa  -196   289}\special{pa  -200   306}\special{pa  -211   352}\special{pa  -223   395}%
\special{pa  -235   436}\special{pa  -246   473}\special{pa  -258   508}\special{pa  -269   541}%
\special{pa  -281   571}\special{pa  -292   598}\special{pa  -304   622}\special{pa  -316   643}%
\special{pa  -327   662}\special{pa  -339   679}\special{pa  -350   692}\special{pa  -362   703}%
\special{pa  -373   711}\special{pa  -385   716}\special{pa  -397   719}\special{pa  -408   719}%
\special{pa  -420   716}\special{pa  -431   711}\special{pa  -443   703}\special{pa  -454   692}%
\special{pa  -466   679}\special{pa  -478   663}\special{pa  -489   644}\special{pa  -501   622}%
\special{pa  -512   598}\special{pa  -524   571}\special{pa  -536   542}\special{pa  -547   509}%
\special{pa  -559   474}\special{pa  -570   437}\special{pa  -582   396}\special{pa  -593   353}%
\special{pa  -605   307}\special{pa  -617   259}\special{pa  -628   208}\special{pa  -640   154}%
\special{pa  -651    97}\special{pa  -663    38}%
\special{fp}%
\special{pa   723  -423}\special{pa   593  -347}%
\special{fp}%
\special{pa   -29    17}\special{pa  -127    74}%
\special{fp}%
\special{pa  -127    74}\special{pa  -595   348}%
\special{fp}%
\special{pa  -595   348}\special{pa  -723   423}%
\special{fp}%
\special{pa  -934  -328}\special{pa  -597  -210}%
\special{fp}%
\special{pa    18     6}\special{pa   620   218}%
\special{fp}%
\special{pa   620   218}\special{pa   934   328}%
\special{fp}%
\special{pa     0  1044}\special{pa     0    55}%
\special{fp}%
\special{pa     0   -65}\special{pa     0  -607}%
\special{fp}%
\special{pa     0  -607}\special{pa     0 -1053}%
\special{fp}%
\special{pa   -45  -228}\special{pa   -31  -263}\special{pa   -16  -297}\special{pa    -1  -328}%
\special{pa    14  -356}\special{pa    29  -382}\special{pa    43  -405}\special{pa    58  -425}%
\special{pa    73  -443}\special{pa    88  -458}\special{pa   103  -470}\special{pa   118  -480}%
\special{pa   133  -487}\special{pa   148  -491}\special{pa   163  -492}\special{pa   178  -491}%
\special{pa   193  -487}\special{pa   208  -481}\special{pa   223  -471}\special{pa   238  -459}%
\special{pa   253  -445}\special{pa   268  -427}\special{pa   283  -407}\special{pa   297  -384}%
\special{pa   312  -359}\special{pa   327  -331}\special{pa   342  -300}\special{pa   357  -266}%
\special{pa   372  -230}\special{pa   387  -191}\special{pa   402  -150}\special{pa   417  -105}%
\special{pa   432   -58}\special{pa   447    -9}\special{pa   462    44}\special{pa   477    99}%
\special{pa   492   157}\special{pa   507   217}\special{pa   522   280}\special{pa   536   346}%
\special{pa   551   415}\special{pa   566   486}%
\special{fp}%
\special{pa  -277   560}\special{pa  -262   500}\special{pa  -247   442}\special{pa  -232   387}%
\special{pa  -217   334}\special{pa  -202   284}\special{pa  -187   237}\special{pa  -172   192}%
\special{pa  -158   150}\special{pa  -158   150}\special{pa  -147   124}\special{pa  -145   117}%
\special{pa  -143   111}\special{pa  -142   110}\special{pa  -141   106}\special{pa  -136    95}%
\special{fp}%
\special{pa   -13  -116}\special{pa    -8  -121}\special{pa     7  -133}\special{pa    22  -143}%
\special{pa    37  -149}\special{pa    52  -154}\special{pa    67  -155}\special{pa    82  -154}%
\special{pa    96  -150}\special{pa   111  -143}\special{pa   126  -134}\special{pa   141  -122}%
\special{pa   156  -107}\special{pa   171   -90}\special{pa   186   -70}\special{pa   201   -47}%
\special{pa   216   -22}\special{pa   231     6}\special{pa   246    37}\special{pa   261    71}%
\special{pa   276   107}\special{pa   291   146}\special{pa   306   188}\special{pa   321   232}%
\special{pa   335   279}\special{pa   350   329}\special{pa   365   381}\special{pa   380   436}%
\special{pa   395   494}\special{pa   410   554}\special{pa   425   618}\special{pa   440   683}%
\special{pa   455   752}\special{pa   470   823}%
\special{fp}%
\special{pa  -373   711}\special{pa  -359   650}\special{pa  -344   592}\special{pa  -329   537}%
\special{pa  -314   484}\special{pa  -299   434}\special{pa  -284   387}\special{pa  -269   342}%
\special{pa  -254   300}\special{pa  -239   261}\special{pa  -224   224}\special{pa  -209   191}%
\special{pa  -194   159}\special{pa  -179   131}\special{pa  -164   105}\special{pa  -149    82}%
\special{pa  -134    62}\special{pa  -120    44}\special{pa  -105    29}\special{pa   -90    17}%
\special{pa   -83    13}\special{pa   -78     9}\special{pa   -75     7}%
\special{fp}%
\special{pa    19     9}\special{pa    30    16}\special{pa    45    28}\special{pa    60    43}%
\special{pa    75    60}\special{pa    90    80}\special{pa   105   103}\special{pa   120   128}%
\special{pa   134   156}\special{pa   149   187}\special{pa   164   221}\special{pa   179   257}%
\special{pa   194   296}\special{pa   209   338}\special{pa   224   382}\special{pa   239   429}%
\special{pa   254   479}\special{pa   269   531}\special{pa   284   586}\special{pa   299   644}%
\special{pa   314   704}\special{pa   329   768}\special{pa   344   834}\special{pa   359   902}%
\special{pa   373   973}%
\special{fp}%
\special{pa  -470   674}\special{pa  -455   613}\special{pa  -440   555}\special{pa  -425   500}%
\special{pa  -410   447}\special{pa  -395   397}\special{pa  -380   350}\special{pa  -365   305}%
\special{pa  -350   263}\special{pa  -335   224}\special{pa  -321   187}\special{pa  -306   154}%
\special{pa  -291   122}\special{pa  -276    94}\special{pa  -261    68}\special{pa  -246    45}%
\special{pa  -231    25}\special{pa  -216     7}\special{pa  -201    -8}\special{pa  -186   -20}%
\special{pa  -171   -30}\special{pa  -156   -37}\special{pa  -141   -41}\special{pa  -126   -42}%
\special{pa  -111   -41}\special{pa   -96   -37}\special{pa   -82   -31}\special{pa   -67   -21}%
\special{pa   -52    -9}\special{pa   -37     5}\special{pa   -22    23}\special{pa   -18    28}%
\special{pa   -16    30}\special{pa   -12    36}\special{pa    -9    40}%
\special{fp}%
\special{pa    51   146}\special{pa    53   150}\special{pa    68   184}\special{pa    83   220}%
\special{pa    98   259}\special{pa   113   300}\special{pa   128   345}\special{pa   143   392}%
\special{pa   158   442}\special{pa   172   494}\special{pa   187   549}\special{pa   202   607}%
\special{pa   217   667}\special{pa   232   730}\special{pa   247   796}\special{pa   262   865}%
\special{pa   277   936}%
\special{fp}%
\special{pa  -566   450}\special{pa  -551   389}\special{pa  -536   331}\special{pa  -522   275}%
\special{pa  -507   223}\special{pa  -492   173}\special{pa  -477   125}\special{pa  -462    81}%
\special{pa  -447    39}\special{pa  -432    -0}\special{pa  -417   -37}\special{pa  -402   -71}%
\special{pa  -387  -102}\special{pa  -372  -130}\special{pa  -357  -156}\special{pa  -342  -179}%
\special{pa  -327  -199}\special{pa  -312  -217}\special{pa  -297  -232}\special{pa  -283  -244}%
\special{pa  -268  -254}\special{pa  -253  -261}\special{pa  -238  -265}\special{pa  -223  -266}%
\special{pa  -208  -265}\special{pa  -193  -261}\special{pa  -178  -255}\special{pa  -163  -246}%
\special{pa  -148  -234}\special{pa  -133  -219}\special{pa  -118  -201}\special{pa  -103  -181}%
\special{pa   -88  -159}\special{pa   -73  -133}\special{pa   -58  -105}\special{pa   -43   -74}%
\special{pa   -29   -41}\special{pa   -14    -4}\special{pa     1    35}\special{pa    16    76}%
\special{pa    31   121}\special{pa    46   168}\special{pa    48   173}\special{pa    50   181}%
\special{pa    56   200}%
\special{fp}%
\special{pa    83   296}\special{pa    91   325}\special{pa   106   383}\special{pa   121   443}%
\special{pa   136   506}\special{pa   151   572}\special{pa   166   641}\special{pa   180   711}%
\special{fp}%
\special{pa    40  -725}\special{pa    29  -652}\special{pa    17  -582}\special{pa     6  -514}%
\special{pa    -6  -450}\special{pa   -18  -388}\special{pa   -29  -328}\special{pa   -41  -272}%
\special{pa   -48  -237}%
\special{fp}%
\special{pa  -145   117}\special{pa  -147   123}\special{pa  -148   126}\special{pa  -155   142}%
\special{pa  -156   146}\special{pa  -162   160}\special{pa  -168   173}\special{pa  -180   198}%
\special{pa  -191   219}\special{pa  -203   238}\special{pa  -214   254}\special{pa  -226   268}%
\special{pa  -237   279}\special{pa  -249   287}\special{pa  -261   292}\special{pa  -272   295}%
\special{pa  -284   295}\special{pa  -295   292}\special{pa  -307   287}\special{pa  -318   279}%
\special{pa  -330   268}\special{pa  -342   255}\special{pa  -353   239}\special{pa  -365   220}%
\special{pa  -376   198}\special{pa  -388   174}\special{pa  -399   147}\special{pa  -411   118}%
\special{pa  -423    85}\special{pa  -434    50}\special{pa  -446    13}\special{pa  -457   -28}%
\special{pa  -469   -71}\special{pa  -480  -117}\special{pa  -492  -165}\special{pa  -504  -216}%
\special{pa  -515  -270}\special{pa  -527  -327}\special{pa  -538  -386}%
\special{fp}%
\special{pa   165  -962}\special{pa   153  -889}\special{pa   142  -819}\special{pa   130  -751}%
\special{pa   119  -687}\special{pa   107  -625}\special{pa    95  -565}\special{pa    84  -509}%
\special{pa    72  -455}\special{pa    61  -403}\special{pa    49  -355}\special{pa    38  -309}%
\special{pa    26  -266}\special{pa    14  -225}\special{pa     3  -188}\special{pa    -9  -153}%
\special{pa   -17  -129}%
\special{fp}%
\special{pa   -96    25}\special{pa  -101    31}\special{pa  -108    37}\special{pa  -113    42}%
\special{pa  -124    50}\special{pa  -136    55}\special{pa  -148    58}\special{pa  -159    58}%
\special{pa  -171    55}\special{pa  -182    50}\special{pa  -194    42}\special{pa  -206    31}%
\special{pa  -217    18}\special{pa  -229     2}\special{pa  -240   -17}\special{pa  -252   -39}%
\special{pa  -263   -63}\special{pa  -275   -90}\special{pa  -287  -119}\special{pa  -298  -152}%
\special{pa  -310  -187}\special{pa  -321  -224}\special{pa  -333  -265}\special{pa  -344  -308}%
\special{pa  -356  -354}\special{pa  -368  -402}\special{pa  -379  -453}\special{pa  -391  -507}%
\special{pa  -402  -564}\special{pa  -414  -623}%
\special{fp}%
\special{pa   289 -1012}\special{pa   278  -939}\special{pa   266  -869}\special{pa   255  -801}%
\special{pa   243  -736}\special{pa   231  -674}\special{pa   220  -615}\special{pa   208  -559}%
\special{pa   197  -505}\special{pa   185  -453}\special{pa   174  -405}\special{pa   162  -359}%
\special{pa   150  -316}\special{pa   139  -275}\special{pa   127  -238}\special{pa   116  -202}%
\special{pa   104  -170}\special{pa    93  -140}\special{pa    81  -113}\special{pa    69   -89}%
\special{pa    58   -68}\special{pa    46   -49}\special{pa    35   -32}\special{pa    23   -19}%
\special{pa    14   -10}%
\special{fp}%
\special{pa   -47     5}\special{pa   -54     2}\special{pa   -58     0}\special{pa   -69    -8}%
\special{pa   -81   -19}\special{pa   -93   -32}\special{pa  -104   -48}\special{pa  -116   -67}%
\special{pa  -127   -89}\special{pa  -139  -113}\special{pa  -150  -140}\special{pa  -162  -169}%
\special{pa  -174  -202}\special{pa  -185  -237}\special{pa  -197  -274}\special{pa  -208  -315}%
\special{pa  -220  -358}\special{pa  -231  -404}\special{pa  -243  -452}\special{pa  -255  -503}%
\special{pa  -266  -557}\special{pa  -278  -614}\special{pa  -289  -673}%
\special{fp}%
\special{pa   414  -874}\special{pa   402  -801}\special{pa   391  -731}\special{pa   379  -664}%
\special{pa   368  -599}\special{pa   356  -537}\special{pa   344  -478}\special{pa   333  -421}%
\special{pa   321  -367}\special{pa   310  -316}\special{pa   298  -267}\special{pa   287  -222}%
\special{pa   275  -178}\special{pa   263  -138}\special{pa   252  -100}\special{pa   240   -65}%
\special{pa   229   -33}\special{pa   217    -3}\special{pa   206    24}\special{pa   194    48}%
\special{pa   182    70}\special{pa   171    89}\special{pa   159   105}\special{pa   148   118}%
\special{pa   136   129}\special{pa   124   137}\special{pa   113   143}\special{pa   101   145}%
\special{pa    90   145}\special{pa    78   143}\special{pa    67   137}\special{pa    55   129}%
\special{pa    45   120}%
\special{fp}%
\special{pa     4    61}\special{pa     1    57}\special{pa    -0    53}\special{pa    -3    49}%
\special{pa    -3    49}\special{pa    -4    46}\special{pa    -9    36}\special{pa   -14    25}%
\special{pa   -26    -2}\special{pa   -38   -32}\special{pa   -49   -64}\special{pa   -61   -99}%
\special{pa   -72  -137}\special{pa   -84  -177}\special{pa   -95  -220}\special{pa  -107  -266}%
\special{pa  -119  -315}\special{pa  -130  -366}\special{pa  -142  -420}\special{pa  -153  -476}%
\special{pa  -165  -535}%
\special{fp}%
\special{pa   538  -550}\special{pa   527  -477}\special{pa   515  -407}\special{pa   504  -339}%
\special{pa   492  -275}\special{pa   480  -213}\special{pa   469  -153}\special{pa   457   -97}%
\special{pa   446   -43}\special{pa   434     8}\special{pa   423    57}\special{pa   411   103}%
\special{pa   399   146}\special{pa   388   186}\special{pa   376   224}\special{pa   365   259}%
\special{pa   353   292}\special{pa   342   321}\special{pa   330   348}\special{pa   318   373}%
\special{pa   307   394}\special{pa   295   413}\special{pa   284   429}\special{pa   272   443}%
\special{pa   261   454}\special{pa   249   462}\special{pa   237   467}\special{pa   226   470}%
\special{pa   214   470}\special{pa   203   467}\special{pa   191   462}\special{pa   180   454}%
\special{pa   168   443}\special{pa   156   430}\special{pa   145   414}\special{pa   133   395}%
\special{pa   122   373}\special{pa   110   349}\special{pa    99   322}\special{pa    87   292}%
\special{pa    76   262}%
\special{fp}%
\special{pa    55   197}\special{pa    53   191}\special{pa    52   188}\special{pa    52   187}%
\special{pa    49   176}\special{pa    46   166}\special{pa    44   157}\special{pa    41   147}%
\special{pa    41   147}\special{pa    40   143}\special{pa    29   104}\special{pa    18    58}%
\special{pa     6    10}\special{pa    -6   -41}\special{pa   -17   -95}\special{pa   -29  -152}%
\special{pa   -40  -210}%
\special{fp}%
\special{pn 8}%
\special{pa -85 -297}\special{pa -83 -304}\special{fp}\special{pa -78 -324}\special{pa -77 -331}\special{fp}%
\special{pn 8}%
\special{pa -84 -304}\special{pa -85 -297}\special{fp}\special{pa -90 -265}\special{pa -91 -257}\special{fp}%
\special{pa -96 -226}\special{pa -97 -218}\special{fp}\special{pa -103 -186}\special{pa -104 -178}\special{fp}%
\special{pa -109 -147}\special{pa -111 -139}\special{fp}\special{pa -116 -107}\special{pa -117 -99}\special{fp}%
\special{pa -123 -68}\special{pa -124 -60}\special{fp}\special{pa -130 -28}\special{pa -131 -21}\special{fp}%
\special{pa -137 11}\special{pa -139 19}\special{fp}\special{pa -145 50}\special{pa -146 58}\special{fp}%
\special{pa -152 89}\special{pa -154 97}\special{fp}\special{pa -160 129}\special{pa -162 136}\special{fp}%
\special{pa -168 168}\special{pa -170 176}\special{fp}\special{pa -177 207}\special{pa -179 215}\special{fp}%
\special{pa -186 246}\special{pa -187 254}\special{fp}\special{pa -195 285}\special{pa -197 292}\special{fp}%
\special{pn 8}%
\special{pn 8}%
\special{pa 597 -349}\special{pa 590 -345}\special{fp}\special{pa 562 -329}\special{pa 555 -325}\special{fp}%
\special{pa 528 -309}\special{pa 521 -305}\special{fp}\special{pa 493 -289}\special{pa 486 -285}\special{fp}%
\special{pa 458 -268}\special{pa 452 -264}\special{fp}\special{pa 424 -248}\special{pa 417 -244}\special{fp}%
\special{pa 389 -228}\special{pa 382 -224}\special{fp}\special{pa 355 -208}\special{pa 348 -204}\special{fp}%
\special{pa 320 -187}\special{pa 313 -183}\special{fp}\special{pa 286 -167}\special{pa 279 -163}\special{fp}%
\special{pa 251 -147}\special{pa 244 -143}\special{fp}\special{pa 217 -127}\special{pa 210 -123}\special{fp}%
\special{pa 182 -107}\special{pa 175 -103}\special{fp}\special{pa 148 -86}\special{pa 141 -82}\special{fp}%
\special{pa 113 -66}\special{pa 106 -62}\special{fp}\special{pa 79 -46}\special{pa 72 -42}\special{fp}%
\special{pa 44 -26}\special{pa 37 -22}\special{fp}\special{pa 9 -6}\special{pa 3 -2}\special{fp}%
\special{pa -25 15}\special{pa -32 19}\special{fp}\special{pn 8}%
\special{pa -601 -211}\special{pa -593 -208}\special{fp}\special{pa -564 -198}\special{pa -557 -196}\special{fp}%
\special{pa -528 -186}\special{pa -521 -183}\special{fp}\special{pa -492 -173}\special{pa -485 -170}\special{fp}%
\special{pa -456 -160}\special{pa -448 -157}\special{fp}\special{pa -420 -147}\special{pa -412 -145}\special{fp}%
\special{pa -384 -135}\special{pa -376 -132}\special{fp}\special{pa -347 -122}\special{pa -340 -119}\special{fp}%
\special{pa -311 -109}\special{pa -304 -107}\special{fp}\special{pa -275 -97}\special{pa -267 -94}\special{fp}%
\special{pa -239 -84}\special{pa -231 -81}\special{fp}\special{pa -203 -71}\special{pa -195 -69}\special{fp}%
\special{pa -166 -58}\special{pa -159 -56}\special{fp}\special{pa -130 -46}\special{pa -123 -43}\special{fp}%
\special{pa -94 -33}\special{pa -87 -30}\special{fp}\special{pa -58 -20}\special{pa -50 -18}\special{fp}%
\special{pa -22 -8}\special{pa -14 -5}\special{fp}\special{pa 14 5}\special{pa 22 8}\special{fp}%
\special{pn 8}%
\special{pa 0 59}\special{pa 0 51}\special{fp}\special{pa 0 19}\special{pa 0 11}\special{fp}%
\special{pa 0 -21}\special{pa 0 -29}\special{fp}\special{pa 0 -61}\special{pa 0 -69}\special{fp}%
\special{pn 8}%
\special{pn 8}%
\special{pa -182 228}\special{pa -180 220}\special{fp}\special{pa -172 189}\special{pa -170 182}\special{fp}%
\special{pa -163 151}\special{pa -161 144}\special{fp}\special{pa -153 113}\special{pa -151 106}\special{fp}%
\special{pa -143 75}\special{pa -141 68}\special{fp}\special{pa -132 37}\special{pa -130 30}\special{fp}%
\special{pa -122 0}\special{pa -119 -8}\special{fp}\special{pa -110 -38}\special{pa -108 -46}\special{fp}%
\special{pa -99 -76}\special{pa -96 -83}\special{fp}\special{pa -87 -113}\special{pa -84 -121}\special{fp}%
\special{pa -74 -150}\special{pa -71 -158}\special{fp}\special{pa -61 -187}\special{pa -58 -195}\special{fp}%
\special{pa -47 -224}\special{pa -44 -232}\special{fp}\special{pn 8}%
\special{pn 8}%
\special{pa -137 98}\special{pa -134 91}\special{fp}\special{pa -122 61}\special{pa -118 53}\special{fp}%
\special{pa -105 24}\special{pa -101 16}\special{fp}\special{pa -86 -13}\special{pa -82 -20}\special{fp}%
\special{pa -66 -48}\special{pa -61 -55}\special{fp}\special{pa -43 -82}\special{pa -38 -88}\special{fp}%
\special{pa -16 -113}\special{pa -11 -119}\special{fp}\special{pn 8}%
\special{pn 8}%
\special{pa -79 8}\special{pa -71 6}\special{fp}\special{pa -32 -5}\special{pa -24 -5}\special{fp}%
\special{pa 15 8}\special{pa 23 10}\special{fp}\special{pn 8}%
\special{pn 8}%
\special{pa -11 36}\special{pa -7 43}\special{fp}\special{pa 11 70}\special{pa 15 77}\special{fp}%
\special{pa 31 106}\special{pa 35 113}\special{fp}\special{pa 49 142}\special{pa 53 149}\special{fp}%
\special{pn 8}%
\special{pn 8}%
\special{pa 55 196}\special{pa 57 204}\special{fp}\special{pa 64 228}\special{pa 66 236}\special{fp}%
\special{pa 73 260}\special{pa 76 268}\special{fp}\special{pa 82 293}\special{pa 84 300}\special{fp}%
\special{pn 8}%
\special{pn 8}%
\special{pa -47 -241}\special{pa -49 -233}\special{fp}\special{pa -56 -201}\special{pa -58 -193}\special{fp}%
\special{pa -65 -161}\special{pa -67 -153}\special{fp}\special{pa -75 -121}\special{pa -76 -114}\special{fp}%
\special{pa -84 -82}\special{pa -86 -74}\special{fp}\special{pa -95 -43}\special{pa -97 -35}\special{fp}%
\special{pa -106 -3}\special{pa -108 4}\special{fp}\special{pa -118 36}\special{pa -120 43}\special{fp}%
\special{pa -130 75}\special{pa -133 82}\special{fp}\special{pa -143 113}\special{pa -146 120}\special{fp}%
\special{pn 8}%
\special{pn 8}%
\special{pa -16 -133}\special{pa -19 -126}\special{fp}\special{pa -31 -92}\special{pa -34 -85}\special{fp}%
\special{pa -49 -52}\special{pa -52 -45}\special{fp}\special{pa -69 -14}\special{pa -73 -7}\special{fp}%
\special{pa -94 22}\special{pa -99 28}\special{fp}\special{pn 8}%
\special{pn 8}%
\special{pa 17 -12}\special{pa 11 -8}\special{fp}\special{pa -10 5}\special{pa -18 7}\special{fp}%
\special{pa -42 6}\special{pa -50 5}\special{fp}\special{pn 8}%
\special{pn 8}%
\special{pa 48 123}\special{pa 43 117}\special{fp}\special{pa 25 95}\special{pa 20 89}\special{fp}%
\special{pa 6 64}\special{pa 2 57}\special{fp}\special{pn 8}%
\special{pn 8}%
\special{pa 77 266}\special{pa 75 258}\special{fp}\special{pa 67 234}\special{pa 64 226}\special{fp}%
\special{pa 56 201}\special{pa 54 193}\special{fp}\special{pn 8}%
\end{picture}}%
\end{center}

\item[Remark]Wires can be added if necessary with command \verb|Wireparadata| as seen in the upper right side figure.
The line-style also can be changed.

See \verb|KeTCindyreferenceE.pdf| or \verb|samples/s09surfaceC| for more information.
\end{description}

\subsection{Generating Files in Obj Format}

\ketcindy\ can generate files of 3D figures in obj format. Moreover, \ketcindy\ also can call Meshlab which is a 3D viewer.

\begin{description}
\item[Examples]\mbox{}\\
\verb|Xyzax3data("","x=[-5,5]","y=[-5,5]","z=[-5,5]");|\\
\verb|fd=["p","x=4*sin(V)*cos(U)","y=4*sin(V)*sin(U)","z=4*cos(V)",|\\
\verb|     "U=[pi/2,4*pi/2]","V=[0,pi]","we"];|\\
\verb|Mkobjcmd("1",fd,[40,40,"-"]);|\\
\verb|Mkobjcrvcmd("2","ax3d",[0.05,"xy"]);|\\
\verb|Mkobjsymbcmd("x",0.2,0,[0,1,0],[5.2,0,0]);|\\
\verb|Mkobjsymbcmd("y",0.2,0,[1,0,0],[0,5.2,0]);|\\
\verb|Mkobjsymbcmd("z",0.2,0,[0,1,0],[0,0,5.2]);|\\
\verb|SetObj();|
\item[Remark]See \verb|KeTCindyreferenceE.pdf| or \verb|samples/s13meshlab| for more information.
\end{description}

% -------------- Making slide --------------

\newpage

\section{Making Slides}

\subsection{Overall Flow}

\ketcindy\ has functions to make slides for presentation. 
For this, follow steps below:
\begin{enumerate}[1)]
\item Copy and rename \verb|template2slide.cdy| in ketcindy folder to a work folder and double-click the file (here, we call sample.cdy). Then the followings will appear.
\begin{center}
\includegraphics[bb=0.00 0.00 743.00 632.00,height=60mm]{fig/slidescreen.pdf}\hspace{5mm}
\includegraphics[bb=0.00 0.00 802.00 633.00,height=60mm]{fig/slidescript.pdf}
\end{center}
\item Edit \verb|"Settitle"|, for example,\\
\verb|  Settitle([|\\
\verb|     "s{60}{20}{How To Use}",|\\
\verb|     "s{60}{50}{a \ketcindy\ member}",|\\
\verb|     "s{60}{60}{\ketcindy\ project}",|\\
\verb|     "s{60}{70}{Aug. 20th}"|\\
\verb|    ],["Color=[1,1,0,0]"]);|
\item Press button \verb|"Title"|, then the title page will be displayed. At the same time,  text file \verb|"sample.txt"| will be created if it does not exist. This \verb|sample.txt| is a template file for making slides.
\item Press button \verb|"Slide"|, then \ketcindy\ will make \verb|sample.tex| from \verb|sample.txt|, 
typeset it, and display \verb|sample.pdf| which contains slides for presentation.
\end{enumerate}
 
\subsection{Editing Text File}

\begin{enumerate}[1)]
\item Put \verb|//| at the last of each line.\\
\hspace*{10mm}Rm) Use \verb+||||+ for \verb|//|.
\item Commands are\\
\verb|    title::titleslidename(::wallpaper)//|\\
\verb|        Rem) Put only once at the first line.|\\
\verb|    main::(main title)//|\\
\verb|    new::(page title)//|\\
\verb|    enumerate//|\\
\verb|           =\begin{enumerate}|\\
\verb|        Rem) Add the option such as [(1)] using :: .|\\
\verb|    itemize//|\\
\verb|           =\begin{itemize}|\\
\verb|    layer::{xsize}{ysize}//|\\
\verb|           =\begin{layer}{xsize}{ysize}|\\
\verb|         Rem) "layer" is an environment defined in ketlayer.sty.|\\
\verb|    item::sentence//|\\
\verb|           =\item sentence|\\
\verb|    putnote::dir{xpos}{ypos}::filename(,scale)//|\\
\verb|           =putnotedir{xpos}{ypos}{\input{fig/filename}}||\\
\verb|         Rem) "putnote" is a command defined in ketlayer.sty|\\
\verb|    end//|\\
\verb|           =\end{itemize,enumerate,layer}|\\
\verb|    ...//|\\
\verb|          To insert a blank line.|\\
\verb|    Rem) Any other TeX command is available.|\\
\end{enumerate}

\subsection{Display of Page step by step}

\begin{enumerate}[1)]
\item Put just after new,\\
\verb|    %repeat=number of steps//|
\item Put at the head of each line as\\
\verb|    %[2,-]::sentence|\\
\verb|          display at all steps from 2|\\
\verb|    %[-,2]::sentence|\\
\verb|           display at all steps until 2|\\
\verb|    %[1..3,5]::sentence|\\
\verb|           display at steps of 1,2,3 and 5|
\item  Use \verb|%thin| to display with thin letters.\\
\verb|    %thin::[2,-]::sentence|
\item The dencity can be changed with Setslidebody or \verb|\setthin|.
\end{enumerate}

\subsection{Making Flip Animation}

\begin{enumerate}[1)]
\item Define function \verb|Mf(s)|, the state at s.
\item Put command \verb|Setpara| in the script editor as\\ 
\verb|    Setpara(subfolder,funcitonstr(mf(s)),range,options);|\\
\verb|        options=["m/r", "Div=25"];|
\item Describe in the text file as\\
\verb|    %repeat=, para=subfolder:{0}:s{60}{10}:input(:scale)//|
\item Press buttons \verb|ParaF| and \verb|Flip|, then \verb|subfolder| will be generated.
\item Press button \verb|Slide|.
\end{enumerate}

\subsection{Making Animation}

\begin{enumerate}[1)]
\item Add the following in the script editor\\ 
\verb|    Addpackage(["[dvipdfmx]{animate}"]);|
\item Add in the second option of Setpara,\\
\verb|    "Frate=num of frame in the second,"Scale=scale,"OpA=option of animation" |
\item Press buttons \verb|ParaF| and \verb|Anime|, then \verb|subfolder| will be generated.
\item Use \verb|\input|, not layer, to display.
\end{enumerate}

\subsection{Changing Style}

The default styles such as size and color of letters can be changed.
See \verb|KeTCindyReferenceE,pdf| or \verb|samples/s07slides|.

\end{document}