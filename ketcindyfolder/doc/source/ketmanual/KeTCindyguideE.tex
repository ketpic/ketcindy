\documentclass[papersize,a4paper,12pt]{article}
\usepackage{ketpic,ketlayer}
\usepackage{amsmath}
% \usepackage{amsmath,newtxmath}
%\usepackage[dvipdfmx]{graphicx,color}
\usepackage{graphicx,color}
\usepackage{wrapfig}
%\usepackage[dvipdfmx,bookmarks=false,colorlinks=true,linkcolor=blue]{hyperref}
\usepackage[bookmarks=false,colorlinks=true,linkcolor=blue]{hyperref}
\setmargin{20}{20}{15}{25}
\usepackage{setspace}
\usepackage{comment}
\usepackage{bm,enumerate}

%\newcommand{\cmd}[1]{
%\begin{center}{\bf\large #1}\end{center}
%\hypertarget{#1}{}
%}

\newenvironment{cmd}[2]{
\hypertarget{#2}{}
\begin{center}{\bf\large #1}\end{center}
\begin{description}
}{
\end{description}
\begin{flushright} \hyperlink{functionlist}{$\Rightarrow$Command List}\end{flushright}
}

% item command for this documentation
\newcommand{\itemket}[1]{
\item[\Ltab{27mm}{#1}]
}


\begin{document}
\title{Guide to \ketcindy}
\author{\ketcindy\ Project Team}
\maketitle

\begin{center}  - ver.3.2 -\end{center}

\hypertarget{index}{}
\tableofcontents

\newpage

\section{About \ketcindy}

\subsection{Overview}

\ketcindy\ is a library of Cindyscript 
which is a programming language of Cinderella. 
It converts the data computed 
for generating dynamic graphics on Cinderella 
into \TeX\ graphical codes. 
Synchronized use of 
interactive graphics capabilities of Cinderella 
and well-structured programming capabilities of Cindyscript 
enables ordinary \TeX\ users to efficiently embed 
high-quality graphics into \TeX\ documents. 
Moreover, the collaborative use of \ketcindy\ 
and other software such as R, Maxima and C 
has been enabled.

\begin{center}
%%% fig.tex 2016-4-30 21:58
%%% fig.sce 2016-4-30 21:58
{\unitlength=3.5mm%
\begin{picture}%
(  38.24000,  21.88000)( -24.72000, -11.80000)%
\special{pn 8}%
%
\special{pa 933 -397}\special{pa 929 -446}\special{pa 920 -494}\special{pa 905 -541}%
\special{pa 884 -586}\special{pa 858 -628}\special{pa 826 -665}\special{pa 790 -699}%
\special{pa 750 -728}\special{pa 707 -752}\special{pa 661 -770}\special{pa 614 -782}%
\special{pa 565 -788}\special{pa 516 -788}\special{pa 467 -782}\special{pa 419 -770}%
\special{pa 373 -752}\special{pa 330 -728}\special{pa 290 -699}\special{pa 254 -665}%
\special{pa 223 -628}\special{pa 196 -586}\special{pa 175 -541}\special{pa 160 -494}%
\special{pa 151 -446}\special{pa 148 -397}\special{pa 151 -348}\special{pa 160 -299}%
\special{pa 175 -252}\special{pa 196 -208}\special{pa 223 -166}\special{pa 254 -128}%
\special{pa 290 -94}\special{pa 330 -66}\special{pa 373 -42}\special{pa 419 -24}\special{pa 467 -11}%
\special{pa 516 -5}\special{pa 565 -5}\special{pa 614 -11}\special{pa 661 -24}\special{pa 707 -42}%
\special{pa 750 -66}\special{pa 790 -94}\special{pa 826 -128}\special{pa 858 -166}%
\special{pa 884 -208}\special{pa 905 -252}\special{pa 920 -299}\special{pa 929 -348}%
\special{pa 933 -397}%
\special{fp}%
\special{pa 1624 -628}\special{pa 1622 -665}\special{pa 1615 -700}\special{pa 1604 -735}%
\special{pa 1588 -768}\special{pa 1569 -799}\special{pa 1545 -827}\special{pa 1519 -852}%
\special{pa 1489 -873}\special{pa 1457 -891}\special{pa 1424 -904}\special{pa 1388 -913}%
\special{pa 1352 -918}\special{pa 1316 -918}\special{pa 1280 -913}\special{pa 1244 -904}%
\special{pa 1210 -891}\special{pa 1178 -873}\special{pa 1149 -852}\special{pa 1122 -827}%
\special{pa 1099 -799}\special{pa 1080 -768}\special{pa 1064 -735}\special{pa 1053 -700}%
\special{pa 1046 -665}\special{pa 1044 -628}\special{pa 1046 -592}\special{pa 1053 -556}%
\special{pa 1064 -522}\special{pa 1080 -489}\special{pa 1099 -458}\special{pa 1122 -430}%
\special{pa 1149 -405}\special{pa 1178 -383}\special{pa 1210 -366}\special{pa 1244 -352}%
\special{pa 1280 -343}\special{pa 1316 -339}\special{pa 1352 -339}\special{pa 1388 -343}%
\special{pa 1424 -352}\special{pa 1457 -366}\special{pa 1489 -383}\special{pa 1519 -405}%
\special{pa 1545 -430}\special{pa 1569 -458}\special{pa 1588 -489}\special{pa 1604 -522}%
\special{pa 1615 -556}\special{pa 1622 -592}\special{pa 1624 -628}%
\special{fp}%
\special{pa 1580 176}\special{pa 1577 128}\special{pa 1568 79}\special{pa 1553 33}%
\special{pa 1532 -11}\special{pa 1506 -53}\special{pa 1475 -90}\special{pa 1439 -124}%
\special{pa 1399 -153}\special{pa 1357 -176}\special{pa 1311 -194}\special{pa 1264 -207}%
\special{pa 1215 -213}\special{pa 1166 -213}\special{pa 1118 -207}\special{pa 1070 -194}%
\special{pa 1025 -176}\special{pa 982 -153}\special{pa 942 -124}\special{pa 906 -90}%
\special{pa 875 -53}\special{pa 849 -11}\special{pa 828 33}\special{pa 813 79}\special{pa 804 128}%
\special{pa 801 176}\special{pa 804 225}\special{pa 813 273}\special{pa 828 320}\special{pa 849 364}%
\special{pa 875 406}\special{pa 906 443}\special{pa 942 477}\special{pa 982 506}\special{pa 1025 529}%
\special{pa 1070 547}\special{pa 1118 559}\special{pa 1166 565}\special{pa 1215 565}%
\special{pa 1264 559}\special{pa 1311 547}\special{pa 1357 529}\special{pa 1399 506}%
\special{pa 1439 477}\special{pa 1475 443}\special{pa 1506 406}\special{pa 1532 364}%
\special{pa 1553 320}\special{pa 1568 273}\special{pa 1577 225}\special{pa 1580 176}%
\special{fp}%
\special{pn 32}%
\special{pa 784 -90}\special{pa 865 -39}%
\special{fp}%
\special{pn 8}%
\special{pn 32}%
\special{pa 931 -435}\special{pa 1058 -540}%
\special{fp}%
\special{pn 8}%
\special{pn 24}%
\special{pa 1225 -360}\special{pa 1204 -227}%
\special{fp}%
\special{pn 8}%
\special{pa 1189 -291}\special{pa 1202 -213}\special{pa 1237 -284}\special{pa 1213 -287}%
\special{pa 1189 -291}\special{sh 1}\special{ip}%
\special{pn 1}%
\special{pa 1189 -291}\special{pa 1202 -213}\special{pa 1237 -284}\special{pa 1213 -287}%
\special{pa 1189 -291}\special{pa 1202 -213}%
\special{fp}%
\special{pn 8}%
\special{pn 24}%
\special{pa 1271 -205}\special{pa 1305 -326}%
\special{fp}%
\special{pn 8}%
\special{pa 1312 -261}\special{pa 1309 -339}\special{pa 1265 -274}\special{pa 1288 -267}%
\special{pa 1312 -261}\special{sh 1}\special{ip}%
\special{pn 1}%
\special{pa 1312 -261}\special{pa 1309 -339}\special{pa 1265 -274}\special{pa 1288 -267}%
\special{pa 1312 -261}\special{pa 1309 -339}%
\special{fp}%
\special{pn 8}%
\special{pa 89 155}\special{pa 85 93}\special{pa 72 32}\special{pa 50 -27}\special{pa 19 -83}%
\special{pa -21 -136}\special{pa -68 -184}\special{pa -123 -228}\special{pa -183 -265}%
\special{pa -250 -297}\special{pa -320 -321}\special{pa -393 -338}\special{pa -469 -347}%
\special{pa -545 -349}\special{pa -621 -343}\special{pa -695 -330}\special{pa -766 -308}%
\special{pa -834 -280}\special{pa -896 -246}\special{pa -953 -205}\special{pa -1002 -159}%
\special{pa -1044 -108}\special{pa -1078 -53}\special{pa -1103 5}\special{pa -1119 65}%
\special{pa -1125 127}\special{pa -1122 188}\special{pa -1109 249}\special{pa -1086 308}%
\special{pa -1055 364}\special{pa -1016 417}\special{pa -968 466}\special{pa -914 509}%
\special{pa -853 547}\special{pa -787 578}\special{pa -716 602}\special{pa -643 619}%
\special{pa -567 628}\special{pa -491 630}\special{pa -416 624}\special{pa -341 611}%
\special{pa -270 590}\special{pa -203 562}\special{pa -140 527}\special{pa -84 486}%
\special{pa -34 440}\special{pa 8 389}\special{pa 42 334}\special{pa 67 276}\special{pa 82 216}%
\special{pa 89 155}%
\special{fp}%
\special{pa -2437 142}\special{pa -2441 96}\special{pa -2453 50}\special{pa -2471 7}%
\special{pa -2496 -35}\special{pa -2527 -73}\special{pa -2565 -108}\special{pa -2607 -139}%
\special{pa -2655 -166}\special{pa -2706 -188}\special{pa -2760 -204}\special{pa -2816 -215}%
\special{pa -2874 -220}\special{pa -2932 -220}\special{pa -2989 -213}\special{pa -3045 -201}%
\special{pa -3099 -184}\special{pa -3150 -161}\special{pa -3197 -134}\special{pa -3239 -102}%
\special{pa -3275 -66}\special{pa -3306 -27}\special{pa -3330 15}\special{pa -3348 59}%
\special{pa -3358 105}\special{pa -3361 151}\special{pa -3357 197}\special{pa -3346 242}%
\special{pa -3328 285}\special{pa -3302 327}\special{pa -3271 365}\special{pa -3234 400}%
\special{pa -3191 432}\special{pa -3144 458}\special{pa -3093 480}\special{pa -3039 496}%
\special{pa -2982 507}\special{pa -2925 512}\special{pa -2867 512}\special{pa -2809 506}%
\special{pa -2753 494}\special{pa -2699 476}\special{pa -2649 453}\special{pa -2602 426}%
\special{pa -2560 394}\special{pa -2523 358}\special{pa -2493 319}\special{pa -2468 277}%
\special{pa -2451 233}\special{pa -2440 188}\special{pa -2437 142}%
\special{fp}%
\special{pa -1461 901}\special{pa -1463 870}\special{pa -1472 839}\special{pa -1486 808}%
\special{pa -1505 779}\special{pa -1530 752}\special{pa -1560 727}\special{pa -1594 705}%
\special{pa -1633 685}\special{pa -1674 669}\special{pa -1719 656}\special{pa -1765 647}%
\special{pa -1813 641}\special{pa -1861 640}\special{pa -1908 643}\special{pa -1955 649}%
\special{pa -2000 659}\special{pa -2043 673}\special{pa -2082 691}\special{pa -2118 711}%
\special{pa -2149 734}\special{pa -2175 760}\special{pa -2197 788}\special{pa -2212 817}%
\special{pa -2222 848}\special{pa -2226 879}\special{pa -2224 911}\special{pa -2216 942}%
\special{pa -2202 972}\special{pa -2182 1001}\special{pa -2157 1028}\special{pa -2127 1053}%
\special{pa -2093 1076}\special{pa -2055 1095}\special{pa -2013 1112}\special{pa -1969 1124}%
\special{pa -1922 1134}\special{pa -1875 1139}\special{pa -1827 1140}\special{pa -1779 1138}%
\special{pa -1732 1131}\special{pa -1687 1121}\special{pa -1645 1107}\special{pa -1605 1090}%
\special{pa -1570 1069}\special{pa -1539 1046}\special{pa -1512 1020}\special{pa -1491 992}%
\special{pa -1475 963}\special{pa -1465 932}\special{pa -1461 901}%
\special{fp}%
\special{pa 201 1172}\special{pa 200 1142}\special{pa 191 1113}\special{pa 177 1084}%
\special{pa 156 1056}\special{pa 130 1030}\special{pa 98 1005}\special{pa 61 983}%
\special{pa 21 963}\special{pa -24 947}\special{pa -71 934}\special{pa -121 924}\special{pa -172 918}%
\special{pa -224 915}\special{pa -275 916}\special{pa -325 921}\special{pa -374 930}%
\special{pa -420 942}\special{pa -462 957}\special{pa -501 976}\special{pa -535 997}%
\special{pa -564 1020}\special{pa -587 1046}\special{pa -605 1074}\special{pa -616 1102}%
\special{pa -620 1132}\special{pa -619 1162}\special{pa -610 1191}\special{pa -596 1220}%
\special{pa -575 1248}\special{pa -549 1274}\special{pa -517 1299}\special{pa -480 1321}%
\special{pa -439 1341}\special{pa -395 1357}\special{pa -347 1370}\special{pa -298 1380}%
\special{pa -247 1386}\special{pa -195 1389}\special{pa -144 1388}\special{pa -94 1383}%
\special{pa -45 1374}\special{pa 1 1362}\special{pa 43 1347}\special{pa 82 1328}\special{pa 116 1307}%
\special{pa 145 1283}\special{pa 168 1258}\special{pa 186 1230}\special{pa 197 1202}%
\special{pa 201 1172}%
\special{fp}%
\special{pa 1001 926}\special{pa 999 888}\special{pa 992 849}\special{pa 978 812}%
\special{pa 959 776}\special{pa 935 742}\special{pa 905 711}\special{pa 872 684}\special{pa 834 659}%
\special{pa 792 639}\special{pa 749 623}\special{pa 703 612}\special{pa 655 605}\special{pa 608 604}%
\special{pa 560 607}\special{pa 513 615}\special{pa 468 627}\special{pa 426 644}\special{pa 386 666}%
\special{pa 350 691}\special{pa 319 720}\special{pa 292 751}\special{pa 270 786}\special{pa 254 822}%
\special{pa 244 860}\special{pa 239 898}\special{pa 241 937}\special{pa 249 975}\special{pa 262 1013}%
\special{pa 281 1048}\special{pa 305 1082}\special{pa 335 1113}\special{pa 369 1141}%
\special{pa 406 1165}\special{pa 448 1185}\special{pa 492 1201}\special{pa 538 1212}%
\special{pa 585 1219}\special{pa 633 1221}\special{pa 680 1218}\special{pa 727 1210}%
\special{pa 772 1197}\special{pa 814 1180}\special{pa 854 1159}\special{pa 890 1134}%
\special{pa 921 1105}\special{pa 948 1073}\special{pa 970 1039}\special{pa 986 1003}%
\special{pa 996 965}\special{pa 1001 926}%
\special{fp}%
\special{pa -768 1296}\special{pa -770 1262}\special{pa -780 1228}\special{pa -795 1195}%
\special{pa -816 1163}\special{pa -843 1133}\special{pa -875 1107}\special{pa -912 1082}%
\special{pa -953 1062}\special{pa -997 1045}\special{pa -1044 1031}\special{pa -1093 1022}%
\special{pa -1143 1018}\special{pa -1194 1017}\special{pa -1244 1021}\special{pa -1294 1029}%
\special{pa -1341 1041}\special{pa -1385 1058}\special{pa -1427 1078}\special{pa -1464 1101}%
\special{pa -1496 1127}\special{pa -1524 1156}\special{pa -1546 1188}\special{pa -1562 1220}%
\special{pa -1571 1254}\special{pa -1575 1289}\special{pa -1572 1323}\special{pa -1563 1357}%
\special{pa -1548 1391}\special{pa -1526 1422}\special{pa -1499 1452}\special{pa -1467 1479}%
\special{pa -1431 1503}\special{pa -1390 1523}\special{pa -1345 1540}\special{pa -1298 1554}%
\special{pa -1249 1563}\special{pa -1199 1567}\special{pa -1148 1568}\special{pa -1098 1564}%
\special{pa -1049 1556}\special{pa -1002 1544}\special{pa -957 1527}\special{pa -916 1507}%
\special{pa -879 1484}\special{pa -846 1458}\special{pa -819 1429}\special{pa -797 1398}%
\special{pa -781 1365}\special{pa -771 1331}\special{pa -768 1296}%
\special{fp}%
\special{pa -925 -621}\special{pa -930 -652}\special{pa -942 -682}\special{pa -960 -711}%
\special{pa -984 -738}\special{pa -1015 -763}\special{pa -1051 -785}\special{pa -1091 -805}%
\special{pa -1136 -821}\special{pa -1185 -834}\special{pa -1236 -843}\special{pa -1289 -848}%
\special{pa -1343 -849}\special{pa -1397 -847}\special{pa -1451 -840}\special{pa -1504 -830}%
\special{pa -1554 -816}\special{pa -1601 -799}\special{pa -1644 -779}\special{pa -1682 -756}%
\special{pa -1716 -730}\special{pa -1744 -702}\special{pa -1766 -673}\special{pa -1781 -643}%
\special{pa -1790 -612}\special{pa -1792 -581}\special{pa -1787 -550}\special{pa -1776 -520}%
\special{pa -1757 -491}\special{pa -1733 -464}\special{pa -1702 -439}\special{pa -1667 -416}%
\special{pa -1626 -397}\special{pa -1581 -381}\special{pa -1533 -368}\special{pa -1482 -359}%
\special{pa -1429 -354}\special{pa -1374 -352}\special{pa -1320 -355}\special{pa -1266 -361}%
\special{pa -1214 -371}\special{pa -1164 -385}\special{pa -1117 -402}\special{pa -1074 -423}%
\special{pa -1035 -446}\special{pa -1001 -471}\special{pa -973 -499}\special{pa -952 -528}%
\special{pa -936 -559}\special{pa -927 -590}\special{pa -925 -621}%
\special{fp}%
\special{pa -1457 142}\special{pa -1460 103}\special{pa -1470 65}\special{pa -1486 28}%
\special{pa -1508 -7}\special{pa -1535 -40}\special{pa -1568 -69}\special{pa -1605 -96}%
\special{pa -1646 -118}\special{pa -1690 -137}\special{pa -1737 -150}\special{pa -1786 -160}%
\special{pa -1836 -164}\special{pa -1886 -163}\special{pa -1936 -158}\special{pa -1985 -148}%
\special{pa -2031 -133}\special{pa -2075 -114}\special{pa -2115 -90}\special{pa -2152 -63}%
\special{pa -2184 -33}\special{pa -2210 0}\special{pa -2231 36}\special{pa -2246 73}%
\special{pa -2255 112}\special{pa -2258 150}\special{pa -2255 189}\special{pa -2245 227}%
\special{pa -2229 264}\special{pa -2207 299}\special{pa -2180 332}\special{pa -2147 362}%
\special{pa -2110 388}\special{pa -2069 410}\special{pa -2025 429}\special{pa -1978 442}%
\special{pa -1929 452}\special{pa -1879 456}\special{pa -1829 455}\special{pa -1779 450}%
\special{pa -1730 440}\special{pa -1684 425}\special{pa -1640 406}\special{pa -1599 382}%
\special{pa -1563 355}\special{pa -1531 325}\special{pa -1505 292}\special{pa -1484 256}%
\special{pa -1469 219}\special{pa -1460 181}\special{pa -1457 142}%
\special{fp}%
\special{pn 16}%
\special{pa 11 -94}\special{pa 183 -234}%
\special{fp}%
\special{pn 8}%
\special{pn 16}%
\special{pa 89 141}\special{pa 804 127}%
\special{fp}%
\special{pn 8}%
\special{pn 16}%
\special{pa -1291 143}\special{pa -1139 141}%
\special{fp}%
\special{pn 8}%
\special{pa -1199 166}\special{pa -1125 141}\special{pa -1200 117}\special{pa -1200 142}%
\special{pa -1199 166}\special{sh 1}\special{ip}%
\special{pn 1}%
\special{pa -1199 166}\special{pa -1125 141}\special{pa -1200 117}\special{pa -1200 142}%
\special{pa -1199 166}\special{pa -1125 141}%
\special{fp}%
\special{pn 8}%
\special{pn 16}%
\special{pa -1291 143}\special{pa -1443 146}%
\special{fp}%
\special{pn 8}%
\special{pa -1382 120}\special{pa -1457 146}\special{pa -1382 169}\special{pa -1382 145}%
\special{pa -1382 120}\special{sh 1}\special{ip}%
\special{pn 1}%
\special{pa -1382 120}\special{pa -1457 146}\special{pa -1382 169}\special{pa -1382 145}%
\special{pa -1382 120}\special{pa -1457 146}%
\special{fp}%
\special{pn 8}%
\special{pn 16}%
\special{pa -1282 585}\special{pa -983 469}%
\special{fp}%
\special{pn 8}%
\special{pa -1031 513}\special{pa -970 464}\special{pa -1049 468}\special{pa -1040 491}%
\special{pa -1031 513}\special{sh 1}\special{ip}%
\special{pn 1}%
\special{pa -1031 513}\special{pa -970 464}\special{pa -1049 468}\special{pa -1040 491}%
\special{pa -1031 513}\special{pa -970 464}%
\special{fp}%
\special{pn 8}%
\special{pn 16}%
\special{pa -1282 585}\special{pa -1580 701}%
\special{fp}%
\special{pn 8}%
\special{pa -1532 656}\special{pa -1593 706}\special{pa -1514 701}\special{pa -1523 678}%
\special{pa -1532 656}\special{sh 1}\special{ip}%
\special{pn 1}%
\special{pa -1532 656}\special{pa -1593 706}\special{pa -1514 701}\special{pa -1523 678}%
\special{pa -1532 656}\special{pa -1593 706}%
\special{fp}%
\special{pn 8}%
\special{pn 16}%
\special{pa -266 757}\special{pa -298 613}%
\special{fp}%
\special{pn 8}%
\special{pa -261 668}\special{pa -301 600}\special{pa -308 678}\special{pa -285 673}%
\special{pa -261 668}\special{sh 1}\special{ip}%
\special{pn 1}%
\special{pa -261 668}\special{pa -301 600}\special{pa -308 678}\special{pa -285 673}%
\special{pa -261 668}\special{pa -301 600}%
\special{fp}%
\special{pn 8}%
\special{pn 16}%
\special{pa -266 757}\special{pa -234 902}%
\special{fp}%
\special{pn 8}%
\special{pa -271 847}\special{pa -231 915}\special{pa -224 837}\special{pa -248 842}%
\special{pa -271 847}\special{sh 1}\special{ip}%
\special{pn 1}%
\special{pa -271 847}\special{pa -231 915}\special{pa -224 837}\special{pa -248 842}%
\special{pa -271 847}\special{pa -231 915}%
\special{fp}%
\special{pn 8}%
\special{pn 16}%
\special{pa 173 549}\special{pa 342 681}%
\special{fp}%
\special{pn 8}%
\special{pa 279 662}\special{pa 353 689}\special{pa 309 624}\special{pa 294 643}\special{pa 279 662}%
\special{sh 1}\special{ip}%
\special{pn 1}%
\special{pa 279 662}\special{pa 353 689}\special{pa 309 624}\special{pa 294 643}\special{pa 279 662}%
\special{pa 353 689}%
\special{fp}%
\special{pn 8}%
\special{pn 16}%
\special{pa 173 549}\special{pa 4 417}%
\special{fp}%
\special{pn 8}%
\special{pa 67 436}\special{pa -7 409}\special{pa 37 474}\special{pa 52 455}\special{pa 67 436}%
\special{sh 1}\special{ip}%
\special{pn 1}%
\special{pa 67 436}\special{pa -7 409}\special{pa 37 474}\special{pa 52 455}\special{pa 67 436}%
\special{pa -7 409}%
\special{fp}%
\special{pn 8}%
\special{pn 24}%
\special{pa -2258 143}\special{pa -2423 143}%
\special{fp}%
\special{pn 8}%
\special{pa -2362 119}\special{pa -2437 143}\special{pa -2362 168}\special{pa -2362 143}%
\special{pa -2362 119}\special{sh 1}\special{ip}%
\special{pn 1}%
\special{pa -2362 119}\special{pa -2437 143}\special{pa -2362 168}\special{pa -2362 143}%
\special{pa -2362 119}\special{pa -2437 143}%
\special{fp}%
\special{pn 8}%
\special{pn 16}%
\special{pa -909 -237}\special{pa -1052 -419}%
\special{fp}%
\special{pn 8}%
\special{pa -995 -386}\special{pa -1061 -430}\special{pa -1034 -356}\special{pa -1015 -371}%
\special{pa -995 -386}\special{sh 1}\special{ip}%
\special{pn 1}%
\special{pa -995 -386}\special{pa -1061 -430}\special{pa -1034 -356}\special{pa -1015 -371}%
\special{pa -995 -386}\special{pa -1061 -430}%
\special{fp}%
\special{pn 8}%
\special{pn 16}%
\special{pa -907 808}\special{pa -775 597}%
\special{fp}%
\special{pn 8}%
\special{pa -787 662}\special{pa -767 585}\special{pa -828 636}\special{pa -807 649}%
\special{pa -787 662}\special{sh 1}\special{ip}%
\special{pn 1}%
\special{pa -787 662}\special{pa -767 585}\special{pa -828 636}\special{pa -807 649}%
\special{pa -787 662}\special{pa -767 585}%
\special{fp}%
\special{pn 8}%
\special{pn 16}%
\special{pa -907 808}\special{pa -1040 1019}%
\special{fp}%
\special{pn 8}%
\special{pa -1028 954}\special{pa -1047 1031}\special{pa -987 980}\special{pa -1007 967}%
\special{pa -1028 954}\special{sh 1}\special{ip}%
\special{pn 1}%
\special{pa -1028 954}\special{pa -1047 1031}\special{pa -987 980}\special{pa -1007 967}%
\special{pa -1028 954}\special{pa -1047 1031}%
\special{fp}%
\special{pn 8}%
\Large%
\settowidth{\Width}{KeTCindy概念図}\setlength{\Width}{-0.5\Width}%
\settoheight{\Height}{KeTCindy概念図}\settodepth{\Depth}{KeTCindy概念図}\setlength{\Height}{-0.5\Height}\setlength{\Depth}{0.5\Depth}\addtolength{\Height}{\Depth}%
\put(-19.0000,6.0000){\hspace*{\Width}\raisebox{\Height}{KeTCindy概念図}}%
%
%
\normalsize%
\settowidth{\Width}{CindyScript}\setlength{\Width}{-0.5\Width}%
\settoheight{\Height}{CindyScript}\settodepth{\Depth}{CindyScript}\setlength{\Height}{-0.5\Height}\setlength{\Depth}{0.5\Depth}\addtolength{\Height}{\Depth}%
\put(3.9200,2.8800){\hspace*{\Width}\raisebox{\Height}{CindyScript}}%
%
%
\settowidth{\Width}{CindyLab}\setlength{\Width}{-0.5\Width}%
\settoheight{\Height}{CindyLab}\settodepth{\Depth}{CindyLab}\setlength{\Height}{-0.5\Height}\setlength{\Depth}{0.5\Depth}\addtolength{\Height}{\Depth}%
\put(9.6800,4.5600){\hspace*{\Width}\raisebox{\Height}{CindyLab}}%
%
%
\settowidth{\Width}{作図}\setlength{\Width}{-0.5\Width}%
\settoheight{\Height}{作図}\settodepth{\Depth}{作図}\setlength{\Height}{-0.5\Height}\setlength{\Depth}{0.5\Depth}\addtolength{\Height}{\Depth}%
\put(8.6400,-1.2800){\hspace*{\Width}\raisebox{\Height}{作図}}%
%
%
\settowidth{\Width}{KeTpic}\setlength{\Width}{0\Width}%
\settoheight{\Height}{KeTpic}\settodepth{\Depth}{KeTpic}\setlength{\Height}{-\Height}%
\put(-18.6300,0.7900){\hspace*{\Width}\raisebox{\Height}{KeTpic}}%
%
%
\settowidth{\Width}{R}\setlength{\Width}{-0.5\Width}%
\settoheight{\Height}{R}\settodepth{\Depth}{R}\setlength{\Height}{-0.5\Height}\setlength{\Depth}{0.5\Depth}\addtolength{\Height}{\Depth}%
\put(-13.3800,-6.4600){\hspace*{\Width}\raisebox{\Height}{R}}%
%
%
\settowidth{\Width}{Maxima}\setlength{\Width}{-0.5\Width}%
\settoheight{\Height}{Maxima}\settodepth{\Depth}{Maxima}\setlength{\Height}{-0.5\Height}\setlength{\Depth}{0.5\Depth}\addtolength{\Height}{\Depth}%
\put(-1.5200,-8.3600){\hspace*{\Width}\raisebox{\Height}{Maxima}}%
%
%
\settowidth{\Width}{Risa/Asir}\setlength{\Width}{-0.5\Width}%
\settoheight{\Height}{Risa/Asir}\settodepth{\Depth}{Risa/Asir}\setlength{\Height}{-0.5\Height}\setlength{\Depth}{0.5\Depth}\addtolength{\Height}{\Depth}%
\put(4.5000,-6.6200){\hspace*{\Width}\raisebox{\Height}{Risa/Asir}}%
%
%
\settowidth{\Width}{FriCAS}\setlength{\Width}{0\Width}%
\settoheight{\Height}{FriCAS}\settodepth{\Depth}{FriCAS}\setlength{\Height}{-0.5\Height}\setlength{\Depth}{0.5\Depth}\addtolength{\Height}{\Depth}%
\put(-10.0900,-9.3600){\hspace*{\Width}\raisebox{\Height}{FriCAS}}%
%
%
\settowidth{\Width}{MeshLab}\setlength{\Width}{0\Width}%
\settoheight{\Height}{MeshLab}\settodepth{\Depth}{MeshLab}\setlength{\Height}{-0.5\Height}\setlength{\Depth}{0.5\Depth}\addtolength{\Height}{\Depth}%
\put(-11.6900,4.2400){\hspace*{\Width}\raisebox{\Height}{MeshLab}}%
%
%
\Large%
\settowidth{\Width}{KeTCindy}\setlength{\Width}{-0.5\Width}%
\settoheight{\Height}{KeTCindy}\settodepth{\Depth}{KeTCindy}\setlength{\Height}{-0.5\Height}\setlength{\Depth}{0.5\Depth}\addtolength{\Height}{\Depth}%
\put(-3.7600,-1.0200){\hspace*{\Width}\raisebox{\Height}{KeTCindy}}%
%
%
\settowidth{\Width}{TeX}\setlength{\Width}{-0.5\Width}%
\settoheight{\Height}{TeX}\settodepth{\Depth}{TeX}\setlength{\Height}{-0.5\Height}\setlength{\Depth}{0.5\Depth}\addtolength{\Height}{\Depth}%
\put(-21.0400,-1.0600){\hspace*{\Width}\raisebox{\Height}{TeX}}%
%
%
\settowidth{\Width}{Cinderella}\setlength{\Width}{0\Width}%
\settoheight{\Height}{Cinderella}\settodepth{\Depth}{Cinderella}\setlength{\Height}{-0.5\Height}\setlength{\Depth}{0.5\Depth}\addtolength{\Height}{\Depth}%
\put(2.0500,7.0000){\hspace*{\Width}\raisebox{\Height}{Cinderella}}%
%
%
\settowidth{\Width}{Scilab}\setlength{\Width}{0\Width}%
\settoheight{\Height}{Scilab}\settodepth{\Depth}{Scilab}\setlength{\Height}{-0.5\Height}\setlength{\Depth}{0.5\Depth}\addtolength{\Height}{\Depth}%
\put(-15.2700,-1.0800){\hspace*{\Width}\raisebox{\Height}{Scilab}}%
%
%
\end{picture}}%
\end{center}

Firstly, dynamic figure is generated on Cinderella. 
Secondly, \ketcindy\ generates 
a source file of R and makes R execute it 
for the generation of \TeX\ graphical codes. 
Thirdly, those codes are formatted into 
\TeX\ file which is input in the targetting \TeX\ document 
via the command \verb|\input|. 
Finally, usual compilation procedure of \TeX\ results in 
the generation of final PDF output 
including the corresponding figure. 
A batch file \verb|kc.bat| for Windows 
or a shell file \verb|kc.sh| for Mac or Linux 
is generated via \ketcindy\ 
in order to batch-process all the steps 
from the second to the last. 
Also by using these files, 
collaboration of Cinderella and other software 
as shown in the schematic diagram above 
is processed. 

Summarizingly, specific steps to generate a \TeX\ figure 
are listed as follows.

\begin{enumerate}[(1)]
\item 
Generate the needed geometric elements 
on the Euclidean view of Cinderella 
using its drawing tools. 
These elements can be moved interactively. 

\hspace{30mm}\includegraphics[bb=0.00 0.00 408.02 347.02,width=6cm]{Fig/incenter01.pdf}

\item 
Input the \ketcindy\ codes into Cindyscript editor 
to specify the graphical elements to be displayed 
in \TeX\ final output. 
Also \ketcindy\ codes are used 
to generate supplementary graphical elements 
and handle them. 

\hspace{10mm}\includegraphics[bb=0.00 0.00 811.04 257.01,width=12cm]{Fig/incenter02E.pdf}

In this stage, the programming capabilities 
inherently implemented to Cindyscript can be used simultaneously. 
Execute the whole program by clicking the "Run" button. 
For more details, see section 3. 

\item 
Click the button named \verb|Figures| in Euclidean view 
to automatically generate the following files 
in the folder named "fig". 
Here, "incenter" is the name specified 
via the command \verb|Setfiles("incenter")| 
in step (2). 

\begin{tabbing}
12\=1234567897890123456\=\kill

 \> \verb|kc|.sh or \verb|kc.bat| \> shell script file(Mac) or batch file(Windows) \\
 \> \verb|incenter.r| \> \\
 \> \verb|incenter.tex| \> \TeX\ file composed of graphical codes\\
 \> \verb|incentermain.aux| \> \\
 \> \verb|incentermain.log| \> \\
 \> \verb|incentermain.pdf| \> PDF file to display the resulting graphical image\\
 \> \verb|incentermain.tex| \> \TeX\ file temporarily used to generate 
the file \verb|incentermain.pdf|
\end{tabbing}

Subsequently, the file \verb|incentermain.pdf| 
is automatically displayed as shown below. 

\hspace{30mm}\includegraphics[bb=0.00 0.00 348.02 284.51,width=6cm]{Fig/incenter03.pdf}

We can manipulate this final output 
by modifying the inputs in steps (1) and (2) 
before processing the step (3) again. 

\vspace{\baselineskip}
\item  
Using \ketpic\ package of \TeX , 
\verb|incenter.tex| can be read 
into the targetting \TeX\ document 
via the command 
\begin{center}
\verb|\input{incenter}|
\end{center} 
Then the same figure is embedded in the targetting PDF output. 

\end{enumerate}


\newpage

\subsection{The drawing procedure of \ketcindy}

\subsubsection{Geometric figure}

\subsubsection{Graph of function}

\subsubsection{Spatial figure}

\subsubsection{Table}

\subsubsection{Collaboration with other software}


\newpage

\subsection{Plotting data}
Here we call the data computed 
to generate the graphs of functions and geometric elements 
"Plotting data" which is abbreviated as PD. 
The PD to draw segment is the list of coordinates 
of its two endpoints. 
For example, 
when the coordinates of the points A and B 
are (1, 1) and (3, 2) respectively, 
PD of the segment AB named \verb|Listplot ([A,B])| 
is stored in the form \verb|[[1,1],[3,2]]|. 
Also the PD to draw a curve is the collection of 
those for drawing small segments 
which connect contiguous dividing points of the curve. 
PD are automatically given names via \ketcindy\ 
following the rules below.

\begin{itemize}
\item 
The beginning part of the PD's name 
depends on the kind of the corresponding graphical element. 
For instance, 
\verb|sg| is associated to segments and 
\verb|cr| is associated to circles. 

\item 
When some extra name is specified 
as the first argument in the definition of PD, 
it is added to the beginning part given above. 
For instance, the PD defined below 
is given the name \verb|sg1|. 
\begin{center}
\verb|Listplot("1",[[0,0],[1,2]]);| 
\end{center}

\item 
When the extra name is not needed, 
the names of the points are added 
to the beginning part given above. 
For instance, the PD defined below  
is given the name \verb|sgABC|. 
\begin{center}
\verb|Listplot([A,B,C]);|
\end{center}

\end{itemize}

\noindent 
Once PD are generated, 
their names are displayed on the console view of Cinderella. 
For instance, when the PD named \verb|sgABCA| is generated, 
the corresponding message is displayed as shown below. 

\begin{center}
\includegraphics[bb=0.00 0.00 298.02 115.01,width=6cm]{Fig/pdtoconsole.pdf}
\end{center}
Also the content of PD is displayed 
via the function \verb|println()| of Cindyscript. 
For instance, inputting the command 
\verb|println(sgABCA)| makes the following list displayed. 
\begin{center}
\verb| [[1,3],[-1,0],[3,0],[1,3]] |
\end{center}
This list is composed of the coordinates of the points A, B, and C. 

These names of PD are used 
when the corresponding PD need to be transformed. 
For instance, 
PD to draw the parallel transport of the segment AB 
is generated via the \ketcindy\ command 
\begin{center}
\verb|Translatedata("1","sgAB",[2,3]);|
\end{center}

PD can be generated also 
by using the programming capability of Cindyscript 
which can be subsequently used in \ketcindy .  
For more details, 
see the example of \verb|Listplot()| 
in the command reference. 
Inclusion of too much elements into a single PD 
may cause some error. 
To prevent such error, 
PD should be divided into several PD 
each of which is composed of 200 elements or so. 


\newpage

\section{Cindyscript}

\subsection{Cindyscript editor}

Choose "Cindyscript" in the "Scripting" menu 
or push keybuttons Ctrl+9 (Windows) / Command+9 (Mac), 
then Cindyscript editor opens as shown below. 

\begin{layer}{150}{0}
\putnotese{7}{15}{\includegraphics[bb=0.00 0.00 703.04 425.02,width=14cm]{Fig/slotE.pdf}}
\arrowlineseg[16]{30}{20}{10}{90}
\putnotese{25}{5}{Slots}
\arrowlineseg[16]{50}{20}{10}{100}
\putnotese{42}{5}{Page name}
\arrowlineseg[16]{90}{20}{10}{110}
\putnotese{80}{5}{Font size}
%\arrowlineseg[16]{107}{20}{15}{140}
%\putnotese{80}{5}{描画面を前面に}
\arrowlineseg[16]{135}{20}{10}{110}
\putnotese{125}{5}{Run}
\arrowlineseg[16]{142}{20}{10}{100}
\putnotese{135}{5}{Help}
\putnotese{100}{35}{Text field}
\putnotese{100}{80}{Console}
\end{layer}

\vspace{105mm}

Commands can be input into preferred "slot". 
Specific timing for execution of commands 
is assigned to each slot. 
The slot for current work can be chosen 
only by clicking the corresponding tab in the menu. 
Users can add extra pages to each slot. 
For instance, 
when some initialization other than 
those included in \verb|KETlib| is needed, 
clicking the folder icon of "Initialization" makes a new page open 
in which extra commands can be input. 
The name of each page can be given 
by directly inputting it into the "Page name" column. 
The font size of the scripts can be tuned 
by changing the number in the "Font size" column. 
Frequently used slots are listed below. 

\begin{itemize}

\item 
Draw

The commnds in this slot are executed 
when some change, like movement of point, 
occurs in the Euclidean view. 
In \verb|templatebasic1.cdy|, 
the protoype page named \verb|figure| 
including the \ketcindy\ commnads 
like \verb|Ketinit();| and \verb|Windispg();| 
which are unconditionally necessary 
has been prepared.  
The \ketcindy\ commands for drawing 
should be input into this slot. 

\item 
Initialization 

The definitions of functions 
and the initial values of variables 
are input here. 
The commands in this slot are exected 
only once just after the "Run" button is clicked. 
Thus, the initial data in this slot is changed 
when some modifications are made in other slots. 
In \verb|templatebasic1.cdy|, 
the protoype page named \verb|KETlib| 
including the default setting of \ketcindy\ 
has been prepared. 

\item 
Key Typed

The commnds in this slot are executed 
when some key is pushed. 

\end{itemize}

Clicking "Run" button or pushing the keybuttons Shift+Enter 
makes the whole program be executed. 
The results derived from executing the function \verb|print()| 
and error messages are displayed on the console view 
which is put at the bottom part of Cindyscript editor. 
Each error and its location 
is displayed together with the message 
"WARNING" or "syntax error". 
The outputs displayed on the console 
can be copied to other usual text editors. 

Click the "Help" button, 
then reference manual of Cinderella opens 
as shown below. \\

\includegraphics[bb=0.00 0.00 712.04 577.03,width=14cm]{Fig/CindyhelpE.pdf}




\subsection{Input}


\subsection{Variables and constants}


\subsection{Frequently used commands}





% -------------- Making slide --------------

\newpage

\section{Making Slides}

\subsection{Overall Flow}

\ketcindy\ has functions to make slides for presentation. 
For this, follow steps below:
\begin{enumerate}[1)]
\item Copy and rename \verb|template2slide.cdy| in ketcindy folder to a work folder and double-click the file (here, we call sample.cdy). Then the followings will appear.
\begin{center}
\includegraphics[bb=0.00 0.00 743.00 632.00,height=60mm]{fig/slidescreen.pdf}\hspace{5mm}
\includegraphics[bb=0.00 0.00 802.00 633.00,height=60mm]{fig/slidescript.pdf}
\end{center}
\item Edit \verb|"Settitle"|, for example,\\
\verb|  Settitle([|\\
\verb|     "s{60}{20}{How To Use}",|\\
\verb|     "s{60}{50}{a \ketcindy\ member}",|\\
\verb|     "s{60}{60}{\ketcindy\ project}",|\\
\verb|     "s{60}{70}{Aug. 20th}"|\\
\verb|    ],["Color=[1,1,0,0]"]);|
\item Press button \verb|"Title"|, then the title page will be displayed. At the same time,  text file \verb|"sample.txt"| will be created if it does not exist. This \verb|sample.txt| is a template file for making slides.
\item Press button \verb|"Slide"|, then \ketcindy\ will make \verb|sample.tex| from \verb|sample.txt|, 
typeset it, and display \verb|sample.pdf| which contains slides for presentation.
\end{enumerate}
 
\subsection{Editing Text File}

\begin{enumerate}[1)]
\item Put \verb|//| at the last of each line.\\
\hspace*{10mm}Rm) Use \verb+||||+ for \verb|//|.
\item Commands are\\
\verb|    title::titleslidename(::wallpaper)//|\\
\verb|        Rem) Put only once at the first line.|\\
\verb|    main::(main title)//|\\
\verb|    new::(page title)//|\\
\verb|    enumerate//|\\
\verb|           =\begin{enumerate}|\\
\verb|        Rem) Add the option such as [(1)] using :: .|\\
\verb|    itemize//|\\
\verb|           =\begin{itemize}|\\
\verb|    layer::{xsize}{ysize}//|\\
\verb|           =\begin{layer}{xsize}{ysize}|\\
\verb|         Rem) "layer" is an environment defined in ketlayer.sty.|\\
\verb|    item::sentence//|\\
\verb|           =\item sentence|\\
\verb|    putnote::dir{xpos}{ypos}::filename(,scale)//|\\
\verb|           =putnotedir{xpos}{ypos}{\input{fig/filename}}||\\
\verb|         Rem) "putnote" is a command defined in ketlayer.sty|\\
\verb|    end//|\\
\verb|           =\end{itemize,enumerate,layer}|\\
\verb|    ...//|\\
\verb|          To insert a blank line.|\\
\verb|    Rem) Any other TeX command is available.|\\
\end{enumerate}

\subsection{Display of Page step by step}

\begin{enumerate}[1)]
\item Put just after new,\\
\verb|    %repeat=number of steps//|
\item Put at the head of each line as\\
\verb|    %[2,-]::sentence|\\
\verb|          display at all steps from 2|\\
\verb|    %[-,2]::sentence|\\
\verb|           display at all steps until 2|\\
\verb|    %[1..3,5]::sentence|\\
\verb|           display at steps of 1,2,3 and 5|
\item  Use \verb|%thin| to display with thin letters.\\
\verb|    %thin::[2,-]::sentence|
\item The dencity can be changed with Setslidebody or \verb|\setthin|.
\end{enumerate}

\subsection{Making Flip Animation}

\begin{enumerate}[1)]
\item Define function \verb|Mf(s)|, the state at s.
\item Put command \verb|Setpara| in the script editor as\\ 
\verb|    Setpara(subfolder,funcitonstr(mf(s)),range,options);|\\
\verb|        options=["m/r", "Div=25"];|
\item Describe in the text file as\\
\verb|    %repeat=, para=subfolder:{0}:s{60}{10}:input(:scale)//|
\item Press buttons \verb|ParaF| and \verb|Flip|, then \verb|subfolder| will be generated.
\item Press button \verb|Slide|.
\end{enumerate}

\subsection{Making Animation}

\begin{enumerate}[1)]
\item Add the following in the script editor\\ 
\verb|    Addpackage(["[dvipdfmx]{animate}"]);|
\item Add in the second option of Setpara,\\
\verb|    "Frate=num of frame in the second,"Scale=scale,"OpA=option of animation" |
\item Press buttons \verb|ParaF| and \verb|Anime|, then \verb|subfolder| will be generated.
\item Use \verb|\input|, not layer, to display.
\end{enumerate}

\subsection{Changing Style}

The default styles such as size and color of letters can be changed.
See \verb|KeTCindyReferenceE|.

\end{document}