\documentclass{ujarticle}
\usepackage{ketpic,ketlayer}
\usepackage{amsmath,amssymb}
\usepackage{graphicx}
\usepackage{xcolor}
\usepackage{bm,enumerate}
\usepackage[dvipdfmx,colorlinks=true,urlcolor=blue]{hyperref}

\setmargin{20}{20}{20}{20}

\西暦

\renewcommand{\labelitemi}{・}
\renewcommand{\labelitemii}{・}

\begin{document}

\begin{center}
KETCindyのインストール (Windows)
\end{center}

\hfill 修正日:\today

\begin{enumerate}[\bf\large 1.]
\item Cinderella, R, Maxima をインストールする.
 \begin{itemize}
 \item \url{https://beta.cinderella.de}  (Cinderella)
 \item \url{https://cran.r-project.org}   (R)
 \item \url{https://sourceforge.net/projects/maxima/files}  (Maxima)\\
\hspace*{1zw}・v5.38.0以前の方がv5.40.0以降より軽い.
 \item \url{https://www.sumatrapdfreader.org/download-free-pdf-viewer.html} (Sumatra)\\ \hspace*{10mm}注)SumatraはWindowsだけである.
 \end{itemize}
 注) ダブルクリックして,起動するかを確認する.

\item TeXをインストールしていない場合はインストールする.
 \begin{enumerate}[(1)]
 \item TeXLiveを推奨
    \begin{itemize}
    \item 2018以降ではketcindyが既に入っている.
    \end{itemize}
 \item KeTTeXはTeXLiveの軽量版
    \begin{itemize}
    \item kettex.exeを以下からダウンロードできる.\\
    \hspace*{10mm}\url{https://www.dropbox.com/s/fthw4btjqqs33tc/kettex.exe?dl=0}
    \item ダブルクリックしてできるkettexを\verb|C:\|に入れる.
    \end{itemize}
 \item 他のTeXの場合は,{\bf 3.}(2)を参照する.
 \end{enumerate}

\item KeTCindyのインストール
  \begin{enumerate}[(1)]
  \item ketcindyをCTAN(\url{https://ctan.org})からダウンロードする.\\
  \hspace*{10mm}ketcindyで検索 \verb|>| Package ketcindy \verb|>| download
    \begin{itemize}
    \item[注)]最新版は,Repositoryのサイト\\
        \hspace*{10mm}\url{https://github.com/ket­pic/ketcindy}\\
       から以下のようにダウンロードできる.\\
        \hspace*{10mm}Clone or download \verb|>| Download ZIP\\
        この場合は,ketcindy-masterになる.
    \item[注)]解凍してできるフォルダをデスクトップに移動する.
    \end{itemize}
  \item ketcindy(-master)/forWindowsを開く.
    \begin{itemize}
    \item[注)]Windows10の場合,セキュリティブロックを解除する.\\
\hspace*{5mm}右クリックして「プロパティ」>「セキュリティ」>「許可」
    \item[注)]他のTeX(TeXLive,W32tex以外)を使っている場合
      \begin{itemize}
      \item setketcindy.bat をテキストエディタで開く.
      \item パスを修正する.
      \end{itemize}
    \end{itemize}
  \item 次のバッチファイルを右クリック「管理者として実行」を選ぶ.\\
  \hspace*{3zw}\Ltab{5zw}{KeTTeX} setketcindy.bat\\
  \hspace*{3zw}\Ltab{5zw}{TeXLive} setketcindytexlive.bat\\
  \hspace*{3zw}\Ltab{5zw}{W32TeX} setketcindyw32J.bat\\
  \hspace*{3zw}\Ltab{5zw}{他のTeX} setketcindy.bat(要修正)
   \begin{itemize}
    \item scriptsの中身がTeXにコピーされる
    \item ketcindyのstyleファイルがTeXにコピーされmktexlsrが実行される.
    \item CinderellaのPluginsにKetcindyPlugin.jarをコピー,ketcindy.iniが作成される.
    \end{itemize}
  \item setwork.batをダブルクリック
    \begin{itemize} 
    \item 作業ディレクトリketcindyがユーザホームに作成される.
    \item タイプセットの方法(TeXの種類)\\
    \hspace*{10mm}通常は,platex (p)またはuplatex(u)を選ぶ.
    \item R,\ Maximaでは,バージョン番号(3.4.2,\ 5.37.3など)を入れる.
    \item ketcindyフォルダにworkフォルダの中身がコピーされる.
    \item \verb|.ketcindy.conf|(編集可能)がユーザホームに作成される.\\
    \hspace*{10mm}注)TeXを切り替えるときなどはこのファイルを修正する.
    \item マニュアルもコピーされる.
    \item 作業ディレクトリにketincy.confの雛形がコピーされる.
    \item KeTCindyを立ち上げたとき,設定ファイルは次の順に読み込まれる.
      \begin{enumerate}[1)]
      \item ketoutset.txt
      \item ユーザホームの\verb|.ketcindy.conf|
      \item 作業ディレクトリketcindyの ketcindy.conf
      \end{enumerate}
    \end{itemize}
  \end{enumerate}
  \item KeTCindyのテストラン
    \begin{enumerate}[(1)]
    \item 作業ディレクトリketcindyを開く.
    \item ketcindyの中のtemplate1basic.cdyを選び,「情報を見る」を開く.
      \begin{itemize}
      \item アプリケーションが所定のCinderella2になっていることを確かめる.
      \item「情報」を閉じて,template1basic.cdyをダブルクリックする.
      \item 画面に白い枠が出れば,ライブラリの読み込みは成功.
      \end{itemize}
    \item スクリーンの左上部にあるFigureボタンを押して,PDFが表示されれば成功.
     \begin{itemize}
     \item[注)]PDFの表示後,ターミナル画面を閉じるようにするには
       \begin{itemize}
        \item アプリケーション \verb|/| ユーティリティ \verb|/| ターミナルを開く
        \item トップメニューから\\
          \hspace*{5mm}ターミナル>環境設定 \verb|>|(プロファイル)\verb|>| シェル\\
          \hspace*{10mm}「シェルが正常に終了した場合閉じる」を選択
        \end{itemize}
    \item[注)]インストール用のketcindy(-master)フォルダは削除した方が混乱しない.
     \end{itemize} 
  \end{enumerate}

\item TeXworksの設定(kettexの場合)
  \begin{itemize}
  \item \url{https://github.com/TeXworks/texworks/releases/} からダウンロードできる.
  \item TeXworksを立ち上げる
  \item TeXworks \verb|>| ファイル \verb|>| 設定 \verb|>| タイプセット
  \item 上の欄(パス)に以下を追加\\
  \hspace*{5mm}\verb|C:\kettex\texlive\bin\win32|\\
  \hspace*{10mm}注) 上の行を上の欄の先頭になるように移動する.
  \item 下の欄の横にある + をクリック
    \begin{itemize}
    \item 名前:uplatex(ptex2pdf)またはplatex(ptex2pdf)
    \item プログラム : ptex2pdf
    \item 引数:\\
    \hspace*{10mm} \verb|-u|(uplatexの場合のみ)\\
    \hspace*{10mm} \verb|-l|\\
    \hspace*{10mm} \verb|-ot|\\
    \hspace*{10mm}  \verb|$synctexoption|\\
    \hspace*{10mm}  \verb|$fullname|
    \item[]OKボタンを押し,デフォルトを変更してOKボタンを押す.
    \end{itemize}
  \end{itemize}
\item gccのインストール
  \begin{itemize}
    \item 曲面描画のためには, gccが必要である.
    \item minGWのホームページ\url{http://www.mingw.org}から\\
    \hspace*{10mm}download \verb|>| Install \verb|>| mingw-get-setup.exe\\
    をダウンロードして実行\\
    \hspace*{10mm}注) パッケージは,mingw32-base, mingw32-gcc-g++だけでよい.
  \end{itemize}

\item 手動でインストールする場合(KeTTeX)\\
\hspace*{1zw}注)他のTeXの場合は,適宜パスを置き換える.\\
\hspace*{3zw}\verb|C:\kettex\texlive| $=>$ \verb|C:\texlive\2018| など
  \begin{enumerate}[(1)]
  \item \verb|ketcindy(-master)\ketcindyfolder|を開いておく.
  \item scriptsフォルダの中身を以下にコピーする.\\
 \verb|C:\kettex\texlive\texmf-dist\scripts\ketcindy|
  \item styleフォルダの中身を以下にコピーする.\\
 \verb|C:\kettex\texlive\texmf-dist\tex\latex\ketcindy|
  \item コマンドプロンプトで以下を実行する\\
  \hspace*{1zw}\verb|C:\kettex\texlive\bin\win32\mktexlsr|
  \item \verb|C:\Program files\Cinderella2(.exe)|をダブルクリック
  \item メニューから 「スクリプト」>「プラグインを開く」を選び,Pluginsフォルダを開く.
  \item \verb|scripts\ketjava\KetCindyPlugin.jar|をPluginsにコピーする.
  \item \verb|ketcindy(-master)\forWindows\Forsettingmanually\ketcindy.ini|をPluginsにコピーする.\\
  \hspace*{2zw}注)他のTeX の場合は,ketcindy.iniをテキストエディタで開いて修正する.
  \item Puginsを閉じ,Cinderella2を終了する.
  \item workを適当な場所にコピーして,名前を変更する.\\
\hspace*{2zw}注)標準は パス:ユーザーホーム(の直下) 名前:ketcindy
  \item 上の作業ディレクトリ(ketcindy) に \verb|doc\ketmanual| のマニュアルをコピーする.
  \item forWindowsにある ketcindy.conf をテキストエディタで開き,適宜修正する.
  \item ユーザホームにコピーして,名前を \verb|.ketcindy.conf| に変える.
  \end{enumerate}
\end{enumerate}

\end{document}