\documentclass[a4j]{jarticle}
%あいう

\usepackage{ketpic,ketlayer}
\usepack

\input{/tools/reference/refstyle}

\setmargin{20}{20}{15}{25}

\begin{document}

\begin{flushright}
2016.12.11
\end{flushright}


\begin{center}
{\bf \huge \ketpic\ --ketpic--}\vspace{3mm}\\
\end{center}

\section{スタイル名}
\tab{ketpic.sty,\ ketpic2e.sty(pict2e必要)}

\section{概要}

\bs Width,\ \bs Height,\ \bs Depthを定義する.

また,10個の一時カウンタ ketpicctra,\ $\cdots$,\ ketpicctrj を定義する.

\section{マクロ一覧}

\subsection{プリアンブル用}

%\tab{\bs usepack}amsmath, amssymb,graphicx,colorを読み込む

\tab{\bs setmargin\br{left}\br{right}\br{top}\br{bottom}}\\
\tab{}マージン設定(単位はmm)\\
\rei \bs setmargin\br{20}\br{20}\br{15}\br{25}

\subsection{本文用}

\tab{\bs ketpic}ロゴ \ketpic

\tab{\bs Ltab\br{W}\br{S}}Wの幅を確保して左寄せでSを書く\\
\rei \bs Ltab\br{5cm}\br{abc}xyz

\tab{\bs Rtab\br{W}\br{S}}Wの幅を確保して右寄せでSを書く

\tab{\bs Ctab\br{W}\br{S}}Wの幅を確保して中央寄せでSを書く

%\tab{\bs vbackward\br{C}}文字Cの高さ分だけ上に戻る\\
%\rei \bs vbackward\br{\bs includegraphics\br{ex.eps}}

%\tab{\bs vforward\br{C}}文字Cの高さ分だけ下に進む

\tab{\bs ketcalcwidth[0]\br{C}}文字Cの幅を単位長で計ってカウンタ1に返す\\
\chuu オプションが1のときは,値を表示

\tab{\bs ketcalcheight[0]\br{C}}文字Cの高さを単位長で計ってをカウンタ1に返す\\
\chuu オプションが1のときは,値を表示

\tab{\bs ketcalcdepth[0]\br{C}}文字Cの深さを単位長で計ってカウンタ1に返す\\
\chuu オプションが1のときは,値を表示

\tab{\bs ketcalcwh\br{C}}文字Cの横幅と縦幅をmm単位で返す\\
\chuu \{横幅\}\{縦幅\}の形式で

\subsection{シンボル}

\begin{layer}[120]{120}{0}
\putnotee{5}{5}{\dangerbendmark}
\putnotee{10}{10}{\cautionmark}
\putnotee{15}{15}{\circlemark{1}}
\putnotee{20}{25}{\circleshade{1}{0.5}}
\putnotee{5}{30}{\NEarrow,\ \NELarrow,\ \NERarrow}
\end{layer}

\tab[6cm]{\bs dangerbendmark[size]}ブルバキの「危険な曲がり角」

\tab[6cm]{\bs cautionmark[size]}注意書きのマーク

\tab[6cm]{\bs circlemark[thickness]\br{size}} 円\\
\chuu サイズ1= 4mm(直径)

\tab[6cm]{\bs circleshade[thickness]\br{size}\br{density}}中塗りの円

\tab[6cm]{\bs NEarrow[倍率],\bs NELarrow[倍率]など} 増減矢印

\end{document}
